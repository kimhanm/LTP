\section{Introduction}

Introduction is stolen from \url{https://golem.ph.utexas.edu/category/2021/06/duality_in_transport_problems.html}.


Suppose we want to transport some material from $s$ suppliers $S_{1}, \ldots, S_{s}$ to $r$ receivers $R_{1}, \ldots, R_{n}$, where the supply available from supplier $S_i$ is $\sigma_i$ and the demand at receiver $R_j$ is $\rho_j$.


If the cost of moving one part of material from $S_i$ to $R_j$ is $k_{ij} \in \R_{\geq 0}$, 
then we are interested in finding a \textbf{transport plan}, which can be given by a matrix $(\alpha_{ij})_{i,j}$ where $\alpha_{ij}$ denotes the amount of material moved from supplier $S_i$ to receiver $R_j$, such that
\begin{align*}
  \forall j: \sum_{i} \alpha_{ij} \geq \rho_j, \quad \forall  i: \sum_{j} \alpha_{ij} \leq \sigma_i
\end{align*}
and the total cost of the transport plan $\sum_{i,j}k_{ij} \alpha_{ij}$ is minimized.


Because the cost of moving material is positive, it is clear that the demand constraint is an equality $\sum_{i} \alpha_{ij} = \rho_j$.


\begin{ex}[]
Let's say that there are three supplies and three receivers, with
\begin{align*}
  \sigma = (\sigma_1,\sigma_2,\sigma_3) = (350,100,200), \quad \rho = (\rho_1,\rho_2,\rho_3) = (200,200,250)
\end{align*}
and where the transport cost is given by
\begin{align*}
  K =  \begin{pmatrix}
    39 & 44 & 47\\
    22 & 22 & 30\\
    14 & 25 & 29
  \end{pmatrix}
\end{align*}
\end{ex}
