
%   Macros
%       Math
%           Symbols
\newcommand{\N}{\mathbb{N}}
\newcommand{\Z}{\mathbb{Z}}
\newcommand{\Q}{\mathbb{Q}}
\newcommand{\R}{\mathbb{R}}
\newcommand{\C}{\mathbb{C}}
\newcommand{\K}{\mathbb{K}}
\newcommand{\QED}{\ensuremath{\square}}
\newcommand{\realS}{\ensuremath{\mathbb{S}}}
\newcommand{\obda}{o.B.d.A.}
\newcommand{\dbrack}[1]{\llbracket #1 \rrbracket}
%           Expressions
\newcommand{\vfrac}[2]{\ensuremath{\frac{#1}{#2}}}
\newcommand{\integ}[2]{\ensuremath{\int_{#1}^{#2}}}
\newcommand{\summe}[2]{\underset{#1}{\overset{#2}{\sum}}}
\newcommand{\ilsumme}[2]{\sum_{#1}^{#2}}
\newcommand{\product}[2]{\underset{#1}{\overset{#2}{\prod}}}
\newcommand{\vprod}[3]{\ensuremath{\prod_{#1}^{#2}{#3}}}
\newcommand{\abs}[1]{\vert #1 \vert}
\newcommand{\calO}{\ensuremath{\mathcal{O}}}
\newcommand{\calU}{\ensuremath{\mathcal{U}}}
\newcommand{\fracexp}[3]{\ensuremath{\frac{#1^{#3}}{#2^{#3}}}}
\newcommand{\bracket}[1]{\ensuremath{\left(#1\right)}}

 
%       Tikz
\newcommand{\tikznode}[2]{%
\ifmmode%
\tikz[remember picture,baseline=(#1.base),inner sep=0pt] \node (#1) {$#2$};%
\else
\tikz[remember picture,baseline=(#1.base),inner sep=0pt] \node (#1) {#2};%
\fi}
%tnode, trem

\newcommand{\twodaxisgrid}[6]{ % x-min, x-max, y-min, y-max, x-name y-name
    \tikzset{x={(0:1cm)}, y={(90:1cm)}}
    \draw[->] (#1,0) -- (#2,0) node[right]{#5};
    \draw[->] (0,#3) -- (0,#4) node[above]{#6};
    \draw[very thin,color=gray] (#1, #3) grid (#2,#4);
}
\newcommand{\twodaxis}[6]{ % x-min, x-max, y-min, y-max, x-name y-name
    \tikzset{x={(0:1cm)}, y={(90:1cm)}}
    \draw[->] (#1,0) -- (#2,0) node[right]{#5};
    \draw[->] (0,#3) -- (0,#4) node[above]{#6};
}
\newcommand{\coord}[4]{% {Name}{x}{y}{position}
    \coordinate[label=#4:#1] (#1) at (#2,#3);
}
\def\centerarc[#1](#2)(#3:#4:#5)% Syntax: [draw options] (center) (initial angle:final angle:radius)
    { \draw[#1] ($(#2)+({#5*cos(#3)},{#5*sin(#3)})$) arc (#3:#4:#5); }
%       Formatting
%           Formats
\def\Loesung{\begin{center}\textbf{Lösung}\end{center}}
\newcommand{\Korollar}{\textbf{Korollar \quad}}
\newcommand{\Beispiel}{\textbf{Beispiel\quad}}
\newcommand{\Beweis}{\textbf{Beweis \quad}}
\newcommand{\Proposition}{\textbf{Proposition \quad}}
\newcommand{\Satz}{\textbf{Satz\quad}}
\newcommand{\Definition}{\textbf{Definition}\quad}
\newcommand{\Lemma}{\textbf{Lemma}\quad}
\newcommand{\boldline}[1]{\textbf{\underline{#1}}}
%           Shortcuts
\newcommand{\XinR}{X \subseteq \R}
\newcommand{\realfunc}[2]{{#1}: {#2} \rightarrow \R}
\newcommand{\realfuncab}[1]{\ensuremath{{#1}: [a,b] \rightarrow \R}}
\newcommand{\intervalldelta}[1]{\ensuremath{(#1-\delta, #1+\delta)}}

% Integral
\newcommand{\Zerlegung}[2]{\ensuremath{a = {#1}_0 < {#1}_1 < \ldots < {#1}_{#2} = b}}
\newcommand{\Intervallstueck}[2]{\ensuremath{({#1}_{#2} - {#1}_{#2 -1})}}
\newcommand{\openintervalindex}[2]{\ensuremath{({#1}_{#2-1},{#1}_{#2})}}

\newcommand{\Integralab}{\int_{a}^{b}}
\newcommand{\Summeaufn}[1]{\summe{#1=1}{n}}
\newcommand{\TFab}{\ensuremath{\mathcal{TF}([a,b])}}
\newcommand{\intervalab}{[a,b]}


\newcommand{\xoabs}[1]{\abs{#1 - x_0}}
\newcommand{\fxoabs}[1]{\abs{#1 - f(x_0)}}
\newcommand{\defstetigkeitx}{
\begin{align*}
	\forall \epsilon > 0 \exists \delta > 0 \xoabs{x} < \delta \implies \fxoabs{f(x)} < \epsilon
\end{align*}
}

\newcommand{\linearkomb}[2]{\ensuremath{#1_1 v_1 + \ldots + #1_n v_{#2}}}
\newcommand{\customlinearkomb}[4]{\ensuremath{#1_{#2} #3_{#2} + \ldots + #1_{#4} #3_{#4}}}
\newcommand{\listindex}[2]{\ensuremath{#1_1, \ldots #1_{#2}}}
\newcommand{\customindex}[3]{\ensuremath{#1_{#2}, \ldots #1_{#3}}}

% Metrische Räume
\newcommand{\folge}[2]{({#1})_{{#2} = 0}^{\infty}}
\newcommand{\xfolge}{(x_n)_{n=0}^\infty}
\newcommand{\yfolge}{(y_n)_{n=0}^\infty}
\newcommand{\zfolge}{(z_n)_{n=0}^\infty}
\newcommand{\sfolge}{(s_n)_{n=0}^\infty}
\newcommand{\afolge}{(a_n)_{n=0}^\infty}
\newcommand{\bfolge}{(b_n)_{n=0}^\infty}
\newcommand{\cfolge}{(c_n)_{n=0}^\infty}
\newcommand{\superiorlim}{\underset{\ensuremath{n\rightarrow\infty}}{\text{lim sup }}}
\newcommand{\inferiorlim}{\underset{\ensuremath{n\rightarrow\infty}}{\text{lim inf }}}
\newcommand{\supremum}{\text{sup}}
\newcommand{\infimum}{\text{inf}}
\newcommand{\teilfolge}[1]{\ensuremath{(#1_k)_{k=0}^\infty}}
\newcommand{\inflim}[1]{\underset{#1\rightarrow\infty}{\lim}}
\newcommand{\metricS}[1]{\ensuremath{(#1,d)}}
\newcommand{\cauchy}{\ensuremath{\mathcal{C}}}
\newcommand{\inverse}[1]{{#1}^{-1}}
\newcommand{\vfolge}{(v_n)_{n=0}^\infty}
\newcommand{\wfolge}{(w_n)_{n=0}^\infty}
\newcommand{\ufolge}{(u_n)_{n=0}^\infty}
\newcommand{\Kernel}{\text{Ker}} % Kernel
\newcommand{\Image}{\text{Im}} % Image
\newcommand{\rang}{\text{rang}} % Rang
\newcommand{\spn}{\text{span}} % Span
\newcommand{\id}{\text{id}} % Identity
\newcommand{\End}{\text{End}} % Endomorphism
\newcommand{\Eig}{\text{Eig}}

%Grenzwerte
\newcommand{\xlim}{{\underset{x\rightarrow x_0}{\lim}}}
\newcommand{\einschlim}[1]{{\underset{{#1 \neq x_0}}{\underset{x\rightarrow x_0}{\lim}}}}
\newcommand{\rightlim}{\ensuremath{{\underset{x_0 < x}{\underset{x\rightarrow x_0}{\lim}}}}}
\newcommand{\neglim}[1]{\underset{#1\rightarrow -\infty}{\lim}}

% Exp
\newcommand{\exponential}[2]{\ensuremath{\bracket{1 + \frac{#1}{#2}}^{#2}}}
\DeclarePairedDelimiter\ceil{\lceil}{\rceil}
\DeclarePairedDelimiter\floor{\lfloor}{\rfloor}

% Norm
\newcommand{\Norm}[1]{||#1||}


% Differentialrechnung
\newcommand{\pderiv}[2]{\frac{\partial #1}{\partial #2}}
\newcommand{\eintegral}[3]{\left[#1\right]_{#2}^{#3}}