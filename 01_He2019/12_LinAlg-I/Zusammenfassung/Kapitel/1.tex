\section{Gruppen, Ringe, Körper, Polynome, Matrizen}
\begin{mdframed}
Eine \textbf{Gruppe} ist ein Tupel $(G,*,e)$ bestehend aus einer Menge $G$, einer Verknüpfung $*$ und einem neutralem Element $e \in G$ sodass gilt:
\begin{itemize}
    \item[G1)]  Assoziativität: $a *(b*c) = (a*b)*c = a*b*c, \forall a,b,c \in G$
    \item[G2)] Neutrales Element: $e*a=a, \forall a \in G$
    \item[G3)] Inverses Element: $\exists a' \in G: a'*a = e, \forall a \in G$ 
\end{itemize}
    
\end{mdframed}
Eine Gruppe heisst \textbf{abelsch} falls 
\begin{itemize}
    \item[G4)] Kommutativität: $\forall a,b \in G: a*b = b*a$
\end{itemize}
\emph{Bemerkung:}
\begin{enumerate}[{(}a{)}]
    \item $e$ ist eindeutig und rechts-neutral
    \item Das Inverse ist eindeutig und auch rechts-inverse
    \item Es gelten die Kürzungsregeln:\\
     $a * \tilde{x} = a * x \implies \tilde{x} = x$\\
     $\tilde{x} * a = x * a \implies \tilde{x} = x$
\end{enumerate}
Sei $(G,\cdot,e)$ eine Gruppe. Eine nichtleere Teilmenge $H \subseteq G$ heisst \textbf{Untergruppe}, falls gilt
\begin{itemize}
    \item $\forall a, b \in H: a\cdot b \in H$
    
    \item $\forall a \in H: a^{-1} \in H$
\end{itemize}
Seien $(G,\cdot_G,e_G), (H,\cdot_H,e_H)$ Gruppen. Eine Abbildung $\phi: G \rightarrow H$ heisst Gruppenhomomorphismus, wenn gilt
\begin{align*}
    \forall a,b \in G: \phi(a \cdot_G b) = \phi(a)\cdot_H \phi(b)
\end{align*}
\emph{Bemerkung:}
\begin{itemize}
    \item $\phi(e_G) = e_H$
    \item $\phi(a^{-1}) = \phi(a)^{-1}$
    \item falls $\phi$ bijektiv ist, ist $\phi^{-1}: H \rightarrow G$ auch ein Gruppenhomomorphismus
\end{itemize}

\begin{mdframed}
Ein \textbf{Ring} ist ein Tupel $(R,+,\mathbf{\cdot},0)$ bestehend aus einer Menge $R$, zwei Verknüpfungen $+$ und $\cdot$, und einem ausgezeichnetem Element $0 \in R$, sodass gilt:
\begin{itemize}
    \item[R1)] $(R,+,0)$ ist eine abelsche Gruppe
    \item[R2)] die Multiplikation $\cdot$ ist assoziativ.
    \item[R3)] Distributivität $a \cdot (b+c) = a\cdot b + a\cdot c$ \quad $(a+b)\cdot c = a\cdot c + b\cdot c$  
\end{itemize}
\end{mdframed}
Ist die Multiplikation kommutativ, so heisst $(R,+,\mathbf{\cdot},0)$ \textbf{kommutativer Ring}. Hat ein Ring dazu noch ein Einselement $1 \in R$, sodass gilt $\forall a \in R: 1 \cdot a = a\cdot 1 = a$, so heisst es \textbf{Ring mit Eins}\\
Ein Ring heisst \textbf{Ringteilerfrei}, wenn 
\begin{align*}
    \forall a,b \in R: a\cdot b = 0 \implies a= 0 \lor b = 0
\end{align*}
Eine Teilmenge $R' \subseteq R$ heisst \textbf{Unterring}, falls $(R',+,0)$ eine Untergruppe ist: $ (a,b \in R' \implies a + b \in R' \land - a \in R')$ 
\begin{mdframed}
Ein \textbf{Körper} ist ein Tupel $(K, +, \cdot, 0,1)$ mit einer Menge, zwei Verknüpfungen, $+$, $\cdot$ und zwei ausgezeichneten Elementen $0, 1 \in R$, sodass gilt:
\begin{itemize}
    \item[K1)] $(K,+,0)$ ist eine abelsche Gruppe
    \item[K2)] $(K^*, \cdot,1)$ ist eine abelsche Gruppe
    \item[K3)] Distribivgesetz  
\end{itemize}
\end{mdframed}
\textbf{Rechenregeln}
\begin{enumerate}[{(}a{)}]
    \item $1 \neq 0$
    \item $0 \cdot a = a \cdot 0 = 0$
    \item Nullteilerfreiheit
    \item $a \cdot (-b) = (-a)\cdot b = -(a\cdot b)$ \quad $(-a)\cdot (-b) = a \cdot b$
    \item $x\cdot a = \tilde{x}\cdot a \land a \neq 0 \implies x = \tilde{x}$
\end{enumerate}
Ist $R$ ein Ring mit $1$, so ist seine \textbf{Charakteristik} die Zahl
\begin{align*}
    \text{char}(R):= \begin{cases}
        0, \text{\quad falls } n\cdot 1 \neq 0, \forall n \in \N^*\\
        \min\{n \in \N^*: n\cdot 1 = 0\}, \quad \text{sonst}
    \end{cases}
\end{align*}
$\Z_{/p\Z}$ ist genau dann ein Körper, wenn $p$ Prim ist.
\textbf{Lemma}: \quad ist $K$ ein Körper, so ist $\text{char}(K)$ entweder $0$ oder Prim.
\begin{mdframed}

Sei $K$ ein Körper. Ein \textbf{Polynom} $f$ in einer Variable $T$ und Koeffizienten in $K$ ist ein Ausdruck der Form
\begin{align*}
    f(T) = \summe{k = 0}{n} a_{k} T^{k} = a_0 \cdot T^0 + a_1\cdot T + a_2 \cdot T^2 + \ldots + a_n \cdot T^n
\end{align*}
\end{mdframed}
$a_n \neq 0$ heisst Leikoeffizient.\\
Der Grad von $f$ ist $\text{deg}(f) := \begin{cases}
    -\infty, \quad \text{falls } f = 0 \\
    \max\{k \in N, a_k \neq 0\}, \quad \text{sonst}
\end{cases}$
Man schreibt $K[T]$ für die Menge aller Polynome über $K$. \quad $K[T]$ ist mit der Polynommultiplikation und der Polynomaddition ein Kommutativer Ring (mit Eins) und es gilt $\deg(p\cdot q) = \deg(p) + \deg(q)$
\mdfsetup{backgroundcolor=black!10}
\begin{mdframed}
\textbf{Satz} (Polynomdivision): \quad Sind $f,g \in K[T], g \neq 0$ So gibt es eindeutige Polynome, $q$(Quotient),$r$(Rest) $\in K[T]$, sodass $f = q \cdot g + r$ und $\text{deg}(r) <\text{deg}(g)$
\end{mdframed}
\boldline{Nullstellen von Polynomen} Sei $K$ ein Körper, $f \in K[T]$
\begin{enumerate}[{(}1{.)}]
    \item	Ist $\lambda \in K$ eine Nullstelle von $f$, so gibt es ein eindeutiges Polynom $g \in K[T]$ mit $f = (T-\lambda)\cdot g$ und $\text{deg}(g) = \text{deg}(f)-1$

    \item Sei $k$ die Anzahl Nullstellen von $f$. Ist $f\neq 0$, so ist $k\leq \text{deg}(f)$. Ist $k$ unendlich, so ist die Abbildung
    \begin{align*}
        \tilde{\cdot}: K(T) &\rightarrow \text{Abb}(K,K)\\
        f &\mapsto \tilde{f}
    \end{align*}
    injektiv. Ist $f\neq 0$ und $\lambda \in K$, so ist $\mu(f;\lambda)$ die Vielfachheit der Nullstelle $\lambda$ in $f$.
    \begin{align*}
        \mu(f;\lambda) := \max\{r \in \N \big\vert f(\lambda) = f^2(\lambda) = \ldots = f^{r-1}(\lambda) = 0\}
    \end{align*}
    \item Sind $\lambda_1, \ldots, \lambda_k \in K$ die verschiedenen Nullstellen von $f$ und $r_i = \mu(f,\lambda_i)$ ihre Vielfachheiten, so ist 
    \begin{align*}
        f = (T-\lambda)^{r_1} \cdots (T-\lambda_k)^{r_k}\cdot g \text{\quad mit deg}(g) = \text{deg}(f) - (r_1 + \ldots + r_k) \text{und $g$ ohne Nullstellen}
    \end{align*}
    Falls deg$(g) = 0$, zerfällt $f$ in Linearfaktoren. 
    
    \item \textbf{Fundamentalsatz der Algebra}: Jedes Polynom $f \in \C[t]$ mit deg$(f) > 0$ hat mindestens eine Nullstelle in $\C$ 
    
    \item Jedes Polynom über $\C$ zerfällt in Linearfaktoren.
    
    \item Ist $f \in \R[t]$ und $\lambda \in \C$ eine Nullstelle von $f$, so ist $\overline{\lambda}$ auch eine Nullstelle von $f$ und es gilt $\mu(\lambda) = \mu(\overline{\lambda})$
    
    \item Jedes Polynom $f \in \R[t]$ mit deg$(f) = n \geq 1$ besitzt eine Zerlegung
    \begin{align*}
        f = a \cdot (T-\lambda_1) \cdots (T - \lambda_r) \cdot g_1 \cdots g_m
    \end{align*}
    mit $a, \lambda_1, \ldots , \lambda_r \in \R, a \neq 0, g_1, \ldots, g_m \in \R[t]$ normierte Polynome mit Grad $2$ ohne relle Nullstellen.
    
    \item Jedes Polynom $f \in \R[t]$ von ungeradem Grad hat mindestens eine Nullstelle
\end{enumerate}
