\section{Differential Geometry}

\subsection{Implicit function theorem}


\begin{bthm}[Implicit function theorem]
  Let $U \subseteq \R^{n} \times \R^{m}$ open, $(x_0,y_0) \in U$, $F: U \to \R^{m}$ continuous that satisfies
  \begin{enumerate}
    \item $F(x_0,y_0) = 0$
    \item For all $k = 1, \ldots, m$ the partial derivatives $\del_{y_k}F: U \to \R^{m}$ exist and are continuous.
    \item The matrix $A = \left(
        \del_{y_k}F_j(x_0,y_0))
    \right)_{j,k} \in \R^{m \times m}$ is invertible.
  \end{enumerate}
  Then the equation $F(x,y) = 0$ can ``solved for $y$'' as a function of $x$ in a small region around $x_0$.

  That is, there exist $r,s >0$ and a continuous function $f:B(x_0,r) \to B(y_0,s)$ such that for all $(x,y) \in B(x_0,r) \times B(y_0,s)$:
  \begin{align*}
    F(x,y) = 0 \iff y=f(x)
  \end{align*}

  Additionally if $F \in C^{d}(U)$, then $f \in C^{d}(B(x_0,r))$ and has derivative
  \begin{align*}
    Df(x) = - \left(
      (D_yF)(x,f(x))
    \right)^{-1} \circ (D_xF)(x,f(x))
  \end{align*}
\end{bthm}



\begin{bthm}[Inverse function theorem]
Let $U \subseteq \R^{n}$ open, $f: U \to  \R^{n}$ $d$-times differentiable and $x_0 \in U$ such that $Df(x_0)$ is invertible.

Then in a small region round $x_0$, $f$ is invertible. That is, there exists open neighborhoods $U_0 \subseteq U$ of $x_0$ and $V_0 \subseteq \R^{n}$ if $y_0 = f(x_0)$. such that $f|_{U_0}$ is bijective.

Its inverse $f^{-1}$ has derivative
\begin{align*}
  (Df^{-1})(y) = (Df(x))^{-1}
\end{align*}
for all $x \in U_0$ and $y = f(x) \in V_0$.


\end{bthm}



\subsection{Submanifolds}

\begin{bdfn}[]
  A subset $M \subseteq \R^{n}$ is called a $k$-dimensional \textbf{smooth submanifold} (of $\R^{n}$), if $M$ is locally isomorphic to $\R^{k}$.

  That is, for every point $p \in M$, there exists an open neighborhood $U_p \subseteq \R^{n}$ of $p$ and a diffeomorphism $\phi_p: U_p \to V_p = \phi_p(U_p) \subseteq \R^{n}$ such that
  \begin{align*}
    \phi_p(U_p \cap M) = \{y \in V_p \big\vert y_i = 0 \text{ for all } i > k\}
  \end{align*}
  We call $\phi$ a \textbf{map} of $M$ around $p$ and its inverse $\phi^{-1}: V_p \to  U_p$ a \textbf{parametrisation} of $M$ around $p$.
\end{bdfn}


\begin{bthm}[Constant rank theorem]
Let $U \subseteq \R^{n}$ open, $F: U \to  \R^{m}$ smooth.
If for all $p \in U$ the derivative $DF(p): \R^{n} \to  \R^{m}$ is surjective, then the niveau set of zeros
\begin{align*}
  M = \{p \in U \big\vert F(p) = 0\}
\end{align*}
is an $(n-m)$-dimensional smooth submanifold.
\end{bthm}

\begin{bdfn}[]
Let $M \subseteq \R^{n}$ be a $k$-dimensional submanifold.
\begin{itemize}
  \item The \textbf{tangent space} of $M$ at $p \in M$ is the $k$-dimensional vector space
    \begin{align*}
      T_pM = \{\gamma'(0) \big\vert \gamma: (-1,1) \to  M \text{ differentiable with } \gamma(0) = p\} \subseteq \R^{n}
    \end{align*}
  \item The \textbf{tangent bundle} of $M$ is the collection of tangent spaces of every point $p \in M$:
    \begin{align*}
      TM = \{(p,v) \in \R^{n} \times \R^{m} \big\vert p \in M, v \in T_pM\}
    \end{align*}
  \item The \textbf{canonical projection} is the map
    \begin{align*}
      \pi: TM \to  M, \quad (p,v) \mapsto p
    \end{align*}
    A map $s: M \to  TM$ is called a \textbf{section} (or vector field) if $\pi \circ s = \id_M$.
    \begin{center}
    \begin{tikzcd}[ ] %\arrow[bend right,swap]{dr}{F}
      M \arrow[]{r}{s}& TM \arrow[]{r}{\pi}& M
    \end{tikzcd}
    \end{center}
\end{itemize}
\end{bdfn}

\begin{bprop}[]
  Let $U \subseteq \R^{n}$ open, $F: U \to  \R^{m}$ smooth and $M = F^{-1}(0)$.
  If for all $p \in M$ the derivative $DF(p): \R^{n} \to  \R^{m}$ is surjective,
  then
  \begin{align*}
    TM = \{(p,v) \in U \times \R^{n} \big\vert F(p) = 0 \text{ and } DF(p)(v) = 0\}
  \end{align*}
  and in particular, for all $p \in M$ 
  \begin{align*}
    T_p M = \Ker DF(p)
  \end{align*}
\end{bprop}
