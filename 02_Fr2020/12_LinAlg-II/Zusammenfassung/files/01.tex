\section{LinAlg I}

% \begin{definition}{Charakteristische Polynom}
%     Sei $A \in M(n\times n,K)$. Das \textbf{charakteristische Polynom} von $\mathbf{A}$ ist
%     \begin{align*}
%         P_A(t) = \det(A - t \cdot E_n)
%     \end{align*}
%     Für $F$ in $\End(V)$ mit $\dim V = n < \infty$ mit einer Basis $\mathcal{B} = (v_{1}, \ldots, v_{n})$ ist das charakteristische Polynom (unabhängig von der Wahl der Basis)
%     $P_F(t) = P_{\mathcal{M}_{\mathcal{B}}(F)}(t)$.
% \end{definition}
% Ist $P_A(t) = \sum_{i = 0}^{n} a_i t^i$ das charakteristische Polynom, so hat es folgende Eigenschaften
% \begin{itemize}
    % \item   $\lambda$ ist Nullstelle von $P_A(t) \Leftrightarrow \lambda$ ist Eigenwert von $A$.
    % \item   (Cayley-Hamilton)   $P_F(A) = 0$
    % \item   $a_n = (-1)^n, a_0 = \det(A), a_{n-1} = (-1)^{n-1} \trace(A)$
    % \end{itemize}
    Ist $P_A(t) := \det(A - t \cdot E_n) = \sum_{i = 0}^{n}\alpha_it^i$ das Charakteristische Polynom einer Matrix $A \in M(n\times n,K)$, so gilt
\begin{align*}
    a_n = (-1)^n, a_0 = \det(A), a_{n-1} = (-1)^{n-1} \trace(A), \quad \trace(AB) = \trace(BA)
\end{align*}

% \begin{itemize}
%     \item   Die Eigenwerte von $F$ sind genau die Nullstellen von $P_F(t)$.
%     \item   Die \textbf{algebraische Vielfachheit} eines Eigenwertes $\lambda$ von $F \in \End(V)$ ist die Vielfachheit der Nullstelle $\lambda$ von $P_F(T)$
%     \item   Die \textbf{geometrische Vielfachheit} von $\lambda$ ist 
%     \begin{align*}
%         \dim \Eig(F;\lambda) = \dim \Ker(F - \lambda \id_V)
%     \end{align*}
%     und es gilt $1 \leq \text{ geometrische Vielfachheit } \leq \text{ algebraische Vielfachheit}$
% \end{itemize}

\begin{definition}{Quotientenraum}
    Sei $V$ ein $K$-Vektorraum, $U \subseteq V$ ein Untervektorraum. Definiere die Äquivalenzrelation auf $V$ durch $v \sim_{U} v' \Leftrightarrow v - v' \in U$. Der \textbf{Quotientenraum} $V/_{U}$ ist die Menge der Äquivalenzklassen von $\sim_U$.\\ 
    Der Quotientenraum mit der Abbildung $\rho: V \to  V/_U, v \mapsto [v]$ hat die \textbf{universelle} Eigenschaft, dass es für jede lineare Abbildung $F: V \to W$ mit $U \subseteq \Ker(F)$ ein eindeutig bestimmte lineare Abbildugn $\overline{F}: V/_U \to W$ gibt, sodass $F = \overline{F} \circ \rho$. 
\end{definition}


\begin{definition}{Äquivalenz und Ähnlichkeit}
Zwei Matrizen $A, B \in M(m\times n,K)$ heissen \textbf{äquivalent}, wenn es matrizen $S \in GL(m,K), T \in GL(n,K)$ gibt, sodass $B = S A T^{-1}$\\
Im Falle von $m = n$ heissen zwei Matrizen $A, B \in M(n\times n,K)$ \textbf{ähnlich}, falls es ein $S \in GL(n,K)$ gibt, sodas $B = S A S^{-1}$\\
Analog heissen $F,G \in \Hom(V,W)$ \textbf{äquivalent}, falls es Isomorphismen $\Phi, \Psi$ von $V,W$ gibt, sodass $G = \Psi \circ F \circ \Phi$ und für $W = V$ heissen $F,G \in \End(V)$ \textbf{ähnlich}, wenn es einen Isomorphismus $\Phi: V \to V$ gibt, sodass $G = \Phi \circ F \circ \Phi^{-1}$.
\end{definition}
Es sind dann äquivalent: 
\begin{enumerate}[{(}i{)}]
    \item	$F, G$ sind ähnlich.
    \item   Für jede Basis $\mathcal{B}$ von $V$ sind $\mathcal{M}_{\mathcal{B}}(F)$ und $\mathcal{M}_{\mathcal{B}}(G)$ ähnlich.
    \item   $F$ und $G$ haben (bis auf Vertauschung) die gleichen Jordan'sche Normalform.
\end{enumerate}



% \begin{proposition}{Diagonalsierbarkeit }
%     Sei $F \in \End(V)$, $\mathcal{B}$ eine Basis, $\dim(V) = n < \infty$. Dann gilt
%     \begin{align*}
%         \mathcal{M}_{\mathbf{B}}(F) \text{ ist diagonalmatrix } \Leftrightarrow \mathcal{B} \text{ besteht aus Eigenvektoren von $F$}
%     \end{align*}
%     Gibt es so eine $\mathcal{B}$, so heisst $F$ \textbf{diagonalsierbar}. 
% \end{proposition}
% Es sind äquivalent:

% \begin{enumerate}[{(}i{)}]
%     \item   $F$ ist diagonalisierbar
%     \item   $P_F(t)$ zerfällt in Linearfaktoren und für alle Eigenwerte gilt: geometrische Vielfacheit $=$ algebraische Vielfachheit.
%     \item   $V = \Eig(F;\lambda_1) \oplus \ldots \oplus \Eig(F;\lambda_k)$
% \end{enumerate}


\begin{definition}{Simultan Diagonaliserbar}
    Zwei Endomorphismen $F,G \in \End(V)$ heissen \textbf{simultan} diagonalisierbar, wenn es eine Basis $\mathcal{B}$ von $V$ gibt, sodass beide $\mathcal{M}_{\mathcal{B}}(F)$ und $\mathcal{M}_{\mathcal{B}}(G)$ diagonal sind.\\
    Das ist genau dann der Fall, wenn $F \circ G = G \circ F$
\end{definition}


\begin{definition}{Trigonalisierbare Endomorphismen}
    Eine Abbildung $F \in \End(V)$ heisst \textbf{trigonalisierbar}, falls eine Basis $\mathcal{B}$ von $V$ gibt, sodass $\mathcal{M}_{\mathcal{B}}(F)$ eine obere Dreiecksmatrix ist.
\end{definition}
Äquivalent dazu sind
\begin{enumerate}
    \item	Es gibt eine $F$-invariante \textbf{Fahne} in $V$: eine Kette von Untervektorräumen $\{0\} = V_0 \subseteq V_1 \subseteq \ldots \subseteq V_n = V$, sodass $F(V_i) \subseteq V_i, \forall i = \{1, \ldots, n\}$
    \item   Das charakteristische Polynom $P_F(t)$ zerfällt in Linearfaktoren.
\end{enumerate}


\begin{rezept}{Triagonalisierung}
    \begin{enumerate}[{(}i{)}]
        \item	Charakteristisches Polynom $P_F(t)$ berechnen und Eigenwerte von $F$ bestimmen. Zerfällt es nicht in Linearfaktoren, so ist $F$ nicht trigonalisierbar. 
        \item Einen Eigenvektor $v_1$ zu einem $\lambda$ bestimmen. 
        \item   Ersetze in der Kanonischen Basis $\mathcal{K} = (e_1, e_2, e_3)$ einen Vektor mit $v_1$, sodass immer noch alle linear unabhängig sind. $\mathcal{K} \to \mathcal{B}_1$
        \item   Setze $S_1 = T_{\mathcal{K}}^{\mathcal{B}_1}$ z.B. $T_{\mathcal{K}}^{\mathcal{B}_1} = (v_1|e_2|e_3)$ 
        \item   Setze $A_2 = S_1^{-1} A S_1$ und wiederhole $(ii) - (iv)$ mit der unteren Teilmatrix $A_2' \in M(n-1\times n-1,K)$ %(d.h. ohne die Eigenspalte)
        \begin{align*}
            \begin{pmatrix}
                & &\\
                & A &\\
                & & 
            \end{pmatrix}
            \to S_1^{-1}AS_1 = A_2 = \begin{pmatrix}
                \lambda_1 & * \\
                0 & \begin{matrix}
                    A_2'
                \end{matrix}
            \end{pmatrix}
        \end{align*}
        \item   ''Verlängere'' die Vektoren $v_2' \in \R^{n-1}, v_3'' \in \R^{n-2}$ usw. durch hinzufügen von $0$ in den verlorengegangenen Koordinaten.
    \end{enumerate}
\end{rezept}


% \begin{satz}{Cayley-Hamilton}
%     Sei $V$ ein endlich dimensionaler $K$-Vektorraum, $F \in \End(V)$ und $P_F$ das charakteristische Polynom von $F$. Dann gilt $P_F(F) = 0 \in \End(V)$ 
% \end{satz}


\begin{definition}{Minimalpolynom}
    Das Minimalpolynom von einem $F \in \End(V), \dim V = n < \infty$ ist das eindeutig bestimmte (kleinste) normierte Polynom $M_F \in K[t]$ sodass $M_F(F) = 0 \in \End(V)$ und gilt
    \begin{align*}
        \forall g \in K[t] \text{ mit } g(F) = 0 \implies \deg(M_F) \leq \deg(g)
    \end{align*}
    % Es gilt 
    % \begin{enumerate}[{(}i{)}]
    %     \item   $M_F | P_F$
    %     \item   $P_F  | M_F^n$
    % \end{enumerate}
\end{definition}
Ist für ein invertierbares $A \in GL(n,K)$ das charakteristische Polynom gegben durch $P_A(t) = \sum_{i = 0}^{n}a_i t^i$, so gilt
\begin{align*}
    P_A(A) = \sum_{i = 0}^{n}a_i A^{i} = 0 =  a_0 A^{-1} + \sum_{i = 1}^{n} a_i A^{i-1}
\end{align*}
Da $a_0 = \det(A) \neq 0$ gilt insbesondere.
\begin{align*}
    A^{-1} = - \sum_{i = 1}^{n} \frac{a_i}{a_0}A^{i-1} \quad \text{und} \quad  A^n = -\sum_{i = 0}^{n-1} a_i A^i
\end{align*}

% \begin{definition}{Ideal}
%     Sei $R$ ein Ring. Ein \textbf{Ideal} $I \subseteq R$ ist eine Teilmenge, sodass
%     \begin{enumerate}
%         \item   $p,q \in I \implies p+q, -p \in I$
%         \item   $\forall p \in I, r \in R: r \cdot p \in I$     
%     \end{enumerate}
%     Die Menge $I_F = \{p \in K[t] \big\vert p(F) = 0 \in \End(V)\}$ ist ein Ideal und es wird erzeugt von $M_F$, d.h. $I_F = \{q \cdot M_F \big\vert q \in K[t]\}$
% \end{definition}

\section{Jordan-Normalform}

\begin{definition}{Nilpotente Endmorphismen}
    Man nennt $F \in \End(V)$ \textbf{nilpotent}, falls es ein $d \in \N$ gibt, sodass $F^d = 0$
\end{definition}
Äquivalent dazu sind
\begin{enumerate}
    \item	Das charakteristische Polynom ist $P_F(t) = (-t)^n$
    \item   Es gibt eine Basis $\mathcal{B}$ von $V$, sodass $\mathcal{M}_{\mathcal{B}}(F)$ eine \underline{strikte} obere Dreiecksmatrix ist.
\end{enumerate}

% \begin{definition}{Hauptraum}
%     Sei $F \in \End(V), \dim V = n < \infty$ sodass das charakteristische Polynom in Linearfaktoren zerfällt: $P_F(t) = \pm(t- \lambda_1)^{r_1} \dots (t- \lambda_k)^{r_k}$
%     Der \textbf{Hauptraum} zum Eigenwert $\lambda$ von $F$ ist
%     \begin{align*}
%         V_i = \Hau(F;\lambda_i) := \Ker\ (F - \lambda_i \cdot \id_V)^r_i
%     \end{align*}
% \end{definition}


\begin{lemma}{Fitting}
    Sei $G \in \End(V), \dim V = n < \infty$ und 
    \begin{align*}
        d: = \min\{l \in \N \big\vert \Ker G^l = \Ker G^{l+1}\}
    \end{align*}
    Dann gilt
    \begin{enumerate}
        \item   $d = \min \{l \in \N \big\vert \Image G^l = \Image G^{l+1}\}$
        \item   $\Ker G^{d+i} = \Ker G^d$ und $\Image G^{d+i} = \Image G^d, \forall i \in \N$
        \item   $U := \Ker G^d$ und $W := \Image G^d$ sind $G$-invariante Untervektorräume.
        \item   $(G|_{U})^d = 0$. und $G_W$ ist ein Isomorphismus.
        \item   $M_{G|_{U}} = t^d$
        \item   $V = U \oplus W$ und $\dim U = r \geq d, \dim W = n - r$ mit $r = \mu(P_G,0)$
    \end{enumerate}
    Insbesondere gibt es eine Basis $\mathcal{B}$ von $V$, sodass
    \begin{align*}
        \mathcal{M}_{\mathcal{G}} = \begin{pmatrix}
            N & 0\\
            0 & C
        \end{pmatrix} \quad \text{mit} \quad N^d = 0 \quad \text{und} \quad C \in GL(n-r,K)
    \end{align*}
\end{lemma}


\begin{nosatz}{Satz über die Hauptraumzerlegung}
    Sei $F \in \End(V)$ und $P_F(t) = \pm (t - \lambda_1)^{r_1} (t - \lambda_2)^{r_2} \dots (t - \lambda_k)^{r_k}$ mit den $\lambda_{1}, \ldots, \lambda_{k}$ paarweise verschiedenen Eigenwerten von $F$. Sei $V_i = \Hau(F;\lambda_i)$. Dann gilt
    \begin{enumerate}
        \item   $V_i$ ist $F$-invariant, $\dim V_i = r_i$ und $F|_{V_i}$ hat char. Polynom $(t - \lambda_i)^{r_i}$
        \item   $V = V_1 \oplus \ldots \oplus V_k$
        \item   $F$ lässt sich eindeutig zerlegen: $F = F_D + F_N$ mit 
        \begin{itemize}
            \item   $F_D$ ist diagonalisierbar und $F_N$ ist nilpotent und es gilt $F_D \circ F_N = F_n \circ F_D$
            \item   $F_D$ und $F_n$ sind Linearkombinationen von $\id, F, F^2, \ldots$ 
        \end{itemize}
    \end{enumerate}
\end{nosatz}


% Jede nilpotente Matrix $A$ kann in eine strikte obere Dreiecksmatrix transformiert werden.\\
% Wir definieren die \textbf{Jordan-Matrizen} der grösse $k$
% \begin{align*}
    % \end{align*}
    % Und den Jordanblock zum Eigenwert $\lambda$ definiert durch $J_{\lambda,k} := J_k + \lambda \cdot E_k$.
    
    \begin{satz}{Nilpotente Endomorphismen}
        Sei $G \in \End(V)$ nilpotent und $d = \min \{l \in \N \big\vert G^l = 0\}$. Dann gibt es eindeutig bestimmte Zahlen $s_1, \ldots, s_d \in \N$ sodass
        \begin{align*}
            \dim V = n = d s_d + (d-1) s_{d-1} + \ldots s_1
        \end{align*} 
        und eine (nicht eindeutige) Basis $\mathcal{B}$ von $V$, sodass
        \begin{align*}
            \mathcal{M}_{\mathcal{B}}(g) = \begin{pmatrix}
                {J_d}_{\ddots_{ J_d}}\\
                & {J_{d-1}}_{\ddots_{J_{d-1}}}\\
                & & \ddots \\
                & & & {J_1}_{\ddots_{J_1}}
            \end{pmatrix}
            \quad \text{mit} \quad
                J_k = \begin{pmatrix}
                    0 & 1 & 0 & \dots & 0\\
                    0 & 0 & 1 & \dots & \vdots\\
                    \vdots & & \ddots & \ddots & 0\\
                    \vdots & & & \ddots & 1\\
                    0 & \dots & & & 0
                \end{pmatrix} \in M(k\times k,K)
    \end{align*}
\end{satz}


% Es gilt $\dim V = n = \sum_{i = 1}^{d} d \cdot s_d$ und
%     \begin{align*}
%         \underbrace{s_l(\lambda)}_{\text{\# J-Blöcke d. Grösse $l$ zum EW $\lambda$}} 
%         %&= \underbrace{(\dim\Ker(A - \lambda E_n)^l - \dim \Ker(A - \lambda E_n)^{l-1})}_{\text{\# J-Blöcke d. Grösse mind. $l$}} - \underbrace{(\dim \Ker (A - \lambda E_n)^{l+1} - \dim \Ker(A - \lambda E_n)^l)}_{\text{\# J-Blöcke d. Grösse mind. $l+1$}}\\
%         =  2 \dim \Ker (A - \lambda E_n)^l - \dim \Ker (A - \lambda E_n)^{l-1} - \dim \Ker (A - \lambda E_n)^{l+1}
    % \end{align*}



In $\R$ zerfällt jedes Polynom in Lineare und Quadratische Faktoren
\begin{align*}
    P_F(t) = (t-\lambda_1)^{r_1} \dots (t - \lambda_k)^{r_k} g_1^{q_1} \dots g_m^{q_m}
\end{align*}
wobei $g_j = (t - a_j)^2 + b_j^2$ mit $z_j = a_j + i b_j$ und $\overline{z_j}$ den komplexen Nullstellen von $g_j$.
Definiere die Matrizen
\begin{align*}
    A_j = \begin{pmatrix}
        a_j & b_j\\
        -b_j & a_j
    \end{pmatrix}
    \quad \tilde{J}_r(A) := \begin{pmatrix}
        A & E_2\\
        & A & E_2\\
        & & \ddots & \ddots\\
        & &  & \ddots & E_2\\
        & & & & A
    \end{pmatrix} \in M(2r\times 2r,K)
\end{align*}

\begin{satz}{Reelle Jordan-Normalform}
    Sei $F \in \End(V)$ und $P_F(t) = (t-\lambda_1)^{r_1} \dots (t - \lambda_k)^{r_k} g_1^{q_1} \dots g_m^{q_m}$
    Dann gibt es eine Basis $\mathcal{B}$ von $V$ sodass
    die Abbildungsmatrix folgende Form hat:\\
    Die ersten Blöcke haben die gewöhnliche Jordan-Normalform von den Reellen Nullstellen, und die restlichen Jordan-Blöcke haben die Form $\tilde{J}_r(A)$.\\
    Ist $s_k(\lambda)$ die Anzahl Jordan-Blöcke der Grösse $k$ zum Eigenwert $\lambda$ und $s_{jk}'$ die Anzahl der $2k$-Blöcke zum Polyom $g_j$, so gilt:
    \begin{align*}
        s_{k} = 2a_k - a_{k-1} - a_{k+1}, \quad 
        s_{jk}' = \dim \Ker(g_j(F)^{k} - \frac{1}{2} \left(
            \dim \Ker (g_j(F)^{k-1}) + \dim \Ker (g_j(F)^{k+1})
        \right)
    \end{align*}
    wobei $a_k(\lambda) = \dim \Ker(A - \lambda E_n)$.
\end{satz}


Finde ein $w \in \Eig(B;a+ib)$. Und setze $v_1 = Re(w), v_2 = Im(w)$ für die Basis bzw. in die Transformationsmatrix.

