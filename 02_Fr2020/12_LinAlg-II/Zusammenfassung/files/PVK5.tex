\section{Polynome}
\begin{itemize}
    \item   Ein Ring $(R,+,\cdot,0)$ heisst \textbf{nullteilerfrei}, falls $a \cdot b = 0 \implies a = 0$ oder $b = 0$.\\
    Für jeden Körper $K$ ist $K[t]$ nullteilerfrei.
    \item   Ist $R$ ein Ring mit Eins, so ist seine \textbf{Charakteristik} definiert durch
    \begin{align*}
        \text{char}(R) := \left\{ \begin{array}{rcl}
            0, &\text{falls}& n \cdot 1 \neq 0, \forall n \geq 1\\
            \min\{n \in \N^*\}: n \cdot 1 = 0, &\text{sonst}
        \end{array} \right.
    \end{align*}
    \item   Sei $R$ ein kommutativer Ring, $f,g \in \R \setminus \{0\}$. Dann heisst $f$ \textbf{teilt} $g$ oder $f | g$, falls ein $h \in R$ exisitert mit $hf = g$.
    \item   Ein kommutativer Ring heisst \textbf{Hauptidealring}, falls es zu jedem Ideal $\{0\} \neq I \subseteq R$ ein eindeutiges Element $M_I \in I$ gibt, sodass $I = RM_I = \{fM_I \big\vert f \in R\}$.\\
    Der Polynomring $K[t]$ ist ein Hauptidealring und $M_I$ heisst \textbf{Minimalpolynom} von $I$.
    \item   (Polynomdivision)
    Sind $f,g \in K[t]$ mit $g \neq 0$, so gibt es eindeutig bestimmte Polynome $q,r \in K[t]$, sodass
    \begin{align*}
        f = q \cdot g + r \quad \text{und} \quad \deg r < \deg g
    \end{align*}
\end{itemize}

\begin{definition}{ggT, kgV}
    Für Polynome $f,g$ ist der \textbf{ggT} von $f,g$ das normierte Polynom $q$ vom höchsten Grad, sodass $q|f$ und $q|g$. Der $\text{ggT}(f,g)$ ist auch das Minimalpolynom vom Ideal $fK[t] + gK[t]$. und es existieren $a,b \in K[t]$, sodass $\text{ggT}(f,g) = af + bg$\\
    Das \textbf{kgV} von $f,g$ ist gleich das Minimapolynom des Schnittes der von $f$ und $g$ erzeugten Ideale: $\kgV(f,g) = M_{fK[t] \cap gK[t]}$
\end{definition}

\begin{satz}{Primfaktorzerlegung}
    Ein Polynom $f \in K[t]$ heisst \textbf{irreduzibel}, falls 
    \begin{align*}
        f = gh \implies \exists 0 \neq c \in K: f = cg \text{ oder } f = ch
    \end{align*}
    $f$ heisst \textbf{prim}, falls für $g \neq 0$ 
    \begin{align*}
        f| gh \implies f|h \text{ oder } f|h
    \end{align*}
    $f$ prim $\implies f$ irreduziblel und falls $\degree f > 0$ gilt auch die Umkehrung.\\
    Sei $0 \neq f \in K[t]$ ein Polynom. Dann gibt es eindeutig bestimmte irreduzible, normierte Polynome positiven Grades $p_1, \ldots, p_n \in K[t]$ ($n \geq 0$) und ein $0 \neq c \in K$, sodass $f = c p_1 \cdots p_n$. Also $K[t]$ ist ein \textbf{faktorieller Ring}.
\end{satz}

\begin{lemma}{Maximale- und Primideale}
    Sei $R$ ein kommutativer Ring. Ein Ideal $I \subset R$ heisst \textbf{maximal}, falls $I \neq R$ und es kein Ideal $J$ gibt, sodass $I \subset J \subset R$.\\
    $I$ heisst \textbf{prim}, falls $I \neq R$ und für alle $f,g \in R$ gilt 
    \begin{align*}
        fg \in I \implies f \in I \text{ oder } g \in I
    \end{align*}
    \begin{itemize}
        \item   $I$ ist genau dann maximal, wenn $R/I$ ein Körper ist.
        \item   $I$ ist genau dann prim, wenn $R/I$ nullteilerfrei ist.
    \end{itemize}
    Hierbei ist der Quotient durch die Äquivalenzrelation $\sim$ definiert:
    \begin{align*}
        R/I = R/_{\sim}, \quad f \sim g \Leftrightarrow f - g \in I
    \end{align*}
\end{lemma}


Definiere für $A \in M(n\times n,K)$ die Matrizen
\begin{align*}
    \tilde{E}_n := \begin{pmatrix}
        0 & 0 & \cdots & 0\\
        \vdots & \ddots & & \vdots\\
        0 & 0 & \cdots & 0\\
        1 & 0 & \cdots & 0
    \end{pmatrix} \in M(n\times n,K), \quad \tilde{J}_r(A) := \begin{pmatrix}
        A & {\tilde{E}_n}\\
        & A \tilde{E}_n\\
        & & \ddots & \ddots\\
        & &  & \ddots & \tilde{E}_n\\
        & & & & A
    \end{pmatrix} \in M(nr\times nr,K)
\end{align*}
\begin{nosatz}{Allgemeine Jordansche Normalform}
    Sei $V$ ein endlich dimensionaler $K$-Vektorraum und $F \in \End(V)$. Sei $P_F = \pm p_1^{r_1} \cdots p_k^{r_k}$ die Faktorisierung des charakteristischen Polynoms in irreduzible normierte Polynome $p_1, \ldots p_k$. Dann gibt es eine Basis $\mathcal{B}$ von $V$, sodass $\mathcal{M}_{\mathcal{B}}(F)$ blockdiagonal ist, wobei die Blöcke auf der Blockdiagonalen alles Jordanblöcke der Form $\tilde{J}_r(B_{p_j})$ für $j = 1, \ldots, k$ und $r = 1, \ldots, r_j$. Ist $s_{jr}$ die Anzahl der Jordanblöcke der Form $\tilde{J}_r(B_{p_j})$, dann gilt
    \begin{align*}
        \sum_{r = 1}^{r_j} s_{jr} = r_j \cdot \degree p_j, \quad \text{ für alle } j = 1, \ldots, k
    \end{align*}
    \begin{empheq}[box=\bluebase]{align*}
        s_{jr} = \frac{1}{\deg p_j} \left(
            2 \dim \Ker(p_j(F)^r)
        \right)
    \end{empheq}
\end{nosatz}