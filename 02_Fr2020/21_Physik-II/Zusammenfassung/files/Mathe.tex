\section{Mathematische Formeln}

\subsection{Koordinatensysteme}

\textbf{Kugelkoordinaten}
\begin{align*}
    \Phi \begin{pmatrix}
        r\\ \theta \\\phi
    \end{pmatrix} = \begin{pmatrix}
        r \sin \theta \cos \phi\\
        r \sin \theta \sin \phi\\
        r \cos \theta
    \end{pmatrix}, \quad \det D \Phi = r^2 \sin \theta
\end{align*}
\begin{align*}
    \bm{\nabla \cdot V} = \frac{1}{r^2} \frac{{\del}(r^2V_r)}{\del r} + \frac{1}{r \sin \theta} \frac{\del(\sin \theta V_{\theta})}{\del \theta} + \frac{1}{r\sin \theta} \frac{\del V_{\phi}}{\del \phi}
\end{align*}
\begin{align*}
    \bm{\nabla \times V} = \begin{pmatrix}
        \frac{1}{r \sin \theta} \left[\frac{\del(\sin \theta V_\phi)}{\del \theta} - \frac{\del V_{\theta}}{\del\phi}\right]\\
        \frac{1}{r} \left[\frac{1}{\sin \theta} \frac{\del V_r}{\del\phi} - \frac{\del(r V_\phi)}{\del r}\right]\\
        \frac{1}{r} \left[\frac{\del(r V_\theta)}{\del r} - \frac{\del V_r}{\del \theta}\right]
    \end{pmatrix}
\end{align*}
\textbf{Zylinderkoordinaten}
\begin{align*}
    \Phi \begin{pmatrix}
        r\\ \theta \\ z
    \end{pmatrix}
    = \begin{pmatrix}
        r \cos \theta\\
        r \sin \theta\\
        z
    \end{pmatrix}, \quad \det D \Phi =  r 
\end{align*}
\begin{align*}
    \bm{\nabla \cdot V} = \frac{1}{r}\frac{\del(rV_r)}{\del r} + \frac{1}{r} \frac{\del V_\phi}{\del \phi} + \frac{\del V_z}{\del z}
\end{align*}
\begin{align*}
    \bm{\nabla \times V} = \begin{pmatrix}
        \frac{1}{r} \frac{\del V_z}{\del \phi} - \frac{\del V_\phi}{\del z}\\
        \frac{\del V_r}{\del z} - \frac{\del V_z}{\del r}\\
        \frac{1}{r} \left[\frac{\del(r V\phi)}{\del r} - \frac{\del V_r}{\del \phi}\right]
    \end{pmatrix}
\end{align*}

\subsection{Trigonometrie}

\begin{align*}
    \sin(\alpha + \beta) &= \sin \alpha \cos \beta \pm \cos \alpha \sin \beta\\
    \cos(\alpha \pm \beta) &= \cos \alpha \cos \beta \mp \sin\alpha \sin \beta\\
    2 \cos \alpha \cos \beta &= \cos(\alpha - \beta) + \cos(\alpha + \beta)\\
    2 \sin \alpha \sin \beta &= \cos(\alpha - \beta) - \cos(\alpha + \beta)\\
    2 \sin\alpha \cos \beta &= \sin(\alpha + \beta) + \sin(\alpha - \beta)\\
    \sin \alpha \pm \sin \beta &= 2 \sin \left(\frac{\alpha \pm \beta}{2}\right) \cos \left(\frac{\alpha - \beta}{2}\right)\\
    \cos\alpha + \cos \beta &= 2 \cos \left(\frac{\alpha + \beta}{2}\right) \cos \left(\frac{\alpha - \beta}{2}\right)
\end{align*}

\subsection{Differentialgleichungen}

Ein \textbf{Harmonischer Oszillator} erfüllt die folgende Differentialgleichung
\begin{empheq}[box=\bluebase]{align*}
    \overset{\cdot \cdot}{x} + \omega x = 0
\end{empheq}
und hat die Allgemeine Lösung
\begin{align*}
    x(t) = A \cos(\omega t + \phi) = C_1 \sin(\omega t) + C_2 \cos(\omega t)
\end{align*}
der \textbf{gedämpfte} harmonische Oszillator besitzt einen Dämpfungsterm $2\tau$
\begin{empheq}[box=\bluebase]{align*}
    \overset{\cdot \cdot}{x} + 2\tau \dot{x} + \omega^2 x = 0
\end{empheq}
mit der von $\tau$ und $\omega$ abhängigen Allgemeinen Lösungen
\begin{align*}
    \text{Schwache Dämpfung } \tau < \omega: &\quad x(t) = C e^{- \tau t} \cos(\omega t + \phi)\\
    \text{Starke Dämpfung } \tau > \omega: &\quad x(t) = C_1 e^{- k_1 t} + C_2 e^{- k_2 t}\\
    \text{Kritische Dämpfung } \tau = \omega: &\quad x(t) = \left(C_1 t + C_2\right)e^{-\tau t}
\end{align*}
Für $k_1 = - \lambda_1, k_2 = -\lambda_2$ Lösungen der Gleichung $\lambda^2 + 2\tau \lambda + \omega^2 = 0$

\subsection{Vektoranalysis}
\begin{empheq}[box=\bluebase]{align*}
    A \times (B \times C) = B (A \cdot C) - C(A \cdot B)
\end{empheq}