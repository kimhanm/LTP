
\section{Felder in Materie}
\subsection{Dipole}
Für zwei mit dem Vektor $\bm{\ell}$ getrennte Punktladungen $\pm q$ ist das \textbf{elektrische Dipolmoment} $\bm{p}$ in Richtung der positiven Ladung zeigend gegeben durch
\begin{align*}
    \bm{p} = q \bm{\ell} \quad \text{bzw.} \quad \bm{p} = \int_{\R^3} \rho \vec{r} dV \text{ für ausgedehnte Ladungen}
\end{align*}
Analog ist das \textbf{magnetische Moment} $\bm{\mu}$ einer Leiterschleife (bzw. Spule mit $n$ Windungen) der zur Fläche  $A$ senkrecht stehende Vektor
\begin{align*}
    \bm{\mu} = (n) I \bm{A}
\end{align*}

Auf elektrische und magnetische Dipole wirkt in externen $\bm{E}$ und $\bm{B}$-Feldern ein Drehmoment $\bm{\tau}$ gegeben durch
\begin{empheq}[box=\bluebase]{align*}
    \bm{\tau} = \bm{p} \times \bm{E}_{\text{ext}}, \quad \tau = \bm{\mu} \times \bm{B}_{\text{ext}} 
\end{empheq}

\subsection{Polarisierte Materie}

Dielektrisches Material, welches sich in einem Eletkrischen Feld befindet wird \textbf{polarisiert}. Dabei enstehen darin elementare elektrische Dipole $\bm{p}$. Liegt eine Teilchendichte $N$ dieser Dipole vor, so definieren wir die \textbf{Polarisationsdichte}
\begin{align*}
    \bm{P} = N \bm{p}
\end{align*}
Weiter definieren wir die dimensionslosen Konstanten \textbf{elektrische Suszeptibilität} $\chi_{\text{el}}$ und die relative Dielektrizitätskonstante $\epsilon$ als
\begin{empheq}[box=\bluebase]{align*}
        \chi_{\text{el}} = \frac{P}{\epsilon_0 E_{\text{res}}}, \quad \epsilon = \frac{E_{\text{vak}}}{E_{\text{res}}} = 1 + \chi_{\text{el}}
\end{empheq}
Somit können wir die \textbf{Dielektrische Verschiebung} $\bm{D}$ definieren durch
\begin{align*}
    \bm{D} = \epsilon_0 \bm{E} + \bm{P} = \epsilon \epsilon_0 \bm{E}, \quad \text{für} \quad \bm{P} = \epsilon_0(\epsilon - 1)\bm{E}
\end{align*}
Daraus folgen die Beziehungen
\begin{empheq}[box=\bluebase]{align*}
    \bm{\nabla \cdot D} = \rho_{\text{frei}}, \quad - \bm{\nabla \cdot P} = \rho_{\text{geb}}
\end{empheq}

\subsection{Magnetisierte Materie}
Stoffe, wie z.B. Eisen, oder Wasser können das Magnetische Feld eines Magneten verstärken oder abschwächen und dabei zum Teil selber magnetische Eigenschaften aufweisen.

Ein Magnetfeld kann also in einem Festkörper ein Magnetfeld induzieren, das aus elementaren magnetische Dipolen $\bm{\mu}$ mit Teilchendichte $N$ besteht. Dazu definieren wir die \textbf{Magnetisierung} als 
\begin{align*}
    \bm{M} = N \bm{\mu}
\end{align*}
Auch hier definieren wir die \textbf{magnetische Suszeptibilität} $\chi_{\text{mag}}$ und die \textbf{relative Permeabilität} $\mu$ durch
\begin{empheq}[box=\bluebase]{align*}
    \chi_{\text{mag}} = \frac{\bm{M}}{\bm{H}}, \quad \mu = 1 + \chi_{\text{mag}}
\end{empheq}
Hier beschreibt das Magnetische \textbf{H-Feld} die effektive magnetische Feldstärke in einem magnetisierten Medium und ist gegeben durch
\begin{empheq}[box=\bluebase]{align*}
    \bm{H} = \frac{\bm{B}}{\mu_0} - \bm{M} = \frac{\bm{B}}{\mu \mu_0}
\end{empheq}
Für para- und ferromagnetische ist $\mu > 1$ und für diagmagnetische ist $\mu < 1$.

Dazu gilt
\begin{empheq}[box=\bluebase]{align*}
    \bm{\nabla \times H} = \frac{\del \bm{D}}{\del t} + \bm{J}_{\text{frei}}
\end{empheq}

bezeichnen wir mit $\bm{J}_{\text{geb}}$ die im Medium gebundenen Ströme, so gilt Beziehung
\begin{empheq}[box=\bluebase]{align*}
    \bm{\nabla \times M} = \bm{J}_{\text{geb}}  
\end{empheq}

Im Vergleich zum Vakuum gelten dann folgende Beziehungen
\begin{align*}
    \mathbf{E}_{\text{res}} = \frac{1}{\epsilon} \mathbf{E}_{\text{vak}}, \quad \mathbf{D}_{\text{res}} = \mathbf{D}_{\text{vak}}
\end{align*}

Wir erhalten dabei die modifizierten Maxwell Gleichungen

\begin{empheq}[box=\bluebase]{align*}
    \bm{\nabla \cdot D} = \rho_{\text{frei}}
\end{empheq}
\begin{empheq}[box=\bluebase]{align*}
    \bm{\nabla \times H} = \bm{J}_{\text{frei}} + \frac{\del \bm{D}}{\del t}
\end{empheq}

Auf Plattenkondensatoren hat dies den Effekt, dass $C = \epsilon C_{\text{vak}}$\\
Elektromagnetische Wellen in einem polarisierten/magnetisierten Medium breiten sich mit der Geschwindigkeit 
\begin{align*}
    c' = \frac{1}{\sqrt{\epsilon\epsilon_0\mu\mu_0}} = \frac{c}{\sqrt{\epsilon \mu}}
\end{align*}
Dabei breiten sich die Wellen mit Amplituden $\abs{\bm{E}} = c \abs{\bm{B}}$ und Ausbreitungsrichtung des Poynting-Vektors $\bm{S} = \frac{1}{\mu_0} \bm{E} \times \bm{B}$. 