\section{Schaltkreise}
\textbf{Ohmesches Gesetz}
Ist $\sigma$ die Leitfähgkeit, $\rho$ der spezifischer Widerstand, so ist die Stromfluss dichte gegeben durch
\begin{align*}
    \vec{J} = \sigma \vec{E} = \frac{\vec{E}}{\rho}, \quad R = \frac{\rho \cdot \ell}{A} = \frac{U}{I}
\end{align*}

\subsection{Kirchhoffschen Regeln}
Für jeden Knoten gilt
\begin{align*}
    \sum I_{\text{in}} = \sum I_{\text{out  }}
\end{align*}
Für jede geschlossene Schlaufe gilt
\begin{align*}
    \sum V_i = 0
\end{align*}
Für Parallelschaltung gilt
\begin{align*}
    \frac{1}{R_{\text{tot}}} = \sum_{i} \frac{1}{R_i}, \quad C_{\text{tot}} = \sum_{i} C_i, \quad \frac{1}{L_{\text{tot}}} = \sum_i \frac{1}{L_i}
\end{align*}
Für Serieschaltung
\begin{align*}
    R_{\text{tot}} = \sum_i R_i, \quad \frac{1}{C_{\text{tot}}} = \sum_i \frac{1}{C_i}, \quad L_{\text{tot}} = \sum_i L_i
\end{align*}
Im Spezialfall von zwei Leiterlementen gilt dann
\begin{align*}
    \frac{1}{\text{tot}} = \frac{1}{a} + \frac{1}{b} \implies \text{tot} = \frac{ab}{a+b}
\end{align*}

\subsection{Leistung}
Die dissipierte Leistung $P$ die durch einen Stromkreis fliesst ist gegeben durch

\begin{align*}
    P = \frac{dW}{dt} = \dot Q V = I V = I^2R = \frac{V^2}{R}
\end{align*}


\textbf{Vorzeichenregeln}

\begin{tabular}{lL}
    Resistor & V_i = - RI\\
    Spannung $- \to +$ & V_i  = \epsilon\\
    Kondensator $- \to +$ & V_i = \frac{Q}{C}\\
\end{tabular}



\subsection{Kondensator \& Spule}
Die in einem geladenen Kondensator gespeicherte Energie ist gegeben durch
\begin{empheq}[box=\bluebase]{align*}
    dW &= \Phi dQ = \frac{Q}{C} dQ \implies\\
     W &= \int_{0}^{Q_0} V(q) dQ = \frac{Q_0^2}{2C} = \frac{CV_0^2}{2} = \frac{Q_0V_0}{2}
\end{empheq}
Für einen Spule ist dies
\begin{empheq}[box=\bluebase]{align*}
    dW = P dt = L I d I \implies W = \int_{0}^{t} P(t) dt = \frac{L I_0^2}{2}
\end{empheq}
In einem Schaltkreis bestehend aus einem Widerstand und entweder einer Spule oder einem Kondensator, so kann mit den Differentialgleichungen
\begin{align*}
    U_0 -IR - \frac{Q(t)}{C} = 0, \quad \text{bzw.} \quad U_0 - IR - L \frac{d I}{dt} =0
\end{align*}
Strom, Spannung und Ladung bei durch exponentielle Funktionen beschrieben werden.\\

Sind $Q_0, I_0, U_0$ die Ladung, Stromstärke und Spannung, die der Kondensator im \emph{Vollgeladenen} Zustand aufweist, so ist für $\tau_C = RC$ beim \textbf{Aufladen}
\begin{empheq}[box=\bluebase]{align*}
    Q_C(t) = Q_0 \cdot \left(1 - e^{- \frac{t}{\tau_C}}\right)\\
    U_C(t) = U_0 \cdot \left(1 - e^{-\frac{t}{\tau_C}}\right)\\
    I_C(t) = \frac{U_0}{R} \cdot e^{-\frac{t}{\tau_C}}
\end{empheq}
und beim \textbf{Entladen} des Kondensators:
\begin{empheq}[box=\bluebase]{align*}
    Q_C(t) = Q_0 \cdot e^{-\frac{t}{\tau_C}}\\
    U_C(t) = U_0 \cdot e^{- \frac{t}{\tau_C}}\\
    I_C(t) = - \frac{U_0}{R} \cdot e^{- \frac{t}{\tau_C}}
\end{empheq}

Für eine Spule, ist dann die Spannung über die Spule und über den Widerstand anders (mit $U_L$ und $U_R$ gekennzeichnet).

Beim \textbf{Einschaltevorgang} erhält man mit $\tau_L = \frac{L}{R}$ die Gleichungen

\begin{empheq}[box=\bluebase]{align*}
    U_L(t) = U_0 e^{- \frac{t}{\tau_L}}\\
    I_L(t) = \frac{U_0}{R} \left(1 - e^{- \frac{R}{L}t}\right)\\
    U_R(t) = U_0 \left(1 - e^{- \frac{t}{\tau_L}} \right)
\end{empheq}
und beim \textbf{Ausschaltevorgang}
\begin{empheq}[box=\bluebase]{align*}
    U_L(t) = - U_0 e^{- \frac{t}{\tau}}\\
    I_L(t) = I_0 e^{- \frac{t}{\tau_L}}\\
    U_R(t) = U_0 \cdot e^{- \frac{t}{\tau}}
\end{empheq}

Die Halbwertszeit $T$ der Exponentialfunktion ist $T_{\text{halb}} = \tau \cdot \ln 2$.
Sind beides Spule und Kondensator vorhaden, so ist bei einem Wechselstromkreis die höchste Leistung bei $\omega_{\text{max}} = \frac{1}{\sqrt{LC}}$ erhältlich.


\subsection{Bio-Salat}  
$\vec{r}' - \vec{r}$ vom Leiter zum Beobachtungspunkt
\begin{align*}
    d \vec{B} = \frac{\mu_0}{4\pi r^2} I d \vec{l} \times \hat{\vec{r}}
\end{align*}

% \begin{rezept}{Bio-Salat}
%     Zutaten:
%     \begin{itemize}
%         \item   1 Bataviasalat
%         \item   $\frac{1}{2}$ Salatgurken
%         \item   1 Tomate
%         \item   1 Paprikaschote
%         \item   $\frac{1}{2}$ Zuccini
%         \item   4 Radieschen
%         \item   $\frac{1}{2}$ Bund Frühlingszwiebeln
%         \item   200 ml Salatöl
%         \item   100 ml Balsamico Bianco
%         \item   100 ml Apfelsaft
%         \item   Kräuter
%         \item   Salz, Pfeffer
%     \end{itemize}
%     Zubereitung: Bataviasalat putzen, abtropfen lassen und in mundgerechte Stücke pflücken. Gurken und Radieschen in Scheiben schneiden. Zucchini in Stife hobel und Paprika würfeln. Frühlingszwiebeln in Röllchen scneiden.

%     Alles in eine Schlüssel geben und mit Kräutern bestreuen und mischen.

%     Für das Dressing Öl, Essig, Saft, Salz, Pfeffer im aufschlagen.
% \end{rezept}

\begin{rezept}{Bio-Salat}
    \begin{enumerate}[{(}i{)}]
        \item   Leiter parametrisieren. $\Phi: V \to $Leiter. (Kugel-, Zylinder- oder sonstige Koordinaten wählen)
        \item   Leiterelement $d \vec{l}$ in Koordinaten angeben. $\vec{r}$ ist die Position des Betrachtungspunktes, $\vec{r}'$ ist die Position des Leiterelements.
        \item   Symmetrien erkennen und womöglich Komponenten wegkürzen.
        \item   $B = \int_{\text{Leiter}} d B = \frac{\mu_0}{4 \pi}\int_{V} \frac{1}{\abs{\vec{r} - \vec{r}'}} I d \vec{l} \times \frac{\vec{r} - \vec{r}'}{\abs{\vec{r} - \vec{r}'}}$
    \end{enumerate}
\end{rezept}

