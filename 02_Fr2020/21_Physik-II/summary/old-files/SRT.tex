\section{SRT}
\subsection{Vierervektor}
Der \textbf{Vierervektor} ist der Posititionsvektor eines Ereignisses im \textbf{Minkowski-Raum}. Dazu gehören der Vektor $\beta$ und der Skalar $\gamma$
\begin{align*}
    x^{\mu} = \begin{pmatrix}
        x_0\\
        x_1\\
        x_2\\
        x_3
    \end{pmatrix} = \begin{pmatrix}
        ct\\
        \vec{x}
    \end{pmatrix}, \quad \vec{\beta} = \frac{\vec{v}}{c}, \quad \gamma = \frac{1}{\sqrt{1 - \beta^2}}
\end{align*}


Das \textbf{Viererskalarprodukt} im Minkowski-Raum ist die symmetrische bilineare Abbildung definiert durch
\begin{align*}
        \left<x^{\mu},y^{\nu}\right> = g_{\mu \nu}x^{\mu} y^{\nu} \begin{pmatrix}
        x_0\\x_1\\ x_2\\ x_3
    \end{pmatrix}^T \begin{pmatrix}
        1\\
        0 & -1\\
        0 & 0 & -1\\
        0 & 0 & 0 & -1
    \end{pmatrix} \begin{pmatrix}
        y_0\\ y_1 \\ y_2 \\ y_3
    \end{pmatrix}
\end{align*}
Sie lässt sich auch mit der gewöhnlichen Skalarprodukt notation leicht verändert aufschreiben
\begin{align*}
    \Norm{x^{\mu}}^2 = x^{\mu} \cdot x_{\mu} = \begin{pmatrix}
        x_0\\
        x_1\\
        x_2\\
        x_3
    \end{pmatrix} \cdot \begin{pmatrix}
        x_0\\
        -x_1\\
        -x_2\\
        -x_3
    \end{pmatrix}
\end{align*}
Das \textbf{Raumzeitintervall} ist dann unter Lorentz-Transformationen erhalten:
\begin{align*}
        (\Delta s)^2 := (c\Delta t)^2 - (\Delta x_1^2 + \Delta x_2^2 + \Delta x_3^2) = (\Delta s')^2
\end{align*}

\subsection{Zeit \& Länge}

Die \textbf{Zeitdilatation} hat den Effekt, dass die Zeit für bewegende Uhren scheinbar langsamer läuft. Die Eigenzeit $\Delta t'$ des mitbewegten Systems verläuft schneller als die gemessene Zeit $\Delta t$ im Laborsystem.
\begin{empheq}[box=\bluebase]{align*}
    \Delta t' =   \frac{\Delta t}{\gamma}
\end{empheq}
Bewegte Objekte erscheinen wegen der \textbf{Längenkontraktion} kürzer im Laborsystem ($\Delta x$) als die gemessene Eigenlänge im mitbewegten System ($\Delta x'$)
\begin{empheq}[box=\bluebase]{align*}
    \Delta x' = \gamma \Delta x
\end{empheq}


Die Transformation der Koordinaten vom Laborsystem $K$ zum mitbewegten System $K'$, das sich mit der konstanten Geschwindigkeit $\vec{v} = \vec{\beta}c$ in $x$-Richtung bewegt ist gegeben durch 
\begin{align*}
    \Lambda_{K'}^K = \begin{pmatrix}
        \gamma & - \beta \gamma & 0 & 0\\
        -\beta \gamma & \gamma & 0 & 0\\
        0 & 0 & 1\\
        0 & 0 & 0& 1
    \end{pmatrix}, \quad 
    \Lambda_{K}^{K'} = \begin{pmatrix}
        \gamma & + \beta \gamma & 0 & 0\\
        +\beta \gamma & \gamma & 0 & 0\\
        0 & 0 & 1\\
        0 & 0 & 0& 1
    \end{pmatrix}
\end{align*}


Bewegt sich $K'$ mit Geschwindigkeitet $v$ relativ zu $K$ und misst die Geschwindigkeit $u'$ in die gleiche Richtung, so gilt

\begin{empheq}[box=\bluebase]{align*}
    u = \frac{u' + v}{1 + \frac{u' v}{c^2}} < c
\end{empheq}


\subsection{Energie, Impuls}
Die \textbf{Vierergeschwindigkeit} die von der Eigenuhr gemessene Zeit $\tau$ berechneter Vierervektor.
\begin{align*}
    u^{\mu} = \frac{d x^{\mu}}{d \tau} = c\gamma \begin{pmatrix}
        1\\
        \bm{\beta}
    \end{pmatrix}, \quad \Norm{u}^2 = u^{\mu}u_{\mu} = c^2
\end{align*}

\textbf{Viererimpuls}
\begin{align*}
    P^{\mu} = m \cdot u^{\mu} = \begin{pmatrix}
        m \gamma c\\
        m \gamma c \vec{\beta}
    \end{pmatrix} = \begin{pmatrix}
        \frac{E}{c}\\
        \vec{p}
    \end{pmatrix}\\
    \implies \abs{P}^2 = P_{\mu}P^{\mu} = \frac{E^2}{c^2} - \abs{\vec{p}}^2 = m^2 c^2
\end{align*}
\begin{empheq}[box=\bluebase]{align*}
    E_{\text{tot}} = \underbrace{E_0}_{= mc^2} + E_{\text{kin}} = m \gamma c^2
\end{empheq}

\begin{empheq}[box=\bluebase]{align*}
    E^2 = \abs{\vec{p}}^2c^2 + m^2c^4
\end{empheq}


Das Photon hat keine Masse. Darum gilt $E_{\nu} = \abs{\vec{p}}_{\nu} c$


\begin{empheq}[box=\bluebase]{align*}
    m \gamma \vec{a} = \vec{F} - \frac{\vec{F} \cdot \vec{v}}{c^2} \vec{v}
\end{empheq}


Die Taylor-Entwicklung von $\gamma$ ist 
\begin{align*}
    \gamma = \frac{1}{\sqrt{1 - \beta^2}} \simeq 1 + \frac{1}{2} \beta^2
\end{align*}
\subsection{Doppler-Effekt}
Allgemein: Setze $\theta$ als den Winkel zwischen $v_q$ und der Verbindungsstrecke $B-Q$:


\begin{empheq}[box=\bluebase]{align*}
    f_B = f_Q \frac{\sqrt{1 - \beta^2}}{1 - \beta \cos \theta}
\end{empheq}
\textbf{Longitudinaler Doppler-Effekt}: Spezialfälle $Q$ bewegt sich zu $B$ ($\theta = 0$) und $Q$ bewegt sich von $B$ weg ($\theta = \pi$)
\begin{align*}
    f_B|_{\theta = 0} = f_Q\sqrt{\frac{1 + \beta}{1 - \beta}}, \quad f_B|_{\theta = \pi} = f_Q \sqrt{\frac{1 - \beta}{1 + \beta}}
\end{align*}

\textbf{Transversaler Doppler-Effekt}: Quelle bewegt sich senkrecht zum Beobachter $\theta = \frac{\pi}{2}$
\begin{align*}
    f_B = f_Q \sqrt{1 - \beta^2}
\end{align*}


Im \textbf{Minkowski-Diagramm} ist der Winkel der transformierten Raum-Zeitachsen $\alpha$ gegeben durch $\tan(\alpha) = \beta = \frac{ct}{x}$

\subsection{Relativistische Transformationen}

Die $\vec{E}$ und $\vec{B}$-Felder transformieren sich mit $\vec{E} = \vec{E}_{\parallel} + \vec{E}_{\bot}$ wie folgt: 

\begin{empheq}[box=\bluebase]{align*}
    \mathbf{E}_{\parallel}' = \mathbf{E}_{\parallel}, \quad \mathbf{E}_{\bot}' = \gamma \left(\mathbf{E}_{\bot} + c \vec{\beta} \times \mathbf{B}_{\bot} \right)\\
    \mathbf{B}_{\parallel}' = \mathbf{B}_{\parallel}, \quad \mathbf{B}_{\bot}' = \gamma (\mathbf{B}_{\bot} - \frac{\vec{\beta}}{c} \times \mathbf{E}_{\bot})
\end{empheq}
