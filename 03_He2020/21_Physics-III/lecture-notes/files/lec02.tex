


\subsubsection{Lenses}
Imagine a situtation, where a point $A$ is emitting light in all directions and you want to ``send'' all the light rays towards $B$.\\ Normally, Fermat's principle would tell us that there is only one path form $A$ to $B$ that the light would reach $B$.\\

Now what a lens would do is to change the optical path lengths for some paths such that more than one of them is extremal. This results in ``more Light'' reaching $B$. (See Figure 1.6)\\


One example of this is the \textbf{single surface lens}, which an arced surface between two materials. (For example light and glass).

We can describe the surface of the surface using some function $g(h)$, which allows us the write down the optical path length for the straight line and the point going trough the point $(g(h),h)$.\\


For the straight line, we simply have $D_{ACB} = n_1d_1 + n_2d_2$, whereas for the path $AC'B$ we have
\begin{align*}
				D_{AC'B} = n_1 \sqrt{h^2 + (d_1 + g(h))^2} + n_2 \sqrt{h^2 + (d_2 - g(h))^2}
\end{align*}
Using the taylor approximation of $(1 + \epsilon)^{\alpha}$ for $\abs{\epsilon} \ll 1$, we have $(1 + \epsilon)^{\alpha} \simeq 1 + \alpha \epsilon$.\\
Using the so called \emph{Paraxial approximation}, we are considering only rays that make a small angle with the optical axis ($ACB$). This means that we consider $h$ and $g(h)$ to be small in comparison to $d_1$ and $d_2$.\\
This allows us to simplify the optical path length $D_{AC'B}$. After dividing out the $(d_i + g(h))$ term and taking the taylor approximation of this term we get
\begin{align*}
				D_{AC'B}(h) &= n_1(d_1 + g(h)) \sqrt{1 + \frac{h^2}{(d_1 + g(h))^{2}}} + n_2(d_2 - g(h)) \sqrt{1 + \frac{h^2}{(d_2 - g(h))^{2}}}\\
										&\simeq n_1(d_1 + g(h)) \left(1 + \frac{h^2}{2(d_1 + g(h))^{2}}\right) + n_2(d_2 - g(h)) \left(1 + \frac{h^2}{2(d_2 - g(h))^{2})}\right)\\
										&= n_1d_1 + n_2d_2 - (n_2-n_2)g(h) + \frac{h^2}{2} \left(\frac{n_1}{d_1} + \frac{n_2}{d_2}\right)
\end{align*}
where the errors are of order $\epsilon^2 \propto h^4$.\\
After taking the difference between the two pathlengths and setting it to zero, we can get the right curvature of the lens surface.
\begin{align*}
				\Delta D = D_{AC'B} - D_{ACB} = \frac{h^2}{2} \left(\frac{n_1}{d_1} + \frac{n_2}{d_2}\right) - (n_2 - n_1) g(h) = 0\\
				\implies g(h) = \frac{h^2}{2(n_2 - n_1)} \left(\frac{n_1}{d_1} + \frac{n_2}{d_2}\right)
\end{align*}
which is a parabola.\\
In practice, creating parabolic surfaces can be hard so we can further approximate the surfac as a sphere with Radius $R \gg h$:
\begin{align*}
				g(h) = R - \sqrt{R^2 - h^2} \simeq \frac{h^2}{2R}
\end{align*}
for $\frac{1}{R} = \frac{1}{n_2 - n_1} \left(\frac{n_1}{d_1} + \frac{n_2}{d_2}\right)$.\\

Next we may consider a light source living infinitely far away from you. Here, it's emitted light rays are coming in paralel to you and we want to focus all light rays into a single point using a lens.\\


Using our previous equations, we would have $d_1 = \infty$ and $d_2 = f$ for the distance to the focal point we want to get

\begin{empheq}[box=\bluebase]{align}\label{eq:redstar} 
		R = \frac{f(n_2 - n_1)}{n_2}, \quad f = \frac{R n_2}{n_2 - n_1}, \quad \frac{n_2}{f} = \left(\frac{n_1}{d_1} + \frac{n_2}{d_2}\right) 
\end{empheq}

Check the images in the slides for lecture 2.\\
When the light enters the lens, we can have multiple scenarios. If we get a positive focal distance $f > 0$, we wee that the light gets converges into a single point.\\
If however we get $f < 0$, we get a \emph{virtual focal point} at distance $f$ \emph{before} the lens.\\
If we look from the other side of the lens, it would look like the light is coming from the virtual focal point.\\

If we chose a spherical surface, we see that the light rays don't intersect at exactly the same point. We call this phenomenon the \emph{spherical abberations}
\begin{empheq}[box=\bluebase]{align*}
				\Delta D_{\text{err}} = \phi(n_1,n_2) h \left(\frac{h}{f}\right)^{3}
\end{empheq}

Next we consider a lens with two surfaces. The first surface with Radius $R_1$ will have focal length $f'$. Here we will look at the rays hitting the second surface as coming from a virtual source from behind the lens and use the equation \ref{eq:redstar} with $d_1 = -f$ and we use $n_2 = 1, n_1 = n$. We then get
\begin{align*}
				\frac{1}{R_2} &= \frac{1}{1-n} \left(\frac{n}{d_1} + \frac{1}{d_2}\right)\\
											&= \frac{1}{1-n} \left(- \frac{n-1}{R_1} + \frac{1}{f}\right) \\
				\frac{1}{f} &= (n-1) \left(\frac{1}{R_1} - \frac{1}{R_2}\right)
\end{align*}
This accurately describes our Biconvex lense $R_1 > 0, R_2 < 0 \implies f > 0$


This gives rise to the \textbf{lensmaker's equation}
\begin{empheq}[box=\bluebase]{align*}
				\frac{1}{f} = \frac{1}{d_1} + \frac{1}{d_2}
\end{empheq}
where we an object $A$ at distance $d_1$ before the lens and Image of the object at point $B$ at distance $d_2$ after the lens.\\
This gives us the following sign conventions

\begin{tabular}{ll}
	$f > 0$ & Lightrays converge\\
	$f < 0$ & Lightrays diverge\\
	$d_2 > 0$ & Real Image\\
	$d_2 < 0$ & virtual image\\
\end{tabular}


\subsubsection{Ray Tracing}
If we have an object which does not lie on the lens axis, we can use the three rules to find out, where the image is.\\
Rule 1 gives us a straight line through the center of the lens. Using either rule 2 or rule 3, we intersect rule 1 with the line that goes parallel until it hits the lens and go through the focal point.\\
For a convex lens, we use the focal point of the other side for rule 2, and for concave lenses, we use the focal point on the same side as the object or rule 2. For rule 3, we use the other focal point\\

To better understand how big the image (virtual or real) will be, we introduce the \textbf{Magnification factor} $M := \frac{h'}{h} = - \frac{d_2}{d_1}$.\\
It's absolute value tell us its size and the sign tells us the orientation.\\


If we are interested in seeing things far away from us, we dont' necessarily want to increase Magnification factor as much as we want to increase the angular size of the object. One way to do this is to use multiple lenses.\\

Let's say an object with angular size $\Delta \theta$ whose rays travel trough lens 1 and form an image $h'$ at focal point $f_1$. We can then magnify the image by lens 2 whose focal point lies at the image $h'$.\\
This would give us
\begin{align*}
				h' &= f_1 \tan(\Delta \theta) \simeq f_1 \Delta \theta\\
				\Delta \theta' &= \arctan \left(\frac{h'}{f_2}\right) \simeq \frac{f_1}{f_2} \Delta \theta
\end{align*}
A microscope does something even better: Instead of putting the image $h'$ at the focal point $f_2$, we can place the image $h'$ between the focal point and the lens to create a virtual image that is bigger, but further away than $h'$.\\

Some important things to consider is that the lensmaker's equation assumes that the index of refraction of the surrounding medium is air. If we place a lens in, for example water. The properties of a lens can change!
