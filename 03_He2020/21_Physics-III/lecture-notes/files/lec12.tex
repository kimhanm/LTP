So by integrating with respect to time we get that
\begin{align*}
	\int_{t_1}^{t_2} \frac{d v_{\bot}}{dt} dt = \int_{t_1}^{t_2}\frac{k}{mv_0b}\sin(\theta) \frac{d \Phi}{dt}dt\\
	\implies \int_{0}^{v_0 \sin\theta}d v_{\bot} = \int_{0}^{\pi - \theta} \frac{k \sin \Phi}{m v_0 b} d \Phi
\end{align*}
So we can find out how far apart the particle was, with respect to the measured angle 
\begin{align*}
	v_0 \sin \theta	&= \frac{k}{mv_0b} (1 - \cos(\pi - \theta)) = \frac{k}{mv_0b}(1 + \cos \theta)\\
	\implies b &= \frac{k}{mv_0^2} \cot \frac{\theta}{2}
\end{align*}

Now we can ask how the angle changes, as the height of the path with respect to the gold atom was. I the flux of particles per Area $A$ is $F = \frac{R}{A}$, then
\begin{align*}
	d R_B &= F \cdot 2 \pi b db\\
	db &= -\frac{k}{2mv_0^2 \sin^2 \frac{\theta}{2}} d \theta\\
	dR_b &= dR_{\theta} = \pi F \left(\frac{k}{mv_0^2}\right)^2 \frac{\cos \frac{\theta}{2}}{\sin^3 \frac{\theta}{2}}d \theta
\end{align*}
Where $dR_{\Omega}$ is the rate of particles going trough $d \Omega$. So what is the rate of particles being scattered in the detector solid angle $d \Omega$ for $d \Omega_{\theta} = 2 \pi \sin \theta d \theta$? We then have
\begin{align*}
	\frac{d R_{\Omega}}{d R_{\theta}} &= \frac{d \Omega}{d \Omega_{\theta}} \\
	d R_{\Omega}&= \frac{d R_{\theta}}{d \Omega_{\theta}} d \Omega \\
							&= \pi F \left(\frac{k}{mv_0^2}\right)^2 \frac{\cos \frac{\theta}{2}}{\sin^3 \frac{\theta}{2}} \frac{d \Omega}{2 \pi \sin \theta}\\
							&= \frac{F}{4} \left(\frac{k}{mv_0^2}\right)^2 \frac{1}{\sin^4 \frac{\theta}{2}} d \Omega
\end{align*}

Now if we have a differential cross section $d \sigma$, then the rate at which particles scatter of of $d \sigma$ will be $d R = F d \sigma$.
So for the rutherford scattering we will have
\begin{empheq}[box=\bluebase]{align*}
	d \sigma = \frac{1}{4} \left(\frac{k}{mv_0^2}\right) \frac{1}{\sin^4 \frac{\theta}{2}} d \Omega
\end{empheq}
So by sending $N$ $\alpha$-particles in the area $A$, $n$ atoms in the target area $A$. THen the number of particles scattered into $d \Omega$ will be $\frac{nN}{A} d \sigma$.\\

So the Rutherford scattering shows us that if we keep making the target area smaller, when does the measured amount follows this prediction. So this gives us a bound on how big the nucleus is. If the measurement follows the prediction, this tells us that the coulomb potential is the only thing scattering the particles. If the measurement deviates, this is telling us that there must be other forces at play, i.e. the particles hitting the nucleus.\\


\subsection{Photons}
We saw that the Blackbody gave us some energy of radiation which comes in bundles of $hf = \hbar \omega$.\\

In an experiment, we charged up a plate with electrons. Then we shined light on it and we saw that the charge decreases until the plate is neutral again. If we try this again but we put plastic glass in front of the plate, we see that this time, the plate does not get discharged.\\

If the try the same experiment again, but the charge is positive, then nothing happens.\\

We call this the \textbf{photoelectric effect}, where Photons have the ability to ``remove'' electrons from atoms.\\

In another experiment, we create a circuit consisting of two metal plates, on of which gets shined on with light. We then can measure the amount of electrons displaced by measurig the current.\\

We start out with yellow light and measure some voltage. If we go to green light, the voltage increases a bit. Then blue light gives an even higher voltage, with ultraviolet light giving an even higher voltage.\\

We also see that the speed at which the voltage reaches the maxium voltage depends on the intensity of the light.

If we plot the frequencies with their voltages, we see a linear correspondence. If we were to continue this linear relationship until the frequency $\nu = 0$, then we would write
\begin{align*}
	E_{\max} = h \nu - \Phi
\end{align*}
where $\Phi$ is called the \textbf{work-function}\\ 

Einsteins explained the photoelectric effect as follows: The energy contained in light can only be absorbed by multiples of $h \nu$ and he called this quantum of light the \textbf{photon}.

From this $\Phi$ can be seen as the energy needed to free electrons.\\

The photoelectric effect allows us to convert light into energy, with one method being the \textbf{photomultiplier}, where the ejected electron is accelerated by a strong electric field, crashing into a metal plate which causes more electrons to be ejected and further accelerated.


We can also look at the inverse photoelectric effect which is used in x-ray scanning for example, where electrons get smaehd into an anode, which ejects xray photons.


\subsection{Rayleigh-Scattering}

We can approximate the binding potential of an electron in an atom of second order and get the quadratic terms in the taylor expansion. From this we just get a harmonic oszillator

\begin{align*}
	m \frac{d^2 x}{dt^2} + m \omega_0^2 x = -e E \cos(\omega t)
\end{align*}
which has the solution
\begin{align*}
	x = x_0 \cos(\omega t), \quad \text{where} \quad x_0 = \frac{e E}{m(\omega^2 - \omega_0^2)}
\end{align*}

Which creates an oszillating dipole moment $p$ given by
\begin{align*}
	p = p_0 \cos(\omega t) \propto x_0 \cos(\omega t)
\end{align*}
This in turn generates an electric field that gets emitted
\begin{align*}
	E_{\text{emitted}} \propto \omega^2 p_0 = \frac{\omega^2 e E}{m(\omega^2 - \omega_0^2)} \cos(\omega t)
\end{align*}

Considering that the aproximation only holds for small deviations, we can assume that for our purposes $\omega \ll \omega_0$, which means
\begin{align*}
	E_{\text{emitted}}	\propto \omega^2 \cos(\omega t)
\end{align*}




