\section{Quantum Mechanics}
In the past lectures we saw many experiments which do not fit with classical mechanics, where we are working under the assumption that matter consists of particles and that we could measure properties of a particle without any concern of how the particle will be affected.

Quantum mechanics was created to reconcile the classical view with the experiments covered in the past sections. It is a \emph{mathematical tool} to help us predict the behaviour of the objects we consider.


\subsection{The wavefunction}
In quantum mechanics, matter is described as particle waves, where the state of a quantum system is completely described by the \textbf{wavefunction}, which is a complex valued function with spacetime as its domain.

In the case where we have 1D space, we will write $\psi(x,t)$ which has the following properties.

\begin{enumerate}
	\item The probability of finding the particle at the time $t$ between $x$ and $x + dx$ is $\abs{\psi(x,t)}^2 dx$

	\item The probability density will also have the \emph{normalisation condition}
		\begin{align*}
			\int_{-\infty}^{\infty} \abs{\psi(x,t)}^2 dx = \int_{-\infty}^{\infty}P(x,t)dx = 1
		\end{align*}
\end{enumerate}

For example, if we consider a free particle with momentum $p_0 = \hbar k_0$ this can be described by
\begin{align*}
	\psi(x,t) = A e^{i(k_0x - \omega_0 t)}, \quad \text{for} \quad \hbar w_0 = \frac{p_0^2}{2m}
\end{align*}

Notice that this particle has perfectly described momentum, but in turn its position is spread out through space, which makes it very non-physical. Also notice that the constant $A$ is determined by the normalisation condition.

A useful tool when describing wavefunctions are \textbf{basis sets of functions}. Which is a collection of functions $\left(\phi_{n}\right)$ that are orthonormal:
\begin{align*}
	\int_{D}\phi_n^*(x) \phi_m(x) dx = \delta_{mn}
\end{align*}
and such that any function $\psi(x)$ can be written as a linear combination of these vectors. 
\begin{align*}
	\psi(x) = \sum \psi^n \phi_n(x) \quad \text{where} \quad \psi^n = \int_{D} \phi_n^*(x) \psi(x) dx
\end{align*}
, since from the linearity of the integral we have that
\begin{align*}
	\int_{D}\phi_n^*(x) \psi(x) dx &= \int_D \phi_n^*(x) \sum_{m} \psi^m \phi_m(x) dx\\
																 &= \sum \psi^m \underbrace{\int_{D}\phi_n^*(X) \phi_m(x) dx}_{\delta_mn} = \psi^n
\end{align*}
Now in order to show that the set of functions is complete, it must be that
\begin{align*}
	\psi(x) &\stackrel{!}{=} \sum_{n} \left[\int_D \phi_n^*(x') \psi(x') dx'\right]\phi_n(x)\\
					&= \int_D \left[\sum_{n} \phi_n^*(x') \phi_n(x)\right] \psi(x't) dx'
\end{align*}
Recall that the dirac delta distribution has the property
\begin{align*}
	\int \delta(x - x') g(x') dx' = g(x)
\end{align*}
so in order for the equation to hold, it must be that
\begin{align*}
	\sum_n \phi_n^*(x') \phi_n(x) = \delta(x - x')
\end{align*}
Which gives us the \textbf{completeness condition} for the set of functions $\left(\phi_{n}\right)$.\\


An example of an orthonormal basis of functions is given in the  Fourier series. For the Domain $\left(- \frac{L}{2}, \frac{L}{2}\right)$ we can write a function $\psi(x)$ as
\begin{align*}
	\psi(x = \frac{1}{\sqrt{L}} \sum_{n} \psi^n e^{\frac{i2\pi n}{L}x}
\end{align*}
Where the factor $\frac{1}{\sqrt{L}}$ is needed such that
\begin{align*}
	\int_{-L/2}^{L/2} \abs{\phi_n(x)}^2 dx = 1
\end{align*}
To find the expansion coefficients we can calculate
\begin{align*}
	\int_{-L/2}^{L/2}\psi(x) \phi_m^*(x) dx = \frac{1}{L} \sum_{n} \psi^n \int_{-L/2}^{L/2}e^{\frac{2\pi (n-m)}{L}x}dx\\
	\text{for} \quad \frac{1}{L} \int_{-L/2}^{L/2} e^{\frac{i2\pi (n-m)}{L}x}dx = \delta_{mn}
\end{align*}

If the dimension of our vector space is large enough, we can label the basis through a continuous variable $\alpha$. 
Here the orthonormality of the basis vector is described in terms of the dirac delta:
\begin{align*}
	\int \phi^*(\alpha,x) \phi(\alpha',x) dx = \delta(\alpha - \alpha')
\end{align*}
So instead of having a sum over the coefficients, we integrate
\begin{align*}
	\phi(x) = \int \tilde{\psi(\alpha)}\phi(\alpha,x)d \alpha
\end{align*}
The the ``coefficients'' can be determined by
\begin{align*}
	\psi(\alpha) = \int \phi^*(\alpha,x) \psi(x) dx
\end{align*}


\subsection{Operators}
Operators represent observables, which are procedures that are ``applied'' to functions. For example on the free particle
\begin{align*}
	\psi_k(x,t) = A e^{i(kx - \omega t)}
\end{align*}
we can use the identity operator $\hat{I}$ which multiplies the function by $1$.\\
Another operator is the one that takes the derivative with respect to $x$, so
\begin{align*}
	\frac{\del }{\del x} \psi(x,t) = ik A e^{i(kx - \omega t)} = \frac{i p}{\hbar} \psi(x,t)
\end{align*}
notice that the operator returned a scalar multiple of the function. In this case we call $\frac{ip}{\hbar}$ an \textbf{eigenvalue} and the free particle $\psi(x,t)$ an \textbf{eigenfunction}.
Also notice that we were able to obtain the momentum from this operator. So we can modify it slightly to get the \textbf{momentum operator} $\hat{P} = -i \hbar \frac{\del }{\del x}$.

When we want to measure the position of a particle, we can use the \textbf{position operator} $\hat{x} = x$. Now what are the Eigenfunctions for this operator? We can find this out by solving
\begin{align*}
	\hat{x} \psi_{\alpha}(x) = x \psi_{\alpha}(x) \stackrel{!}{=} \alpha \psi_{\alpha}(x)
\end{align*}
Which can only be satisfied by the dirac delta functions.
\begin{align*}
	\psi_{\alpha}(x) = A \psi(x - \alpha)
\end{align*}

It turns out that every physical, measurement is represented by \textbf{hermitian}  operators, whose eigenvalues are the possible measureable results.
Recall from Linear Algebra that the hermitian operators have real eigenvalues.

That is if you measure an observable, and the system has a wavefunction that is an eigenfunction of the corresponding operator, then the result will always be the eigenvalue.

\subsection{Change of basis}
We know that the eigenfunctions of the momentum operator $\hat{p}$
\begin{align*}
	\phi_p(x) = \frac{1}{\sqrt{2 \pi \hbar}}e^{\frac{ip}{\hbar}x}
\end{align*}
aswell as the Eigenfunctions of the position operator $\hat{x}$
\begin{align*}
	\phi_{\alpha}(x) = \delta(x-\alpha)
\end{align*}
Form basis of the function space. 
This means that we can write any function using a linear combinaion of these basis functions.
If we want to find out the expansion coefficients in terms of the eigenfunctions of $\hat{x}$, then we get
\begin{align*}
\tilde{\psi}(\alpha,t) = \int \delta(x-a) \psi(x,t) dx = \psi(\alpha,t)
\end{align*}
and in terms of the eigenfucntions of $\hat{p}$ we obtain the coefficients by through
\begin{align*}
\tilde{\psi}(p,t)	= \frac{1}{\sqrt{2 \pi \hbar}} \int e^{\frac{-ip}{\hbar}x} \psi(x,t) dx
\end{align*}

Notice that both representation gives us the exact same information about the particle, since the coefficients $\tilde{\psi}(p,t)$ and $\tilde{\psi}(\alpha,t)$ give us the original function $\psi(x,t)$ or its fourier transform $\tilde{\psi}(p,t)$ back.

If we want to know the probability of measuring the system between $x$ and $x + dx$, then we want to use the expansion with respect to $\hat{x}$, since
\begin{align*}
	P(x) dx = \abs{\psi(x,t)}^2 dx
\end{align*}
and if we ant to know the probabilty of measuring the momenum between $p$ and $p + dp$, then we use the expansion with respect to $\hat{p}$
\begin{align*}
	P(p) dp = \abs{\tilde{\psi}(p,t)}^2 dp
\end{align*}
In general for any other basis, we obtain the following theorem.

\begin{ntheorem}[Spectral Theorem]
	The Eigenfunctions of an operator $\hat{O}$ form a complete, orthonormal basis and the probability of measuring an observable to be between $O$ and $O + dO$ is
	\begin{align*}
		P(O)dO = \abs{\psi_O(O,t)}^2 dO
	\end{align*}
\end{ntheorem}

