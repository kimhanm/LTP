\subsection{3D Schrödinger equation}
The generalisation of the Schrödinger Equation in 3D is still
\begin{align*}
	i \hbar \frac{\del \Delta}{\del t} = \hat{H} \Psi
\end{align*}
, but the Hamiltonian has changed a bit. It is now given by

\begin{empheq}[box=\bluebase]{align*}
	\hat{H} = \frac{1}{2m} \abs{\hat{\bm{p}}}^2 + V(\hat{\bm{r}})
\end{empheq}
where $\hat{\bm{r}}$ and $\hat{\bm{p}}$ are vectors made up of operators for each spatial component
\begin{align*}
	\hat{\bm{r}} = (\hat{x},\hat{y},\hat{z}) \quad \text{and} \quad \hat{\bm{p}} = (\hat{p_x}, \hat{p_y}, \hat{p_z})
\end{align*}
In the position representation, we would write
\begin{align*}
		\hat{p_x} = -i \hbar \frac{\del }{\del x},
	\quad
		\hat{p_y} = -i \hbar \frac{\del }{\del y},
	\quad
		\hat{p_z} = -i \hbar \frac{\del }{\del z},
\\
	\implies \hat{\bm{p}} = - i \hbar \nabla
\end{align*}
So if we use the laplacian
\begin{align*}
	\Delta = \nabla^2 = \frac{\del^2}{\del x^2} + \frac{\del^2}{\del y^2} + \frac{\del^2}{\del z^2}
\end{align*}
we can write the schrödinger Equation as follows
\begin{empheq}[box=\bluebase]{align*}
	i \hbar \frac{\del \Delta}{\del t} = - \frac{\hbar^2}{2m} \Delta \Psi + V \Psi
\end{empheq}

These new operators commute/ don't commute in the following way
\begin{enumerate}
	\item For different cartesian coordinates, they commute:
		\begin{align*}
			[\hat{x},\hat{p_y}] = [\hat{x},\hat{p_z}] = [\hat{y},\hat{p_z}] = \ldots = 0
		\end{align*}
	\item Since we assume smoothness of our wavefunction, the position operators also commute by Schwarz-Clairaut's theorem.
		\begin{align*}
			[\hat{p_x},\hat{p_y}] = [\hat{p_y},\hat{p_z}] = [\hat{p_z},\hat{p_x}] = 0
		\end{align*}
	\item The position and momentum in the same catesian coordinate do not commute, just like in the 1D case
		\begin{align*}
			[\hat{x},\hat{p_x}] = [\hat{y},\hat{p_y}] = [\hat{z},\hat{p_z}] = i \hbar
		\end{align*}
\end{enumerate}

THe normalisation condition for the 3D case is now
\begin{align*}
	\int \abs{\Psi(\bm{r},t}^2 dx dy dz = 1
\end{align*}
, where $\int \abs{\Psi(\bm{r},t}^2 dx dy dz$ is the probability of finding the particle in the volume $dxdydz$.


Often, the potential will be have radial symmetry, which depend solely on the distance form the origin. We call these \textbf{central potentials}.

For those it makes sense to use spherical coordinates, where the Laplacian has the form
\begin{align*}
		\Delta 
	= 
		\frac{1}{r^{2}} \frac{\del }{\del r} \left(
			r^{2} \frac{\del }{\del r}
		\right)
		+
		\frac{1}{r^{2}\sin \theta} \frac{\del }{\del \theta}\left(
			\sin \theta \frac{\del }{\del \theta}
		\right)
		+
		\frac{1}{r^{2} \sin^{2} \theta} \frac{\del^2}{\del \phi^2} 
\end{align*}

The symmetry means that we can seperate solutions for the problem into products of functions which depend only on the radial cordinate and the angular ones, so we can write
\begin{align*}
	\Psi(r,\theta,\phi) = R(r) Y(\theta,\phi)
\end{align*}
so the TISE looks like this
\begin{align*}
	-\frac{\hbar^{2}}{2m} \left[
		\frac{Y}{r^{2}} \frac{\del }{\del r}\left(r^{2} \frac{\del R}{\del r}\right)
		+
		\frac{R}{r^{2}\sin \theta} \frac{\del }{\del \theta} \left(
			\sin \theta \frac{\del Y}{\del \theta}
		\right)
		+
		\frac{R}{r^{2}\sin^{2}\theta} \frac{\del^2 Y}{\del \phi^2}
	\right]	+ VRY = ERY
\end{align*}
Dividing by $RY$ and multiplying by $- \frac{2m r^{2}}{\hbar^{2}}$ we get
\begin{align*}
	{\color{orange}\left[
			\frac{1}{R}\frac{\del }{\del r}\left(
				r^{2}\frac{\del R}{\del r}
			\right)
			-
			\frac{2mr^{2}}{\hbar^{2}} \left(
				V(r) - E
			\right)
	\right]}
	+
	{\color{blue}\frac{1}{Y}\left[
			\frac{1}{\sin \theta} \frac{\del }{\del \theta} \left(
				\sin \theta \frac{\del Y}{\del \theta}
			\right)
			+
			\frac{1}{sin^{2} \theta} \frac{\del^2 Y}{\del \phi^2}
	\right]} = 0
\end{align*}
Where we seperated the terms that only depend on ${\color{orange}r}$ and the terms that depend on ${\color{blue}\theta}$ and ${\color{blue}\phi}$.

For this to hold true for all $r,\theta,\phi$, it must be orange and blue terms are constant. Let's say that they equal $l(l+1)$ and $-l(l+1)$ respectively.

We first solve the ${\color{blue}\text{angular}}$ part.
If we separate the variables 
\begin{align*}
	Y(\theta,\phi) = T(\theta) \Phi(\phi)
\end{align*}
and plug this into the angular equation, we multiply everything by $\sin^{2}\theta$ which gives us
\begin{align*}
	\frac{1}{T}\left[
		\sin \theta \frac{\del }{\del \theta} \left(
			\sin \theta \frac{\del T}{\del \theta}
		\right)
		+
		l(l+1) \sin^{2}\theta
	\right]
	= - \frac{1}{\Phi}\frac{\del^2 \Phi}{\del d \phi^2}
\end{align*}
where both sides again only depend on one variable. This also means that they equal a constant which we call $m^2$.

THe part in $\Phi$ is a simply differential equation with solution
\begin{align*}
	\frac{1}{\Phi} \frac{\del^2 \Phi}{\del \phi^2} = -m^2 \implies \Phi(\phi) = e^{im \phi}
\end{align*}
But since the exponential is $2\pi i$ periodic, it means that $m$ must be an integer.

If we look at the equation for $T(\theta)$:
\begin{align*}
		\sin \theta \frac{\del }{\del \theta} \left(
			\sin \theta \frac{\del T}{\del \theta}
		\right)
		+
		\left[
			l(l+1) \sin^{2}\theta - m^2
		\right]
	=
		0
\end{align*}
we can show that the solutions are given by the \textbf{associated Legendre functions} $P_l^{m}(x)$, which are defined as
\begin{empheq}[box=\bluebase]{align*}
	P_l^{m}(x) &:= (-1)^{m}(1- x^{2})^{\frac{m}{2}} \left(
		\frac{\del }{\del x}
	\right)^{m}
	P_l(x)\\
		P_l^{-m}(x) &:= (-1)^{m} \frac{(l-m)!}{(l+m)!}P_l^{m}(x)
\end{empheq}
, where $P_l(x)$ is called the $l$-th \textbf{Legendre Polynomal}
\begin{align*}
	P_l(x) := \frac{1}{2^{l}l!} \left(
		\frac{\del }{\del x}
	\right)^{l} (x^{2} - 1)^{l}
\end{align*}
Note that the solutions only make sense if $l$ is an integer.

Also if $\abs{m} > l$, then since $P_l(x)$ is a $l$-th order Polynomial $P_l^{m}(x) = 0$.
The first couple associated polynomials look like this
\begin{align*}
	P_0^{0} = 1, \quad P_1^{0} = \cos \theta, \quad P_1^{1} = - \sin \theta\\
	P_2^{0} = \frac{1}{2}(3 \cos^{2}\theta -1), \quad P_2^{1} = -3 \sin \theta \cos \theta, \quad P_2^{2} = 3 \sin^{2}\theta
\end{align*}

Combining these with $\Phi(\phi)$, we get the angular wavefunctions, which are called \textbf{spherical harmonics} $Y_l^{m}(\theta,\phi)$, which are given by
\begin{align*}
	Y_l^{m}(\theta,\phi) = \sqrt{\frac{(2l+1)(l-m)!}{4\pi (l+m)!}} e^{im \phi}P_l^{m}(\cos \theta)
\end{align*}
which is defined for $l \in \N, m \in \{-l, \ldots l\}$.
The first few spherical harmonics are
\begin{align*}
	Y_0^{0} &= \sqrt{\frac{1}{4\pi}}\\
	Y_1^{0} &= \sqrt{\frac{3}{4\pi}} \cos \theta, \quad Y_1^{\pm 1} = \mp \sqrt{\frac{3}{8\pi}}\sin \theta e^{\pm i \phi}\\
	Y_2^{0} &= \sqrt{\frac{5}{16\pi}}, \quad Y_2^{\pm 1} = \mp \sqrt{\frac{15}{8\pi}} \sin \theta \cos \theta e^{\pm i \phi}\\
	Y_2^{\pm 2} &= \sqrt{\frac{15}{32\pi}}\sin^{2}(\theta)e^{\pm 2i \phi}
\end{align*}

Note that the spherical harmonics are ortho-normal:
\begin{align*}
	\int_{0}^{\pi}\int_{0}^{2 \pi} \left(Y_{l'}^{m'}\right)^\ast Y_{l}^{m} \sin \theta d \theta d \phi = \delta_{ll'}\delta_{mm'}
\end{align*}


The differential equation for the radial part of the wavefunction is
\begin{align*}
	\frac{\del }{\del r}\left(
		r^2 \frac{\del R}{\del r}
	\right) - \frac{2Mr^{2}}{\hbar^{2}}\left(
	V(r) - \Phi
	\right)
	R = l(l+1)R
\end{align*}
And let $u(r) = r R(R)$, then
\begin{align*}
	\frac{\del }{\del r}\left(
		r^{2} \frac{\del R}{\del r}
	\right) = r \frac{\del^2 u}{\del r^2}
\end{align*}
and we get the equivalent formulation
\begin{align*}
	-\frac{\hbar^{2}}{2m} \frac{\del^2 u}{\del r^2} + \left[V(r) + \frac{\hbar^{2}}{2m} \frac{l(l+1)}{r^2}\right] u = Eu
\end{align*}
Which also simplifies the normalisation condition
\begin{align*}
	\int_{0}^{\infty}r^2 \abs{R}^2 dr = 1 \iff \int_{0}^{\infty}\abs{u}^2 dr = 1
\end{align*}
