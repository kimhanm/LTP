\subsection{Angular momentum}
In classical mechanics, we are often interested in finding conserved quantities and symmetries as they can greatly help us when analyzing problems.

In quantum mechanics, the conserved quantities come in the form of measurements of observables that commute with the Hamiltonian when the system is in an eigenstate of both the eigenstate and the Hamiltonian.
In other words: we can measure the eigenvalues of eigenstates of operators that commute with the Hamiltonian.

A trivial operator would the Hamiltonian itself, so energy would be a conserved quantity.


Here we will define the \textbf{angular momentum} Operator(s), using the position operator $\vec{x}$ and momemtum Operator $\vec{p}$. 
\begin{align*}
	\vec{\hat{L}} = \vec{\hat{x}} \times \vec{\hat{p}}
\end{align*}
, or for each of the three components:
\begin{align*}
	\hat{L}_x = \hat{y}\hat{p}_z - \hat{z} \hat{p}_y, \quad \hat{L}_y = \hat{z} \hat{p}_x - \hat{x} \hat{p}_z, \quad \hat{L}_z = \hat{x} \hat{p}_y - \hat{y} \hat{p}_x
\end{align*}
This definition follows the classical definition of angular momentum $\vec{L} = \vec{r} \times \vec{p}$.

we will from now on stop writing the hats as it should be clear from context.

Let's consider the commutator between two of the components. We see that they do not commute.
\begin{align*}
	[L_x,L_y] &= [yp_z - zp_y, zp_x - xp_z]\\
						&= [yp_z, z p_x]
						- \underbrace{[yp_z - xp_z]}_{=0}
					- \underbrace{[zp_y, zp_x]}_{=0}
						+ [zp_y,xp_z]\\
						&= yp_x[p_z,z] + xp_y[z,p_z] = i \hbar (xp_y - y p_x)\\
						&= i \hbar L_z
\end{align*}
This means that the different components of angular momentum are simultaneously observable, so it doesn't really make sense to think of angular momentum as a vector as it's direction is fuzzy.

Similarily, we get the other commutators
\begin{align*}
	[L_y,L_z] = i \hbar L_x, \quad \text{and} \quad [L_z,L_x] = i \hbar L_y
\end{align*}
Note that the square $L^2 = L \circ L$ does commute with each of the components
\begin{align*}
    [L^2,L_x] 
  &= 
    [L_x^2,L_x] + [L_y^2, L_x] + [L_z^2,L_x]\\
  &= 
    L_y[L_y,L_x] + [L_y,L_x]L_y + L_z[L_z,L_x] + [L_z,L_x]L_z\\
  &= L_y(-i \hbar L_z) + (-i \hbar L_z)L_y + L_z(i \hbar L_y) + (i \hbar L_y)L_z = 0\\
  \text{similarly:}\quad [L^2,L_y] = [L^2,L_z] = 0
\end{align*}
This means that sould be able to find simultaneous eigenstates of $L^2$ and $L_z$.
Let's denote the eigenstates as $\ket{l,m}$, so the equaion will look like
\begin{align*}
  L^2 \ket{l,m} = \hbar^{2}l(l+) \ket{l,m} \quad \text{and} \quad L_z \ket{l,m} = \hbar m \ket{l,m}
\end{align*}
where the $\hbar$ are such that $l$ and $m$ are dimensionless.

Now recall the ladder operators $a$ and $a^{\dagger}$ from the quantum harmonic oscillator, with which were able to create new eigenstates for our system.
They satisfied the commutator relations
\begin{align*}
  [a^{\dagger}a,a^{\dagger}] = a^{\dagger} \quad \text{and} \quad [a^{\dagger}a,a] = -a
\end{align*}
We can find a similar relation between the operator pairs
\begin{align*}
  L_{\pm} := L_x \pm iL_y
\end{align*}
which gives the following commutator relations
\begin{align*}
  [L_z,L_{\pm}] =\pm \hbar L_{\pm}, \quad [L^2, L_{\pm}] = 0
\end{align*}
where we can think of $L_z$ as being like $a^{\dagger}a$ and $L_{\pm}$ as $a^{\dagger}$ and $a$.

So we can create new states like this:
\begin{align*}
    L^2(L_{\pm} \ket{l,m}) 
  &= 
    L_{\pm} L^2 \ket{l,m} 
  = 
    \hbar^{2}l(l+1) (L_{\pm} \ket{l,m})
\\
    L_z(L_{\pm}\ket{l,m})
  &= 
    (\pm \hbar L_{\pm} + L_{\pm}L_z) \ket{l,m}
  =
  \hbar(m \pm 1)(L_{\pm} \ket{l,j})
\end{align*}
which shows that $L_{\pm}\ket{l,m}$ is also an eigenstate of $L^{2}$ with eigenvalue $\hbar^{2} l(l+1)$ and that $L_{\pm}\ket{l,m}$ is an eigenstate of $L_z$ with eignevalue $\hbar(m\pm 1)$
We can bring this into a nicer form by setting
\begin{align*}
  A_{l,\pm}^{m} := \hbar \sqrt{l(l+1) - m(m\pm 1)}
\end{align*}
to get the ladder relation similar to what we found in the harmonic oscillator
\begin{align*}
  L_{\pm} \ket{l,m} = A_{l,\pm}^{m}\ket{l,m \pm 1}
\end{align*}
likewise, we can again show what kinds of values $l$ and $m$ can take by considering the following:

The operator $L_{+}$ lets us increase the eigenvalue of $L_z$ by $\hbar$ and $L_-$ decreases the eigenvalue. But this process has to end somewhere, or else we would have measured that the the $z$-component of the angular momentum has a bigger magnitude that the total angular momentum itself.

So there must be some maximal value of $m$, say $\mu:= m_{\max}$ aswell als a minimal value $\mu' := m_{\min}$ such that the Ladder operators become zero:
\begin{align*}
  L_{+}\ket{l,\mu} = 0 \quad \text{and} \quad L_{-} \ket{l,\mu'} = 0
\end{align*}
we are also able to find out these relations
\begin{align*}
  L_{\pm} L_{\mp} = (L_x \pm i L_y)(L_x \mp i L_y) = L_x^2 + L_y^2 \mp i(L_xL_y - L_yL_x) = L^2 - L_z^{2} \pm \hbar L_z
\end{align*}
solving the above for $L^2$, we see that
\begin{align*}
  L^2 \ket{l,\mu} = (L_{-}L_{+}+ L_z^{2} + \hbar L_z) \ket{l,\mu} = \hbar^{2} \mu(\mu + 1) \ket{l,\mu}
\end{align*}
but we also knew that $L^2$ fulfilled the eigenstate equation
\begin{align*}
  L^2 \ket{l,\mu} = \hbar^{2} l (l+1) \ket{l,\mu}
\end{align*}
so we immediately see that $m_{\max} = \mu = l$. In a similar fashion we can show
\begin{align*}
  L^2 \ket{l,\mu'} = \hbar^{2} \mu'(\mu'-1) \ket{l,\mu'}
\end{align*}
This equation on itself has two solutions, $\mu' = -l$ and $\mu' = l+1$, but the latter would mean $\mu' > \mu$.  

To see what values $l$ can take on we notice that the interval $[-l,l]$ must have the length of an integer because we take an integer multiple times the application of $L_{+}$ to increase $m$ by one starting from $-l$ to $l$.

So the possible values for $l$, and $m$ are
\begin{align*}
  l = 0, \tfrac{1}{2}, 1, \tfrac{3}{2}, \ldots \quad \text{and} \quad m = -l, -l+1, \ldots l+1,l
\end{align*}
which look suspiciously like the possibles values of $l$ and $m$ for the spherical harmonics except that now we can have half-integer values.
Also notice that the maximum measured result for $L_z$, given by
\begin{align*}
  L_z \ket{l,m} = \hbar m \ket{l,m} \implies L_{z,\max} = \hbar l
\end{align*}
from before is always lass than the total magnitude $\hbar\sqrt{l(l+1)}$ (except for $l = 0$) which we got from
\begin{align*}
  L^{2}\ket{l,\mu} = \hbar^{2} \mu(\mu+1) \ket{l,\mu}
\end{align*}

To further strenghten the connection between the eigenstates $\ket{l,m}$ and the spherical harmonics $Y_l^{m}$, let's re-write the angular momentum operators in spherical coordinates. In the position representation, it is:
\begin{align*}
  \vec{L} = - i \hbar (\vec{r} \times \vec{\nabla})
\end{align*}
so using the spherical nabla operator
\begin{align*}
  \vec{\nabla} = \vec{r} \frac{\del }{\del r} + \vec{\theta} \frac{1}{r}\frac{\del }{\del r} + \vec{\phi} \frac{1}{r \sin \theta} \frac{\del }{\del \phi}
\end{align*}
we can find that the cartesian components $L_x,L_y,L_z$ of the angular momentum are given by
\begin{align*}
  L_x &= i \hbar\left(
    \sin \phi \frac{\del }{\del \theta} + \cos \phi \cot \theta \frac{\del }{\del \phi}
  \right)\\
  L_y &= i \hbar\left(
    -\cos \phi \frac{\del }{\del \theta} + \sin \phi \cot \theta \frac{\del }{\del \phi}
  \right)\\
    L_z &= -i \hbar \frac{\del }{\del \phi}
\end{align*}
aswell as the square of the Angular momentum operator
\begin{align*}
  L^2 = - \hbar \left[
    \frac{1}{\sin \theta} \frac{\del }{\del \theta} \left(
      \sin \theta \frac{\del }{\del \theta}
    \right)
    + \frac{1}{\sin^{2} \frac{\del^2}{\del \phi^{2}}}
  \right]
\end{align*}
now recall from spherical harmonics that our equation for the angular part was
\begin{align*}
  \left[
    \frac{1}{\sin \theta} \frac{\del }{\del \theta} \left(
      \sin \theta \frac{\del }{\del \theta} Y_{lm}  
    \right)
    + \frac{1}{\sin^{2} \theta} \frac{\del^2}{\del \phi^{2}} Y_{lm}
  \right]
  =
  -l(l+1)Y_{lm}
\end{align*}
which is equivalent to the eigenvalue equation
\begin{align*}
  L^2 \ket{l,m} = \hbar^{2} l(l+1)\ket{l,m}
\end{align*}
and likewise for the phi component, the equation 
\begin{align*}
  \frac{\del^2}{\del \phi^{2}Y_{lm}} = -m^{2} Y_{lm}
\end{align*}
is equivalent to the eigenvalue equation
\begin{align*}
  L_z \ket{l,m} = \hbar m \ket{l,m}
\end{align*}
Also, the Hamiltonian can be written as
\begin{empheq}[box=\bluebase]{align*}
  H = \frac{1}{2mr^{2}} \left[
    -\hbar^{2} \frac{\del }{\del r} \left(
      r^{2} \frac{\del }{\del r}
    \right)
    + L^{2}
  \right] + V
\end{empheq}
so any eigenstate of the central potential hamiltonian $H$ is simultanously an eigenstate of $L^{2}$ and $L_z$
\begin{align*}
  H \psi = E \psi, \quad L^{2} \psi = \hbar^{2}l(l+1)\psi,\quad L_z \psi = \hbar m \psi
\end{align*} 

\subsection{Spin}
The connection between angular momentum and spherical harmonics only works for the integer values of $l$, so we still need to discuss the half-integer values of $l$.

Although half-integer values might seem odd, there are physical experiments that point toward the existence of a quanttity that behaves like angular momentum aswell as other experimental observations that hint at the existence of half-integer values of $l$.

In particular, when taking into account special relativity in a high-resolution spectroscopy of hydrogen, we see two spectral lines for the Balmer series transition $l = 1 \to l = 0$, where we would only expect $2l + 1$ of them, which is an odd number for integer values of $l$.


Another example that we will look at is an experiment performed by Otto Stern and Walther Gerlach in 1922, in which they set out to measure the magnetic moment of atoms.
Recall that the magnetic moment $\vec{\mu}$ of a current $I$ in a loop with radius $r$ is given by
\begin{align*}
  \vec{\mu} = \pi r^{2} I \hat{n}
\end{align*}
, where $\hat{n}$ is the unit normal vector perpendicular to the plane in counter-clockwise direction. In the Bohr model of the atom, the electron has a circular orbit around the nucleus. 

So if $v$ is the speed of the electron, $e$ its charge and $r$ the radius of the orbit, then the current $I$ is
\begin{align*}
  I = -e \left(
    \frac{v}{2 \pi r}
  \right) \implies \vec{\mu} = - \pi r^{2} e \frac{v}{2 \pi r}\hat{n} = -\frac{e}{2m}\vec{r} \times \vec{p} = - \frac{e}{2m} \vec{L}
\end{align*}
so when the atom is placed in a magnetic field $\vec{B}$ along the $z$ direction, it would experience a force $\vec{F}$ equal to the gradient of the Potential $- \vec{\mu} \cdot \vec{B}$
\begin{align*}
  \vec{F} = - \nabla(-\vec{\mu} \cdot \vec{B}) = \mu_z \frac{\del B_z}{\del z}\hat{z} \propto L_z \frac{\del B_z}{\del z}\hat{z}
\end{align*}
Here we assume that the field gradient are mostly along the $z$ direction, so atoms with different $z$ component of angular momentum will experience different forces.
But also, we know that $L_z$ can only take on $2l+2$ values which should be an odd number for integer $l$

The Stern-Gerlach experiment used an apparatus, where silver atoms pass through an inhomogeneous filed and land on a glass plate with the location where tey land indicating the force that the atoms experienced from the magnetic field.

They observed that not only were the measured outcomes quantized, they found that there were two distinct locations.

To explain this, a new property called \textbf{spin} was proposed which behaved similar to angular momentum in that the \textbf{spin operator} $S$ satisfied the commutator and eigenvalue equations
\begin{align*}
  [S_x,S_y] = i \hbar S_z, \quad
  [S_y,S_z] = i \hbar S_x, \quad
  [S_z,S_x] = i \hbar S_y\\
  S^{2} \ket{s,m_s} = \hbar^{2} s(s+1) \ket{s,m_s} \quad \text{and} \quad S_z \ket{s,m_s} = \hbar m \ket{s,m_s}
\end{align*}
where $s$ and $m_s$ can take on the values
\begin{align*}
  s = 0, \tfrac{1}{2}, 1, \frac{3}{2}, \ldots \quad \text{and} \quad m_s = -s, -s + 1, \ldots, s-1, s
\end{align*}
It turns out that unlike angular momentum, each elementary particle has a specific value of $s$ that is conserved, i.e. never changes. Electrons have spin $\tfrac{1}{2}$, photons have spin $1$ and so on.

The most important case is $s = \tfrac{1}{2}$. Most particles that make up ordinary matter are leptons and have spin $\tfrac{1}{2}$. 
This also means that they can have only two eigenstates $m = - \tfrac{1}{2}, \tfrac{1}{2}$ which is nice because qunatum systems that only have two eigenstates can be described by spin-$\tfrac{1}{2}$ particles, even if the system has no particles with such spin.
It also features a quantum analogue of the classical bit with two states $0$ and $1$, which in the quantum world are called qubits.

We often describe spin-$\tfrac{1}{2}$ particles in vector-notation using the eigenstates as a basis:
\begin{align*}
  \ket{l = \tfrac{1}{2}, m_s = -\tfrac{1}{2}} \leftrightarrow \begin{pmatrix}
  0 \\ 1
  \end{pmatrix}, \quad
  \ket{l = \tfrac{1}{2}, m_s = \tfrac{1}{2}} \leftrightarrow \begin{pmatrix}
  1 \\ 0
  \end{pmatrix} 
\end{align*}
where a general state $\ket{\psi}$ can be written as
\begin{align*}
  \ket{\psi} = \alpha \ket{\tfrac{1}{2}, \tfrac{1}{2}} + \beta \ket{\tfrac{1}{2}, -\tfrac{1}{2}} \leftrightarrow \begin{pmatrix}
  \alpha \\ \beta
  \end{pmatrix}
\end{align*}
We've seen how operators acting on states can be viewed as matrices. 
In our case, we have a $2$-dimensional vector space so for example, we can write 
\begin{align*}
  \hat{S}_z = \frac{\hbar}{2} \begin{pmatrix}
  1 & 0\\
  0 & -1
  \end{pmatrix} =: \frac{\hbar}{2} \sigma_z
\end{align*}
we see that the eigenstates to $s = \tfrac{1}{2}$ are indeed eigenvectors are indeed eigenvectors of the matrix $\sigma_z$/eigenstates of the $\hat{S}_z$ operator
\begin{align*}
 \hat{S}_z \ket{\tfrac{1}{2}, -\tfrac{1}{2}} = \frac{\hbar}{2} \begin{pmatrix}
 1 & 0\\
 0 & -1
 \end{pmatrix}
 \begin{pmatrix}
 0 \\ 1
 \end{pmatrix}
 = - \frac{\hbar}{2} \begin{pmatrix}
 0 \\ 1
 \end{pmatrix}\\
 \hat{S}_z \ket{\tfrac{1}{2}, \tfrac{1}{2}} = \frac{\hbar}{2} \begin{pmatrix}
 1 & 0\\
 0 & -1
 \end{pmatrix}
 \begin{pmatrix}
 1 \\ 0
 \end{pmatrix}
 = - \frac{\hbar}{2} \begin{pmatrix}
 1 \\ 0
 \end{pmatrix}
\end{align*}
Recall from linear algebra if we have an operator $\hat{O}$ and some orthonormal basis $\{\ket{\phi_n}\}$, then we can recover the matrix coefficents $o_{ij}$ as follows
\begin{align*}
  o_{ij} = \braket{\phi_i| \hat{O}|\phi_j}
\end{align*}
so if we want to find out what the matrix $\sigma_{+}$ representing the $S_+$ operator looks like, we can calcululate $\braket{\tfrac{1}{2},m | S_+| \tfrac{1}{2}, m'}$ for $m,m' = - \tfrac{1}{2}, \tfrac{1}{2}$.

Because the ladder operator satisfies
\begin{align*}
  S_{\pm} \ket{s,m} = \hbar \sqrt{s(s+1) - m(m\pm 1)} \ket{s, m \pm 1}
\end{align*}
we know that
\begin{align*}
  \braket{\tfrac{1}{2}, \tfrac{1}{2}|\hat{S}_+|\tfrac{1}{2},-\tfrac{1}{2}} =  \hbar \braket{\tfrac{1}{2},\tfrac{1}{2}} = \hbar
\end{align*}
and it vanishes for all the others. Therefore, the matrix representation $\sigma_+$ is given by
\begin{align*}
  \hat{S}_+ = \hbar \begin{pmatrix}
  0 & 1\\
  0 & 0
  \end{pmatrix}, \quad \implies
  \begin{pmatrix}
  1 & 0
  \end{pmatrix}
  \cdot \hbar \begin{pmatrix}
  0 & 1\\
  0 & 1
  \end{pmatrix}
  \cdot \begin{pmatrix}
  0 \\ 1
  \end{pmatrix}
  = \hbar \checkmark
\end{align*}
Similarly, for $S_-$ we find 
\begin{align*}
  S_- = \hbar \begin{pmatrix}
  0 & 0\\
  1 & 0
  \end{pmatrix}
\end{align*}
and using the relations 
$S_x = \frac{1}{2}\left(S_+ + S_-\right), 
S_y = \frac{1}{2i}\left(S_+ - S_-\right)$
, we get 
\begin{align*}
  S_x = \frac{\hbar}{2}\begin{pmatrix}
   0 & 1\\
   1 & 0
  \end{pmatrix}
  =: \frac{\hbar}{2} \sigma_x \quad \text{and} \quad S_y = \frac{\hbar}{2} 
  \begin{pmatrix}
    0 & -i\\ i & 0
  \end{pmatrix}
  =: \frac{\hbar}{2}\sigma_y
  \end{align*}
We call $\sigma_x, \sigma_y, \sigma_z$ the \textbf{Pauli matrices}, which are just the matrix representations of the corresponding Operators.

For example, if we ask what the probability of a state $\ket{\psi} = \begin{pmatrix}
a \\ b
\end{pmatrix}$ measuring as an eigenvector of $S_x$ is, we take the dot product with the normed eigenstate
\begin{align*}
  \abs{\braket{\phi|\psi}}^{2} = \frac{1}{2} \begin{pmatrix}
  1 & 1
  \end{pmatrix}
  \cdot \begin{pmatrix}
  a \\ b
  \end{pmatrix}
  = \frac{\abs{a + b}^{2}}{2}
\end{align*}

Now we can explain the Stern-Gerlach experiment by particles which have spin $\tfrac{1}{2}$. The two eigenstates with this spin experience a force in opposite directions.

By doing this experiment with electrons, we can determine the magnetic moment of the electrons, which was found to be
\begin{align*}
  \mu = \pm \frac{e \hbar}{2 m_e} =: \pm \mu_B
\end{align*}
where $\mu_B$ is the called the \textbf{Bohr magneton}
