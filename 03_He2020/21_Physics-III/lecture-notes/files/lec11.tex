In order to better under stand matter better, let's first look at some key facts: Atoms are made up of
\begin{tabular}{lll}
	& Charge & Mass\\
	Protons & $1.6 \cdot 10^{-19}$ Coulomb & $1.66 \cdot 10^{-27}\kilo\gram$\\
	Neutrons & $0$ & $1.66 \cdot 10^{-27}\kilo\gram$\\
	Electrons & $-1.6 \cdot 10^{-19}\kilo\gram$ & $9.11 \cdot 10^{-31}\kilo\gram$
\end{tabular}

The different atoms can be ordered in the periodic table, where many of their properties can be seen, such as their atomic mass, (which is not quite an integer since atoms can have multiple \emph{isotopes}) and the number of protons (which determines the name we give an atom).

For example, Carbon has $6$ protons and electrons, but there are two isotopes. One with $6$ neutrons and one with $7$.

So how can we tell different isotopes apart? We can find out their atomic mass, by making a gas and using the ideal gas law $pV = Nk_B T$ and measuring the pressure, volume and temperatur, we can infer $N$ and after weighing the gas, we can find out the average mass per atom.

This method is obviously pretty unrealistic, as you would need a gigantic amount of them.

When calculating these masses, we found that the mass is about an integer multiple of the mass of a Hydrogen atom.

But what's bad about using Hydrogen as the ground base is that it is very reactive, so people have opted to base it around the carbon atom and defined the \textbf{atomic mass unit} $u := \frac{1}{12}$-th of the mass of a carbon atom.


A better way of measuring the mass of an atom is to heat up the atoms and shoot the ion beam through a canal where we have an electromagnetic field and measure how much it gets redirected by the field.

The force on an ion is going to be given be the Lorentz equation
\begin{align*}
	\vec{F} = q(\vec{E} + \vec{v} \times \vec{B})
\end{align*}
The effect of the electiv field will then be
\begin{align*}
	m \frac{d^2 y}{d t^2} = qE \implies y = \frac{q E t^2}{2m} \simeq \frac{qEl^2}{2mv^2}
\end{align*}
where used the approximation that the velocity in the forward directionn will not change much, as the canal will be short.

The effect of the magnetic field will be 
\begin{align*}
	m \frac{d^2 x}{d t^2} = -qvB \implies x = -\frac{1vB}{2m}t^2 \simeq -\frac{qBl^2}{2mv}
\end{align*}
Next we want to elminitate the velocity from the equation by writing out the relationship between the $x$ and $y$ deflections and we get
\begin{align*}
	y = \frac{m}{q} \frac{2E}{B^2l^2}x^2
\end{align*}
which is a parabola! So by looking where the the point lies on the parabola, regardless of its velocity, we can find out its mass, assuming we know $q, E, B, l$.

Now after we can measure the mass of atoms, we want to know how we can measure the size of atoms. This turns out to not be as simple as placing a ruler beside it as we can't really create a ruler (which is made of atoms) small enough.

The idea is that we measure the size by throwing other things at it and seeing what happens.

We could, for example, throw other atoms, electros or an $\alpha$-particle (nucleus of helium) at it.

How close can atoms get to each other? By using Van der Waals equation
\begin{align*}
	(P + \frac{q}{V^2})(V - b) = Nk_B T
\end{align*}
By measuring the relations between $P,V,N,T$ we can sketch out what $a$ and $b$ look like, which tells us the size of an atom.

Another way is by shining X-rays onto the atom and looking at the diffraction pattern to determine the size of the atom.

Another, more modern way is called \textbf{scattering}. The basic idea is that we heat up the gas in an oven and open a small hole in the oven, so that the gas can escape through a chamber with another gas in it, called the scattering chamber. By varying the length of the second gas in the chamber and by measuring how much gas makes it throgh we can find out how much area the second gas takes away from the path to the detector which gives us the size of the atoms.

The area that will be covered is called the \emph{scattering cross section} and is $\gamma = \pi(r_1 + r_2)^2$, where the gas beam will be deflected.\\

Then we ask what is the probabilty that an atom in the beam will get scatted by an atom in the chamber when it flies trough? It will be
\begin{align*}
	P_s &= \frac{\text{total cross section of all atoms in $A dx$}}{A}\\
			&= n \sigma dx
\end{align*}
so $N \cdot P_s$ atoms will get scattered. So the change in the number of atons will be
\begin{align*}
	dN = -nP_S = - n \sigma N dx
\end{align*}
so after the distance $x$ the number of remaining atoms will be
\begin{align*}
	N(x) = N_0 \cdot e^{-n \sigma x}
\end{align*}
What we found from these measurements was that the size of the atom was on the order of a few Anstroms, ($\simeq 10^{-10}\meter$).\\

The problem with this measurement was that we were assuming that the atoms were ball shaped or had a hard shell around them, which was unsure at the time.

To probe further, we need to find out the internal structure of an atom.\\
Some tested the scattering experiment with electrons. And from the scattering equation above we expect that an electron can travel for $10^{-7}\meter$ without being disturbed.\\
But what we found instead was the the electrons could travel centimeters before betting scattered by a significant amount. Moreover, we found that faster electrons were less likely to scatter than slower moving electrons.\\

Another experiment was done be Rutherford. Instead of shooting electrons into a gas, he shot $\alpha$ particles at a thin gold foil. These particles will then be scattered by the foil and we will measure the scattering distribution.

If $b$ is the height above which the electron flies above the gold atom and we denote $\theta$ to be the angle of deflection and $\Phi$ to be the angle from the gold atom to the electron, when we have
\begin{align*}
	\vec{F} = \frac{2ze^2}{4 \pi \epsilon_0 r^2} \hat{r} = \frac{k}{r^2}\hat{r}
\end{align*}
So since the force is always radial, it means that the angular momentum $\vec{L} = m \vec{r} \times \vec{r}$ will be preserved. So
\begin{align*}
	L_{\text{init}} = mv_0 b = m \dot{\Phi}r^2 = L_{\text{trajectory}} \implies \frac{1}{r^2} = \frac{\dot{\Phi}}{v_0b},
\end{align*}
Which gives us
\begin{align*}
	\vec{F} = \frac{k \dot{\Phi}}{v_0 b}\hat{r} \quad \text{and} \quad \vec{F}_{\bot} = \frac{k \dot{\Phi}}{v_0 b} \sin \Phi\\
	\implies m \dot{v}_{\bot} = \frac{k \dot{\Phi}}{v_0 b} \sin \Phi
\end{align*}
so we can integrate this term from the initial time to som time in the very future where the particle is infinitely far away. 
Since we are dealing with conservative foreces, the final kinetic energy will the qual the initial kinetic energy, so
\begin{align*}
	v_{\bot}(t_2) =  v_0 \sin \theta
\end{align*}
and at time $t_2$ we have $\Phi = \pi - \theta$.
