We saw that we could associate a particle with its \emph{de Broglie Wavelength}. But how exactly can the behaviour of particles be described as a wave?

Ideally we want to define somthing like a \textbf{wave-packet}, which is a localized collection of a wave that exists in some region in space. 
To do this we first model a particle travelling in free 1D space with position $x = vt$. As a monochromatic plane wave we could write
\begin{align*}
	A(z,t) = A \cos(kx - \omega t) = A \text{Re}[e^{i(kx- \omega)}]
\end{align*}
Where the particle would be delocalised, as it exists in a streched region in space.
To resolve this difference between wave and particle behaviour, we will use superposition of multiple waves to achieve localisation.
Recall that the intensity profile from the diffraction pattern of light going through some aperture with size $a$ was
\begin{align*}
	I \propto \abs{\int_{-a/2}^{a/2}dx \int_{-a/2}^{a/2}e^{-ik(Xx + Yy))}}^2
\end{align*}
And if $a \gg 1/k$, then this integral averages out expect when $X=Y=0$, where we recover Ray-optics again.

So if we take the superposition of waves with different $k$ values, then we can get something that behaves how we want a wave packet to behave.

For example if we take $k_0 - \Delta k < k_0 + \Delta k$, then it's associate Frequency willl be
\begin{align*}
	\omega(k) = \frac{E}{\hbar}
\end{align*}
where $E = \frac{p^2}{2m}$ is the kinetic energy. And since $p = \frac{h}{\lambda} = \hbar k$, and $k = \frac{2 \pi}{\lambda}$ we can write
\begin{align*}
	\omega = \frac{\hbar k^2}{2m}
\end{align*}
If we assume a small range of $k$ values (and momentum values): $\Delta k \ll k_0$ we can Taylor expand the frequency
\begin{align*}
	\omega = \omega(k_0) + \frac{\del \omega}{\del k}\big\vert_{k_0}(k - k_0) + \ldots = \frac{\hbar k_0^2}{2m} + \frac{\hbar k_0}{m}(k - k_0)
\end{align*}
And by adding up all the waves (over the continuous variable $k$) we get that the resulting wave will be
\begin{align*}
	\psi(x,t) &= \int_{k_0 - \Delta k}^{k_0 + \delta k}A e^{i(kx - \omega(k)t)}	dk\\
						&\simeq A e^{i\left(\frac{\hbar k_0^2}{m}t - \omega_0 t\right)} \int_{k_0 - \Delta k}^{k_0 + \Delta} e^{ik \left(x - \frac{\hbar k_0}{m}t\right)}dk\\
						&= 2A e^{i(k_0x - \omega_0 t)} \sinc \left[\Delta k\left(x - \frac{\hbar k_0}{m}t\right)\right]
\end{align*}
which reminds us of the diffraction pattern mentioned before.
Recall that we cound generalize our square aperture using a transmission function $\tau$, if we do the same here, we could write our boundary $k_0 - \Delta k < k < k_0 + \Delta k$ as a function in the $k$-space:
\begin{align*}
	\tilde{\psi(k,t)} = \left\{\begin{array}{ll}
			A e^{-i \omega(k)t} & \text{for } \abs{k - k_0} < \Delta k \\
			0 & \text{else}
	\end{array} \right.
\end{align*}
Just like with the fourier transform, where we can either describe a function through the values it takes in the ``real space'' or describe it through its fourier transform which describes the frequencies of the function, we can describe wave packets either directly through the function $\psi(x,t)$ or through its frequencies $\tilde{\psi(k,t)}$ in the  $k$-space.

Now let's find out the relationship between the fourier transform and the sinc envelope.
The sinc envelope has a peak at
\begin{align*}
	x = x_{\max} = \frac{\hbar k_0}{m}t
\end{align*}
which allows us to define the \textbf{group velocity} $v_0$ as
\begin{align*}
	v_0 :=\frac{d x_{\max}}{dt} = \frac{\hbar k_0}{m} = \frac{p_0}{m}
\end{align*}
which corresponds to the classical velocity of a moving particle.
The width of the sinc envelope $\Delta x$ corresponds to one period of $\sin$ in $\sinc$, so
\begin{align*}
	\Delta k \Delta x = 2 \pi, \implies \Delta p \Delta x = \hbar \Delta k \Delta x = 2\pi \hbar = h
\end{align*}
where $\Delta k$ is the width of $\tilde{\psi(k,t)}$. 

So what does $\psi(x,t)$ tell us about physical measurements? If we see $P(x,t) dx$ as the probability of finding the particle between $x$ and $x + dx$, then
\begin{align*}
	P(x,t) dx = \abs{\psi(x,t)}^2 dx \quad \text{where} \quad \abs{\psi(x,t)}^2 = \psi^*(x,t) \psi(x,t)
\end{align*}
So since all the probabilities have to add up to $1$ we must have that by Parsevals theorem that
\begin{align*}
	\int_{-\infty}^{\infty}\abs{\tilde{\psi}(k,t)}^2 dk = \int_{-\infty}^{\infty} \abs{\psi(x,t)}^2 dx = \int_{-\infty}^{\infty}P(x,t) dx = 1
\end{align*}
Therefore we can find out that the constant $A$ must satisfy $\abs{A}^2 = \frac{1}{2 \Delta}$.

And since $\psi(x,t)$ describes a single particle, it means that particles can interfere with themselves. 
In an experiment, where we shoot single alpha particles through a double slit, we get an interference pattern, where
\begin{align*}
	\sin \theta = \frac{\lambda}{\Delta x} = \Delta p \frac{x}{p}	\implies \Delta p \Delta x = \lambda p = h
\end{align*}
Using the standard deviation in $x$ and $p$ we see that this is consistent with the Heisenberg uncertainty relation
\begin{align*}
	\Delta x^2 = \left<x^2\right> - \left<x\right>^2 \implies \Delta p x \Delta x \geq \frac{\hbar}{2}
\end{align*}

