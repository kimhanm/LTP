\subsection{The Hydrogen Atom}
When we look at the spectrum of frequencies of the light emitted by atoms, we see that they always come in discrete frequencies instead of a whole range of them.

Hydrogen gas for example, when excited by a high voltage in a gas discharge tube has peaks at around $656.3 \text{nm}, 486.1 \text{nm}, 434 \text{nm}$ and $410.1 \text{nm}$, which is called the Balmer series.

Johann Balmer, without explaining the wavelengths came up with an empirical formula for these wavelengths
\begin{align*}
	\nu = \frac{c}{\lambda} \propto c R_{\infty} \left(
		\frac{1}{n'^2} - \frac{1}{n^2}
	\right)
\end{align*}
,where $R_{\infty} \simeq 1.1 \cdot 10^{7} \m^{-1}$ is the Rydberg constant.

Before quantum mechanics, Niels Bohr constructed a model in which the electrons of the atom sit in stable orbits around the nucleus. The Bohr model however doesn't explain why the electrons emit radiation.


\subsection{Radial equation for the Coulomb potential}
We consider a hydrogen atom consisting of a proton and an electron. Since the mass of the proton is much bigger, we assume that it is motionless so we can center the coordinate system at the proton.

The electron experiences a Coulomb potential given by
\begin{align*}
	V(r) = - \frac{e^2}{4\pi \epsilon_0} \frac{1}{r}
\end{align*}
If we insert this in to the radial equation from before, we get
\begin{align*}
	-\frac{\hbar^{2}}{2m_{e}} \frac{\del^2 u}{\del r^2} + \left[\frac{e^2}{4\pi \epsilon_0}\frac{1}{r} + \frac{\hbar^{2}}{2m_{e}} \frac{l(l+1)}{r^2}\right] u = Eu
\end{align*}

Remember from the finite square well problem, that in order for the wave function to have a bound state (i.e. discrete Eigenvalues), the energy of the Eigenstate has to be less than the maximum value of the potential. 

Here the potential is negative and approaches $0$, for $r \to \infty$. Since we assume that the electron cannot ``escape'' the hydrogen atom, we assume that it's total energy is also less than zero. So we can focus on Eigenstates with $E < 0$.

So now let's define
\begin{align*}
	\kappa := \frac{\sqrt{- 2 m_e E}}{\hbar} \in \R
\end{align*}
We can then rewrite the radial equation as
\begin{align*}
		\frac{1}{\kappa^2} \frac{\del^2 u}{\del r^2} 
	= 
		\left[
			1 - \frac{
				m_{e} e^{2}
			}{
				2\pi \epsilon_0 \hbar^{2} \kappa
			} 
				\frac{1}{\kappa r} 
			+ 
				\frac{l(l+1)}{(\kappa r)^2}
		\right]u
\end{align*}
To make the equation prettier, we remove the dimensions by introducing the variables
\begin{align*}
	\rho := \kappa r \quad \text{and} \quad \rho_0 := \frac{m_e e^{2}}{2 \pi \epsilon_0 \hbar^{2} \kappa}
\end{align*}
which allows us to rewrite the differential equation as
\begin{align*}
	\frac{\del^2 u}{\del \rho^2} = \left[1 - \frac{\rho_0}{\rho} + \frac{l(l+1)}{\rho^{2}}\right]u
\end{align*}
To further simplify the problem, we look at the asymptotic behavour $\rho \to \infty$ and $\rho to 0$. So
\begin{align*}
	\frac{\del^2 u}{\del \rho^2} = u \quad \text{for} \quad \rho \to \infty\\
	\frac{\del^2 u}{\del \rho^2} = \frac{l(l+1)}{\rho^2}u \quad \text{for} \quad \rho \to 0
\end{align*} 
because in each case, one term will dominate the other two. 

For $\rho \to \infty$, the general solution will be of the form
\begin{align*}
	u(\rho) = Ae^{-\rho} + Be^{\rho}
\end{align*}
, for $u(\rho)$ to be finite, the $Be^{\rho}$ must be zero.

For $\rho \to 0$, the general solution has the form
\begin{align*}
	u(\rho) = C \rho^{l+1} + D \rho^{-l}
\end{align*}
but because $l = 0,1,2, \ldots$ is positive, $D$ must be zero in order for $u(\rho)$ to be finite for $\rho \to 0$.

Knowing the asymptotic behaviour, we can write the solution of the full differential equation as a product of the asymptotic solutions multiplied by a new function $v(\rho)$:
\begin{align*}
	u(\rho) = e^{-\rho} \rho^{l+1} v(\rho)
\end{align*}
We also know that $\rho^{l+1}$ is the term with the smallest order, or else it would blow up for $\rho \to 0$, so we can write $v(\rho)$ as a polynomial of the form
\begin{align*}
	v(\rho) = \sum_{j=0}^{\infty}c_j \rho^{j}
\end{align*}
We then calculate the partial derivatives:
\begin{align*}
	\frac{\del u}{\del \rho} &= \rho^{l}e^{-\rho}\left[
		(l+1-\rho)v + \rho \frac{\del v}{\del \rho}
	\right]	\\
		\frac{\del^{2}v}{\del \rho^{2}}
	&=
	\rho^{l}e^{-\rho} \left(
		\left[
			-2(l+1) + \rho + \frac{l(l+1)}{\rho}
		\right]v
		+
		2(l+1- \rho) \frac{\del v}{\del \rho}
		+
		\rho \frac{\del^{2} v}{\del \rho^{2}}
	\right)
\end{align*}
which when plugged into the full differential equation yields
\begin{align*}
	\rho \frac{\del^{2}v}{\del \rho^{2}} + 2(l+1 - \rho) \frac{\del v}{\del \rho} + \left[\rho_0 - 2(l+1)\right]v = 0
\end{align*}
Using the polynomial representation of $v(\rho)$
\begin{align*}
	\frac{\del v}{\del \rho} = \sum_{j = 0}^{\infty}(j+1)c_{j+1}\rho^{j} \quad \text{and} \quad \frac{\del^{2} v}{\del \rho^{2}} = \sum_{j=0}^{\infty}j(j+1)c_{j+1}\rho^{j-1}
\end{align*}
the differential equation becomes
\begin{align*}
	\sum_{j=0}^{\infty}j(j+1)c_{j+1}\rho^{j} + 2 (l+1) \sum_{j=0}^{\infty} j(j+1)c_{j+1}\rho^{j} - 2 \sum_{j=0}^{\infty}jc_j \rho^{j} + [\rho_0 - 2(l+1)] \sum_{j=0}^{\infty}c_j \rho^{j} = 0
\end{align*}
which, when evaluating the coefficients for each $\rho^{j}$, gives us
\begin{align*}
	j(j+1)c_{j+1} + 2(l+1)(j+1)c_{j+1} - 2jc_j + [\rho_0 - 2(l+1)]c_j = 0
\end{align*}
This gives us a recursive formula for the coefficents:
\begin{align*}
	c_{j+1} = 
	\left[
		\frac{
			2(j+l+1) - \rho_0
		}{
		(j+1)(j+2l + 2)
		} 	
	\right]c_j
\end{align*}
This looks like an infinite series, but it can't be. For $u(\rho) = 0$ as $\rho \to \infty$, we must have that the $e^{-\rho}$ domiates the polynomials.

This can only be true if the series terminates at some $j = N$ with $c_N = 0$ and $c_m = 0 \forall m \geq N$.
The recursion formula then says
\begin{align*}
	2(N+l) - \rho_0 = 0
\end{align*}
, so if we define $n := N + l \in \N$ we get $\rho_0 = 2n$.

Therefore we can trace back our definitions
\begin{align*}
	\rho_0 = \frac{m_e e^{2}}{2 \pi \epsilon_0 \hbar^{2} \kappa} \quad \text{and} \quad \kappa = \frac{\sqrt{-2m_e E}}{\hbar}
\end{align*}
to find that
\begin{align*}
	E_n = - \frac{\hbar^{2} \kappa_n^{2}}{2m} 
	=
	-\left[
		\frac{m_e}{2 \hbar^2} \left(
			\frac{e^{2}}{4\pi \epsilon_0}
		\right)^{2}		
	\right]
	\frac{1}{n^2} =: \frac{E_1}{n^2} \quad \text{for} \quad n = 1,2,3, \ldots
\end{align*}

We just were able to find out all the eigenenergies without even writing the wavefunctions. This shouldn't really be surprising, as in linear algebra we calculate the eigenvalues before the eigenvectors by finding the characteristic polynomial.

In some cases, knowing the eigenenergies is enough as we will see when we talk about spin for example.

But for the coulomb potential, we want to actually find the wave equations.

Before doing so, we can find the characteristic length scale for the wavefunctions.
\begin{align*}
	a := \frac{\kappa}{n} = \frac{4\pi \epsilon_0 \hbar^{2}}{m_e e^{2}} \simeq 0.53 \cdot 10^{-10} \m 
\end{align*}
the \textbf{Bohr radius}.

To find the wave equations, recall that we split it into a radial and a spherical part.
\begin{align*}
	\Psi_{nlm}(r,\theta,\rho) = R_{nl}(r) \cdot Y_l^{m}(\theta,\phi)
\end{align*}
which are labeled by what we call the \textbf{quantum numbers} $n,l,m$ for
\begin{align*}
	n \in \N, \quad l \in \left\{0,1,2, \ldots, n-1\right\}, \quad m \in \left\{-l, -l+1, \ldots, 0, \ldots, l-1, l\right\}
\end{align*}
We will see later that they are associated to the angular momentum. 
Also notice how the energy only depends on $n$. In fact the number of degenerate eigenstates with the same energy is
\begin{align*}
	d(n) = \sum_{l=0}^{n-1}(2l+1) = n^{2}
\end{align*}
The ground state energy $E_1$ is
\begin{align*}
	E_1 = - \left[\frac{m_e}{2hb^{2}} \left(
		\frac{e^{2}}{4\pi \epsilon_0}
	\right)^{2}\right] = - 13.6 \text{eV} 
\end{align*}
And the only wave function with this energy is
\begin{align*}
	\Psi_{100}(r,\theta,\phi) = R_{10}(r)Y_0^{0}(\theta,\phi)
\end{align*}
Recall the recursion relation for the coefficients $c_j$ from before:
\begin{align*}
	c_{j+1} = \frac{2(j+l+1-n)}{(j+1)(j+2l+2)}c_j
\end{align*}
so for $n = 1, j = 0$, this gives us $c_1 = 0$ and with $R = \frac{u}{r}$ and $\rho = \kappa r = \frac{r}{a}$ we get
\begin{align*}
	u_{10} = c_0 \rho e^{-\rho} \implies R_{10}(r) = \frac{c_0}{a}e^{-\frac{r}{a}}
\end{align*}
Applying the normalisation condition, we can find out $c_0$ to be
\begin{align*}
	\int_{0}^{\infty}\abs{R_{10}}^{2}r^2 dr \implies c_0 = \frac{2}{\sqrt{a}}
\end{align*}
And combining it with $Y_0^{0} = \frac{1}{\sqrt{4 \pi}}$ which we found in the last lecture we get that
\begin{align*}
	\Psi_{100}(r,\theta,\phi) = \frac{1}{\sqrt{\pi a^{3}}} e^{-\frac{r}{a}}
\end{align*}

We can now do this for all $n,m,l$. Let's see for example the wavefunctions with quantum number $n = 2$.


The radial wavefunction $R_{20}$ is given by
\begin{align*}
	R_{20} = \frac{c_0}{2a}(1 - \frac{r}{2a}) e^{-\frac{r}{2a}}
\end{align*}
and the states $\ket{\Psi_{211}}, \ket{\Psi_{210}}$ and $\ket{\psi_{21-1}}$ have the radial wavefunction
\begin{align*}
	R_{21} = \frac{c_0}{4a^2}re^{-\frac{r}{2a}}
\end{align*}
Note that the constants $c_0$ are not the same for each state and have to be determined by the normalisation condition.

More generally, the radial wavefunctions can be found by calculating the so called \textbf{associated Legendre Polynomials}
\begin{align*}
	L_q^{p}(x) = \frac{x^{-p}e^{x}}{q!}
	\left( \frac{d}{dx}	\right)^{q} (e^{-x}e^{p+q})
\end{align*}
and by setting
\begin{align*}
	U(\rho) = L_{n-l-1}^{2l +1} (2 \rho)
\end{align*}
Bringing everything together, the wavefunction (including normalisation) is given by
\begin{align*}
	\Psi_{nlm}(r,\theta,\phi) 
	=
	\sqrt{\left(
		\frac{2}{n a}
	\right)^{3}
	\frac{(n-l-1)!}{2n(n+l)!}
}
e^{-\frac{r}{na}}
\left(
	\frac{2r}{na}
\right)^{l}
L_{n-l-1}^{2l + 1} \left(
	\frac{2r}{na}
\right)
Y_{l}^{m}(\theta,\phi)
\end{align*}

Note that $v(\rho)$ is a polynomial or order $N-1 = n -l - 1$, so the \emph{radial} wavefunction has this many roots, and the \emph{angular} wavefunction has $l-m$ roots as $\theta$ goes from $0$ to $\pi$.

