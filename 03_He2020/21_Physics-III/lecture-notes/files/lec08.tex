\subsection{Laws of thermodynamics}
\begin{itemize}
	\item[Zeroeth Law:] Two systems in thermal equilibrium have the same temperature.
	\item[First Law:] Heat is a form of energy and the change in Energy $\Delta U$ is the sum of applied heat $Q^{\swarrow}$ and the work done on the system $\delta W^{\swarrow}$.
		A reversible process is when all systems involved remain in thermal equilibrium.
		For a reversible process we have the relationship between Heat, Temperature and Entropy
		\begin{align*}
			\delta Q = T\ dS
		\end{align*}
	\item[Second Law:] The Entropy of an insolated system never decreases.
\end{itemize}

\subsection{Boltzman Factor and the partition function}

THe first question we will ask is that if we have a system with Temperature $T$, what is the probability for it to be in a microstate with energy $\epsilon$.\\

We do this by attaching the system with a reservoir with a much larger total energy. Such that the total energy $U_0 \gg \epsilon$ is much larger (Steve Fig 3.7). If we assume that they are isolated from the outside, but heat can flow freely between them, then the total energy is fixed.

Then the number of microstates in which the reservoir has Energy $U_b$, equals the number of microstates in which the System has the rest of the energy.
\begin{align*}
	\Omega(U_b) = \Omega(U_0 - \epsilon)
\end{align*}
This means that we can express the probability of finding the system in microstate $\epsilon$ is.
\begin{align*}
	P(\epsilon) = \Omega(U_0 - \epsilon)/z
\end{align*}
Expressing this in terms of $\sigma_R(U_R) = \ln \Omega_R(U_R)$ we have that
\begin{align*}
	\frac{P(\epsilon_1)}{P(\epsilon_2)} = \frac{\Omega_R(U_0 - \epsilon_1}{\Omega_R(U_0 - \epsilon_2)} = e^{\sigma_R(U_0 - \epsilon_1) - \sigma_R(U_0 - \epsilon_2)}
\end{align*}
Using $\epsilon_1, \epsilon_2  \ll U_0$ we get
\begin{align*}
	\frac{P(\epsilon_1)}{P(\epsilon_2)} \simeq e^{-(\frac{\del \sigma_R}{\del U})_{N_R}(\epsilon_1 - \epsilon_2)} \frac{e^{-\beta \epsilon_1}}{e^{-\beta \epsilon_2}}
\end{align*}
where $\beta = (\frac{\del \sigma}{\del U}) = \frac{1}{k_B T}$.
By defining the \textbf{Boltzmann factor} to be $e^{-\beta \epsilon}$, we get the \textbf{partition function} $Z$, which sums over all possible microstates $s$ of the system.
\begin{align*}
	Z = \sum_{s} e^{-\beta \epsilon_s}
\end{align*}
so the probability of finding the system in the microstate with energy $\epsilon$ we get
\begin{align*}
	P(\epsilon) = \frac{e^{-\beta \epsilon}}{Z}
\end{align*}
Note that we have the folllowing property:
\begin{align*}
	\frac{\del \ln Z}{\del x} = \frac{1}{Z} \frac{\del Z}{\del x} = \frac{1}{z} \sum_{s} \frac{\del (\beta \epsilon_s)}{\del x} e^{-\beta \epsilon_s} = \left< \frac{\del (-\beta \epsilon_s)}{\del x}\right>
\end{align*}
We can make use of this to calculate the Energy $U$:
\begin{align*}
	U = 	\left< \epsilon_s\right> - \frac{\del \ln(Z)}{\del \beta}
\end{align*}

The second question we will be asking is the following:\\
If we have $N$ atoms with total energy $U$, and each atom can be in one f a number of states, labeled by $i$, where state $i$ has energy $\epsilon_i$ and $E \gg \epsilon_i$.\\
In equilibrium, what is the probability $p_i$ of a particular atom is in the state $i$?\\

If we write $n_i$ for the number of atoms in state $i$, then $\{n_i\}$ defines a macrostate and we am maximaize $\Omega$ over $n_i$ to find the equilibrium macrostate.\\

If we have $N$ atoms, how many ways are there to distribute them to get $\{n_i\}$? It is clear that
\begin{align*}
	\Omega = \frac{N}{n_1! (N-N_1)!} \cdot \frac{(N-n_1)!}{n_2!(N - n_1 - n_2)!} \dots \frac{n_k!}{n_k! 0!} = \frac{N!}{n_1!n_2! \dots n_k!}
\end{align*}
so instead of maximizing $\Omega$ we can instead maximize $\ln(\Omega) = \ln(N!)  - \sum_{i} \ln(n_i!)$ subject to the constraints 
\begin{align*}
	\sum_{i} n_i = N \quad \sum_{i}\epsilon_in_i = E
\end{align*}
Using the method of Lagrange multipliers $\alpha$ and $\beta$ the maximization is found when
\begin{align*}
	0 \stackrel{!}{=} \frac{\del }{\del n_i} \ln(n_i!) + \alpha + \beta \epsilon_i	
\end{align*}
If we assume that $n_i$ is large, using Stirlings approximation we have
\begin{align*}
	\ln(x!) \simeq x \ln x - x
\end{align*}
So for $\ln(n_i) + \alpha + \beta \epsilon_i$ we have
\begin{align*}
n_i = e^{-\alpha}e^{-\beta \epsilon_i} \implies p_i = \frac{n_i}{N} = \frac{e^{-\alpha}e^{-\beta \epsilon_i}}{\sum_{i}e^{-\alpha} e^{-\beta \epsilon_i}}
\end{align*} 
If we identify $\beta = \frac{1}{k_B T}$ this describes the situation where our system is a single atom and the reservoir is a liquid or a gas.

\subsection{Ideal Gas}

In an ideal gas, the energy of a single paticle is just its kintic energy $\epsilon = \frac{p^2}{2m}$. So for the partitiion function $Z$ we cab ask: what is the probability that a particle has a position between $q$ and $q + dq$ and a momentum between $p$ and $p + dq$?.

In 1 dimension, it will $P(q,p) dq dp$ which is just the scaled area spanned by $dq$ and $dp$.

In 3 dimensions, we have six dimensions:
\begin{align*}
	P(\vec{q},\vec{p})d^3q d^3p = \frac{1}{A} e^{-\beta \epsilon}d^3q d^p
\end{align*}
where 
\begin{align*}
	A = \int \int e^{-\beta \epsilon_i}d^3q d^3p = \int d^3q \int e^{- \beta p^2/2m}d^3p
\end{align*}

We introduce the \textbf{Planck's constant} $h$, which has position $\cdot$ momentum as units.\\

For a single particle, the partition function will be of the form
\begin{align*}
	Z_{\text{sp}} = \frac{A}{h^3} = \frac{V}{h^3} \int_{0}^{\infty} 4\pi p^2 e^{- \beta p^2/2m}dp
\end{align*}
And by using $\int_{0}^{\infty}x^2 e^{-ax^2}dx = \frac{\sqrt{\pi}}{4a^{3/2}}$ we find that
\begin{align*}
	Z_{\text{sp}} = \frac{v}{h^3} \left(\frac{2\pi m}{\beta}\right)^{3/2}
\end{align*}
Now if the particles are distinguishable (but identical), we get the many-particle partition function, which can be expressed in terms of the single particle partition function $Z_{\text{sp}}$
\begin{align*}
	Z_{\text{D}} &= \sum_{s_1} \sum_{s_2} \ldots \sum_{s_N}e^{-\beta(\epsilon_{s_1} + \epsilon_{s_2}+ \ldots + \epsilon_{s_n}}\\
							 &= \left(\sum_{s_1}e^{-\beta \epsilon_{s_1}}\right) \left(\sum_{s_2}e^{-\beta \epsilon_{s_2}}\right) \dots \left(\sum_{s_N}e^{-\beta \epsilon_{s_N}}\right)\\
							&= Z_{\text{sp}}^N
\end{align*}
where $s_i$ is the state of particle $i$
