\subsection{Blackbody radiation}
We define a \textbf{Blackbody} as an object that absorbs all light that hits it. This is again an idealized concept as most objects do reflect a small part of light.\\

But since they absorb everything, it must mean that if they were in thermal equilibrum with the surrounding environment, they must also \emph{emit} some radiation.\\


For our model of a blackbody, we imagine a Box with some light trapped inside, having temperature $T$.\\
The Box will have one of its corner on the Origin, with dimensions $L_x, L_y, L_z$.

We then poke a hole in the box and add a spectrometer measuring how much light can escape.

Now what can we say about the light inside? If we imagine a metallic box, the wave equation together with the condition, that there are no charges in the box will be
\begin{align*}
	\nabla^2 \vec{E} = \frac{1}{c^2} \frac{\del^2 E}{\del t^2} \quad \text{and} \quad \vec{\nabla} \cdot \vec{E} = 0
\end{align*}
But since we als have a metal case, the boundary condition is that the parallel component of the electric field vanishes. ($E_{\parallel} = 0$).



In the $1D$ case we have that
\begin{align*}
	\frac{\del^2 \vec{E}}{\del x^2} = \frac{1}{c^2} \frac{\del^2 E}{\del t^2}, \quad \vec{E} \cdot \hat{x} = 0, \quad E(x = 0) = 0,\ E(x = L_x) = 0
\end{align*}
A general solution for this is 
\begin{align*}
	\vec{E}(x,t) = \vec{E}_0 \sin(k_x x) e^{i \omega t}, \quad \text{where} \quad k_x = \frac{n \pi}{L_x}, n = 1, 2, 3, \ldots \quad \omega = c_0 x_k, \quad \vec{E}_0 = E_0 \hat{r}
\end{align*}
Each (linearly) independent solutions can be characterized by its wave-vector $k_x$ and one of two polarisations:


Now for the $3D$ case, the general solutions will be of the form
\begin{align*}
	\vec{E}(\vec{r},\vec{t}) = \begin{pmatrix}
		E_{0x} \cos(k_x x) \sin(k_y y) \sin(k_z z)\\
		E_{0y} \sin(k_x x) \cos(k_y y) \sin(k_z z)\\
		E_{0z} \sin(k_x x) \sin(k_y y) \cos(k_z z)
	\end{pmatrix} \cdot e^{i \omega t}
\end{align*}
Where we have the following conditions
\begin{align*}
	\vec{k} = \left(\frac{n_x \pi}{L_x}, \frac{n_y \pi}{L_y}, \frac{n_z \pi}{L_z}\right), \quad \text{for} \quad n_x,n_y,n_z = 0, 1, 2, \ldots \quad \omega = c \abs{\vec{k}}, \quad \vec{E}_0 \cdot \vec{k} = 0
\end{align*}
Here again each solution can be characzerized by its wave vector $\vec{k}$ and one of 8 possible polarisations.\\

If we recall the harmonic oscillator, its energy $\upvarepsilon$ will be of the form
\begin{align*}
	\upvarepsilon = \frac{1}{2} k \dot{x}^2 + \frac{p^2}{2m}
\end{align*}
which can also be expressed in terms of the electormagnetic field and Volume $V$:
\begin{align*}
	\upvarepsilon = \frac{V}{2} \left(\epsilon_0 \abs{\vec{E}}^2 + \frac{1}{\mu_0} \abs{\vec{B}}^2\right)
\end{align*}
Using the equipartition theorem, we see that the two quadratic terms give us that the average energy per mode at temperature $T$ is $k_B T$.

We can now ask what the energy of the emitted light is with frequencyy between $\omega$ and $\omega + d \omega$.

We can also ask how many modes there are that have frequency between $\omega$ and $\omega + d \omega$ or between $k$ and $k + d k$.\\

In the space of all possible $k$ vectors we have that a shell with radius $k$ and thickness $d k$ will have the volume
\begin{align*}
	\frac{1}{8} 4 \pi k^2 d k
\end{align*}
And since all the modes are equally spaced, we can ascribe each mode its own volume. So the volume per mode will be
\begin{align*}
	\frac{\pi}{L_x}\cdot \frac{\pi}{L_y}\cdot	\frac{\pi}{L_z} = \frac{\pi^3}{V}
\end{align*}
We also need to account that there are 2 polarisations per $k$ vector, so the total number of modes between $k$ and $k + dk$ will be 
\begin{align*}
	g(k) dk = \frac{1}{8} 4\pi k^2 dk \cdot \frac{V}{\pi^3} \cdot 2 = \frac{V}{\pi^2} k^2 dk
\end{align*}
Writing $\omega = c k$, we get the \textbf{Rayleigh-Jeans Law} 
\begin{empheq}[box=\bluebase]{align*}
	\rho(\omega) d \omega = k_B T \cdot \frac{g(\omega) d \omega}{V} = k_B T \frac{\omega^2}{\pi^2 c^3} d \omega
\end{empheq}
Which presents a problem! As $\omega$ goes to infinity, we would have that $\rho(\omega) \to \infty$ which means that we expect an infinite energy density.

This problem is also called the \emph{Ultraviolet Catastrophy}. So how do we resolve it?\\

\subsection{Planck distribution}
What Planck proposed was that for each mode, the energy in that mode can only take on discrete values, which are non-negative integer values of a certain quality.
\begin{align*}
	\epsilon = n \cdot h \nu = n \hbar \omega, \quad n = 1, 2, 3, \ldots
\end{align*}
where $\hbar = \frac{h}{2 \pi}$ is the reduced Planck's constant.

In this case, the partition function of a mode will be
\begin{align*}
	Z = \sum_{n=0}^{\infty}e^{-\beta n \hbar \omega} = \frac{1}{1 - e^{-\beta \hbar \omega}}
\end{align*}
where the equipartition theorem gives us that the average energy will be
\begin{align*}
	\left<\epsilon\right> &= -\frac{\del }{\del \beta}(\ln Z) = -\frac{1}{Z} \frac{\del }{\del \beta}(Z)\\
												&= (1 - e^{-\beta \hbar \omega}) \frac{\hbar \omega e^{-\beta \hbar \omega}}{(1 - e^{-\beta \hbar \omega})^2} \\
												&= \frac{\hbar \omega}{e^{\beta \hbar \omega} - 1} \simeq k_B T
\end{align*}
when $k_B T \gg \hbar \omega$ so using the Planck distribution we get
\begin{align*}
	\rho(\omega) d \omega = \frac{\hbar \omega^3}{\pi^2 c^3} \frac{d \omega}{e^{\beta \hbar \omega} - 1}
\end{align*}
which behaves normally in the sense that for $\omega \to \infty$ we have $\rho(\omega) \to 0$ (See Steve 3.15). Here the curve has a peak at
\begin{align*}
	\rho_{\max}(\nu) \text{ is at } \hbar \omega \simeq 2.822 k_B T
\end{align*}
Which is called the \emph{Wien displacedment Law}.

From the Plank distribution, we can calculate the total Energy density $u$, given by the \textbf{Stefan-Boltzmann Law}
\begin{empheq}[box=\bluebase]{align*}
	u(T) = \int_{0}^{\infty} \frac{\hbar \omega^3}{\pi^2 c^3} \frac{d \omega}{e^{\beta \hbar \omega} - 1} = a T^4
\end{empheq}
where $a$ is the constant $a = \frac{\pi^2 k_B^4}{15 \hbar^3 c^3}$

This allows us to describe the Power radiated per unit area of the blackbody surface:
\begin{align*}
	j(T) = u(T) \frac{c}{4} = \sigma T^4
\end{align*}
where $\sigma = \frac{ac}{4} = 5.67 \times 10^{8} \frac{W}{m^2K^2}$ is called \emph{Stefan's constant}.\\

Now we want to see what the energy inside an infinitesimal volume looks like:

\begin{align*}
	dE = R^2 \sin \theta d \theta d \phi dr \cdot u(T)
\end{align*}
then de energy that will make it out of of the small hole $d S$ is
\begin{align*}
	d \epsilon = d E \frac{dS \cdot \cos \theta}{4 \pi R^2} 
\end{align*}
And the energy in the volume that moves through $dS$ in a tmie $dt$, with $\frac{dr}{dt} = c$. So the total power radiated will be
\begin{align*}
	j(T) = \frac{1}{dS} \frac{d \epsilon}{dt} = \frac{c u(T)}{4 \pi} = \int_{0}^{\pi/2}d \theta \int_{0}^{2 \pi} d \phi \cos \theta sin \phi = \frac{1}{4 \pi} \frac{1}{2} 2 \pi c u(T) = \frac{1}{4} c u(T)
\end{align*}

