\subsection{Spectroscopy}
Electro-magnetic waves can come in a huge spectrum of wavelengths. When we have a mixture of light, we might be interested in finding out what wavelengths are present.\\
Prisms are able to separate some frequencies, but they aren't really good to make accurate measurements, as they the index of refraction doesn't differ alot for waves, whose wavelengths only differ by a small amount. We could work the difference in angles using Snell's Law.\\

A more effective way to seperate by wavelength is using \textbf{Gratings}.\\
Consider an array of small mirrors with centers distance $d$ apart with some gaps inbetween. (Steve Fig. 1.44)\\
Now if we were assuming that there would be no gaps, we could argue with symmetry and find out that the angle of incoming and outgoing light is the same. But because we have gaps here, we can't do that.\\
When looking at the light reflected from neighboring mirrors, we can calculate the their phase difference as
\begin{align*}
	\phi(\lambda) = k \cdot \left( d \sin \theta_i - d \sin \theta_j\right), \quad \text{where} \quad k = \frac{2\pi}{\lambda}
\end{align*}
Now look at the contribution of all $N$ strips and we can see that the Intensitiy will be
\begin{align*}
	U &\propto \sum_{n=0}^{N-1} e^{i\phi n} = \frac{1 - e^{iN\phi}}{1 - e^{i\phi}} \\
		&= \frac{e^{i\phi (\frac{N}{2}} - e^{i\phi \frac{N}{2}})}{e^{i \frac{\phi}{2}} (e^{- \frac{\phi}{2}} - e^{i \frac{\phi}{2}})}	
		= e^{i(N-1) \frac{\phi}{2}} \left(\frac{\sin(N \frac{\phi}{2})}{\sin (\frac{\phi}{2})}\right)
\end{align*}

So the Intensitiy will have
\begin{align*}
	I = \abs{U}^2 \propto \frac{\sin^2 \left(N \frac{\phi}{2}\right)}{\sin^2 \left(\frac{\phi}{2}\right)}
\end{align*}
which will have its maxima at 
\begin{align*}
\frac{\phi}{2} = m \pi \quad \text{for} \quad m \in \Z
\end{align*}
At these points, the Intensity will be
\begin{align*}
	\lim_{\epsilon \to 0} \frac{\sin^2 \left(N \frac{\phi}{2} + \epsilon\right)}{\sin^2 \left(\frac{\phi}{2} + \frac{\epsilon}{N}\right)} = N^2
\end{align*}
See (Steve Fig. 1.47)
\begin{align*}
	\sin \theta_i - \sin \theta_j = \frac{2 \pi m}{dk} = \frac{m \lambda}{d}
\end{align*}
Now we can ask when we will be able to resolve the difference between two wavelenegths. Looking at the (Fig. 1.47), we can frame the condition (similar to the Rayleigh criterium) as follows:\\
The conditions for the peak of wavelength $\lambda_1$ sitting at the zero of wavelength $\lambda_2$ will be
\begin{align*}
	\frac{\phi(\lambda_1)}{2} = m\pi, \quad \text{and} \quad \frac{\phi(\lambda_2)}{2}N = m\pi N + \pi\\
	\implies ND(\sin \theta_i - \sin \theta_j) (k_2 - k_1) = 2\pi
\end{align*}
So the smallest resolvable frequency difference wil be
\begin{empheq}[box=\bluebase]{align*}
	\Delta \omega_{\text{min}} &= (k_2 - k_1)c\\
														 &= \frac{2\pi c}{ND(\sin \theta_i - \sin \theta_j} > \frac{2 \pi c}{2Nd}
\end{empheq}
So the more Mirrors we have the tighter we can resolve different frequencies.\\

We have observed that the resolution depends on the total \emph{path-difference}. The next tool will try to maximize this value\\

The \textbf{Michelson Interferometer} will have the setup (Steve Fig. 1.49)\\
There the path difference will be $2x = 2(d_2 - d_1)$ which can be adjusted however we want. When looking at the field, we have
\begin{align*}
	U = \frac{U_0}{2}(1 + 2^{2ikx}) \implies I = \abs{U}^2 = \frac{\abs{U_0}^2}{4}(2 + 2 \cos(2kx)) = \frac{I_0}{2}(1 + \cos \left(\frac{2 \omega x}{c}\right)
\end{align*}
We see that the shorter the wavelength is, the more sensitive the Interferometer in $x$. It is in fact so sensitive, that it can be used to detect gravitational waves at LIGO.\\

In order to talk about the \emph{distribution} of frequences inn a light bundle, we can define the \textbf{spectral density} $S(\omega)$ of a bundle of light by measuring the added Intensity of all frequences between $\omega$ and $\omega + d\omega$.\\
So in general we will have
\begin{align*}
	I(x) &= \frac{I_0}{2} \int_{0}^{\infty} S(w) (1 + \cos \left(\frac{2\omega x}{c}\right)d \omega\\
			 &= \frac{I_0}{2} + \frac{I_0}{2} \int_{0}^{\infty} S(\omega) \cos \left(\frac{2\omega x}{c}\right)d \omega
\end{align*}
We can invert this relation using the inverse fourier/cosine transform to get
\begin{align*}
	S(\omega) = \int_{0}^{\infty}(2I(x) - I_0) \cos \left(\frac{2\omega x}{c}\right) dx
\end{align*}
Which corresponds to measuring the intensity as we move our mirror further and further away. In reality, we obviously can't move our mirro infitely far away, so we can only take the integral from $0$ to some $x_{\max}$\\

This means that depending on how much ``room'' we have for our Intereferometer, we can resolve frequencies up to a certain maximam, with the theoretical limit being infinite precision.\\

In the ideal case with a monochromatic source with some frequency $\omega_0$, we should get the dirac delta function $S(\omega) = \delta(\omega - \omega_0)$, but in reality we would get
\begin{align*}
	S(w) &= \int_{0}^{x_{\max}} \cos \left(\frac{2\omega_0x}{c}\right) \cos \left(\frac{2\omega x}{c}\right)dx\\
			 &\propto \sinc \left(\frac{(\omega - \omega_0) x_{\max}}{c}\right) + \sinc \left( \frac{(\omega + \omega_0) x_{\max}}{c}\right)
\end{align*}
where the second term is negligible around $\omega \sim \omega_0$. So we will obviously have a peak at $\omega = \omega_0$ with the closst zeros at
\begin{empheq}[box=\bluebase]{align*}
	\Delta \omega = \omega - \omega_0 = \frac{\pi c}{x_{\max}} \quad \text{or} \quad \frac{\Delta \omega}{\omega_0} = \frac{\lambda}{2 x_{\max}}
\end{empheq}
Now here the path difference is ``only'' proportional to twice the physical length $x$. So how do we get even more path difference?\\

The \textbf{Fabry-Perot etalon} is a device (Steve Fig. 1.50) where we have two cavity mirrors placed parallel to each other. Here the light will keep bouncing between these mirrors. So what do these Amplitudes look like? Every boucen, some part of of it will be transmitted and ``lost'' and the other part wil be reflected.\\

So every round trip we will accumulate a nice amount of phase differece at the cost of losing some intensity.\\
If we write $A_0$ for the incoming Amplitude, $A_1$ for the light that gets transmitted in the straight line, and $A_2$ for the amplitude if it makes one round trip etc, we will have
\begin{align*}
  A_1 &= A_0 t^2 e^{ikd}\phantom{r^2e^{2i\phi}}, \quad A_2 = A_0t^2 e^{ikd}r^2 e^{i \phi}\\
  A_3 &= A_0t^2 e^{ikd}r^4 e^{2i\phi}, \quad A_l = A_0 t^{2} e^{idk}r^{2(l-1)}e^{i(l-1)\phi}
\end{align*}
where the phase $\phi$ is a function of the frequency
\begin{align*}
\phi = 2kd + \phi_r = \frac{2 \omega d}{c} + 2\phi_r
\end{align*}
If we set $A$ to be the sum of all out-going amplitudes on the right side, we get
\begin{align*}
  A & := \sum_{l = 1}^{\infty}A_l = A_0 t^2 e^{idk} \sum_{n=0}^{\infty} r^{2n} e^{in\phi}\\
		&= A_0 t^{2} e^{idk} \frac{1}{1 - r^2e^{i\phi}}
\end{align*}
So comparing the Intensities we will have
\begin{align*}
	\frac{I_{\text{out}}}{I_{\text{in}}} =: \eta = \frac{\abs{A}^2}{\abs{A_0}^2} = \frac{t^2}{1 + r^4 - 2r^2 \cos\phi}
\end{align*}
If we write $T:= t^2$ and $R:= r^2$ and assume that $t^2 + r^2 =1$ we have
\begin{align*}
	\eta = \frac{(1 - R)^2}{(1-R)^2 + 2R(1 - \cos\phi)}	= \frac{1}{1 + \frac{4R}{(1-R)^2} \sin^2 \left(\frac{\phi}{2}\right)}
\end{align*}
So if we have a higher reflection index $R$, it means that our light can reflect more often (probabilitstically) and therefore we get a more sensitive result in $\omega$.
