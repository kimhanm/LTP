% \section{Organisation}
% 
% What is this class about?
% 
% 
% \begin{enumerate}
% 	\item Optics
% 			\begin{itemize}
% 				\item Geometrical optics
% 				\item Wave optics
% 				\item Interference
% 				\item Polarisation
% 			\end{itemize} 
% 	\item Statistical mechanis
% 			\begin{itemize}
% 				\item States and how to count them
% 				\item Boltzmann factor
% 				\item Ideal gas
% 				\item Blackbody radiation
% 			\end{itemize} 
% 	\item Segue into the quantum world 
% 			\begin{itemize}
% 				\item Atoms
% 				\item Photons
% 				\item Wave-particle duality
% 			\end{itemize} 
% 	\item Quantum mechanics 
% 			\begin{itemize}
% 				\item Wavefunctions
% 				\item Schrödinger equations
% 			\end{itemize}
% \end{enumerate} 
% 
% We generally want to become confortable with uncertainty and having to deal with approximation.\\
% In Optics, we want to know how well we can measure something by looking at it. In stat mech, there are just too many objects we want to consider and in quantum mechanics our measurements directly impact the object we want to study.\\
% 
% Books
% \begin{itemize}
% 	\item ``Optics'', E. Hecht
% 	\item ``Fundamentals of Photonics'', B. Saleh \& M. Teich
% 	\item ``Statistical Physics'', F.Mandl
% 	\item ``Introduction to Quantum Mechanics'', D. Griffiths
% \end{itemize} 
% 

\section{Optics}

Usually, we think of light as a wave. But often, we can simplify and just think of it as a ray. We can do it when the wavelength is much smaller than the size of the objects the light touches.\\

Going in the other direction, the more complete picture is electro-magnetic optics. We can go even further and look at light as a quantum phenomenon.\\

We will start out in the simplest model, Ray optics or also called Geometrical optics.


\subsection{Ray Optics}
\subsubsection{Fermat's Principle}
One of the underlying principles of Ray optics is \textbf{Fermat's principle}:\\
\emph{Light travels the path whose optical path length is extremal to variations in the path}
\\

Intuitively, this means that if we consider a path that is infinitesimally close the the actual path, the optical path lenth does not change.\\
The path is usually a minimum, but that is not always the case.
In a homogeneuous medium, where the speed of light is the same everywhere, this is just saying that light travels in a straight line.\\
Now consider two media, let's say vacuum and glass and Points $A$, $B$ in each medium\\
In vacuum, the speed of light is about $c = 3 \times 10^8 \m/\s$ and in glass, it is about $1.5$ times slower than that. We then can define the \textbf{refractive index} of glass to be the fraction 
\begin{align*}
				n_{\text{glass}} := \frac{c}{v_{\text{glass}}} \simeq 1.5
\end{align*}
This allows us to define the \textbf{optical path length} of a path to be the quantity
\begin{align*}
	D := n \cdot \ell
\end{align*}
, where $l$ is the physical length of the path.\\
If we can parametrize the space of variations from a path with a scalar quantity $\eta$ then \textbf{Fermat's Principle} can be reformulated to say 
\begin{align*}
\frac{d D(\eta)}{d\eta}|_{\eta = 0} = 0
\end{align*}
So what path does the light take if we take the situation depicted in figure \#\#?
We can easily calculate the total optical path length to be
\begin{align*}
	D = n_1 \sqrt{d_1^2 + h_1^2} + n_2 \sqrt{d_2^2 + h_2^2}
\end{align*}
Because $h, d_1, d_2$ are fixed and $h_2 = h - h_1$ we can write it using only the parameter $h_1$ so in order to find the extremum of $D(h_1)$ we need
\begin{align*}
	0 &= \frac{dD(h_1)}{dh_1}\\
	&= \frac{n_1 h_1}{\sqrt{d_1^2 + h_1^2}} - \frac{n_2h_2}{\sqrt{d_2^2 + (h-h_2)^2}}\\
	&= n_1 \sin \theta_1 - n_2 \sin \theta_2
\end{align*}
From this we get \textbf{Snell's Law:}

\begin{empheq}[box=\bluebase]{align*}
	n_1 \sin \theta_1 = n_2 \sin \theta_2
\end{empheq}


