\subsection{Sylow's Theorem}
Recall that Lagrange's theorem wsays that for any subgroup $H < G$ both it's order $\abs{H}$ and index $[G:H]$ are divisors of $\abs{G}$.



\begin{lemma}[]
	Let $p \in \N$, prime, $n = p^{k}m$ with $(m,p)$ coprime. Then $\binom{n}{p^k}$ is not divisible by $p$.
\end{lemma}

Proof: Set $S := \Z/(p^{k}) \times \{1, \ldots, m\}$, $G = \Z/(p^{k})$ and define a group action from $G$ to $S$ by addition in the first component ($g \cdot (a,j) = (a + g, j)$)

Note that the $G$-orbits in $S$ are of the Form $G \times \{i\}$ for some fixed $i \in \{1, \ldots, m\}$. We then define
\begin{align*}
	T:= \left\{A \subseteq S \big\vert \abs{A} = p^{k}\right\}
\end{align*}
Then let define a $G$ action on $T$ with $g \cdot A = \{g \cdot (a,j)\big\vert (a,j) \in A\}$

Since $G$ is a $p$-group. We know from the lemma on the cardinality of $T$ that
\begin{align*}
	\binom{n}{p^k} = \abs{T} = \abs{\text{Fix}_G(T)}
\end{align*}
We know the cardiniality of $\text{Fix}_G(T)$ because
\begin{align*}
	A \in \text{Fix}_G(T) \iff A \subseteq S, \abs{A} = p^{k}, g \cdot A = A \forall g \in G 
\end{align*}
which is then the case if $A$ is a union of $G$-orbits in $S$. We know how the Orbits look, so there are exactly $m$ of those.
\begin{align*}
	\binom{n}{p^{k}} = \abs{T} = \abs{\text{Fix}} = m
\end{align*}
and $m\not| p$.


\textbf{Example}: \quad Let $G = \text{SL}_2(\F_p)$ of order $p(p^{2}-1)$. Then $H_p = \left\{\begin{pmatrix}
1 & a\\
0 & 1
\end{pmatrix} \big\vert a \in \F_p\right\} \simeq \F_p$ is a Sylow $p$-subgroup.

\begin{ntheorem}[Sylow's Theorem]
	Let $G$ be a finite group, $p \in \N$ prime and $n = \abs{G} = p^{k}m$ for some $k \geq 1$ and $(m,p)$ coprime.
	\begin{enumerate}
		\item There exists a maximal $p$-subgroup $H_p$ with $\abs{H_p} = p^{k}$, which we call \textbf{Sylow $p$-subgroups}.
		\item If $H < G$ is a $p$-subgroup, there exists a $p$-Sylow subgroup $H_p$ with $H < H_p$
		\item Any two Sylow $p$-subgroups are conjugates. 
	\end{enumerate}
\end{ntheorem}
Proof: 
\begin{enumerate}
	\item Let $T = \left\{A \subseteq G: \big\vert \abs{A} = p^{k}\right\}$. Then $T$ is a $G$-set with left multiplikation. From the lemma it follows that $\abs{T} = \binom{n}{p^{k}} \neq 0 \mod p$.
		From the corollary on the Orbits and stabilizers, we know that
		\begin{align*}
			\abs{T} = \abs{\text{Fix}_G(T)} + \sum_{\text{non-trivial orbits}} [G: \text{Stab}_G(A)]
		\end{align*}
		If $n = p^{k}$, $H_p = G$ is itself a Sylow $p$-subgroup. Else if $p^{k} < n$. There doesn't exist a $G$-invariant subset $A \subseteq G$ with $\abs{A} = p^{k}$. Since we $g A \in \{\{e\},G\}$. So there are are no fixpoints of the group action.

		Since $\abs{T} \neq 0 \mod p$ the formula on the cardinality says that there exists at least an $A_0 \in T$ such that $[G: \text{Stab}_G(A_0)] \neq 0 \mod p$.

		We want to show that $H_o := \text{Stab}_G(A_0)$ is a Sylow $p$-subgroup with $\abs{H_p} = p^{k}$. Since
		\begin{align*}
			\abs{G} = \abs{H_p} \cdot [G: H_p] = p^{k}m \quad \text{and} \quad p\not| [G \cdot H_p]
		\end{align*}
		it must follow that $p^{k} | \abs{H_p}$ from the definition of th stabilizer, $H_p \cdot A_0 = A_0$.
	
	This just means that for $a_0 \in A_0$ and $h \in H_p$ we have $h \cdot a_0 \in A_0$. So $\abs{H_p} = \abs{H_p \cdot a_0} \leq \abs{A_0}$.
	But from the way we set $A_0$ its cardinality was $p^{k}$. So we have
	\begin{align*}
		p^{k}| \abs{H_p} \leq p^{k} \implies \abs{H_p} = p^{k}
	\end{align*}
	
	\item Let $H$ be a $p$-subgroup and $H_p$ be a Sylow $p$-subgroup. We define $T = T/H_p$ and let $H$ act on $T$ with left multiplication. From the lemma on fixpoints we know that
		\begin{align*}
			\abs{\text{Fix}_H(T)} = \abs{T} = [G: H_p] = \frac{n}{p^{k}} = m \neq 0 \mod p
		\end{align*}
		In particular there exists a Fixpoint $g H_p \in T$ such that
		\begin{align*}
			hgH_p = gH_p \implies hg \in gH_p \implies h \in gH_p g^{-1} \implies H < gH_p g^{-1}
		\end{align*}
		which shows that $gH_pg^{-1}$ is a Sylow $p$-subgroup.
		
	\item Let $H,H_p$ be two Sylow $p$-subgroups. Then from (b) we know that there exist a $g \in G$ such that $H < gH_pg^{-1}$. Therefore
		\begin{align*}
			\abs{H} = \abs{H_p} = p^{k} \implies H = gH_pg^{-1}
		\end{align*}
\end{enumerate}


\subsection{Symmetric and Alternating Groups}

\begin{theorem}[]
	Let $n \geq 1$. We call the elements of $S_n = \text{Bij}(\{1, \ldots, n\})$ \textbf{permutations}.

	On $S_n$ there exists a Homeomorphism $\sgn: S_n \to \{\pm 1\}$ which maps every permuation a sign, where $\sgn(\tau_{ij}) = -1$, for $i\neq j$.

	We say $\sgn \in S_n$ is called \textbf{even}, if $\sgn(\sigma) = 1$ and \textbf{odd}, if $\sgn(\sigma) = -1$.
\end{theorem}


Proof: see Lineare algebra. We can also prove this by looking at $F \in \Z[X_1, \ldots, X_n]$ and then defining $F^{\sigma} = F(X_{\sigma(1)}, \ldots X_{\sigma(n)})$ which defines a group action from $S_n$ to $\Z[X_1, \ldots, X_n]$.

Then define $P = \prod_{1 \leq i < j \leq n}(X_i - X_j)$ and we see that
\begin{align*}
	P^{\sigma} = \prod_{1 \leq i < j \leq}(X_{\sigma(i)} - X_{\sigma(j)}) = \sgn(\sigma)P
\end{align*}
which can be used as a definition of $\sgn$.

\textbf{Notation:} \quad For $\sigma \in S_n$ we often write $\sigma$ in the following way
\begin{align*}
	\sigma =: \begin{pmatrix}
	1 & 2 & \ldots & n\\
	\sigma(1) & \sigma(2) & \ldots & \sigma(n)
	\end{pmatrix}
\end{align*}
Or we can use the better notation using cycles, where we first find out the first non-fixpoint $i_1$ of $\sigma$. Then look at the sequence
\begin{align*}
	\sigma(i_1), \sigma^{2}(i_1), \ldots, \sigma^{k_1}(i_1) = i_1
\end{align*}
If these are all non-fixpoints, then we write
\begin{align*}
	\sigma = \left(i_1, \sigma(i_1), \sigma^{2}(i_1), \ldots \sigma^{k_1-1}(i_1)\right)
\end{align*}
If there are more, then let $i_2 > i_1$ the next non-fixpoint, etc. Then we write
\begin{align*}
	\sigma = \left(i_1, \sigma(i_1), \sigma^{2}(i_1), \ldots \sigma^{k_1-1}(i_1)\right) \left(i_2, \sigma(i_1), \ldots \sigma^{k_2 - 1}(i_2)\right) \ldots \left(i_r, \sigma(i_r), \ldots \sigma^{k_r -1}(i_r)\right)
\end{align*}
In this case we also say that $\sigma$ has cylcestructure $k_1, k_2, \ldots k_r$. (We may also chang the order of the $k_i$.)


\begin{proposition}[]
	Two permuatations are conjuagates in $S_n$ if and only if they have the same cycle structure.1
\end{proposition}
Proof: See page 122: let $\sigma \in S_n$ and $(i_1, \ldots, i_k)$ a cycle. Then
\begin{align*}
	\sigma\ (i_1, \ldots, i_k) \sigma^{-1} = \left(\sigma(i_1), \sigma(i_2), \ldots \sigma(i_k)\right)
\end{align*}
