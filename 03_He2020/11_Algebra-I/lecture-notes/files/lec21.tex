\begin{theorem}[]
For $n \leq 4$, $A_n$ and $S_n$ are resolvable and $A_n$ is simple for $n \geq 5$
\end{theorem}
Proof: For $n = 1,2,3$ we trivally have
\begin{align*}
	A_1 \simeq A_2 \simeq \{e\} \quad \text{and} \quad A_3 \simeq \Z/(3) \text{ is abelian}
\end{align*}
For $n = 4$ we look at the subgroup 
\begin{align*}
	V_4 = \left<(1,2)(3,4),(1,3)(2,4)\right>	= \{e, (1,2)(3,4), (1,3)(2,4), (1,4)(2,3)\}
\end{align*}
wher every non-trivial Element has order $2$. It follows that $V_4 \simeq \Z/(2) \times \Z/(2)$. And since $V_4$ contains all the elements of cycletype $2,2$ we have a subnormal sequence $V_4 \lhd A_4$, which shows that $A_4$ is resolvable.

For $n \geq 5$ we look at a group action on $\{1, \ldots, n\}$ with the following Lemma
\begin{lemma}[]
Let $g \geq 3$, then the Group action of $A_n$ to $\{1, \ldots, n\}$ is transitive
\end{lemma}
Proof: This follows directly from the fact that the orit of $1$ is $\{1, \ldots, n\}$, since
\begin{align*}
	(1,2,3): 1 \mapsto 2, \quad (1,i,2): 1 \mapsto i	
\end{align*}

\begin{lemma}[]
	Let $n \geq 5$ and $H \lhd A_n$ nontrivial. Then $H$ contains a permutation $\sigma \neq e$ with at least one fixpoint.
\end{lemma}
Proof: Let $\sigma \in H$ and $\tau \in A_n$. Then the commutator of them is in $H$, since $[\tau,\sigma] = \tau \sigma \tau^{-1}\sigma^{-1} \in H$.

Let $\sigma$ be non-trivial. If it has a fixed points, we're done. If it has none, then we can find a $\sigma' \in H$ that has a fixed point.
We consider the following cases
\begin{itemize}
	\item $\sigma$ Has a cycle $(i_1,i_2,i_3, \ldots i_k)$ of length $k \geq 4$. Then we can chose $\tau = (i_1,i_2,i_3) \in A_{n}$ and $\sigma' = [\tau,\sigma]$. Then
		\begin{align*}
			\sigma' = (i_1,i_2,i_3)\sigma(i_1,i_2,i_3)^{-1}\sigma^{-1}
		\end{align*}
		which maps
		\begin{align*}
			i_1 \mapsto i_k \mapsto i_k \mapsto i_2 \quad \text{and} \quad i_3 \mapsto i_2 \mapsto i_1 \mapsto i_2 \mapsto i_3
		\end{align*}
		which shows that it is non-trivial and has a fixpoint.
		
	\item $\sigma$ has cycles of length $2$ and $3$. Then $\sigma' = \sigma^{2}$ has cycles of length $3$ and fixpoints.
	\item If $\sigma$ only has cycles $(i_1,i_2,i_3),(i_4,i_5,i_6), \ldots$ of length $3$, then chose $\tau = (i_1,i_2,i_4)$ and $\sigma' = [\tau,\sigma]$, which maps
		\begin{align*}
			i_1 \mapsto i_3 \mapsto i_3 \mapsto i_1 \mapsto i_2 \quad \text{and} \quad i_6 \mapsto i_5 \mapsto i_5 \mapsto i_6 \mapsto i_6
		\end{align*}
		which shows non-triviality and the existence of a fxpoint.

	\item $\sigma$ only has cycles $(i_1,i_2),(i_3,i_4),(i_5,i_6), \ldots$, of which there are at least $3$, since $n \geq 5$. Then chose $\tau = (i_1,i_2,i_3)$ and $\sigma' = [\tau,\sigma]$, which maps
		\begin{align*}
			i_1 \mapsto i_2 \mapsto i_3 \mapsto i_1 \mapsto i_2, \quad \text{and} \quad i_5 \mapsto i_6 \mapsto i_6 \mapsto i_5 \mapsto i_5
		\end{align*}
\end{itemize}
which show all possible cases for $\sigma$.


Now we can prove that $A_5$ is simple:

Let $H \lhd A_5$ be nontrivial and $\sigma \in H \setminus \{e\}$ the permutation with a fixpoint from the lemma. In particular, $\sigma$ cannot be a $5$-cycle (or it would have no fixed points) and it since $\sigma \in H$ it also cannot have $4$-cycles (\#\#\# why?)

So it must either have cycletype $3$ or $2,2$. 
If $\sigma = (i_1,i_2)(i_3,i_4)$, then chose $\tau(i_1,i_2,i_5)$, then
\begin{align*}
	\tau \sigma \tau^{-1} = (i_2,i_5)(i_3,i_4) \quad \text{and} \sigma \tau \sigma \tau^{-1} = (i_1,i_2)(i_3,i-4)(i_2,i_5)(i_3,i_4) = (i_1,i_2,i_5) \in H
\end{align*}
So $H$ also contains a $3$ cycle.

We then show that all $3$-cycles in $A_5$ are conjugates, which means that $H$ must contain all $3$-cycles.

Let $\sigma = (i_1,i_2,i_3)$ then define $\tau$ as
\begin{align*}
	1 \mapsto i_1, \quad 2 \mapsto i_2, \quad 3 \mapsto i_3, \quad 4 \mapsto x, 5 \mapsto y
\end{align*}
By swapping $x$ and $y$, we can always assume that $\tau \in A_n$. Then
\begin{align*}
	\tau(1,2,3)\tau^{1} = (i_1,i_2,i_3)	 
\end{align*}
which shows that indeed all $3$ cycles are conjugates and thus from the previous calculation are in $H$.

Then we know that
\begin{align*}
	(i_1,i_2,i_3)(i_3,i_4,i_5) = (i_1,i_2,i_3,i_4,i_5) \in H \implies H = A_5
\end{align*}
which shows that $A_5$ is simple.

For $n > 5$ we use induction on $n$. Let $H \lhd  A_n$ and $\sigma \in H \setminus \{e\}$ have a fixpoint.

We can assume without loss of generality, that $\sigma(n) = n$ is the fixpoint. From the induction step we can write $\{e\} \neq H \cap A_{n-1} = A_{n-1} \lhd A_{n_1}$ From the first lemma, we know that every Element of $A_n$ with a fixpoint is conjugate to an element of $A_{n-1}$.

This shows that $H$ contains every element with a fixpoint. Then take any $\sigma \in A_n$.

\begin{itemize}
	\item If $\sigma$ has a fixpoint, it is in $H$. If
	\item If a cycle $\tau$ of $\sigma$ has odd length $< k$, then $\tau^{-1} \sigma$ in the cycle, so adding $\tau$ again, which obvously has fixed points outside of $\tau$, 
		\begin{align*}
			\sigma = \tau(\tau^{-1}\sigma) \text{ has a fixpoint}
		\end{align*}
		which shows $\sigma \in H$
	\item If $n$ is odd and $\sigma = (i_1, \ldots, i_n)$ we can write
		\begin{align*}
			(i_1, \ldots i_{n-2}) (i_{n-2},i_{n-1},i_n) \in H
		\end{align*}
		where the first term is in $H$ because of induction and the second term is a $3$-cycle so is in $H$.
	\item If $\sigma$ has a cycle $(i_1, \ldots, i_{2k})$ of even length $2k \geq 4$, then using induction on the following decomposition
		\begin{align*}
			\sigma = \left(
				(i_1, \ldots, i_{2k - 2}) \ldots
			\right) (i_{2k-2}, i_{2k-1}, i_{2k}) \in H
		\end{align*}
		shows $\sigma \in H$
		
	\item If $\sigma$ only has $2$-cycles, we know from $n \geq 6$ and $\sigma \in A_n$ that $n \geq 8$, and we can write $\sigma$ as a product of an element of $A_n$ of cycle type $2,2$ and another element of $H$ with fixpoints.
\end{itemize}

This shows that $H = A_n$ which shows that $A_n$ is simple.

\subsection{Classification of groups of small order}
\begin{theorem}[]
	Let $G$ be a group of order $n = \abs{G} \leq 100$. Then either $G$ is resolvable or $G \simeq A_5$ (and $n = 60$)
\end{theorem}
For this theorem we use a colletion of previously known lemmas with higher and higher complexity.

Recall that we call a group $G$ resolvable if it as a subnormal series 
\begin{align*}
	\{e\} = G_0 \lhd G_1 \lhd  \ldots \lhd G_k = G	
\end{align*}
where the factor groups $G_j/G_{j-1}$ are all abelian.

\begin{proposition}[]
Let $G$ be a group and $N \lhd G$. If $N$ and $G/N$ are resolvable, then so is $G$
\end{proposition}
Proof: Since they are resolvable, we have subnormal series
\begin{align*}
	\{e\} = G_0 \lhd G_1 \lhd \ldots \lhd G_l = N\\
	\{eN\} = H_0 \lhd H_1 \lhd \ldots \lhd H_m = G/N
\end{align*}
with abelian factorgroups. Let $\pi: G \to G/N$ be the canonical projection. Then define
\begin{align*}
	G_j' := \pi^{-1}(H_i) < G	 \implies G_l = N = \pi^{-1}(eN) = G_0' < G_1' < \ldots G_m' = G
\end{align*}

We want to show that the sequence
\begin{align*}
	G_0 < G_1 < \ldots < G_l = N = G_0' < \ldots G_m' = G
\end{align*}
is subnormal.

Let $h \in G_{j-1}',g \in G_j'$. Then because $H_{j-1} \lhd H_j$ we know that
\begin{align*}
	\pi(h) \in H_{j-1}, \pi(g) \in H_j &\implies \pi(g)\pi(h)\pi(g)^{-1} &\in H_{j-1}\\
																		 &\implies \pi(ghg^{-1}) \in H_{j-1}\\
																		 &\implies ghg^{-1} \in \pi^{-1}(H_{j-1}) = G_{j-1}'
\end{align*}
which shows $G_{j-1}' \lhd G_j'$. To show that the factor groups are abelian, we use the third isomorphism theorem to show that
\begin{align*}
	\faktor{G_j'}{G_{j-1}'} \simeq \faktor{\faktor{G_j'}{N}}{\faktor{G_{j-1}'}{N}} = \faktor{\pi(G_j')}{\pi(G_{j-1}')} = \faktor{H_j}{H_{j-1}}
\end{align*}

Which shows that $G$ is resolvable.

From the exercise problems we know the following
\begin{itemize}
	\item A Group $G$ with $a^{2} = e$ for all $a \in G$ is abelian and thus resolvable.
	\item A subgroup with Index $2$ is normal.
	\item All groups of order $\leq 10$ are resolvable.
	\item All groups or order $p \in \N$ prime are cyclic, abelian and thus resolvable.
	\item All Groups of order $p^{2}$ are abelian and resolvable.
	\item All $p$-Groups for $p \in \N$ prime ($\abs{G} = p^{k}$) are niltpotent and thus resolvable.
\end{itemize}

The questions is: how far can we go with this?








 


