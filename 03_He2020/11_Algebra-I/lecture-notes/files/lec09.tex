Before proving (b), we first prove the next proposition
\begin{proposition}[]
	Let $R$ be a principal ideal domain and $p \in R$ irreducible. Then $(p)$ is a maximal ideal, and $p$ is prime.
\end{proposition}
Proof:	Let $J \subseteq R$ be an ideal such that $J \supsetneq (p)$. Since $R$ is a PID, there exists a $d \in R$ such that $J = (d) \supsetneq (p)$, which means that $d|p$, i.e. $\exists c \in R: p = dc$. Because $p$ is irreducible, we know that either $d$ or $c$ is a unit. If $c$ were a unit, we had $d = pc^{-1}$, but that would be that $d \in (p)$ which contradicts our assumption that $(d) \supsetneq (p)$. Therefore $d$ is a unit, which shows that $(d) = R$. In other words, $(p)$ is indeed a maximal ideal, and therefore $p$ is prime.\\

We also need one more proposition for the proof of the theorem
\begin{proposition}[]
	Let $R$ be a PID and let
	\begin{align*}
		J_0 \subseteq J_1 \subseteq J_2 \subseteq \ldots
	\end{align*}
	be an ascending chain of Ideals in $R$. Then there exists an $n \in \N$ such that $J_m = J_n$ for all $m \geq n$
\end{proposition}
Proof: Define $J = \bigcup_{n = \N} J_n$. Since it is the union of ideals containing the ideals lower in the chain, we know that $J$ itself is an ideal. And since $R$ is PID, $J = (d)$. But then there exists an $n \in N$ such that $J_n = (d) = J = J_m$ for $m \geq n$.\\


With this, we can easily prove (b) from the previous theorem:\\
Let $f \in R \setminus \{0\}$. If $f$ is a unit or is irreducible, then $f$ there is nothing to show.\\
Now assume that $f \in R \setminus \{0\}$ can not be written as a finite product of a unit and prime elements.
Assume that $f = f_0 = f_1 \tilde{f}_1$. If both $f_1$ or $\tilde{f}_1$ could be decomposed, then the same would be true for $f$. We may now assume that it is $f_1$ that can't be decomposed. Then we would ahve
\begin{align*}
	f_0 = f_1 \tilde{f}_1, f_1 = f_2 \tilde{f}_2, f_2 = f_3 \tilde{f}_3 \ldots
\end{align*}
In particular we would have a sequence of elements that divide each other: $f_{n+1} | f_n$ which means that we get an ascending chain of ideals
\begin{align*}
	(f_n) \subseteq (f_{n+1}), \quad \forall n \in \N
\end{align*}
Using the second proposition, we would have that there exists an $n \in \N$ such that $(f_{n}) = (f_{n+1})$. But since $R$ is a PID, we know that after looking at the prime elemnts which generate these ideals, that $f_{n} = a f_{n+1}$ for some unit $a \in R^{\times}$. And because the $\tilde{f}_n$ which contradicts the constructin of $f_n, \tilde{f}_n$.\\


For example, let's look at some prime numbers in $\Z[i]$. Some of them are $1 \pm i, 3, 2 \pm i$. Then we can describe the units of this Ring to be
\begin{align*}
	\Z[i]^{\times} = \{z \in \Z[i]\big\vert N(z) = 1\} = \{\pm 1, \pm i\}
\end{align*}
We also know that $2$ is not prime in $\Z[i]$ since $2 = (1+i)(1 -i)$ as well as $5 = (2+i)(2-i)$ is not prime either.


\begin{lemma}[]
	Let $z \in \Z[i]$ such that $N(z) = p \in \Z$ for $p$ prime in $\N$. Then $z$ is irreducible (and prime since $\Z[i]$ is a PID).
\end{lemma}
Proof: Let $z = uv$ for $u,v \in \Z[i]$. Then 
\begin{align*}
	p = N(z) = N(uv) = N(u)N(v) \implies \text{ wlog } N(u) = 1 \implies u \in \Z[i]^{\times}
\end{align*}

\begin{lemma}[]
	Let $p \in \N$ be prime in $\N$ that can not be written as a sum of two squares, then $p$ is also prime in $\Z[i]$
\end{lemma}
Proof: Assume $p = z \cdot w$. Then $N(z) \cdot N(w) = N(p) = p^2$. So $N(z) | p^2$ (in $\N$). So we have $N(z), N(w) \in \{1,p,p^2\}$. But we can remove $p$ from the list since
\begin{align*}
	N(z) = N(a + ib) = a^2 + b^2 = p \lightning
\end{align*}
contradicts the property of $p$, so wlog $N(z) = 1$ and $N(w) = p^2$. So one of them is a unit which shows that $p$ is irreducible (and prime).\\

In another example, we look at $K[X]$ for some field $K$. Then no polynomials of degree $0$ is irreducible, for they are the constants. If the degree is $1$, then every polynomial is irreducible. 
If the degree is $2$, then it is irreducible if and only if it has no zeros. If the degree is $3$, then we can use the same criterion as before. If the degree is $4$ however, we could have it as product of two polynomials of degree $2$ without zeros so the criterium doesn't work here.\\

In general, the question of finding irreducible polynomials depends alot on the field we are working with.\\


\subsection{Unique factorisation domain}

\begin{definition}[UFD]
	An integral domain $R$ is called a \textbf{unique factorisation domain} (or factorial ring) if every element $a \in R \setminus \{0\}$ can be written as a product of a unit and finitely many prime elements of $R$:
	\begin{align*}
		a = up_1 \dots p_n \quad \text{for} \quad u \in R^{\times}, p_1, \ldots p_n \text{ prim}
	\end{align*}
\end{definition}
Note: Every PID (and thus every euclidean Ring) is a UFD.
\begin{proposition}[]
	Let $R$ be a UFD. Then $p \in \setminus \{0\}$ is prime if and only if $p$ is irreducible.
\end{proposition}
Proof: As with any integral domain, prime implies irreducible. So let $p$ be irreducible. Because $R$ is a UFD, $p = u p_1 \dots p_n$, but since $p$ is irreducible, $n$ must be $1$, or else $p$ wouldn't be irreducible so $p = u p_1$. But becase $u$ is a unit, we have $(p) = (up_1) = (p_1)$ which means that $(p)$ is a prime ideal so $p$ is prime.\\


\begin{corollary}[]
	Let $R$ be an integral domain. Then $R$ is a UFD if and only if every element $a \in R \setminus \{0\}$ can be written as a product of a unit and finitely many irreducible elements and if the ring has the property that irreducible $\implies$ prime.
\end{corollary}

\begin{definition}[]
	Let $R$ be a commutative Ring and $a,b \in R$. We say $a$ and $b$ are \textbf{associated} and write $a \sim b$ if there exists a unit $u \in R^{\times}$ such that $a = ub$
\end{definition}

This also induces an equivalence relation. Reflexivity follows by chosing $u = 1$, symmetriy by using $b = u^{-1}a$ and transitivity comes from the product of the units $u_1$ and $u_2$.\\


\begin{lemma}[]
	Let $R$ be an integral domain and let $p,q$ be irreducible such that $p | q$. Then $p \sim q$
\end{lemma}
Proof: Becuase $p = ap$ and because $p$ is irreducible, and $q$ is not a unit, it follows that $a$ is a unit.\\

\begin{definition}[]
	For $n \in \N$ we define the \textbf{symmetric} group $S_n$ to be the set of consisting of the bijections
	\begin{align*}
		S_n := \left\{\sigma: \{1, \ldots, n\} \to \{1, \ldots, n\} \big\vert \sigma \text{ bijective}\right\}
	\end{align*}
\end{definition}
\begin{theorem}[Unique factorisation]
	Let $R$ be a unique factorisation domain. Then the factorisation of every nonzero element $a$ is unique up to permutation of the prime elements. In other words
	\begin{align*}
		up_1 \dots p_n = a = vq_1 \dots q_m \implies n = m \text{ and } \exists \sigma \in S_n: q_i \sim p_{\sigma(j)} \text{ for all } 1 \leq i \leq n
	\end{align*}
\end{theorem}
