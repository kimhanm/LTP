Then we can define a norm on $K=\Q[\sqrt{d}]$. 
\begin{align*}
	N(z) = N(a + b \sqrt{d}) = z \tau(z) = a^2 - db^2
\end{align*}
which is multiplicative, since $\tau$ is an isomorphism:
\begin{align*}
	N(zw) = (zw) \tau(zw) = z \tau(z) w \tau(w) = N(z) N(w)
\end{align*}

When we loook at the Ring $\Z[\sqrt{d}]$ and we can define the degree using the norm function and obtain the following theorem

\begin{theorem}[]
	For $d = -1, -2, 2, 3$ the Ring $R = \Z[\sqrt{d}]$ is a euclidean Ring using the degree function $\phi(z) := \abs{N(z)}$
\end{theorem}

Proof: Let $f,g \in R$, with $f \neq 0$. We calculate division in $\Q[\sqrt{d}]$ such that $z = a + b \sqrt{d} = \frac{g}{f} \in \Q[\sqrt{d}]$ and chose the best approximation in $\Z[\sqrt{d}]$:
\begin{align*}
	q:= [a] + [b] \sqrt{d} \in R
\end{align*}
Then we have
\begin{align*}
	\phi(z-q) = \abs{N(z-q)} = \abs{(a - [a]^2 - d(b- [b])^2} \leq \frac{1}{4} + \frac{1}{4}d < 1\quad \text{for } d = -1,-2,2
\end{align*}
Even for $d = 3$, since we have a minus in the absolute value, the upper bound holds. If $d = -3$, the above argument would be incorrect.

We define $r = g - fq \in \Z[\sqrt{d}]$ and get that $g = fq + r$ and
\begin{align*}
	\phi(r) = \abs{N(g - fq)} = \abs{N(f)N(z-q)} < \abs{N(f)} = \phi(f)
\end{align*}

From this we obtain the following
\begin{lemma}[]
	Let $R = \Z[\sqrt{d}]$. Then we have
	\begin{enumerate}
		\item $u \in R^{\times}$ if and only if $N(u) = \pm 1$
		\item If $z \in R$ such that its Norm is prime in $\Z$, then $z$ is irreducible (in $R$).
		\item If $p \in \Z$ is prime, such that neither $p$ nor $-p$ is a Norm of an element in $R$, then $p$ is irreducible.
	\end{enumerate}
\end{lemma}
Proof: 
\begin{enumerate}
	\item If $u \in R^{\times}$ is a unit, then there exists $v \in R^{\times}$ such that $uv = 1$. Therefore $N(u) \cdot N(v) = N(uv) = 1$. Since $N(u), N(v)$ is an element of $\Z$, it follows that they must be $\pm 1$.
		If $N(u) = \pm 1$, then $u(\pm \tau(u)) = \pm N(u)= 1$, so it has an inverse $u^{-1} = \pm\tau(u)$.
	\item If $N(z) = p \in \Z$ is prime. From the multiplicativity of the Norm we have that
		\begin{align*}
			z = ab \implies N(z) = p = N(a) N(b) \implies N(a) = \pm 1 \text{ or } N(b) = \pm
		\end{align*}
		From the first part, this means that one of $a$ or $b$ is a unit.

	\item Let $p \in \Z$ be prime, such that $p$ and $-p$ are not norms of numbers. If $p = ab$, then 
		\begin{align*}
			N(p) = p^2 = N(a) N(w) \implies N(a), N(b) \in \{\pm 1, \pm p, \pm p^2\}
		\end{align*}
		But since they cant be $\pm p$, of of them must have norm $1$ and must be a unit.
\end{enumerate}

\begin{theorem}[Gaussian Integers]
	Let $R = \Z[i]$ be the Ring of Gaussian integers. Then $R$ is a euclidean ring and we can look at the representation set
	\begin{align*}
		P = \left\{z = a + ib \in R \big\vert z \text{ prime and } -a < b \leq a\right\}
	\end{align*}
	whose elements we can categorize as
	\begin{itemize}
		\item 	$z = 1 + i$ (which divides $2 = -i(1 + i)^2$
		\item (inert)$p \in \N$ prime with $p = 3 \mod 4$ with examples $3,7,11, \ldots$
		\item (split) $z = a \pm bi$ prime in $R$, such that $a,b \in \N$, $b < a$ and
			\begin{align*}
				a^2 + b^2 = p = 1 \mod 4 \quad \text{for $p \in \N$ prime}
			\end{align*}
			which includes $5,13, \ldots$
	\end{itemize}
Note: There are infinitely many inert and split primes in $R$.
\end{theorem}
To prove this, we need the following lemma:

\begin{lemma}[]
	Let $p \in \N$ be prime. Then $(p-1)! = -1 \mod p$
\end{lemma}
Proof: We have that 
\begin{align*}
	(p-1)! = \prod_{k=1}^{p-1}k	= 1 \cdot (p-1) \prod_{\underset{ab = 1 \mod p}{1 < a < b < p-1}} = -1 \mod p
\end{align*}
Which is true, since for any $x \in \F_p^{\times}$ 
\begin{align*}
	x = x^{-1} \iff x^2 = 1 iff	 (x+1)(x-1) = 0 \iff x = \pm 1 
\end{align*}

\begin{proposition}[]
	Let $p \in \N$ such that $p = 1 \mod 4$. Then there are two solutions of the equation $x^2 - 1$ in $\F_p$.	
\end{proposition}
Proof: Define $x = (\frac{p-1}{2})!$ in $\F_p$. Then since $(\frac{p-1}{2})$ is divisble by four, we have
\begin{align*}
	x^2 &= 1 \cdot 2 \dots (\frac{p-1}{2}) \cdot (\frac{p-1}{2}) \dots 2 \cdot 1 \cdot (-1)^{\frac{p-1}{2}}\\
			&= 1 \cdot 2 \dots \left(\frac{p-1}{2}\right) \left(\frac{p+1}{2}\right) \dots (p-2)(p-1)\\
			&= (p-1)! = -1 \in \F_p
\end{align*}

\begin{corollary}[]
	Let $p \in \N$ be congruent to $1 \mod 4$. Then $p$ is not prime in $\Z[i]$.
\end{corollary}
Proof: Consider
\begin{align*}
	\Z[i]/(p) \simeq \F_p[X]/(X^2+1), \quad a + ib + (p) \mapsto a +bX \mod p
\end{align*}

But since $X^2 + 1$ has two roots, it is not irreducible in $\F_p$. Therefore $\Z[i]/(p)$ is not an integral domain and $p$ is not a prime element.

Now we have everything we need to prove the theorem.\\

Proof theorem: $1 + i$ is irreducible, since $N(1+i) = 2$ is prime in $\Z$. Because the ring is euclidean, irreducible also means prime.\\
Now let $p \in \Z$ be congruent to $3 \mod 4$. Then since for any $a,b \in \Z$ we have $a^2 + b^2 \neq 3$ (easy calculation), we have that $p \neq a^2 + b^2 \mod 4$.\\

Therefore $p$ (and $-p$) is not a norm of an element in $R = \Z[i]$ and the lemma shows that it is prime.\\

If $p \in \N$ is congruent to $1 \mod 4$, then the corollary says it it is not prime in $R$. Therefore there exists a $z \in R$ such that $N(z) = p$. So we have found $a,b$ such that $p = a^2 + b^2$. Since $2 \not| p$ we can assume $b < a$. Then
\begin{align*}
	p = (a + ib)(a-ib) \text{ such that} a \pm ib \text{ not associated}
\end{align*}
, since the angle between them is smaller than 90 degrees.\\

We now show that the three cases encompass all prime numbers. Let $z \in \Z[i]$ be prime. Thenn since $n = N(z)$ is a natural number. If $p = 2$, then we already know that $2 = (1+i)(1-i)$.\\
If $p = 3 \mod 4$, and $p | z \tilde{z}$ and $p$ is prime in $\Z[i]$, then $p | z$ or $p | \tilde{z}$.\\
If $p = 1 \mod 4$. Then we already know that
\begin{align*}
	(a + ib) | p | z \tilde{z} \implies (a + ib) | z
\end{align*}

We sat that the rings $R = \Z[\sqrt{d}]$ for $d \in \{-2,-1,,2,3\}$ were euclidean by explicitely writing out the division algorithm. But in the case for $d = -3$, the division algorithm doesnt work anymore, but in the Ring $S = \Z[\frac{1 + \sqrt{3}i}{2}]$ it does.





