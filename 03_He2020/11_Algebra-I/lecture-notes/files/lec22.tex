\begin{center}
	\#\#\# Missing first 45 minutes
\end{center}



\subsection{Free Groups and Relations}

\begin{definition}[]
	Let $n \geq 1$
\end{definition}


\begin{theorem}[]
	Let $n \geq 1$ and $b_1, \ldots, b_n$ be pairwise disjoint. THen there exists a free group $F_n$ generated by $b_1, \ldots, b_n$ with the universal property:

For every group $G$ with elements $a_{1}, \ldots, a_{n} \in G$ there exists a unique group homomorphism 
\begin{align*}
	\Phi: F_n \to G \quad \text{with} \quad \phi(b_j) = a_j, \quad \forall 1, \ldots n
\end{align*}
\end{theorem}

We prove this by finding $F_n$. The construction is as follows
\begin{align*}
	F_n = \left\{\text{reduced words in } b_1,b_1^{-1}, \ldots, b_n,b_n^{-1}\right\}
\end{align*}
where a finite list with entries $b_1^{\pm 1}, \ldots, b_n^{\pm 1}$ is called a word.

A word $w$ is called \textbf{reduced}, if we never have to immediate entries $b_i,b_i^{-1}$ that cancel eachother out. 

Then we can find a group structure on $F_n$ in the following way:

For $w_1,w_2 \in F_n$ we define the group operation as concatenation of the words $w_1 \circ w_2$ with cancellation if neccessary (i.e. if $w_1$ ends in $b_1$ and $w_2$ starts with $b_1^{-1}$.

The universal property follows by defining \#\#\# missing 5 mins


\begin{definition}[Relations]
	Let $F_n$ be the free group with $n$ generators. For $W \subseteq F_n$, let
	\begin{align*}
	N = \left<gwg^{-1} \big\vert g \in F_n, w \in W\right>
	\end{align*}
	be the Normaldivisor of $F_n$ generated by $W$.

	Then $\faktor{F_n}{N}$ is called the group with generators $b_1, \ldots, b_n$ and relations $w \in W$ and is written as
	\begin{align*}
		\left<b_1, \ldots, b_n \big\vert w = e \text{ for} w \in W\right>
	\end{align*}
\end{definition}

Examples:
\begin{enumerate}
	\item $\Z^2 \iso \left<a,b \big\vert ab = ba\right>$
	\item $D_6 \iso \left<D,R \big\vert D^3 = R^2 = e, RDR = D^{-1}\right>$
\end{enumerate}


\section{Modules}
\begin{center}
	Modules are to Rings what Vector spaces are to Field
\end{center}
\subsection{Definitions and Examples}
\begin{definition}[]
	Let $R$ be a Ring. An $R$-module $M$ is an abelian Group with a scalar multiplication
	\begin{align*}
		R \times \to M, \quad (a,m) \mapsto a \cdot m
	\end{align*} 
	missing \#\#\# 5 mins
\end{definition}


\begin{definition}[]
	Let $R$ be a Ring and $M,N$ be $R$-modules. We say $\Phi: M \to N$ is $R$-linear (or Module morphism over $R$), if $\Phi$ is a group homomorphism and 
	\begin{align*}
		\forall a \in R, m \in M: \quad \Phi(am) = a \Phi(m)
	\end{align*}
\end{definition}

\begin{definition}[Submodule]
missing \#\#\# 1 min
\end{definition}


\begin{lemma}[]
	Let $R$ be a ring, $M$ an $R$-module and $N < M$ a submodule. Denn the Module structure on $M$ induces a module strucutre on $\faktor{M}{N}$ \#\#\# missing 2 mins
\end{lemma}
Proof trivial

Examples: 
\begin{enumerate}
	\item If $R = K$ is a field, a module is just a vector space.
	\item If $M,N$ are $R$-modules, then we the Hom-set
		\begin{align*}
			\Hom_R(M,N) := \#\#\# missing 1 min			
		\end{align*}
	\item If $R = \Z$, then every abelian Group $M$ is also a $\Z$-module with the usual additive/multiplicative notation. This means that if we can classify \#\#\# missing 1 min
\end{enumerate}


\begin{proposition}[First Isomorphism Theorem]
Let $R$ be a Ring and $M,N$ be $R$-modules with $\Phi: M \to N$ linear. Then
\begin{align*}
	\Ker \Phi < M \quad \text{and} \quad \Image \Phi < N
\end{align*}
are submodules and $\Phi$ induces an Ismorphism
\begin{align*}
	\overline{\Phi}: \faktor{M}{\Ker \Phi} \to \Image \Phi
\end{align*}
\end{proposition}


\begin{lemma}[]
	Let $R$ be a ring and $M_{1}, \ldots, M_{n}$ be $R$-modules. Then $M_{1} \times \dots \times M_{n}$ is again an $R$-module with coordinatewise scalar multiplication
\end{lemma}
Proof: trivial


\begin{lemma}[]
Let $R,S$ be Rings and $M$ be an $R$-module. Then \#\#\# missing 5 mins
\end{lemma}
Proof: missing



What sort of rings can be Interseting? \#\# missing 5 mins



\begin{theorem}[]
	Let $K$ be field and $M$ a vector space over $K$. The definition of a module strucutre on $M$ over $K[X]$ is equivalent to chosing a $K$-linear mapping $\phi:M \to M$.

	If we have a scalar multiplication $\bm{\cdot}: K[X] \times M \to M$, whose restriction on $K \times M$ is compatible with the given scalar multiplication $\cdot: K \times M \to M$, then we can get a $K$-linear map given by
	\begin{align*}
		\phi: M \to M, \quad \phi(m) = X \bm{\cdot} 
	\end{align*}

	On the other hand, such a linear map $\phi$ induces a scalar mulitplication
	\begin{align*}
		\bm{\cdot}: K[X] \times M \to M: \quad f \bm{\cdot} m = (f(\phi))(m) = \left(
			\sum_{k}a_k \phi^{k}
		\right)(m) \quad \text{for} \quad f = \sum_{k}a_kX^{k} \in K[X]
	\end{align*}
\end{theorem}
The way we converted $\bm{\cdot}$ to $\phi$ and back is inverse to how we converted $\phi$ to $\bm{\cdot}$
Proof:

If $\bm{\cdot}: K[X] \times M \to M$ defines a module strucutre on $M$ over $K[X]$, then  $\phi(m) = X \bm{\cdot} m$ defines a $K$-linear mapping on $M$ since
\begin{align*}
	\#\#\# \text{missing calculation 8 min}
\end{align*}

On the other hand, if $\phi: M \to M$ is $K$-linear, then we the scalar multiplication as defined in the theorem is a Ring homomorphism. This also follows the module axioms.



Now we want to classify Modules over PIDs.



missing \#\#\# 8 mins




