\section{Field Theory}
\subsection{Field Extensions}
Note: A ring homomorphism between two fields is always injective because it's kernel is always an ideal and there are only two ideals of a field $K$.
$K$ itself and $(0) = \{0\}$

\begin{definition}[]
	Let $L$ be a field an $K \subseteq L$ a subring that is also a field. We then say $K$ is a \textbf{subfield} of $L$ and we call $L$ a \textbf{field-extension} of $K$.

	We also write $L/K$ if $L$ is a field-extension of $K$, since $L$ can be thought of as a vector space over $K$. We call the dimension of the vector space $L/K$ the \textbf{degree} and denote it with $[L:K]$. 
	If $[L:K] < \infty$, we say $L$ is a \textbf{finite} field-extension of $K$.
\end{definition}


Examples
\begin{enumerate}
	\item $\Q(\sqrt{2})/\Q$
	\item $\C/\R$
	\item $\Q(\sqrt[3]{2})/\Q \iso \faktor{\Q[T]}{(T^3 - 2)}/\Q$
\end{enumerate}


\begin{lemma}[Multiplicity of degree]
Let $F/L$ and $L/K$ be finite field extensions. Then
\begin{align*}
	[F:K] = [F:L] \cdot [L:k]
\end{align*}
\end{lemma}
Proof: Let $m = [F:L]$ and assume $x_{1}, \ldots, x_{m} \in F$ are a basis of $F$ over $L$. 
Also let $n = [L:K]$ and $y_{1}, \ldots, y_{n} \in L$ are a basis of $L$ over $K$.
Then we can show that the products
\begin{align*}
	x_iy_j \in F, \quad \text{for} \quad i = 1, \ldots m, j = 1, \ldots n
\end{align*}
are a basis of $F$ over $K$.

To show linear independence, let $\alpha_{ij} \in K$ and $\sum_{i,j}\alpha_{ij}x_iy_j = 0$. But because $x_{1}, \ldots, x_{m}$ are linearly independent over $L$
\begin{align*}
	\sum_{i=1}^{m}\underbrace{\sum_{j=1}^{n}\alpha_{ij}y_i}_{\in L}x_i = 0 \implies \sum_{j=1}^{n}\alpha_{ij}y_j = 0 \implies \alpha_{ij} = 0, \forall i,j
\end{align*}
which shows that $x_iy_j$ are linearly independent.

To show that they span $F$, let $z \in F$. Then there exist $\beta_{1}, \ldots, \beta_{m} \in L$ such that $z = \sum_{i = 1}^{m}\beta_{i}x_i$. And because $y_i$ are a basis of $L$, for each $\beta_i \in L$ there exist $\alpha_{i1}, \ldots, \alpha_{in} \in K$ such that
\begin{align*}
	\beta_i = \sum_{j=1}^{n}\alpha_{ij}y_j \implies z = \sum_{i,j}\alpha_{ij} x_iy_j
\end{align*}



\begin{definition}[]
	Let $L/K$ be a field extension and $x \in L$ and 
	\begin{align*}
		\phi_x: K[T] \to L, \quad f \mapsto f(x)
	\end{align*}
	the evaluation mapping. 
	\begin{itemize}
		\item If $\phi_x$ is injective, we say $x$ is \textbf{transcendent} over $K$
		\item If $\phi_x$ is not injective, we say that $x$ is \textbf{algebraic}. In this case $\Ker \phi_x = (m_x(T))$ is an ideal and we call $m_x(T)$ the \textbf{Minimal polynomial} of $x$ and the degree of $m_x(T)$ also the degree of $x$.
	\end{itemize}
\end{definition}
Note that the definition for algebraic is equivalent to saying that $x \in L$ is a root of a polynoial with coefficients in $K$.

Examples
\begin{enumerate}
	\item $e, \pi \in \R$ are transcendent over $\Q$
\item $\sqrt[3]{2} \in \R$ is algebraic over $\Q$ with minimal polynomial $T^3 - 2$
\item $\cos(20^{\circ}) \in \R$ is algebraic because
	\begin{align*}
		\cos(3 \phi) = \cos^3(\phi) - 3 \cos \phi \sin^2 \phi = \cos^3 \phi - 3 \cos \phi + 3 \cos^3 \phi
	\end{align*}
	so for $\phi = 20^{\circ}$ we know that
	\begin{align*}
	\cos(60^{\circ}) \frac{1}{2} = 4 \cos^3 \phi - 3 \cos \phi
 	\end{align*}
	with minimal polynomial $4T^3 - 3T - \frac{1}{2} \in \Q[T]$
\end{enumerate}



\begin{proposition}[]
	Let $L/K$ and $x \in L$. If $x$ is transcendent, then
	\begin{align*}
		K[X] = \Image \phi_x \iso L[T]
	\end{align*}
	and the smallest subfield $K(x)$ of $L$ tht contains $K$ and $x$ satisfies
	\begin{align*}
		K(x) \iso K(T)
	\end{align*}
	If $x$ is algebraic, then
	\begin{align*}
		K[X] = \Image \phi_x \iso \faktor{L[T]}{(m_x(T))}
	\end{align*}
	is already the smallest subring $K(x)$ that contains $K$ aswell as $x$. Then also
	\begin{align*}
		[K(x): K] = \deg m_x(T)
	\end{align*}
\end{proposition}

Proof: The isomorphsm follows directly from the first isomorphism theorem.

If $x$ is transcendent, then 
\begin{align*}
	K(x) = \left\{\frac{f(x)}{g(x)} \big\vert f(T),g(T) \in K[T], g \neq 0\right\} \iso \left\{\frac{f(T)}{g(T)} \big\vert f,g \in K[T], g \neq 0\right\}
\end{align*}
if $x$ is transcendent, then $\Ker \phi_x = (m_x(T))$ is a prime ideal. Because in a PID, every prime ideal $\neq (0)$ is a maximal ideal, we know that the faktor-ring with the maximal idea $\faktor{K[T]}{(m_x(T))}$ is a field, so $K[X]$ is a subfield of $L$.

Furthermore, in $\faktor{K[T]}{(m_x(T))}$ we can use division with rest to show that 
\begin{align*}
	1 + (m_x(T)), T + (m_x(T)), \ldots, \ldots T^{\deg m_x - 1} + (m_x(T)) \in \faktor{K[T]}{(m_x(T))}
\end{align*}
form a basis.



\begin{definition}[]
	Let $L/K$ and $x_{1}, \ldots, x_{n} \in L$. We write the smallest subfield of $L$ that contains $K$ aswell as $x_{1}, \ldots, x_{n}$ as
	\begin{align*}
		K(x_1,\ldots,x_n) = \left\{\frac{f(x_1,\ldots,x_n)}{g(x_{1}, \ldots, x_{n})} \big\vert f,g \in K[T_1,\ldots,T_n], g(x_1,\ldots,x_n) \neq 0\right\}
	\end{align*}
\end{definition}



\begin{corollary}[(Wantzel, 1837)]
With ruler and compass, neither the third root of $2$ nor an angle of $20^{\circ}$ can be constructed.\\
Furthermore, if $p \in \N, p > 2$ is prime, and the regular $p$-gon is constructable with ruler and compass, then $p$ is a Farmat-prime: $p = 2^{2^{n}} + 1$.
\end{corollary}
Sketch of the proof: Assume that after finitely many constructions step starting from a unit length by intersecting straight lines with circleswe end up with a straight line of length $x = \sqrt[3]{2}$ or $x = \cos(20^{\circ})$. Then define $L_0 = \Q$, and then
\begin{align*}
	L_{n+1} = \left\{\begin{array}{ll}
		L_{n} & \text{if in the $n$-th construction step two lines are interected.} \\
		\text{ or }
	\end{array} \right.
\end{align*}
a quadratic field extension of $L_n$ which contains the coordinates of the intersection betwen a cricle with a line or circle with a circle.

Then we have
\begin{align*}
	(x - x_0)^2 + (y - y_0)^2 = A^2 \quad \text{and} \quad ax + by = c
\end{align*}
If it has roots in $L_n$, then set $L_{n+1} = L_n$ and if it has roots, then set $L_{n+1} = L_n(x,y)$. Then
\begin{align*}
	x \in L_N / \Q \implies [L_N:  \Q] = 2^l
\end{align*}
for some $l \in \N$, but $K = \Q[X] /\Q$ has degree $3$.
So since
\begin{align*}
	[L_N:L_0] = [L_N:L_{N-1}] \dots [L_1:L_0]
\end{align*}
it would also have to be divisible by three, which is a contradiction.


\begin{definition}[]
	A field extensions $L/K$ is called \textbf{algebraic}, if every $x \in L$ is algebraic over $K$
\end{definition}
\begin{lemma}[]
	Every finite field extension is algebraic.
\end{lemma}
Proof: For $[L:K] < \infty$ and $x \in L$ the mapping 
\begin{align*}
	\phi_x: K[T] \to L, \quad f \mapsto f(x)
\end{align*}
cannot be injective because $K[T]$ has infinite dimension over $K$, whereas $L$ by assumption only has finite dimension over $K$.


\begin{corollary}[]
Let $L/K$ and $x,y \in L$ be algebraic over $K$. Then so are $x+y, x \cdot y$ and for $x \neq 0, \frac{1}{x}$.
\end{corollary}
Proof: By supposition we know that $[K(x):K] < \infty$. We can write the minimal polynomial $m_y(T) \in K[T]$ also as a polynomial in $K(x)[T]$. 

This implies that $y$ is also algebraic over $K(x)$ and so $[K(x)(y):K(x)] < \infty$.

From the lemma on the multiplicity we know that 
\begin{align*}
	[K(x,y):K] = [K(x)(y):K] = [K(x)(y):K(x)]\cdot [K(x):K] < \infty	
\end{align*}
so $K(x,y)$, which contains all the elements $x + y, x \cdot y, \ldots \in K(x,y)$, is a finite field extension of $K$, and using the previous lemma, it is algebraic over $K$.


\begin{corollary}[]
Let $F/L$ and $L/K$. Then $F/K$ is algebraic if and only if $F/L$ and $L/K$ are algebraic.
\end{corollary}
Proof: $\implies$ is an exercise.
For the reverse, assume $F/L$ and $L/K$ are algebraic field extensions each. Let $x \in L$. Then there exists a minimal polynomial
$m_x^{L}(t) \in L[T]$. Now assume $y_{1}, \ldots, y_{n} \in L$ are the coefficients of $m_x^{L}(T)$. Just like in the prevous corollary, we can show that
\begin{align*}
	[K(y_{1}, \ldots, y_{n}):K] < \infty
\end{align*}
so since $m_x^{L}(T)$ has coefficients in $K(y_{1}, \ldots, y_{n})$, we know that
\begin{align*}
	[K(y_{1}, \ldots, y_{n},x): K(y_{1}, \ldots, y_{n})] \leq \deg m_x^{L} < \infty
\end{align*}
using multiplicativity of the degrees, it also follows that $[K(y_{1}, \ldots, y_{n},x):K] < \infty$.

And since $x \in K(y_{1}, \ldots, y_{n},x)$ and $K(y_{1}, \ldots, y_{n},x) /K$ is a finite field extensions, it must be algebraic.


Example: We trivially know that $\sqrt{2}$ and $\sqrt{3} \in \R$ are algebraic over $\Q$. Then
\begin{align*}
	x = (\sqrt{2} + \sqrt{3}) \implies x^2 = 5 + 2 \sqrt{6}\\
	(x^2 - 5)^2 = (2 \sqrt{6})^2 = 24 \implies m_x = x^4 - 10x^2 + 1
\end{align*}





