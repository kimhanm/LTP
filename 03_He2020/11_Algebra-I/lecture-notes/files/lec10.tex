Proof: Let $a = up_1 \dots p_m = v q_1 \ldots q_n$ be two prime factoristions.\\
We use induction on $n$. If $n = 0$, then $a = v \in R^{\times}$ and thus also $m = 0$ because if $m > 0$ we would have $p_1 | a$ and $a | 1$ which means that $p_1 | 1$ but then $p_1$ would be a unit $\lightning$\\

Since we have $q_n | a$, and $q_n$ is prime, it must divide one of the factors $a = u p_1 \dots p_m$. Say $q_n | p_{\sigma(n)}$ which is again prime, so $q_n \sim p_{\sigma(n)}$. Using the induction hypothesis for
\begin{align*}
	\frac{a}{q_n} = \underbrace{u \frac{p_{\sigma(n)}}{q_n}}_{\in R^{\times}}p_1 \dots p_{\sigma(n)-1} p_{\sigma(n) + 1} \dots p_m = v = q_1 \dots q_{n-1}
\end{align*}
It follows that $m-1 = n-1$ and there exists a bijection
\begin{align*}
	\sigma: \{1, \ldots, n-1\} \to \{1, \ldots, \sigma(n)-1, \sigma(n)+1, \ldots, m\}
\end{align*}
such that $q_j \sim p_{\sigma(j)}$ for $j =1, \ldots, n-1, n$.\\

\begin{definition}[]
	Let $R$ be a UFD. We say $P \subseteq R$ is a \textbf{representation set}  (of prime elements) if every $p \in P$ is prime and for every prime element $q \in R$ there exists a unique $p \in P$ such that $q \sim p$.
\end{definition}
For example, the prime numbers in $\Z$ are $\pm$ the ``prime numbers'' of $\N$ (which isn't a ring). So for $R = \Z$ the set
\begin{align*}
	P = \{p \in \Z \big\vert p \text{ prime and positive}\}
\end{align*}
For $R = K[X]$ we test
\begin{align*}
	P = \{f \in K[X] \big\vert f \text{ is irreducible and normed}\}
\end{align*}
This is possible since the units in $K[X]$ are exactly the constant polynomials, so we can always divide out the leading coefficients.\\

For $R = \Z[i]$ we can set
\begin{align*}
	P = \left\{a + ib \text{ prime in }R, a > 0 \text{ and } -a < b \leq a\right\}
\end{align*}
which corresponds to the quarter plane on the right side.

\begin{lemma}[]
	Let $R$ be a UFD, then it has a representation set.
\end{lemma}
Proof: We use the axiom of choice on the set
\begin{align*}
	\left\{[p]_{\sim} \big\vert p \in R \text{ prime}\right\}
\end{align*}
If we want the unique factorisation domain, then we want the prime factorisation be unique including the unit $u$ or $v$, so we have the following theorem

\begin{theorem}[]
Let $R$ be a UFD and $P \subseteq R$ a representation set. THen every element $a \in R \setminus \{0\}$ has a unique prime factorisation of the form
\begin{align*}
	a = u \prod_{p \in P}' p^{n_p} 
\end{align*}
where $n_p$ is zero for all but finitely many $p \in P$.
\end{theorem}
Proof: If $a \in R^{\times}$ we set $u = a$ and $n_p = 0$ for all $p \in P$. Otherwise we just use the property of UFD's that $a = u p_1 \dots p_m$ and for every $p_j$ there exists a unique $p_i \sim p \in P$ and we get
\begin{align*}
	a = u \frac{p_1 \dots p_n}{\prod_{p \in P}p^{n_p}} \prod_{p \in P}p^{n_p}
\end{align*}
where $n_p$ is the number of indices $i$ such that $p_i \sim p$.\\
To show that the factorisation is unique, we assume that
\begin{align*}
	a = u \prod_{p \in P}p^{n_p} = v \prod_{p \in P}n^{n_p'}
\end{align*}
If $n_p' = 0$, for all $p \in P$, then $a = v \in R^{\times}$ and $n_p = 0$ for all $p$. Else if $n_{p_0}' > 0$ then $p_0$ divides $a$ but since there is only one $p \in P$ that is associated to $p_0$, it follows that $n_{p_0}' = n_{p_0}$.

Using induction on the sum $\sum_{p  \in P}n_{p}'$ the theorem follows.


\begin{lemma}[]
	Let $R$ be a UFD and $P \subseteq R$ a representation set. IF $a = u \prod_{p \in P}p^{m_p}$ and $b = v\prod_{p \in P}p^{n_p}$ then $a$ divides $b$ if and only iff $m_p \leq n_p$ for all $p \in P$.
\end{lemma}
Proof: If $b = ac$ and $c = w \prod_{p \in P}p^{k_p}$. then
\begin{align*}
	b = v \prod_{p \in P}p^{n_p} = uw \prod_{p \in P}p^{m_p + k_p}
\end{align*}
Then $v = uw$ and $n_p = m_p + k_p \geq m_p$.\\
If $m_p \leq n_p$ we chose our $c$ to be 
\begin{align*}
	c = vu^{-1} \prod_{p \in P} p^{n_p - m_p} \in R
\end{align*}
which is well defined, since $u$ is a unit and $n_p - m_p \geq 0$.

\begin{proposition}[GCD]
	Let $R$ be a UFD with representation set $P$. Then for every nonzero pair $(a,b) \neq (0,0)$ there exists a \textbf{greatest common divisor}. If $a = u \prod_{p \in P}p^{m_p}, b = v \prod_{p \in P}p^{n_p}$ then the divisor is given by
	\begin{align*}
		d = \prod_{p \in P}p^{\min(m_p,n_p)} =: \gcd(a,b)
	\end{align*}
	We can show that for any integral domain, the $\gcd$ is unique up to a unit.
\end{proposition}
Proof: We see from the definition that $\gcd(a,b)$ divides both $a$ and $b$. If we have another divisor of $a$ and $b$, then its exponents of $p$ must also be smaller than those of $a$ and $b$ and thus smaller than their mimumum.\\

Analogously we can define the $\gcd$ of multiple elements $a_1, \ldots, a_n \in R$ and the above proposition holds aswell.

\begin{definition}[]
	Let $R$ be a UFD. We say that $a_1, \ldots a_l \in R$ are \textbf{coprime}  if $\gcd(a_1, \ldots, a_l) = 1$ or equivalently if for every prime element $p \in R$ there is a $a_j$ such that $p \not| a_j$.
\end{definition}

\begin{corollary}[]
	Let $R$ be a UFD and $K = \text{Quot}(R)$ its quotient field. Then every $x \in K$ has a representation $x = \frac{a}{b}$ with $a,b \in R$ coprime.
\end{corollary}

Proof: Let $x = \frac{\tilde{a}}{\tilde{b}} \in K$ and let $d = \gcd(\tilde{a},\tilde{b})$. Then set
\begin{align*}
	a := \frac{\tilde{a}}{d} \text{ and } b := \frac{\tilde{b}}{d} \implies a,b \text{ coprime}
\end{align*}


\begin{corollary}[]
	Let $R$ be a UFD with quotient field $K = \text{Quot}(R)$. Then every $x \in K$ has a representation of the Form
	\begin{align*}
		x = u \prod_{p \in P} p^{n_p}, \quad \text{for} \quad n_p \in \Z, n_p \neq 0 \text{ for only finitely many } p \in P
	\end{align*}
\end{corollary}


Not all rings are UFDs. For example we can look at $R = \Z[i \sqrt{5}] \subseteq \Q[i\sqrt{5}] \subseteq \C$. And try to see if we can find two different factorisations of the number $6$.
\begin{align*}
	6 = 2 \cdot 3 = (1 + i \sqrt{5})(1 - i \sqrt{5})
\end{align*}
First of all we can see that all of these factors are irreducible and they are not associated with eachother as the units of this ring are just $\pm 1$.\\


Dedekind found that this counterexample could be resolved if we looked at at better, idealised prime factors in a better ring.
\begin{align*}
	(6) = (2, 1 + i \sqrt{5})^2 (3, 1 + i \sqrt{5}) (3, 1 - \sqrt{5})
\end{align*}


\subsection{Examples of euclidean rings}
All examples we look at here live in a quadratic field $K = \Q[\sqrt{d}]$ where $d \in \Z$ is not a square number.
\begin{align*}
	K &= \Q[\sqrt{d}] = \{a + b \sqrt{d}: \big\vert a,b \in \Q\} \simeq \Q[X]/(X^2-d)
\end{align*}
We define the \textbf{conjugation} 
\begin{align*}
	\tau: K \to K, \quad a + b \sqrt{d} \mapsto a - b \sqrt{d}
\end{align*}
which defines a field automorphism on $K$.

First we prove that the fields $\Q[\sqrt{d}]$ and $\Q[X]/(X^2 - d)$ are indeed isomorphic. We define the evaluation mapping
\begin{align*}
	\ev_{\sqrt{d}}: \Q[X] \to K, \quad f \mapsto f(\sqrt{d}), \quad \ev_{\sqrt{d}}(X^2 - d) = 0
\end{align*}
Since $X^2 - d$ has no roots in $\Q$, it is irreducible/prime in the PID $\Q[X]$. Therefore its principal ideal $(X^2 - d)$ is a maximal ideal. So since $(X^2 = \Ker \ev_{\sqrt{d}}$. THe first isomorphism theorem say that
\begin{align*}
	\Q[X]/(X^2 - d) = \Q[X]/\Ker \ev_{\sqrt{d}} \simeq \Image(\ev_{\sqrt{d}}) = \Q[\sqrt{d}] 
\end{align*}
Using this isomorphism we get that the mapping $\tau$ can be thought of as the quasi-composition $-\sqrt{d} \circ \sqrt{d}$ of isomorphisms
\begin{align*}
	K \stackrel{\sqrt{d}}{\to} \Q[\sqrt{d}] \simeq \Q[X]/(X^2 - d) \stackrel{-\sqrt{d}}{\to} K\\
	\sqrt{d} \mapsto [X]_{\sim_{(X^2 - d)}} = X + (X^2 + d) \mapsto -\sqrt{d} + (-d + d) = -\sqrt{d} = \tau(\sqrt{d})
\end{align*}
So $\tau$ is again an isomorphism.

