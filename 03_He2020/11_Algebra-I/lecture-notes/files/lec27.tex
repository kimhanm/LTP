\subsection{Uniqueness}

We have seen that for every $f \in K[T]$ there is a splitting field aswell as an algebraic closure. 
In this section, we find out if/how they are unique.

\begin{theorem}[]
	Let $K$ be a field, $L/K$ a field extensions and $L$ algebraically closed. Then
	\begin{enumerate}
		\item If $E = K[\alpha]$ is a finite field extension of $K$, then there is at least one and at most $[E:K]$ field embeddings
		\begin{align*}
			\sigma: E \to L \quad \text{such that} \quad \sigma|K = \id_K
		\end{align*}	
		if $\charac K = 0$, then there are exactly $[E:K]$ such embeddings.
	\item If $E/K$ is an algebraic field extension, then there exists a $K$-linear embedding $\sigma: E \to L$.
	\end{enumerate}
\end{theorem}

To prove this, we need the following lemma
\begin{lemma}[]
	Let $K$ be a field, $m(T) \in K[T]$ coprime to its derivative $m'(T)$. Then $m(T)$ has exactly $\deg m(T)$ many roots in an algebraic field extension.

	This is the case if $\charac K = 0$ and $m(T)$ is irreducible in $K[T]$.
\end{lemma}

Proof: We define the derivative as the linear map given by
\begin{align*}
	D: f = \sum_{n=0}^{\infty}a_n T^{n} \mapsto f' = \sum_{n=1}^{\infty}n a_n T^{n-1}
\end{align*}
which satisfies the product rule
\begin{align*}
	(f g)' = f'g + fg'
\end{align*}
the product rule itself is a polynomial equation over $\Z$ in the coefficients of $f$ and $g$. In particular we have
\begin{align*}
	\left(
		(T- \alpha)^{2} g(T)
	\right)'
	=
	2(T - \alpha)g(T) + (T - \alpha)^{2}g'(T)
	=
	\left(
		2g(T) + (T - \alpha) g'(T)
	\right)
\end{align*}
so if $f$ has a root with higher multiplicity, then it is also a root of $f'$


The same holds if $\alpha \in L$ when $L/K$ is a field extension.

Now assume $m(T)$ and $m'(T)$ are coprime, then there exist $h_1,h_2 \in K[T]$ such that
\begin{align*}
	1 = h_1(T)m(T) + h_2(T)m'(T)
\end{align*}
so if $L/K$ is a field extension and $\alpha \in L$ is a root of $m(T)$ such that $m'(\alpha) \neq 0$, then that means that it has to be a root with single multiplicity.

\begin{center}
	Missing 5 mins
\end{center}

Now can prove the first item from the theorem.
Let $m(T)$ be the minimal polynomial of $\alpha$ over $K$ and $\beta = \sigma(\alpha)$ for a $k$-linear field embedding $\sigma: E \to L$.
then
\begin{align*}
	m(\beta) = m(\sigma(\alpha)) = \sigma(m(\alpha)) = 0, \quad \text{for} \quad m(T) = \sum_{n}a_nT^{n}, a_n \in K
\end{align*}
but because $\sigma$ is a ring homormophism, it  must map $a_n$ to $a_n$. 
Furthermore, if we go the other way around: for a $f(\alpha) \in K[\alpha]$ we know that
\begin{align*}
	\sigma(f(\alpha)) = f(\sigma(\alpha)) = f(\beta)
\end{align*}
so $\beta = \sigma(\alpha)$ is a root and $\sigma$ is uniquely determined by this root.

Since $m(T)$ has at most $\deg(m(T)) = [E:K]$ roots in $L$ we know that there can be at most this many $K$-linear field embeddings.

For the converse, let $\beta \in L$ be any root of $m(T)$. Then we can use the lemma to show that the existse excactly $\deg m(T)$ many roots.

We can use $\beta$ to define a $K$-linear field inlcusion
\begin{align*}
	\sigma = \sigma_{\beta}: E = K[\alpha] \to L
\end{align*}
Consider the ways to map $f(T) \in K[T]$ using the evaluation mappings $\ev_{\alpha}$ and $\ev_{\beta}$. 
\begin{align*}
	f(T) \mapsto f(\alpha) \in K[\alpha] = E \quad \text{and} \quad f(T) \mapsto f(\beta) \in K[\beta] \subseteq L
\end{align*}
Then $\Ker(\ev_{\alpha}) = (m(T))$ and since $m(\beta) = 0$ we can conlclude that $(m(T)) \subseteq \Ker \ev_{\beta}$. But because $(m(T))$ is a maximal ideal, we even get equality $(m(T)) = \Ker \ev_{\beta}$.

Using the first isomorphism theroem, we get the mappings $\overline{\ev_{\alpha}}$ and $\overline{\ev_{\beta}}$.
\begin{center}
	\begin{tikzcd}[] %\arrow[bend right,swap]{dr}{F}
		& f(\alpha) \in K[\alpha] = E\\
		f(T) + (m(T)) \in \faktor{K[T]}{(m(T))}
		\arrow[]{dr}{\overline{\ev_{\beta}}}
		\arrow[]{ur}{\overline{\ev_{\alpha}}}\\
		& f(\beta) \in K[\beta] \subseteq L
	\end{tikzcd}
\end{center}	
so we can set
\begin{align*}
	\sigma = \overline{\ev_{\beta}} \circ (\overline{\ev_{\alpha}})^{-1}: E \to L
\end{align*}
as our embedding. And for two different roots $\beta_1 \neq \beta_2 \in L$ we have
\begin{align*}
	\sigma_{\beta_1}(\alpha) = \beta_1 \neq \beta_2 = \sigma_{\beta_2}(\alpha) \implies \sigma_{\beta_1} \neq \sigma_{\beta_2}
\end{align*}
so we see that there exactly as many field embeddings of $E = K[\alpha]$ into $L$ as there roots of $m(T)$ in $L$.


\textbf{Example}: Consider $K = \F_p((X))$ and $m(T) = T^{p} - X$ (which is irreducible).

		For $E = \faktor{K[T]}{(m(T))}$, we have the root $T + (m(T)) = \alpha$ of $m(T)$.
		Here we have
		\begin{align*}
			m(T) = (T - \alpha)^{p} = T^{p} - \alpha^{p} = T^{p} -X
		\end{align*}
		so $m$ has $\alpha$ as a root with multiplicity $p$.

To show the second item in the theorem we will need Zorn's Lemma because the field extension can be infinite dimensional.

We define 
\begin{align*}
	\mathcal{O} = \left\{(F,\sigma) \big\vert F \text{ field with } K \subseteq F \subseteq E, \sigma: F \to L \text{ is $K$-linear field extension}\right\}		
\end{align*}
Naturally we get the partial ordering on $\mathcal{O}$ give by
\begin{align*}
	(F_1,\sigma_1) \leq (F_2,\sigma_2) \iff F_1 \subseteq F_2 \text{ and } \sigma_2|F_1 = \sigma_1
\end{align*}
To be able to use Zorn's lemma, we need to show the following
\begin{itemize}
	\item $\mathcal{O} \neq 0$, because $(K, \id) \in \mathcal{O}$.
	\item Let $T \subseteq \mathcal{O}$ be a totaly ordered chain in $\Omega$. We define
		\begin{align*}
			F_T = \bigcup_{(F, \sigma) \in T} F \subseteq E
		\end{align*}
		which should be a subfield of $E$. The proof of this is trivial. For the field embedding, we define
		\begin{align*}
			\sigma_T: F_T \to L, \quad x \in F \mapsto \sigma(x) \quad \text{for} \quad (F,\sigma) \in T
		\end{align*}
		this (just like $F_T$) is well defined because $T$ is a totally ordered chain: If we had $(F_1, \sigma_1) \leq (F_2,\sigma_2) \in T$, we can pick either of the $\sigma$ because 
	\begin{align*}
		\sigma_2(x) = \sigma_2|F_1(x) = \sigma_1(x)
	\end{align*}
	$\sigma_T$ is also a field embedding, because for $x_1,x_2 \in F_T$ there exist
	\begin{align*}
		(F_1,\sigma_1), (F_2,\sigma_2) \in T \quad \text{such that} \quad x_1 \in F_1, x_2
	\end{align*}
	because $T$ is totally ordered, without loss of generality $(F_1, \sigma_1) \leq (F_2,\sigma_2)$, so
	\begin{align*}
		\sigma_T(x_1 + x_2) = \sigma_2(x_1 + x_2) = \sigma_2(x_1) + \sigma_2(x_2) = \sigma_T(x_1) + \sigma_t(x_2)
	\end{align*}
	and analogously for $x_1 \cdot x_2$ and $\frac{1}{x_1}$ if $x_1 \neq 0$.
\item So $(F_T,\sigma_T)$ is an upper bound of $T$.
\end{itemize}
Zorn's Lemma then says that $\mathcal{O}$ has a maximal Element $(F,\sigma) \in \mathcal{O}$.

With this we set $E = F$ and $\sigma: F \to L$ as the wanted field embedding.

If $E \neq F$, there must be an $\alpha \in E \setminus F$. In this case we use $\sigma: F \to L$ to identify the Elements of $F$ with the elements of $\sigma(F)$.
Then
\begin{align*}
	F \subseteq F[\alpha] \subseteq E, \text{ and } F \subseteq L
\end{align*}
and we are in the setting of the first item in the theorem. Using the first part, we know that there is an $F$-linear field empedding $\phi: F[\alpha] L$.

Since we used $\alpha$ to identify elements of $F$ with elements of $\sigma(F)$, it means that $\phi:F[\alpha] \to L$ extends the mapping $\sigma:F \to L$m which means $(F,\sigma) < (F[\alpha],\phi)$.

But this is a contradiction to the maximality of $(F,\sigma)$.

