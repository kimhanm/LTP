% ==== 18.12.20 ====
\begin{corollary}[]
	Let $K$ be a field. 
	\begin{enumerate}
		\item For every $f \in K[T]$, the splitting field is unique up to a $K$-linear isomorphism.
		\item Every two algebraic closures of $K$ are $K$-linearly isomorphic.
	\end{enumerate}
\end{corollary}
\begin{itemize}
	\item Proof: Let $f(T) \in K[T]$ and $E_1,E_2$ be splitting fields of $f(T)$. Then let $L_2$ be an algebraic closure of $E_2$. 

Then we can use the second part of the theorem to get a $\sigma: E_1 \to L_2 \supseteq E_2$
\begin{center}
	\begin{tikzcd}[] %\arrow[bend right,swap]{dr}{F}
		& E_1 \arrow[swap]{dd}{\sigma}\\
		K \arrow[swap]{ur}{} \arrow[]{dr}{}\\
		& L_2 
	\end{tikzcd}
\end{center}
so we can write for $\alpha_{1}, \ldots, \alpha_{n} \in E_1$ that
\begin{align*}
	f(T) = a \prod_{i = 1}^{n}(T - \alpha_i) = \sigma(f(T)) = a \prod_{i = 1}^{n} (T - \sigma(\alpha_i))
\end{align*}
so roots get mapped to roots.

But 
\begin{align*}
	E_1 = K[\alpha_{1}, \ldots, \alpha_{n}]
	=
	\left\{\frac{q_1(\alpha_{1}, \ldots, \alpha_{n})}{q_2(\alpha_{1}, \ldots, \alpha_{n})} \big\vert q_1,q_2 \in K[T_{1}, \ldots, T_{n}], q_2(\alpha) \neq 0\right\}
\end{align*}

is generated by the roots of $f$. And because $E_2$ is a splitting field, the polynomails $(T - \sigma(\alpha_i))$ must lie in $E_2$, so $E_2 = K[\sigma(\alpha_1),\ldots, \sigma(\alpha_n)]$ which is generated by these roots, is the image of $\sigma$.
\begin{align*}
	\sigma(E_1) = \sigma(K(\alpha_{1}, \ldots, \alpha_{n})) = E_2
\end{align*}
This means that $\sigma: E_1 \to E_2$ must be an isomorphism.

	\item Let $L_1,L_2$ be to algebraic closures of $K$. Using the second part of the Theorem, we get a field embedding 
		\begin{align*}
			\sigma: L_1 \to L_2 \quad \text{with} \quad K \subseteq \sigma(L_1) \subseteq L_2
	\end{align*}
	further $L_2/K$ is algebraic and $\sigma(L_1)$ is algebraically closed. $L_2$ being algebraic just means that every element is the root of a polynomial with coefficients in $K$
	\begin{align*}
		\alpha \in L_2 \implies \exists f \in K[T] \setminus \{0\}, f(\alpha) = 0
	\end{align*}
	so $L_2 \subseteq \sigma(L_1) \subseteq L_2$, which shows that $\sigma: L_1 \to L_2$ is an isomorphism.
\end{itemize}

\subsection{Finite Fields}
We know that $\F_p = \faktor{\Z}{(p)}$ for $p \in \N$ prime is a finite field. Are there more and can we classify them?

\begin{theorem}[(Galois, Gauss)]
	\begin{enumerate}
		\item If $K$ is a finite field, then $\abs{K} = p^{n}$ for $p \in \N$ prime and $n \geq 1 \in \N$
		\item For every power of a prime $p^{n}$ there exists an up to isomorphism unique finite field with $p^{n}$ elements.
		\item Let $p \in \N$ prime and $K$ an algebraic closure. Then for every $n \geq 1$, $K$ contains an up to isomorphism unique subfield $\F_{p^{n}}$ with $p^{n}$ Elements. We can even determine this subfield as
			\begin{align*}
				\F_{p^{n}} = \left\{x \in K \big\vert x^{(p^{n})} = x\right\}
			\end{align*}
		\item For $m,n \geq 1$ and fields as in (c)
			\begin{align*}
				\F_{p^{m}} \subseteq \F_{p^{n}} \iff m|n
			\end{align*}
	\end{enumerate}
\end{theorem}
Proof
\begin{enumerate}
	\item Let $\abs{K} < \infty$. Then $\Z \cdot 1_k \iso \F_p = \faktor{\Z}{(p)}$ for a prime $p \in \N$. Therefore $K$ is a finite-dimensional vector space over $\F_p$ so
		\begin{align*}
			K \iso \F_p^{[K:\F_p]} \implies \abs{K} = p^{[K:\F_p]}
		\end{align*}
	\item To construct a field with $q = p^{n}$ Elements we define it as a splitting field. For the polynomial $T = T^{q} - T$, let $L$ be the splitting field of $f$ over $\F_p$ and 
		\begin{align*}
			E = \left\{x \in L \big\vert x^{q} = x\right\}
		\end{align*}
		Consider the \textbf{Frobenius-Homomorphism}
		\begin{align*}
			\Phi: L \to L, \quad x \mapsto x^{p} \implies \Phi^{n}: L \to L, \quad x \mapsto \left(
				\dots \left(
					x^{p}
				\right)^{p} \dots
			\right)^{p} = x^{np} = x^{q}
		\end{align*}
		Because $L$ is the splitting field of a finite field, $L$ itself is also a finite field. Moreover, $\Phi$ is an Automorphism because $\Phi$ is injective and $\abs{L} < \infty$, so
		\begin{align*}
			E = \left\{x \in L \big\vert \Phi^{n}(x) = x\right\} = \left\{x \in L \big\vert f(x) = 0\right\}
		\end{align*}
		which means that $E$ is a subfield of $L$ containing all roots of $f$, which means $E = L$.

		Now we need to show that $E$ has $p^{n}$ Elements. This is equivalent to saying that $f$ doesn't have roots with higher multiplicity. Recall that we can analyze this by looking at the derivative $f'(T)$ and checking if they are co-prime and if $f'(T)$ it has no roots:
		\begin{align*}
			f'(T) = q t^{q -1} - 1 = -1 \neq 0
		\end{align*}
		This means that there are exactly $q$ roots in $L$, so
		\begin{align*}
			\abs{L} = \abs{E} = q = p^{n}
		\end{align*}
		Now let $F$ be a finite field with $p^{n}$ elements. From (a) we know that it extends $\F_p$. Furter, for $x \in F^{\times}$ that $x^{p^{n}-1} = 1$ because $F^{\times}$ is a group.

		Aso $x^{p^{n}} = x$ for all $x \in F$. So $F$ consists of the roots of the polynomial $F(T) = T^{q} - 1$ which makes it the splitting field of $f$.

		The previous corollary proves uniqueness up to isomorphism, since $F \iso L$.
	\item Let $K$ be an algebraic closure of $\F_p$. Then
		\begin{align*}
			\F_{p^{n}} = \left\{x \in K \big\vert x^{p^{n}} = x\right\} \subseteq K
		\end{align*}
		is a subfield. But just like before, we see that it is the splitting field of $T^{q} - T$ and $\abs{\F_{p^{n}}} = p^{n}$

	\item If $m|n$, then $n = m \cdot k$ so $\Phi^{n} = \left( \Phi^{m}\right)^{k}$ and
		\begin{align*}
			\F_{p}^{n} = \left\{x \big\vert \Phi^{n}(x) = x\right\} \supseteq \left\{x \big\vert \Phi^{m}(x) = x\right\} = \F_{p^{m}}
		\end{align*} 
		For the other direction, assume $\F_{p^{m}} \subseteq \F_{p^{n}}$, then we can view this as a field extension and thus see $\F_{p^{n}}$ as a vector space over $\F_{p^{m}}$, so
		\begin{align*}
			p^{n} = \abs{\F_{p^{n}}} = \abs{\F_{p^{m}}}^{k} = p^{mk} \quad \text{for} \quad k = [\F_{p^{n}}: \F_{p^{m}}
		\end{align*}
		which shows divisibility of $n$ by $m$.
\end{enumerate}


\begin{theorem}[]
	Let $K$ be a field and $G \subseteq K^{\times}$ a finite subgroup. Then $G$ is cyclic.

	In particular, $\F_{p^{n}}^{\times}$ is cyclic for every power of a prime $p^{n}$.
\end{theorem}
Proof: Using classification of finite abelian groups we have the isomorphism
\begin{align*}
	(G, \cdot) \iso \faktor{\Z}{(d_1)} \times \ldots \times \faktor{\Z}{(d_1)}
\end{align*}
for $1 < d_1|d_2|\dots|d_n$ with $+$ as its group operation.

It is clear that $x^{d_n} = 1$ for all $x \in G$ because
\begin{align*}
	d_1|\dots|d_n \implies 
	d_n \cdot \left(
		a_1 + (d_1) ,\ldots, a_n + (d_n)
	\right) = (0 + (d_1), \ldots, 0 + (d_n))
\end{align*}
This is equivalent to saying that every $x \in G$ is a root of the polynomial $T^{d_n} - 1$. 

Also $\abs{G} = d_1d_2 \dots d_n$. Which can only be true if $n = 1$, or else we would have more roots ($d_1\dots d_n$ many) than the degree of this polynomial $\deg(T^{d_n - 1} = d_n$.

But then the isomorphism just says $G \iso \faktor{\Z}{(d)}$ which means that $G$ is cyclic.


\begin{corollary}[]
Let $p > 2$ be prime. Then for $a \in \F_p$
\begin{align*}
	a^{\frac{p-1}{2}} = \left\{\begin{array}{ll}
			0 & \text{if } a = 0 \\
		 1& \text{if } a = b^{2}, \text{ for } b \in \F_p^{\times}\\
		 -1 & \text{else}
	\end{array} \right.
\end{align*}
\end{corollary}
Sketch of proof. We show
\begin{align*}
	\F_p^{\times} \iso \faktor{\Z}{(p-1)}
\end{align*}
and we can give the proof in $\faktor{\Z}{(p-1)}$.


\section*{Recap}
\begin{theorem}[Smith Canonical Form]
	For an $A \in \Mat_{mn}(R)$ we can obtain a diagonal matrix $D = gAh^{-1}$ for $g,h \in \GL_m(R), \GL_n(R)$ 
\end{theorem}
\begin{theorem}[]
	Let $G$ be a finitely generated abelian group, then
	\begin{align*}
		G \iso \Z^{r} \times \faktor{\Z}{(d_1)} \times \dots \times \faktor{\Z}{(d_k)}
	\end{align*}
	for $r \geq 0, 1 < d_1|\dots|d_n$.
\end{theorem}
We proved this using the smith normal form where we set
\begin{align*}
	\phi: (a_{1}, \ldots, a_{l}) \mapsto \sum_{i=1}^{l}a_ig_i\\
	G = <g_1, \ldots, g_l> \iso \faktor{\Z^{l}}{\Ker \phi}
\end{align*}
And we could see that $\Ker \phi = A \Z^{n}$ setting $r$ to be the number of zeros in the diagonal of $D$.


Alternatively, we could write
\begin{align*}
	G &\iso \Z^{r} \times G_{\text{tors}}\\
	G_{\text{tors}} &\iso \prod_{p}G_p, \text{ for $G_p$ Sylow $p$-subgroups}\\
	G_p &\iso \faktor{\Z}{(p^{l_1})} \times \dots \times \faktor{\Z}{(p^{l_k})}
\end{align*}
We can explain this as follows. For the prime factorisation $d_j = p_1^{a_1}\dots p_s^{a_s}$ we can use the chinese remainder theorem to show
\begin{align*}
	\faktor{\Z}{(d_j)} \iso \faktor{\Z}{(p_1^{a_1})} \times \dots \times \faktor{\Z}{(p_s^{a_s})}
\end{align*}
Then because $G_{\text{tors}}$ is a finitely generated Torsion module, it is clear that there exists a $D \geq 1$ with $D \cdot g = 0$ for all $g \in G_{\text{tors}}$, because we can take the product of the finitely many $d$ that eliminate $g$.

If $D = a b$ for $a,b \in \N$ coprime, then we have the decomposition
\begin{align*}
	G_{\text{tors}} &\iso G_a \times G_b\\
	G_a &= \left\{g \in G_{\text{tors}} \big\vert a \cdot g = 0\right\}\\
	G_b &= \left\{g \in G_{\text{tors}}| b \cdot g = 0\right\}
\end{align*}
because we can look at the mapping
\begin{align*}
	\Phi: G_a \times G_b \to G_{\text{tors}} \quad (g_1,g_2) \mapsto g_1 + g_2
\end{align*}
Injectivity is equivalent to $G_a \cap G_b$: Because $a,b$ are coprime, there exist $c,f \in \Z$ such that $ca + fb = 1$.
Then for $g \in G_a \cap G_b$ we have
\begin{align*}
	1 \cdot g = e a g + fb g = 0 \implies G_a \cap G_b = 0
\end{align*}
For surjectivity let $g \in G_{\text{tors}}$. Then for $g = eag + fbg$, we just need to show that $eag \in G_b$ and $fbg \in G_a$.
This is true because
\begin{align*}
	b (eag) = eabg = eDg = 0
\end{align*}
and analogously for $fbg \in G_a$.

By iterating this decomposition we get Modules $H_p$ with the property that for some power of a prime we have $p^{n} h = 0 q$ for all $h \in H$.

Using the first part of the theorem we get
\begin{align*}
	H \iso \faktor{\Z}{(e_1)} \times \dots \times \faktor{\Z}{(e_k)}	
\end{align*}
Because $p^{n}h = 0$ for all $h \in H$, we must have $e_k | p^{n} \implies e_j = p^{l_j}$ because \begin{align*}
	p^{n}\left(
		0 + (e_1), \ldots, 1 + (e_k)
	\right) 
	=
	\left(
		0 + (e_1), \ldots, p^{n} + (e_k)
	\right) = 0
\end{align*}
if and only if $e_k |p^{n}$





