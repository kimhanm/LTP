\begin{proposition}[]
				Sei $K$ ein Körper. Dann wird der Quotientenkörper von $K[X]$ als der Körper der rationalen Funktionen $K(X) = \{ \frac{f}{g}: f,g \in K[X], g \neq 0\}$ bezeichnet.\\
Die Elemente sind nicht unbedingt Funktionen, da die Polynome selber nicht Funktionen sind. (Siehe $X^2 + X$ in $\F_2$).\\

Wenn wir die obige Konstruktion iterieren, erhalten wir den Ring der Polynome in mehreren Variablen
\begin{align*}
				R[X_1,X_2, \ldots , X_d] := R[X_1][X_2] \ldots [X_d]
\end{align*}
Falls $R = K$ ein Körper ist, definieren wir auch
\begin{align*}
				K(X_1, X_2, \ldots, X_d) = \{\frac{f}{g} \big\vert f,g \in K[X_1, \ldots, x_d]\}
\end{align*}
\end{proposition}

Bemerkung: Auf $R[X_1, \ldots, X_d]$ gibt es mehrere Grad-Funktionen
\begin{align*}
				\deg_{X_1}, \deg_{X_2}, \ldots, \deg_{X_d}, \quad  \deg_{\text{tot}}(f) = \max \{m_1 + \ldots + m_d \big\vert f_{m_1, \ldots, m_d} \neq 0\}
\end{align*}
Wobei hier $f \in R[X_1, \ldots, X_d]$ folgende Form hat
\begin{align*}
				f = \sum_{m_1, \ldots, m_d} f_{m_1, \ldots, m_d} X_1^{m_1} \dots X_d^{m_d}
\end{align*}


\begin{satz}[]
				Seien $R,S$ zwei kommutative Ringe. Ein Ringhomomorphismus $\Phi: R[X]$ nach $S$ ist eindeutig durch seine Einschränkung $\phi = \Phi|_R$ und durch das Element $x = \Phi(X) \in S$ bestimmt. Des weiteren definiert
\begin{align*}
				\Phi\left(\sum_n a_n X^n\right) = \sum_{n}\phi(a_n)x^n
\end{align*}
einen Ringhomomorphismus falls $\phi: R \to S$ ein Ringhomomorphismus ist und $x \in S$ beliebig ist.
\end{satz}


Beweis: \quad Sei $\Phi: R[X] \to S$ ein Ringhomomorphismus, $\phi = \Phi_R$ und $x = \Phi(X) \in S$. Dann gilt
\begin{align*}
				\Phi \left(\sum_na_nX^n\right) = \sum_n \Phi(a_nX^n) = \sum_n \phi(a_n) \cdot \Phi(x)^n \tag(\ast)
\end{align*}
Sei nun $\phi: R \to S$ ein Ringhomomorphismus und $x \in S$ beliebig.
Wir verwenden  $(\ast)$ um $\Phi$ zu definieren. Es ist klar dass
\begin{itemize}
				\item $\Phi(\bm{1}) = \phi(1_R) x^0 = 1_S$
				\item $\Phi(a+b) = \Phi\left(\sum_{n}(a_n + b_n)X^N)\right) = \sum_n \phi(a_n + b_n)x^n = \ldots =  \Phi(a) + \Phi(b)$
				\item $\Phi(a \cdot b) = \sum_n \phi \left( \sum_{i+j = n} a_ib_j)x^n \right) = \left(\sum_i \phi(a_i)x^i\right) \left(\sum_j \phi(b_j)x^j\right) = \Phi(a) \cdot \Phi(b)$
\end{itemize}



Notation, wir schreiben für zwei kommutative Ringe $R,S$
\begin{align*}
				\Hom_{\text{Ring}}(R,S) = \Hom(R,S) := \{\phi: R \to S \big\vert \phi \text{ ist ein Ringhomomorphismus}\}
\end{align*}
In dieser Notation können wir den obigen Satz in der Form
\begin{align*}
				\Hom(R[X],S) \cong \Hom(R,S) \times S
\end{align*}
beziehungweise für den Fall in mehreren Variablen:
\begin{align*}
				\Hom(R[X_1, \ldots, X_n],S) \cong \Hom(R,S) \times S^n
\end{align*}

Falls wir $R = S$ und $\phi = \id$ setzen, so erhalten wir für jedes $a \in R$ die entsprechende Auswertungsabbildung
\begin{align*}
				\ev_a : f \mapsto f(a) = \sum_n f_n a^n
\end{align*}

Wenn wir $a \in R$ variieren, ergibt sich auch eine Abbildung
\begin{align*}
				\Psi: f \in R[X] \to \left(f(\cdot): R \to R, a \mapsto f(a)\right)  \in  R^R
\end{align*}

Wir statten $R^R$ mit den punktweisen Operationen aus, womit $\Phi: R[X] \to \R^R$ ein Ringhomomorphismus ist.\\
Falls $\abs{R} < \infty$ und $\R \neq \{0\}$, so kann $\Psi$ nicht injektiv sein.


Beispiel: \quad Sei $R = \Z$ und $S = \Z/\Z_m[X]$ für ein $m \geq 1$. Dann gibt es einen Ringhomomorphismus
\begin{align*}
				f \in \Z[X] \mapsto \overline{f} = \sum_n(f_n \mod m)X^n \in \Z/\Z_m[X]
\end{align*}


Beispiel: \quad $R = \C, S = \C[X]$, $\phi(a) = \overline{a}$. Dann ist
\begin{align*}
				f \in \C[X] \mapsto \sum_n \overline{f_n}X^n \in \C[X]
\end{align*}



\section{Ideale und Faktorringe}
\begin{definition}[Ideal]
Sei $R$ ein kommutativer Ring. Ein Ideal in $R$ ist eine Teilmenge $I \subseteq R$, so dass
\begin{enumerate}
\item $0 \in I$
\item $a,b \in I \implies a + b \in I$
\item $a \in I, x \in R \implies xa \in I$
\end{enumerate}
\end{definition}


Beispiel: \quad Seien $R, S$ zwei kommutative Ringe und $\phi: R \to S$ ein Ringhomomorphismus. Dann ist
\begin{align*}
				\Ker \phi = \{a \in \R \big\vert \phi(a) = 0\}
\end{align*}
ein Ideal.

\begin{satz}[Faktorring]
Sei $R$ ein kommutativer Ring und $I \subseteq R$ ein Ideal.
\begin{enumerate}
\item Die Relation $a \tilde b \Leftrightarrow a - b \in I$ ist eine Äquivalenzrelation auf $R$. Wir schreiben auch $a \equiv b \mod I$ und wir schreiben $R/I$ (``$R$ modulo $I$'') für den \textbf{Faktorring} der Äquivalenzklassen
\item DIe Addition, Multiplikation und das Negative induzieren wohldefinierete Abbildungen $R/I \times R/I \to R/I$ bzw. $R/I \to R/I$.
\item Mit diesen Abbildungen, $0_{R/I} = [0]_{\sim}, 1_{R/I} = [1]_{\sim}$ ist $R/I$ ein Ring und die kanonische projektion
\begin{align*}
				\rho: R \to R_i, \quad a \in R \mapsto [a]_{\sim} = a + I
\end{align*}
ist ein surjektiver Ringhomomorphismus.
\end{enumerate}
\end{satz}



Beweis:
\begin{itemize}
				\item 	$a \sim a$, denn $a-a = 0 \in I$. Weiter ist da falls $a-b \in I$ ist auch $b-a = (-1)(a-b) \in I$. Transitivität folgt, da $a-c = a-b + (b-c) \in I$. Also ist $\sim$ eine Äquivalenzrelation.
\item Angenommen $a \sim a', b \sim b'$. Dann gilt
				\begin{align*}
								ab - a'b' = ab - a'b + a'b - a'b' = b(a-a') + a'(b - b') \in I
				\end{align*}
				Der Beweis für die Addition ist trivial und für das additive Inverse folgt aus Multiplikation mit $-1$.
\item Da die Rinaxiome nur Gleichungen enthalten (i.e. $0 \neq 1$ ist kein Axiom) sind die Ringaxiome in $R/I$ direkte Konsequenzen der Ringaxiome in $R$. Des weiteren ist die Projektion $p: R \to R/I, a \mapsto [a]_{\sim}$ ein Ringhomomorphismus.
\end{itemize}



Beispiele:
\begin{itemize}
\item $I = \Z_m \subseteq \Z$ ist ein Ideal.
\item $I = R$ und $I = \{0\}$ sind Ideale in jedem beliebigen kommutativen Ring.
\end{itemize}



\begin{lemma}[]
Sei $I \subseteq R$ ein Ideal in einem kommutativen Ring. Dann gilt
\begin{align*}
I = R \Leftrightarrow 1 \in I \Leftrightarrow I \cap R^\times \neq \es
\end{align*}
\end{lemma}

Beweis: \quad Angenommen $u = v^{-1} \in I$ und $v \in R, a \in R$. Dann gilt $a = avu \in I$. Da $a \in R$ beliebig war, folgt $I = R$.\\


The Lemma answers the following question. What Ideals are there in a field $K$? Just $\{0\}$ and $K$ itself.


\begin{definition}[]
Let $R$ be a commutative Ring and let $a_0, \ldots, a_n \in R$. Then
\begin{align*}
				I = (a_1, \ldots, a_n) = \{x_1 a_1 + x_2 a_1 +, \ldots, x_n a_n \big\vert x_1, \ldots, x_n \in R\} 
\end{align*}
is called  the Ideal \textbf{generated}  by $a_1, \ldots, a-N$. For $a \in R$ we call $I(a) = Ra$ the \textbf{principal ideal} of $a$.
\end{definition}


\begin{lemma}[]
Let $R$ be a commutative Ring.
\begin{enumerate}
				\item $(a) \subseteq (b) \Leftrightarrow b | a$
				\item If $R$ is an integral domain, then $(a) = (b) \Leftrightarrow \exists u \in \R^\times$ such that $b = ua$
\end{enumerate}

Proof: If $(a) \subseteq (b)$, then since $a = 1 \cdot a$, we have $a \in Rb$ which means $b|a$.
\end{lemma}
