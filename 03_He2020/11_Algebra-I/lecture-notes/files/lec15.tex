\section{Group Theory}

\subsection{Definitions and Examples}

\begin{definition}[Group]
	A \textbf{Group} is a set $G$ with an operation $\circ: G \times G \to G$ that satisfies the following axioms.
	\begin{enumerate}
		\item[G1]	Associativity: $\forall a,b,c \in G: \quad (a \circ b) \circ b) = a \circ (b \circ c)$
		\item[G2]	Identity Element: $\exists e \in G: \forall a \in  G: e \circ a = a \circ e = a$
		\item[G3] Inverse Element: $\forall a \in G \exists x \in G: a \circ x = x \circ a = e$
	\end{enumerate}
\end{definition}
Note: The inverse element is uniquely determined for every $a \in G$, since for two inverses $x,y$ of $a$ we have by associativity
\begin{align*}	
	y = (x \circ a) \circ y = x \circ (a \circ y) = x
\end{align*}
We write $a^{-1}$ for the inverse element. 
The identity element is also uniquely determined through left identity $\forall a \in G: e \circ a = a$ or idempotency: $e \circ e = e$, since 
\begin{align*}
	\tilde{e} = \tilde{e} \circ e = e, \quad \text{and} \quad \tilde{e}^{-1} \circ \tilde{e} \circ \tilde{e} = \tilde{e}^{-1} \circ \tilde{e} = e
\end{align*}

\begin{definition}[abelian group]
	Let $G$ be a group. We say that two elements $a,b \in G$ \textbf{commute}, if $ab = ba$. If every pair of elements in $G$ commute, we say that $G$ is \textbf{abelian}, or \textbf{commutative}.
\end{definition}
Note: For abelian Groups, we often use additive Notation: $+: G \to G$ and write $0$ for the identity.


\begin{definition}[powers]
	For a group $G$ and $a \in G$ and for $k \in \Z$  we define the \textbf{powers} of $a$ as
	\begin{align*}
		a^{k} := \left\{\begin{array}{ll}
				\underbrace{a \circ \ldots \circ a}_{k-\text{times}} & \text{for }k > 0\\
				e & \text{for } k = 0\\
				\underbrace{a^{-1} \circ \ldots \circ a^{-1}}_{\abs{k}-\text{times}} & \text{for } k < 0
		\end{array} \right.
	\end{align*}
\end{definition}
Note: We have the following properties for all $a \in G$:
\begin{itemize}
\item $\forall k,l \in \Z: a^{k}a^{l} = a^{k+l}$
\item $\forall k,l \in \Z: \left(a^{k}\right)^{l} = a^{kl}$
\item If $a,b \in G$ commute and $k,l \in \Z$, then $a^{k}$ and $b^{l}$ commute and $\left(ab\right)^{k} = a^{k}b^{k}$
\end{itemize}
The proof is trivial with induction on $k$.\\

Since groups have a inverse operation, we can reduce equations. So for all $a,b,c \in G$ we have
\begin{align*}
	ac = bc \iff a = b \iff ca = cb
\end{align*}
Also the equation $ax = b$ always has a unique solution, $x = a^{-1}b$.\\

Now let's look at how Group s relate to another.
\begin{definition}[Homeomorphism]
	Let $G_1, G_2$ be two groups. A \textbf{homeomorphism} from $G_1$ to $G_2$ is a map $\phi: G_1 \to G_2$ such that
	\begin{align*}
		\phi(ab) = \phi(a) \phi(b), \forall a,b \in G
	\end{align*}
The \textbf{Kernel} and \textbf{Image} of the map are the sets
\begin{align*}
	\Ker \phi = \phi^{-1}\{e_{G_2}\} = \left\{a \in G_1 \big\vert \phi(a) = e_{G_2}\right\}\\
	\Image \phi = \phi(G) = \left\{b \in G_2 \big\vert \exists a \in G_1: \phi(a) = b\right\}
\end{align*}
\end{definition}

We can also talk about (smaller) groups inside groups.
\begin{definition}[subgroup]
	Let $G$ be a group. A \textbf{subgroup} of $G$ is a non-empty subset $H \subseteq G$ such that for any $a,b \in H$ the element $ab^{-1}$ is also in $H$. We write $H < G$.
\end{definition}
The following are equivalent characterisations of subgroups
\begin{enumerate}
\item $H < G$
\item $e \in H$ and $a,b \in H \implies ab \in H$, $a \in H \implies a^{-1} \in H$.
\item $H$ is a group and the inclusion mapping $\iota: H \to G, h \mapsto h$ is a Homeomorphism.
\end{enumerate}
If $\abs{H} < \infty$, then it suffices to show that $H$ is non-empty and $a,b \in H \implies ab \in H$.\\


If $\phi: G_1 \to G_2$ is a homemorphism, then both $\Ker \phi$ and $\Image \phi$ are subgroups of the respective groups.

Examples:
\begin{enumerate}
\item The group of units in a Ring $R^{\times}$ is a subgroup.
\item Let $M$ be a non-empty set. Then the set of bijective maps is a group with respect to composition of maps.
	\begin{align*}
		\text{Bij}(M) := \{f: M \to M \big\vert f \text{ bijective}\}
	\end{align*}
	For $M = \{1, \ldots, n\}$ we write $S_n = \text{Bij}(\{1, \ldots, n\}$.
\item More generally, if $M$ is a set with ``some structure'', then the set of structure preserving maps $\Aut(M)$ is a group.
	\begin{align*}
		\Aut(M) := \left\{\phi: M \to M \big\vert \phi \text{ bijective and structure preserving}\right\}
	\end{align*}
	Some examples of Automophisms\\
	\begin{tabular}{ll}
		$M$ with structure & $\Aut(M)$\\
		\textbf{Set} & $\text{Bij}(V)$\\
		$K$-Vector spaces & $\GL(V)$\\
		$K \supseteq \Q$ & $\text{Gal}(K:\Q) = \left\{\phi: K \to K \big\vert \phi \text{ $\Q$-linear, bijective and }\phi(ab) = \phi(a) \phi(b)\right\}$\\
		\textbf{Grp} & $\Aut(G) = \left\{\phi: G \to G \big\vert \phi \text{ Isomorphism}\right\}$\\
		Affine real plane & $\GL_2(\R) \ltimes \R^2$\\
		Euclidean real plane & $\mathcal{O}_2(\R) \ltimes \R^2$\\
		Spherical Geometry $S^2$ & $\mathcal{O}_3(\R)$\\
		Hyperbolic plane & $\mathcal{SO}_{2,1}, P \GL_2(\R)$\\
		\textbf{Top} & $\text{Homeo}(X) := \left\{\phi: X \to X \big\vert \phi \text{ bijecive, continous, continous inverse}\right\}$\\
		$\textbf{Man}^\infty$ & $\text{Diffeo}(M) = \left\{\phi: M \to M \big\vert \phi \text{ smooth bijective, smooth inverse}\right\}$\\
		Regular polygon in $\R^2$ & Diedral Group $D_{2n}$\\
		Rubik's Cube & Turns on the Rubiks cube
	\end{tabular}
\item For a field $K$ the set of invertible $n \times n$ matrices $\GL_n(K)$ is a group. Furthermore, the determinant $\det: \GL_n(K) \to K^{\times}$ is a group homomorphism and $\Ker \det = \text{SL}_n(K)$
\item $(0,\infty) < \R^\times$ is a subgroup. And $\exp: \R \to \R^{\times}$ is a homeomorphism.
\item If $G_1, G_2$ is a group, then $G_1 \times G_2$ is a group under component wise operation.
\end{enumerate}

\begin{lemma}[]
 Let $G$ be a group and $a \in G$, then the map $\phi: \Z \to G, k \mapsto a^{k}$ a group homomorphism.
\end{lemma}
Proof: The power rule already shows that $\phi$ is a homoeomorphism. And since $\phi(nk) = a^{nk} = \left(a^{k}\right)^{n} = e \implies nk \in \Ker \phi$ we know that $\Ker \phi$ is an ideal. Since $\Z$ is a PID, either
\begin{align*}
	\Ker \phi = (0) \quad \text{or} \quad \Ker \phi = (n_0), n_0 > 0
\end{align*}
So if the kernel is zero, it is injective.

The mapping $\phi$ allows us to define how the cycles of $a$ behave in the group.

\begin{definition}[order]
	Let $G$ be a group, and for $a \in G$, write $\phi_a: \Z \to G, k \mapsto a^k$. If $\phi_a$ is injective, we say $a$ has \textbf{order} infinty and if $\Ker \phi_a = (n_0)$ for some $n_0 > 0$ then a is of order $n_0$
\end{definition}
