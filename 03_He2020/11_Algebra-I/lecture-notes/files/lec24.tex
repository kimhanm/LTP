\begin{ntheorem}[Classification theorem (second part)]
	Let $R$ be a PID and $M_{\text{tors}}$ a finitely generated torsion module. Then there exist $d_{1}|d_2|\ldots| d_{n} \in R \setminus \{0\}$ such that
	\begin{align*}
		M_{\text{tors}} \iso \faktor{R}{(d_1)} \times \ldots \times \faktor{R}{(d_n)}
	\end{align*}
	alternatively, we can write
	\begin{align*}
		M_{\text{tors}} \iso \prod_{j = 1}^{k} M_{\text{tors}}^{(p_i)}
	\end{align*}
	where $p_1, \ldots, p_k \in R$ are non-conjugate primes in $R$ and
	\begin{align*}
		M_{\text{tors}}^{(p_i)} &:= \left\{m \in M_{\text{tors}} \big\vert \exists l \in \N \text{ with }p_i^{l}m = 0\right\}\\
														&\iso \faktor{R}{\left(
															p_j^{n_j,1}
													\right)} \times \ldots \times \faktor{R}{\left(
														p_j^{n_j,k}
													\right)}
	\end{align*}
\end{ntheorem}

\begin{ntheorem}[Smith Canonical Form]
	Let $R$ be a PID, $k,n \geq 1 \in \N$ and $A \in \Mat_{kl}(R)$. Then there exist $g \in \GL_k(R)$ and $h \in \GL_l(R)$ such that
	\begin{align*}
		gAh^{-1} = \begin{pmatrix}
		d_1 &  &  & \\
		 & \ddots &  & \\
		 &  & d_n & \\
		 &  &  & 0_{\ddots}
		\end{pmatrix}, \quad \text{for} \quad d_1|d_2|\ldots|d_n \in R \setminus \{0\}
	\end{align*}
\end{ntheorem}
We will prove this theorem only for euclidean rings.
In the Gaussian elimination algorithm, row operations are left-multiplication with elements of $\GL_k(R)$ and column operations coorrespond with right-multiplication with elements of $\GL_l(R)$

Proof for euclidean rings: We will use induction on $\max(k,l)$.

If $\max(k,l) = 1$, either $A = (0)$ or $A = (d_1)$ for some $d_1 \in R \setminus \{0\}$.

Now let $\max(k,l) \geq 2$. If $A = 0$, we're done. Let
\begin{align*}
	N := \min_{A_ij \neq 0} \phi(a_{ij}) \in \N \quad \text{for} \quad  \phi \text{ the norm on $R$}
\end{align*}
By swapping rows and columns we can assume that 
\begin{align*}
	d_1 = A_{11} \neq 0 \quad \text{and} \quad \phi(d_1) = N
\end{align*}
then use division with rest to get
\begin{align*}
	A_{1j} = a_j d_1 + r_1 \quad \text{for} \quad j = 2, \ldots l \quad \text{and} \quad r_i = 0 \quad \text{or} \quad \phi(r_i) < \phi(d_1)
\end{align*}
and subtract $a_j$ time hte first row of the $j$-th row for $j = 2, \ldots, l$ and we obtain the matrix
\begin{align*}
	A' = \begin{pmatrix}
	d_1 & r_1 & \ldots & r_l\\
	A_{21} &  &  & \\
	\vdots &  &  & \\
	A_{k1} &  &  & 
	\end{pmatrix}
\end{align*}
If $r_i \neq 0$, then $N' = \min_{A'_{ij}} \phi(A_{ij}') < N$ and w can use Induction to let $A'$ have Smith normalform. Therefore without loss of generality, $r_2 = r_3 = \ldots = r_l = 0$

Analogously we can repeat this argument for the first column to get a matrix
\begin{align*}
	A'' = \begin{pmatrix}
	d_1 & 0 & \ldots & 0\\
	0 &  &  & \\
	\ldots &  &  & \\
	0 &  &  & 
	\end{pmatrix}
\end{align*}
If $\max(k,l) = 1$, then $A''$ is already in smith normalform. Else the submatrix of $A''$ has dimension $k-1,l-1$. So the maximum decreases and induction step gives us a matrix of the form
\begin{align*}
	A''' = \begin{pmatrix}
	d_1 &  &  & \\
	 & \ddots &  & \\
	 &  & d_n & \\
	 &  &  & 0_{\ddots}
	\end{pmatrix}
\end{align*}
now we have to show that $d_1|d_2|\ldots|d_n$. If $d_1|d_2$ then smith normalform is reached.

If $d_1 \not|d_2$, then we can add the second row to the first and make division with rest and get a matrix of smaller norm to get
\begin{align*}
	A''' \mapsto \begin{pmatrix}
	d_1 & & 0\\
	0 & d_2 & \\
		&  & {d_3}_{\ddots}
	\end{pmatrix} \mapsto \begin{pmatrix}
	d_1 & r & 0\\
	0 & d_2 & \\
	 &  & 0
	\end{pmatrix} \mapsto A''' \text{ in Smith canonical form.}
\end{align*}


Now we can prove the second classification theorem.

Let $M$ be a finitely generated module and $R$ a euclidean ring. If we assume $x_1, \ldots, x_k \in M$ generate $M$, then
\begin{align*}
	\Phi: a \in R^{k} \mapsto \sum_{i = 1}^{k}a_ix_i \in M
\end{align*}
is surjective. Then $N = \Ker \Phi \subseteq R^{k}$ is a submodule and also a free module itself from a previous proposition. Write $N = <r1,\ldots,r_l>$, so $M \iso \faktor{R^{k}}{N}$.

We define the Matrix
\begin{align*}
	A = (r_1,\ldots,r_k) \in \Mat_{kl}(R)
\end{align*}
and put it in Smith canonical form, so there exist $g \in \GL_k(R), h \in \GL_l(r)$ such that
\begin{align*}
	B := gAh^{-1} = \begin{pmatrix}
	d_1 &  & 0 & \\
	 & \ddots &  & \\
	0 &  & {d_n}_{\ddots} & \\
	 &  &  & 0
	\end{pmatrix} \quad \text{with} \quad d_1|d_2|\ldots|d_n \in R \setminus \{0\}
\end{align*}
	Since $A$ can be identified with a linear map $R^{l} \to R^{k}$, we get that
	\begin{align*}
		N = \Image A = A(R^{l}), \quad \text{and} \quad \Image B = B(R^{k}) = gAh^{-1}(R^{l}) = g \Image A = g N
	\end{align*}
	then let's take a look at what $g$ does to $R^{k}$:
	\begin{align*}
		R^{k} \stackrel{g}{\to} R^{k}, N = \Image A \mapsto gN = \Image B
	\end{align*}
	which induces an isomorphism (First isomorphism theorem)
	\begin{align*}
		M \iso \faktor{R^{m}}{N} \to \faktor{R^{k}}{g_N} = \faktor{R^{k}}{\Image B}
	\end{align*}
	but since $B$ is diagonal, we know that
	\begin{align*}
		\Image B = (d_1) \times (d_2) \times \ldots \times (d_n) \times \{0\}^{k-n}
	\end{align*}
	so we can extend the isomorphism
	\begin{align*}
		M \iso \faktor{R^{k}}{\Image B} \iso \faktor{R}{(d_1)} \times \faktor{R}{(d_2)} \times \ldots \times \faktor{R}{(d_n)} \times R^{k-n}
	\end{align*}
	which is a decomposition in to a torsion and a free part.


	\subsection{Finitely generated abelian gruops}

	\begin{theorem}[]
		Let $G$ be a finitely generated abelian group. Then
		\begin{align*}
			G \iso \faktor{Z}{(d_1)} \times \ldots \times \faktor{\Z}{(d_n)} \times \Z^{k}
		\end{align*}
		where $1 \leq d_1|d_1|\ldots|d_n \neq 0$  and $k \geq 0$. Alternatively we can write
		\begin{align*}
		G \iso \prod_{p \text{prime}} G_p \times \Z^{k} \quad \text{for} \quad G_p \iso \faktor{\Z}{(p^{k_p,1}} \times \ldots \times \faktor{\Z}{(p^{k_p,n}}
		\end{align*}
	\end{theorem}
	This follows directly from the classification theorem because abelian groups are $\Z$-modules.

\subsection{Jordan canonical form}

\begin{theorem}[]
	Let $V$ be a finite-dimensional $\C$-vector space and $\phi: V \to V$ linear. Then there exists a basis of $V$ such that $\phi$ has a matrix representation in jordan canonical form.
\end{theorem}
Proof: Since $V$ is finite-dimensional and $\C[X]$ is infinitely dimension, $V$ must be a torsion-module over $\C[X]$. Further more, $V$ is a finitely generated $\C[X]$-module. Using the classification theorem for modules, we get that
\begin{align*}
	V \iso \prod_{(\lambda,k)} \faktor{\C[X]}{((X-\lambda)^{k})}
\end{align*}
where we used the fundamental theorem of Algebra, to show that the only irreducible elements in $\C[X]$ are of the form $(X - \lambda)$.

We then can describe multiplication with $X$ as aplication of $\phi$ to subspaces on $V$ on $M = \faktor{\C[X]}{((X - \lambda)^k)}$. Which has the basis
\begin{align*}
	1, (X - \lambda), (X - \lambda)^{2}, \ldots, (X - \lambda)^{k_1}
\end{align*}
over $\C$ and so multiplication with $X$ has the following matrix representation with respect to this basis
\begin{align*}
	\begin{pmatrix}
	\lambda & 0 &  &  & 0\\
	1 & \lambda & 0 &  & \\
	0 & 1 & \ddots &  & \\
	\vdots &  &  &  & \\
	0 &  & 1 &  \lambda& 
	\end{pmatrix}
\end{align*}


