
\textbf{Notation:} Let $S \subseteq R$ be a subring Let $a_{1}, \ldots, a_{n} \in R$. We define 
\begin{align*}
	S[a_1, \ldots, a_n] &= \bigcap_{\underset{T \supseteq S}{T \subseteq R}} T\\
											&= \ev_{a_{1}, \ldots, a_{n}} \left(S[X_1, \ldots, X_n]\right)\\
											&:= \left\{ \sum_{k_1, \ldots, k_n \in M} c_{k_{1}, \ldots, k_{n}}a_1^{k_1} \dots a_n^{k_n} \big\vert \abs{M} < \infty, M \subseteq \N^n, c_{k_{1}, \ldots, k_{n}} \in S\right\}
\end{align*}

Proof: We know from the exercises that $S[a_{1}, \ldots, a_{n}]$ is a subring containing, by definition, $S$ and $a_{1}, \ldots, a_{n}$. Further, we know that $\ev_{a_{1}, \ldots, a_{n}}[X_{1}, \ldots, X_{n}]$ is also a subring, since $\ev$ is a Ringhomormorphism. Which shows
\begin{align*}
	S[a_{1}, \ldots, a_{n}] \subseteq \ev_{a_{1}, \ldots, a_{n}} \left(S[X_{1}, \ldots, X_{n}]\right)
\end{align*}
The other inclusion folllows becuase $S[a_{1}, \ldots, a_{n}]$ is a subring. And again, we know that $S$ and $a_{1}, \ldots, a_{n}$ are included which implies
\begin{align*}
	\sum_{(k_{1}, \ldots, k_{n}) \in M}\underbrace{c_{k_{1}, \ldots, k_{n}}}_{\in S}a_1^{k_1} \dots a_n^{k_n} \subseteq S[a_{1}, \ldots, a_{n}]
\end{align*}
This underlines the idea that we can define the span in a vector space as the set containing all linear combinations, or as the vector space equivalent of an ideal generated by the vectors
For example we have
\begin{align*}
	\Z[\frac{1}{2}] &= \{\frac{a}{2^n}: a \in \Z, n \in \Z\}\subseteq \Q\\
	\Z[i] &= \{a + ib: a, b \in \Z\} \subseteq \C\\
	\Z[\sqrt{2}] &= \{a + \sqrt{2}b: a, b \in \Z\}\subseteq \R\\
	\Q[\sqrt{2}] &= \{a + \sqrt{2}b: a, b \in \Q\}\subseteq \R
\end{align*}
The last one is indeed a field as we have
\begin{align*}
	\frac{a + \sqrt{2}}{c + \sqrt{2}} =	\frac{a + \sqrt{2}}{c + \sqrt{2}} \frac{c - \sqrt{2}d}{c - \sqrt{2}d} = \frac{1c - 2bd + \sqrt{2}(ad - bc)}{c^2 - 2d^2}
\end{align*}


\subsection{Matrices}
Let $R$ be a commutative Ring, $m,n \in \N_{>0}$. We define the set $\Mat_{mn}(R)$ as the set of all $m \times n$ matrices
\begin{align*}
	\begin{pmatrix}
	a_{11} & \dots  & a_{1n}\\
	\vdots &  & \vdots\\
	a_{m1} & \dots  & a_{mn}
	\end{pmatrix}
\end{align*}
with coefficients $a_{11}, \ldots, a_{mn} \in R$.\\
For $m = n$ we define Addition and Multiplication as usual, which defines a ringed structure on $\Mat_{nn}(R)$ with identity matrix
\begin{align*}
	I_n = (\delta_{ij})_{i,j} 
\end{align*}
It should be noted that for $n > 1$ the ring is not comutative in general. We denote its unit by
\begin{align*}
	\GL_n(R) := \Mat_{nn}(R)^{\times}
\end{align*}

\begin{nproposition}[Meta-proposition]
	Every calculation-rule for matrices over $R$ that only make use of $+,-,\cdot,0,1$ also apply for any commutative Ring $R$.\\
	They are
	\begin{itemize}
		\item $\det(AB) = \det(A) \cdot \det(B)$
		\item $A \tilde{A} = \tilde{A}A = \det(A) I_m$, where $\tilde{A}$ is the complementary Matrix
			\begin{align*}
				\tilde{A} = \left((-1)^{i+j}\det(A_{ji})\right)_{i,j}
			\end{align*}
	\end{itemize}
\end{nproposition}

\begin{lemma}[]
	If a polynomial $f \in \R[X_{1}, \ldots, X_{n}$ vanishes on $\R^n$, then $f = 0$
\end{lemma}
Proof: Let $f = \sum_{k_{1}, \ldots, k_{n}}c_{k_{1}, \ldots, k_{n}}X_1^{k_1} \dots X_n^{k_n}$ be a polynomial, for which the corresponding polyomial function
\begin{align*}
	f: \R^n \to \R, \quad (a_{1}, \ldots, a_{n}) \mapsto f(a_{1}, \ldots, a_{n})
\end{align*}
vanishes. Then this must also be true for any partial derivative of $f$. Let $(l_{1}, \ldots, l_{n}) \in \N^n$. Then we have
\begin{align*}
	0 &= \del_{X_1}^{l_1} \dots \del_{X_n}^{l_n}f(0) \\
		&= \sum_{k_{1}, \ldots, k_{n}}c_{k_{1}, \ldots, k_{n}}k_1 \cdot(k_1 -1) \dots (k_1 - l_1 + 1)X_1^{l_1} \dots k_n(k_n-1) \dots (k_n - l_n + 1)X^{k_n - l_n}\\
		&= c_{k_{1}, \ldots, k_{n}}l_n! \dots l_n!
\end{align*}
So we can eliminate every $c_{k_{1}, \ldots, k_{n}}$ which shows that $f = 0$.This lemma also holds for any field $K$ with $\abs{K} = \infty$.\\

To prove the proposition, we first note that
\begin{itemize}
	\item 	Every entry in $A(BC) - (AB)C$ a polynomial with integer coefficients in the variables $a_{11}, \ldots, a_{nn}, b_{11} \ldots b_{nn}, c_{11}, \ldots, c_{nn}$.
	\item $\det(AB) - \det(A) \det(B)$ is again a polynomial with integer coefficients in the variables $a_{ij},b_{kl}$.
	\item 	Every entry in $A \tilde{A} - \det(A) I_n$ is a polynomial with integer coefficients in the variables $a_{ij}$.\\
\end{itemize}
For the case $R = \R$ we know that these polynomials evaluated at any point result in zero. Using the previous Lemma, the polynomials also must be zero, so using the Ringhomomorphism $\Z \to R$, on the coefficients, we know that all these equations hold for any matrices over $R$ for any ring $R$.



\section{Faktorisierungen in Ringen}
We want to factorize Rings with unique prime-factorisation. In the following, let $R$ denote an integral domain.\\
Recall the definition of divisibility: $a | b \iff \exists c$ such that $ac = b$ and of the unit: $a \in R^{\times} \iff a|1$.\\

\begin{definition}[irreducible/prime]
	We say an element $p \in R \setminus \{0\}$ is \textbf{irreducible}, if $p \notin R^\times$ and for all $a,b \in R$ we have
	\begin{align*}
		p = ab \implies a \in R^{\times} \quad \text{or} \quad b \in R^{\times}
	\end{align*}
	We say $p \in R \setminus \{0\}$ is \textbf{prime}, if the ideal $(p)$ is a prime ideal. In other words: if $p \notin R^{\times}$ and for all $a,b \in R$ we habe $p|ab \implies p|a$ or $p|b$.
\end{definition}
In a general Ring, the two definitions are not equivalent but we have the following implication
\begin{lemma}[]
	Let $R$ be an integral domain. Then every $p \in R$ prime is also irreducible.
\end{lemma}
Proof: Let $p \in R \setminus\{0\}$ and let $p = ab$ for some $a,b \in R$. Since then also $p | ab$ and because $p$ is prime we can assume without loss of generality, that $p |a$. Then $a = p \cdot c$ for some $c \in R$ and because $R$ is an integral domain, we can show that
\begin{align*}
	p = a b = p c b \implies 1 = cb, \implies b \in R^{\times}
\end{align*}


\subsection{Euclidean Rings}

\begin{definition}[]
	An integral domain $R$ is called a \textbf{euclidean ring}, if there exists a function  $N: \R \setminus \{0\} \to \N$ such that
	the following holds:
	\begin{itemize}
		\item \textbf{Degree inequality:} $N(f) \leq N(fg)$, for all $f,g \in \R \setminus \{0\}$..
		\item \textbf{Division with rest:}	For $f,g \in R$ with $f \neq 0$ there exist $q,r \in R$ such that $g = qf + r$ where $r = 0$ or $N(r) < N(f)$. We call $q$ the \textbf{quotient}  and $r$ the \textbf{rest} of the division.
	\end{itemize}
\end{definition}
Examples:
\begin{itemize}
\item Any field $K$ with $N(f) = 0$ is a euclidean ring. 
\item $R = \Z$ with $N(n) = \abs{n}$ is a euclidean ring.
\item For a field $K$, the Ring $R = K[X]$ and $N(f) = \deg f$ is a euclidean ring.
\item $R = \Z[i]$ with $N(a + ib) = \abs{a + ib}^2 = a^2 + b^2$ is a euclidean ring.
\item $R = \Z[\sqrt{2}]$ with $N(a + \sqrt{2}b) = \abs{a^2 + 2b^2}$ is a euclidean ring.
\end{itemize}
Algebraic number theory works with these objects. Here we prove the division with rest for $R = K[X]$.\\
Let $f \neq 0, g \in R$. If $\deg g < \deg f$ chose $q = 0$ and $r = g$ and we're done. We use induction on the degree of $g$. Let $m \in \N$ be the degree of $g$ and $\deg f = n \leq m$. We define
\begin{align*}
	\tilde{g} = g - \frac{g_m}{f_n}X^{m-n}f
\end{align*}
Since $\tilde{g}$ has $\deg \tilde{g} \leq < \deg g$ there exist $\tilde{q}, \tilde{r}$ such that
\begin{align*}
	\tilde{g} = f \tilde{q} + \tilde{r} \implies g = f(\frac{g_m}{f_n}X^{m-1} + \tilde{q}) + \tilde{r}
\end{align*}
