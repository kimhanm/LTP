\subsection{Conjugation}

\begin{lemma}[]
	Let $G$ be a Group.
	\begin{enumerate}
	\item For every $g \in G$, the mapping
		\begin{align*}
			\gamma_g: G \to G, \quad x \mapsto gxg^{-1}
		\end{align*}
		is an Automorphism on $G$. This is called a \textbf{inner Automorphism}.
	\item The mapping $\Phi: g \in G \mapsto \gamma_g \in \Aut(G)$ is a homomorphism. The Kernel of $\Phi$ is called the \textbf{center}
		\begin{align*}
			Z_G = \left\{g \in G \big\vert \forall x \in G: gx = xg\right\}
		\end{align*}
	\end{enumerate}
\end{lemma}
Proof: For $g,x,y \in G$ we have that
\begin{align*}
	\gamma_g(xy) = gxyg^{-1} = gxg^{-1}gyg^{-1} = \gamma_g(x) \gamma_g(y)
\end{align*}
And for $g,h,x \in G$ we have
\begin{align*}
	\gamma_g(\gamma_h(x)) = g \gamma_h(x)g^{-1} = ghx^{-1}g^{-1} = gh x (gh)^{-1} = \gamma_{gh}(x)
\end{align*}
The mapping is bijective, since
\begin{align*}
	(\gamma_g \circ \gamma_{g^{-1}})(x) = \gamma_{gg^{-1}}(x) = \gamma_{e}(x) = \id(x) 
\end{align*}
For b) we already have shown that $\Phi$ is a homeomorphism, since $\Phi(gh) = \gamma_{gh} = \gamma_g \circ \gamma_h = \Phi(g) \Phi(h)$
Furthermore, we have
\begin{align*}
	\Ker \Phi = \left\{g \in G \big\vert \gamma_g = \id\right\} = \left\{g \in G \big\vert gxg^{-1} = x \forall x \in G\right\}
\end{align*}
And since $gxg^{-1} = x$ if and only if $gx = xg$ the center is indeed $Z_G$.

\begin{definition}[]
Let $G$ be a group and $g \in G$. The set of Fixpoints of $\gamma_g$ is the called the \textbf{centralizer} of $g$
\begin{align*}
	\text{Cent}_G(g) = \left\{x \in G \big\vert gx = xg\right\}
\end{align*}
\end{definition}

\begin{definition}[]
	Let $G$ be a group and $x,y \in G$. We say that $xy$ are \textbf{conjugate}, if there exsts a $g \in G$ such that $\gamma_g(x) = gxg^{-1} = y$
\end{definition}
Note: This defines an equivalence relation on $G$, since
\begin{align*}
	\gamma_e(x) = x, \quad \gamma_g(x) = y \implies \gamma_{g^{-1}}(y) = x, \quad \gamma_g(x) = y, \gamma_h(y) = z \implies \gamma_{hg}(x) = z
\end{align*}
 
Examples:
\begin{enumerate}
	\item For $G = \GL_n(\C)$, two matrices $A,B$ are conjugates, if and only if $A$ and $B$ have the same Jordan-Normal Form.
	\item For $G = \mathcal{U}_n(\C)$ the group of unitary matrices: $A^HA = AA^H = I$, every $g \in G$ is diagonalizable. Therefore we can represent the conjugation classes through the elements of $\left(S^1\right)^n$ modulo permutation of coordinates.
\end{enumerate}
If our Group is too large to study on its own, we might want to understand the conjugation classes first.
The group $S_n$ has $n! \simeq \left(\frac{n}{e}\right)^{n} \sqrt{2\pi n}$ elements. But the number of conjucation classes is much smaller, about $\frac{1}{4 \sqrt{3}n} e^{2\pi \sqrt{\frac{n}{6}}}$ (Hardy-Ramanujan 1918).\\

Examples:
\begin{enumerate}
\item The center of $S_n$ for $n \geq 3$ is $\{1\}$
\item The center of $\GL_n(K)$ is the set $\left\{t \cdot I \big\vert t \in K^{\times}\right\}$
\item The center o $\text{SL}_n(K)$ is the set $\left\{t \cdot I \big\vert t\in K^{\times}, t^n = 1\right\}$
\end{enumerate}

\subsection{Subgroups and Generators}
Recall that a subset $H \subseteq G$ is called a \textbf{subgroup} of $G$, if for all $a,b \in H: ab^{-1} \in H$.
%
%Example:
%\begin{enumerate}
%\item For $G = \Z$ every Subgroup of $\Z$ is an ideal and since $\Z$ is a PID, it is of the Form $H = (n_0)$ for some $n_0 \in \Z$.
%\item Let $n \geq 2 \in \N$. We define the \textbf{Dihedral Group} $D_{2n}$ with $\zeta := e^{\frac{2 \pi i}{n}}$ and $\R$-linear Transformations on $\C$
	%\begin{align*}
		%D_{2n} := \underbrace{\left\{z \mapsto \zeta^{k}z \big\vert k \in \{0, 1, \ldots, n-1\}\right\}}_{C_n \simeq \Z/(n)} \cup \left\{\underbrace{z \mapsto \zeta^{k}\overline{z}}_{\sigma}\big\vert k \in \{0, 1, \ldots, n-1\}\right\}
	%\end{align*}
	%Where $\sigma_k(\sigma_k(z)) = \sigma_k(\zeta^k \overline{z}) = \zeta^k \zeta^{-k}z = z$.
	%The Dihedral group consists of symmetries of the regular $n$-gon.
%\end{enumerate}
%
%\begin{lemma}[]
	%Let $G$ be a group and $I$ a set and $H_i < G$ for every $i \in I$. Then the intersection is a subgroup $\bigcap_{i\in I} H_i < G$.
%\end{lemma}


\begin{definition}[]
	Let $G$ be a group and $X \subseteq G$ a subset. The subgroup \textbf{generated} by $X$ is defined as the smallest subgroup that contains $X$
	\begin{align*}
		\left<X\right> = \bigcap_{\underset{X \subseteq H}{H < G}}H
	\end{align*}
We call $X$ the generating set of $\left<X\right>$. If $\left<X\right> = G$, we say that $G$ is \textbf{generated} by $X$. If $\left<X\right> = \left<g\right>$ for some $g \in G$, we call it the \textbf{cyclic subgroup} generated by $g$
\end{definition}
Note: The generated subgroup can be written as the set of elements
\begin{align*}
	\left<X\right> = \left\{x_1^{k_1} \ldots x_n^{k_n}\big\vert n \in \N, x_{1}, \ldots, x_{n}\in X, k_i \in \{\pm 1\}\right\}
\end{align*}


\begin{lemma}[]
	Let $G$ be a group and $a \in G$. Then $\left<a\right> \simeq \Z/(n_0)$ for some $n_0 \in \N$.
\end{lemma}
Proof: we define the Homomorphism $\phi: n \in \Z \mapsto a^{n} \in G$, which has Kernel $\Ker \phi = I = (n_0)$. Then we can write
\begin{align*}
	\Phi: \Z/(n_0) \to \left<a\right>, \quad k + (n_0) \mapsto a^k
\end{align*}
which is well defined and injective, since
\begin{align*}
	k + (n_0) = l + (n_0) \iff k -l \in (n_0)= \Ker \phi \iff a^{k-l} = e \iff a^k = a^{l}
\end{align*}


Example: The symmetric Group $S_n$ is generated by two elements
\begin{align*}
	\tau_{1,2}: 1 \mapsto 2, 2 \mapsto 1, n \mapsto n, \quad \text{and} \quad \sigma: n \mapsto n+1
\end{align*}
Which have order $2$ and $n$ respectively. Notice that
\begin{align*}
	\sigma \tau_{1,2} \sigma^{-1}: 1 \mapsto n \mapsto n \mapsto 1 \quad 2 \mapsto 1 \mapsto 2 \mapsto 3, \quad 3 \mapsto 2 \mapsto 1 \mapsto 2\\
\implies \sigma \tau_{1,2} \sigma^{-1} = \tau_{2,3}
\end{align*}
By iteration, we can obtain $\tau_{k,k+1}$. Using those we obtain
\begin{align*}
\tau_{i,j} = \tau_{i,i+1} \tau_{i+1,i+2} \ldots \tau_{j-1,j} \tau_{j-2,j-1} \ldots \tau_{i,i+1}
\end{align*}
which clearly generates all of $S_n$.\\

Unlike Subvectorspaces, there is no good way to define \textbf{basis} or \textbf{dimension} for a subgroup. But in $S_6$ there exists a subgroup that is generated by $3$ or more elements and no less.
\begin{align*}
	\left<H\right> = \left<\tau_{1,2}, \tau_{3,4}, \tau{5,6}\right> \simeq \F_2^3
\end{align*}

\begin{definition}[]
	Let $G$ be a group. The \textbf{commutator} of $a,b \in G$ is
	\begin{align*}
		[a,b] := aba^{-1}b^{-1}
	\end{align*}
	and the \textbf{commutator group} is 
	\begin{align*}
		[G,G] = \left<[a,b], a,b \in G\right>
	\end{align*}
\end{definition}
The commutator group ``measures'' how un-abelian the group is.
