% ==== 11.12.20 ====
\subsection{Splitting fields}

\begin{theorem}[(Kronecker)]
	Let $K$ be a field, $f \in K[T]$ with $n = \deg f > 0$. Then there exists a field extension $L/K$ such that
	\begin{align*}
		f(T) = a \prod_{i=1}^{n}(T - \alpha_i)
	\end{align*}
	for $a \in K, \alpha_i \in L $
\end{theorem}

Proof: We can assume without loss of generality that $f$ is normed and prove it over induction on $n$. Since $K[T]$ is a PID, $f(T)$ has an irreducible divisor $p(T)$. We then define.
\begin{align*}
	K_1 = \faktor{K[T_1]}{p(T_1)}
\end{align*}
and look at $K_1$ as a field extension of $K$. Then in $K_1$ we have
\begin{align*}
	p\left(
		T_1 + (p_1(T_1))
	\right) = p(T_1) + (p(T_1)) = 0 + (p(T_1))
\end{align*}
so $f(T)$ has a root in $K_1$, namely $T_1 + (p(T_1)) =: \alpha_1$.

So we can write
\begin{align*}
	f(T) = (T - \alpha_1)f_1(T) \quad \text{for} \quad f_1(T) \in K_1[T]
\end{align*}
if $f_1 = 1$, then we are done because we can just set $L = K_1$. 
Otherwise we can use induction on the degree of $f$ because $\deg f_1 < \deg f$ to get a field extension $L/K_1$
\begin{align*}
	f_1(T) = \prod_{j=2}^{n}(T - \alpha_j), \quad \text{for} \quad \alpha_j \in L
\end{align*}


\textbf{Examples}:
\begin{enumerate}
	\item For $\R$, take the polynomial $f(T) = T^2 + 1$, then $\C = \R[i]$ is such a field extension.
	\item For $K = \Q$ and $f(T) = T^3 - 2$ the field extension is
		\begin{align*}
			L = \Q[\sqrt[3]{2}, \xi \sqrt[3]{2}, \xi^2 \sqrt[3]{2}
		\end{align*}
		for $\xi = \frac{-1 + \sqrt{3}i}{2}$ a third root of unity.
\end{enumerate}

\begin{definition}[Splitting field]
	Let $K$ be a field, $f \in K[T]$ with $\deg f > 0$. A \textbf{splitting field} of $f$ over $K$ is a field extension $L/K$ such that
	\begin{enumerate}
		\item $f$ can be split into linear factors in $L[T]$
		\item For any field $E$ with $K \subseteq E \subsetneq L$, $f$ does not split over $E$.
	\end{enumerate}
\end{definition}
Note: Such a splitting field always exists and is unique up to isomorphism.

\begin{itemize}
	\item If $f \in K[T]$ then we can use Kronecker's theorem to find a field $F/K$ such that $f$ has roots $\alpha_{1}, \ldots, \alpha_{n}\in F$. Then $L := K[\alpha_1,\ldots,\alpha_n]$ is a splitting field.
	\item A splitting field is of course an algebraic field extension of $K$
\end{itemize}


\textbf{Examples}
\begin{enumerate}
	\item For $K = \Q$ and $f(T) = T^2 + 1 \in \Q[T]$ then $\C$ is not a splitting field for this setting, since $Q[i] \subseteq \R[i] = \C$ splits $f$.
	\item For $K = \R$ and $f(T) = T^2 + 1$ as above, then thesplitting field of $f$ over $\R$ is $\C$.
\end{enumerate}

Note: For a field $K, f \in K[T]$ and $L$ a splittig field of $f$ over $K$, then
\begin{align*}
	[L:K] \leq (\deg f)!
\end{align*}
if $f$ over $K$ is irreducible, then $[L K] \geq \deg f$.


\subsection{Algebraic closure}

\begin{definition}[]
	A field $K$ is called \textbf{algebraically closed}, if every polynomial $f \in K[T]$ with $\deg f > 0$ has a root in $K$.

	It follows on induction that $f$ can be split into linear components.
\end{definition}
Note: Every algebraically closed field has ifinitely many elements because if $K= \{k_1, \ldots, k_n\}$, then look at 
\begin{align*}
	f(T) = (T - k_1)\dots (T - k_n) + 1 \in K[T]
\end{align*}


\begin{proposition}[]
	Let $L/K$ be a field extension such that $L$ is algebraically closed. Then the set
	\begin{align*}
		E = \{x \in L \big\vert x \text{ is algebraic over }K\}
	\end{align*}
	is an algebraically closed field extension of $K$.

	We use this as our definition. We call $E$ the \textbf{algebraic closure} $\overline{K}$ of $K$
\end{proposition}
Proof: We need to show that $E$ is a field, that it is algebraically closed, and not dependent on $L$.

It is a field because if $x,y \in L$ are algebraic, then $x+y, x \cdot y$ and $\frac{x}{y}$ are algebraic, for $y \neq 0$

To show that it is algebraically closed, let $f \in E[T]$ with $\deg f > 0$ and let $E_1$ be an algebraic extension of $E$ such that $f$ has a root $\alpha \in E_1$. (This is possible because Kronecker's theorem). Then $E_1 /E$ is algebraically closed. But then $\alpha \in L$ and because $L$ is algebraically closed, $\alpha \in E$.



Note: if $K$ is finite, then $\overline{K}$ is countable because $K[T]$ is countable by going through the polynomials ordered by degree.
The same reasoning can show that if $K$ is countable, then so is $\overline{K}$.

In the proposition, we just assumed that $K$ had an algebraically closed field extension $L/K$ but we can show that it always exists and is unique up to isomorphism.
\begin{theorem}[]
	Let $K$ be a field. Then there exists a field extension $L/K$ such that $L$ is algebraically closed and $L$ is unique up to isomorphism.
\end{theorem}

Proof: For every non-constant polynomial $f \in K[T]$, let $T_f$ be a variable. We consider the polynomialring (in possible infinitely many variables)
\begin{align*}
	R := K\left[(T_f)_{f \in K[T], \deg f > 0}\right]
\end{align*}
Let $I \lhd R$ be the ideal generated by the elements $f(T_f)$. 
\begin{align*}
	I := \left[f(T) = T^{n} + a_{n-1}T^{n-1} + \ldots + a_0, \implies f(T_f) = (T_f)^{n} + a_{n-1}(T_f)^{n-1} + \ldots + a_9\right]
\end{align*}
Then we can show that $I \neq R$:

Chose $1\in I$,
\begin{align*}
	1 = \sum_{i \in X}g_i f_i(T_{f_i}) \in I, \quad \text{for} \quad g_i \in K[T_{f_i}]
\end{align*}
then look at the set $E$ such that every $f_i$ has aroot $\alpha_i$ in $E$. Then evaluate $f_i$ at $T_{f_i} = \alpha_i$. and we get that
\begin{align*}
	1 = \sum_{i \in X} \underbrace{g_i(\ldots)}_{\in E} \underbrace{f_i(\alpha_i)}_{= 0} = 0 \lightning
\end{align*}
Since $R \neq \{0\}$, there exists a maximal ideal $M \subseteq R$, that contains $I$. Then define
\begin{align*}
	L_1 := \faktor{R}{M}
\end{align*}
then $L_1$ is a field (because $M$ is maximal) and $K \to L_1$ is an injective field homomorphism.
By identifiying $K$ with it's image in $L_1$
\begin{align*}
	K \rightarrow K[(T_f)_f] \to \faktor{K[(T_f)_f]}{M} = L_1
\end{align*}
Next we want to show that every $f \in K[T]$ with $\deg f > 0$ has a root in $L_1$ and that $L_1/K$ is an algebraic field extension.

The image of $T_f$ in $L_1$ is a root of $f \in L_1[T]$ because
\begin{align*}
	f(T_p + M) = \underbrace{f(T_f)}_{\in I \subseteq M} + M = 0 + M 
\end{align*}
and every $x \in L_1$ is in the image of $K[T_{f_1}, \ldots, T_{f_m}]$ for a finite set of variables $T_{f_i}$. Because every $T_{f_i}$ is algebraic over $K$, so is $x$.
This shows that $L_1/K$ is an algebraic field extension of $K$.

Lastly, we repeat this process for $L_1$ instead of for $K$ and we get an $L_2/L_1$, and so on to get the stack of field extensions
\begin{align*}
	K = L_0 \subseteq L_1 \subseteq L_2 \subseteq \ldots
\end{align*}
where every $f \in L_i[T]$ iwth $\deg f > 0$ has a root in $L_{i+1}$.

Then take the union of them all
\begin{align*}
	L := \bigcup_{n \geq 0}L_n
\end{align*}
which can be shown to be a field containing $K$ and being algebraically closed over $K$.

This follows from the fact that
\begin{align*}
	F/L, L/K \text{ algebraically closed } \iff F/K \text{ algebraically closed}
\end{align*}

To show that $L$ is algebraically closed, let $f \in L[T]$ with $\deg f > 0$.

Because $f$ only has finitely many coefficients, who each lie in some $L_i$, we can take the maximum of the $L_i$ to find that
\begin{align*}
	f = (T - \alpha_1)f_1, \quad \text{for} \quad L_{i+1}[T]
	f_1 = (T - \alpha_2)f_2, \quad \text{for} \quad L_{i+2}[T]
\end{align*}
so $f$ splits into linear factors.
