\begin{definition}[Convergence of power series]
	We say that the series $\sum_{n \geq 0} f_n(z)$ converges pointwise (uniformly) to the function $f(z)$ if the sequence $s_n(z) = \sum_{k=0}^{n}f_k(z)$ converges pointwise (uniformly) to $f$.\\
	The series $\sum_{n=0}^{\infty}f_n(z)$ is \textbf{absolutely convergent}, if $\sum_{n = 0}^{\infty}\abs{f_n(z)}$ is convergent.\\
	The series $\sum_{n=0}^{\infty}f_n(z)$ is \textbf{absolutely uniformly}  convergent,  if
\end{definition}
Remark: There is no relation between uniform and absolute convergence, i.e. there exist functions which converge uniformly, but not absolutely and vice versa.\\

\begin{proposition}[]
	The sequence $\left(f_{n}\right)_{n \in \N}$ converges uniformly ond $D$, if and only if $\left(f_{n}\right)_{n \in \N}$ is cauchy.\\
	If a sequence of continuous functions converges \emph{uniformly} to a function $f$, then $f$ is also continuous.
\end{proposition}

\begin{ntheorem}[Weierstrass $M$-test]
	If for all $n \geq 0$ there exists an $M_n \geq 0$ such that
\begin{align*}
	\abs{f_n(z)} \leq M_n, \forall z \in D \quad \text{and} \quad \sum_{n \geq 0} M_n \text{ converges}
\end{align*}
Then $\sum_{n \in \N}f_n$ is uniformly and absolutely convergent.
\end{ntheorem}


\begin{nlemma}[Cauchy condition]
The sum $\sum_{n \geq 0}f_n$ converges uniformly on $D$, if and only if 
\begin{align*}
	\forall \epsilon > 0 \epsilon \nu_{\epsilon} > 0 \text{ such that } \abs{\sum_{k = m+1}^{n}f_k(z)} < \epsilon, \forall z \in D, n > m \geq \nu_{\epsilon}
\end{align*}
\end{nlemma}



