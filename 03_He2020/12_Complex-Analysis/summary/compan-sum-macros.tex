% ==== Symbols ====



\newcommand{\N}{\mathbb{N}}
\newcommand{\Z}{\mathbb{Z}}
\newcommand{\Q}{\mathbb{Q}}
\newcommand{\R}{\mathbb{R}}
\newcommand{\C}{\mathbb{C}}
\newcommand{\K}{\mathbb{K}}
\newcommand{\F}{\mathbb{F}}


\DeclareMathOperator{\charac}{char}
\DeclareMathOperator{\degree}{deg}  

\newcommand{\del}{\partial}

\let\oldemptyset\emptyset
\let\emptyset\varnothing
\newcommand{\es}{\text{\o}}
% ==== Delimiters ====

\DeclarePairedDelimiter\abs{\vert}{\vert}
\DeclarePairedDelimiter\ceil{\lceil}{\rceil}
\DeclarePairedDelimiter\floor{\lfloor}{\rfloor}
\DeclarePairedDelimiter\Norm{\vert\vert}{\vert\vert}
\DeclarePairedDelimiter\dbrack{\llbracket}{\rrbracket}


% ==== Math Operators ====
% Algebra
\DeclareMathOperator{\Bil}{Bil}
\DeclareMathOperator{\Hom}{Hom}
\DeclareMathOperator{\Ker}{Ker}
\DeclareMathOperator{\Image}{Im}
\DeclareMathOperator{\spn}{span}
\DeclareMathOperator{\rank}{rank}
\DeclareMathOperator{\rang}{rang}
\DeclareMathOperator{\id}{id}
\DeclareMathOperator{\End}{End}
\DeclareMathOperator{\Eig}{Eig}
\DeclareMathOperator{\Hau}{Hau}
\DeclareMathOperator{\trace}{tr}
\DeclareMathOperator{\rot}{rot} 
\DeclareMathOperator{\ad}{ad}

\DeclareMathOperator{\GL}{GL}


% obot
\DeclareFontFamily{U}{matha}{\hyphenchar\font45}
\DeclareFontShape{U}{matha}{m}{n}{
      <5> <6> <7> <8> <9> <10> gen * matha
      <10.95> matha10 <12> <14.4> <17.28> <20.74> <24.88> matha12
      }{}
\DeclareSymbolFont{matha}{U}{matha}{m}{n}
\DeclareFontFamily{U}{mathx}{\hyphenchar\font45}
\DeclareFontShape{U}{mathx}{m}{n}{
      <5> <6> <7> <8> <9> <10>
      <10.95> <12> <14.4> <17.28> <20.74> <24.88>
      mathx10
      }{}
\DeclareSymbolFont{mathx}{U}{mathx}{m}{n}

\DeclareMathSymbol{\obot}         {2}{matha}{"6B}
\DeclareMathSymbol{\bigobot}       {1}{mathx}{"CB}


% Analysis
\DeclareMathOperator{\grad}{grad}
\DeclareMathOperator{\graph}{graph}
\DeclareMathOperator{\supp}{supp}


% ==== Formatting ====
\newsavebox\MBox
\newcommand\cunderline[2][red]{ % Colored underline
  {\sbox\MBox{$#2$}%
  \rlap{\usebox\MBox}\color{#1}\rule[-1.2\dp\MBox]{\wd\MBox}{0.5pt}}
} 

\newcommand{\specialcell}[2][c]{%
  \begin{tabular}[#1]{@{}c@{}}#2\end{tabular}}



% ==== Tikz ====
\def\centerarc[#1](#2)(#3:#4:#5){ 
    \draw[#1] ($(#2)+({#5*cos(#3)},{#5*sin(#3)})$) arc (#3:#4:#5); 
}
% Syntax: [draw options] (center) (initial angle:final angle:radius)
