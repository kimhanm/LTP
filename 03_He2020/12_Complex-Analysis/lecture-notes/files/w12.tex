% Lecture 01.12.20
Now let's say that $\gamma \subseteq \Omega'$ and $\gamma \sim 0 \mod \Omega$ (not necessarly $\mod \Omega'$).

And for each $a_j$, let $C_j$ be a circle around $a_j$ small enough to only contain $a_j$ and no other $a_k$.

then we can write
\begin{align*}
	\frac{1}{2\pi i}\int_{\gamma}f(z) = \sum_{j} n(\gamma,a_j) R_j
\end{align*}
which is very useful for poles.


Let's say $f$ has a pole at $z = a$ of order $n$. Then 
$
(z - a)^{n}f(z)
$
is analytic in a neighborhood of $a$ and so
\begin{align*}
	f(z) = B_n(z-a)^{-n1} + B_{n-1}(z-a)^{-(n-1)} + \ldots + B_1(z-a)^{-1} \phi(a)
\end{align*}
for some $\phi$ analytic in a neighborhood of $a$.

Then $B_1$ is the residue and $f(z) - \frac{B_1}{z-a}$ is the derivative of a function in a punctured neighborhood of $a$

For example consider the function
\begin{align*}
	f(z) = \frac{e^{ab}}{(z-a)(z-b)}
\end{align*}
for $a \neq b$. It clearly has two poles at $a$ and $b$.
Then
\begin{align*}
	\text{Res}_{z = a} f(z) = \frac{e^{ab}}{z-a}|_{z=a} = \frac{e^{ab}}{a-b} \quad \text{and} \quad \text{Res}_{z=b}f(z) = \frac{e^{ab}}{b-a}
\end{align*}
And let's say we have a curve $\gamma$ with winding numbers $n(\gamma,a) = 2$ and $n(\gamma,b) = -1$, then
\begin{align*}
		\frac{1}{2\pi i}\int_{\gamma}\frac{e^{ab}}{(z-a)(z-b)}dz
	=
	2 \frac{e^{ab}}{a-b} + (-1)\frac{e^{ab}}{b-a}
\end{align*}

Now that we get Cauchy's integal formula for free. If $f: \Omega \to \C$ is analytic and $\gamma \sim 0 \mod \Omega$. Then 
$
\frac{f(z)}{z-a}
$
has a simple pole at $a$. Then simply
\begin{align*}
	\text{Res}_{z=a}\frac{f(z)}{z-a} = f(a)
\end{align*}
and what this says is that
\begin{align*}
	2\pi i \int_{\gamma}\frac{f(z)}{z-a}dz = n(\gamma,a) \text{Res}_{z=a}\frac{f(z)}{z-a} = n(\gamma,a) f(a)
\end{align*}
which shows Cauchy's integral formula. For some closed $\gamma$.


When we counted zeros of a function, we were doing calculus of residues. And if $f: \Omega \to \C$ was analytic except for isolated points where $f$ had a zero of order $h$ and $z = a$, then 
$f(z) = (z-a)^{h}f_h(z)$
and
\begin{align*}
	\frac{f'(z)}{f(z)} &= \frac{\left(
			(z-a)^{h}f_h(z)
	\right)'}{(z-a)^{h}f_h(z)} = \frac{h(z-a)^{h-1}f_h(z) + (z-a)^{h}f_h'(z)}{(z-a)^{h}f_h(z)}\\
										 &= \frac{h}{(z-a)} + \frac{f_h'(z)}{f_h(z)}
\end{align*}
which means
\begin{align*}
	\text{Res}_{z = a} \frac{f'(z)}{f(z)} = h
\end{align*}


This also holds if $f(z)$ has a pole of order $-h$. If $f$ has a pole of order $k$ and $z=a$ then
\begin{align*}
	\text{Res}_{z=a}\frac{f'(z)}{f(z)}= -k
\end{align*}

\begin{theorem}[]
	If $f$ is meromorphic in $\Omega$ with zeros $\{a_j\}$ and poles $\{b_k\}$ counted with multiplicity then
	\begin{align*}
		\frac{1}{2\pi i}\int_{\gamma}\frac{f'(z)}{f(z)}dz = \sum_{\omega \in \{a_j,b_k\}} n(\gamma,\omega) \text{Res}_{z=w} \frac{f'(z)}{f(z)} = \sum_{j}n(\gamma,a_j) - \sum_{k}n(\gamma,b_k)
	\end{align*}
	For all $\gamma \subseteq \Omega$ that don't intersect the poles/zeros
\end{theorem}
This theorem is most useful, when $n(\gamma,z) = 0,1$.

This gives us the \textbf{Argument Principle}.

For any function $g: \Omega \to \C$ analytic. 
\begin{align*}
	g(z) \frac{f'(z)}{f(z)} \text{ has residue } \left\{\begin{array}{ll}
			g(a) h & \text{ if $a$ is a zero of order $h$ of $f$} \\
			-g(a) h& \text{ if $a$ is a pole of order $h$ of $f$}
	\end{array} \right.
\end{align*}

Therefore, if $f(z_0) = \omega_0$ with $f'(z_0) \neq 0$ then there exists some $\delta$ such that
for $\abs{\omega - \omega_0} < \delta$ there exists only one solution for $f(z) = \omega$ in $\abs{z - z_0} < \epsilon$ for some $\epsilon > 0$.

Therefore we can use this for a function $f(z) - \omega$ to get
\begin{align*}
	\frac{1}{2\pi i} \int_{C(z_0,\epsilon)} g(z) \frac{f'(z)}{f(z) - \omega} dz = g(z)
\end{align*}
so in a neighborhood of $f^{-1}(\omega_0)$ we get the \textbf{integral formula for the inverse function}
\begin{align*}
	f^{-1}(\omega) = \frac{1}{2\pi i} \int_{C(z_0,\epsilon)} \xi \frac{f'(\xi)}{f(\xi) - \omega}d \xi
\end{align*}


But what happens when $f'(z_0) \neq 0$?. Then we get multiple branches. See exercise classes.


\section{Series}

This section wants to show that every analytic function has a Taylor series.

\subsection{Sequences of analytic functions}

Let's say we have a sequcene of analytic functions $f_n(z)$ that converge to $f(z)$.

When is it the case that the limit fucntion $f$ is also analytic.

We know that point-wise convergence isn't enough, as the limit function doesn't even have to be continuous. (For example $\lim_{n \to \infty} x^{n}$ on $[0,1]$)

And we want to find the optimal condition that still implies that the limit function is analytic, and is general enough and easy to check.

We will see the following theorem.
\begin{theorem}[Weierstrass]
Let $f_n: \Omega_n \to \C$ be an analytic sequence of functions that converge uniformly in each compact subset of $\Omega$. Then $f$ is analytic is $\Omega$ and $f_n'$ converges uniformly on each compact subset of $\Omega$
\end{theorem}

% 04.12.20

The proof follows from Morerea's theorem and Cauchy's Integral Formula.

Let $a \in \Omega$ and take a compact disk $\overline{\Delta}$ around $a$ and inside $\Omega$. Then
\begin{align*}
	\exists n_0 \text{ such that } \forall n > n_0, \Omega_n \supseteq \overline{\Omega}
\end{align*}
and for any $\gamma \subseteq \overline{\Delta}$ Cauch'y's Integral formula says
\begin{align*}
	\int_{\gamma}f_nz) dz = 0
\end{align*}
Because of uniform convergence in $\overline{\Omega}$,we can take the into the integral
\begin{align*}
	0 = \lim_{n \to \infty} \int_{\gamma} f_n(z)dz = \int_{\gamma}f(z) dz
\end{align*}
which shows that $f$ is analytic in $\Delta$. We will show that $f_n' \to f'$ converges in the exercise class.


This theorem is very useful vor series, for if we have
\begin{align*}
	f(z) = f_1(z) + f_2(z) + \ldots + f_n(z) + \ldots
\end{align*}
so if the series converges uniformly in every compact subset of $\Omega$, then $f$ is analytic and the derivative can be taken termwise.


For the compact set uniform convergence on boundary implies uniform convergece on the set because
$
	\abs{
		f_n(z) - f_m(z)
	}
$
is maximized in the boundary by the Maximum Principle.


\begin{theorem}[Hurwitz]
Let $f_n$ be a sequence of analytic functions and $\neq 0$ in a region $\Omega$ that converges uniformly on every compact subset of $\Omega$. Then either
\begin{enumerate}
	\item $f = 0$
	\item $f(z) \neq 0, \forall z \in \Omega$
\end{enumerate}
\end{theorem}
Proof: Homework. 

