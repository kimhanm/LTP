% ==== 15.12.20 ====
Recall that we showed $\theta(x) = \mathcal{O}(x)$. So if we can show that $\theta(x) \sim x$, then the prime number theorem follows. We will do exactly that now.

Compare the Riemann Zeta Function with 
\begin{align*}
	\phi(s) = \sum_{p} \frac{\log p}{p^s}
\end{align*}
where the sum goes over all $p$ prime.

Notice that the sum converges absolutely and locally uniformly for $\text{Re}(s) > 1$, because it can be bounded by the series going over all natural numbers.
So if we set $\sigma = \text{Re}(s)$, then
\begin{align*}
	\sum_{p} \abs{\frac{\log p}{p^{s}}} \leq \sum_{n}\frac{\log n}{n^{\sigma}} \leq \sum_{n}\frac{\log n}{n^{\sigma}} \leq \sum_{n}\frac{1}{n^{\sigma - \epsilon}} = \zeta(\sigma - \epsilon)
\end{align*}
For $\epsilon > 0$ and $n$ large enough.

Therefore both $\phi(s)$ and $\zeta(s)$ are analytic in $\text{Re}(s) > 1$. 

Next we want to show that
\begin{align*}
	\zeta(s) = \prod_{p}\frac{1}{1 - p^{-s}} \implies \frac{1}{\zeta(s)} = \prod_{p}(1 - p^{-s})
\end{align*}
so this can be used to show that there are infinitely many primes.


Because of the fundamental theorem of number theorey, we can use the prime factorisation to write
\begin{align*}
	\zeta(s) = \sum_{n=1}^{\infty}\frac{1}{n^{s}} = \sum_{r_1 \geq 0, \ldots} (p_1^{r_1}p_2^{r_2} \dots p_k^{r_k} \dots)^{-s} = \prod_{p} \sum_{r \geq 0}p^{-rs} = \prod_{p} \frac{1}{1- p^{-s}}
\end{align*}

Now that we know that the function is analytic in $\text{Re}(s) > 1$, we can try to find the unique analytic continuation. (We proved that two analytic functions that agree on a region have to agree everywhere.)

For this we first look at the region with $\text{Re}(s) < 0$, and then look at the special region with $0 < \text{Re}(s) < 1$.

For the special case where $s_0 = 1$, we obtain the harmonic series, which diverges, so we have a pole there.

Then taking the residue $\zeta(s) - \frac{1}{s - 1}$, which extends analytically to $\text{Re}(s) > 0$ because then we have
\begin{align*}
	\zeta(s) - \frac{1}{s-1} 
	= 
	\sum_{n}\frac{1}{n^{s}} - \int_{1}^{\infty} \frac{1}{x^{s}}dx 
	= 
	\sum_{n}\int_{n}^{n+1} \frac{1}{n^{s}}\frac{1}{x^{s}}dx
\end{align*}
And we need to show that it converges abosolutely. This can be done by writing the summand as a double integral
\begin{align*}
	\abs{\int_{n}^{n+1}\frac{1}{n^{s}} - \frac{1}{x^{s}}dx} 
	= 
	\abs{\int_{n}^{n+1} \int_{n}^{x} \frac{s}{u^{s+1}}du dx} 
	\leq 
	\max_{n\leq u \leq n+1} \frac{\abs{s}}{\abs{u^{s+1}}} = \frac{\abs{s}}{n^{\text{Re}(s) + 1}}
\end{align*}

Then we will show that 
for $s$ with $\text{Re}(s) \geq 1$, the function $\phi(s) - \frac{1}{s-1}$ is analytic and $\zeta(s) \neq 0$, from which we can prove the Prime number theorem.


For $\text{Re}(s) > 1$ the product formula makes it clear that is has no roots. Also
\begin{align*}
	- \frac{\zeta'(s)}{\zeta(s)} = - \sum_{p} \frac{-\left(
		\frac{1}{1-p^{-s}}
\right)^{2}\left(
	\log p p^{-s}
\right)}{\left(
	1 - p^{-s}
\right)^{-1}} = \sum_{p} \frac{\log p p^{-s}}{(1 - p^{-s})} = \sum_{p} \frac{1}{p^{s} - 1} \log p =: (\#)
\end{align*}
recall that $\phi(s) = \sum_{p} \frac{\log p}{p^{s}}$ so
\begin{align*}
	(\#) - \phi(s) = \sum_{p} \log p \left(
		\frac{1}{p^{s} - 1} - \frac{1}{p}^{s}
	\right)\frac{1}{p^{s}(p^{s}-1)}
\end{align*}
which shows absolute and locally uniform convergence for $\text{Re}(s) > \frac{1}{2}$ because
\begin{align*}
	\phi(s) = \frac{-\zeta'(s)}{\zeta(s)} = \sum_{p} \frac{\log p }{p^{s}(p^{s}-1)}
\end{align*}
so $\phi(s)$ extends meromorphically to $\text{Re}(s) > \frac{1}{2}$ with poles at $s = 1$ and zeros of $\zeta(s)$.

This means that the riemann hypothesis is equivalent to saying that the only poles of $\phi$ are at $s = 1$


\begin{center}
	missing last 20 mins
\end{center}
Assume that $\zeta(s)$ has a zero of order $\mu$, then
\begin{align*}
	\lim_{\epsilon \to 0}\epsilon \phi(1 + \epsilon) = 1\\
	\lim_{\epsilon \to 0}
\end{align*}

% ==== 18.12.20 ====


There is an anyltic theorem that says:

Let $f(t)$ for $t \geq 0$ be a bounded locall inttegrable function and suppose that the function
\begin{align*}
	g(z) = \int_{0}^{\infty}f(t) e^{-zt}dt
\end{align*}
defined for $\text{Re}(z) > 0$ extends analytically to $\text{Re}(z) \geq 0$.
Then $\int_{0}^{\infty}f(t) dt$ exists, (and equals $g(0)$.

The idea behind the proof is to consider for $T > 0$ 
\begin{align*}
	g_{T}(z) = \int_{0}^{T}f(t) e^{-zt}dt
\end{align*}
which is analytic for all of $\C$. And then want to show that the limit exists and is given by
\begin{align*}
	\lim_{T \to \infty}	g_T(0) = g(0)
\end{align*}
by drawing curves around zero.

Then we can use this theorem to prove the prime number theorem by showing that
\begin{align*}
	\int_{0}^{\infty}\frac{v(x) - x}{x^2}dx
\end{align*}
converges. For $\text{Re}(s) > 1$, we integrate by parts to get
\begin{align*}
	\Phi(s) = \sum_{p} \frac{\log p}{p^{s}} = \int_{1}^{\infty} \frac{dv(x)}{x^s}dx = - \int_{1}^{\infty}v(x) \left(
		\frac{-s}{x^{s+1}}dx + 0
	\right) 
\end{align*}
\begin{align*}
	\Phi(s) = s \int_{0}^{\infty} \frac{v(e^{t})}{e^{ts + t}}e^{t} dt = s \int_{0}^{\infty} e^{-ts}v(e^{t}) dt
\end{align*}
then use the anaytic theorem for
\begin{align*}
	f(t) = v(e^{t})e^{-t} - 1 \quad \text{and} \quad g(z) = \frac{\Phi(z+1)}{z+1} - \frac{1}{z}
\end{align*}
To check that we can apply the analytic theorem
\begin{align*}
	\frac{\Phi(z+1)}{z+1} - \frac{1}{z} &= \frac{1}{z+1} \left(
		(z+1) \int_{0}^{\infty}e^{-t(t+1)}v(e^{t}) dt 
	\right) - \frac{1}{z}\\
																			&= \int_{0}^{\infty}e^{-tz}e^{-t}v(e^{t}) dt - \frac{1}{z}\\
																			&= \int_{0}^{\infty}e^{-tz}(e^{-t}v(e^{t}) - 1) dt
\end{align*}
notice that $f(t)$ is bounded because $v(x) \leq C x$ which means $v(e^{t}) e^{-t} \leq C$.
Then $g(z)$ extends analyticall to $\text{Re}(z) \geq 0$ and
\begin{align*}
	\frac{\Phi(z+1)}{z+1} - \frac{1}{z}
\end{align*}
\begin{center}
	missing 5 mins
\end{center}
In the end, we have that
\begin{align*}
	\int_{1}^{\infty}\frac{v(x) - x}{x^{2}}dx
\end{align*}
exists.


The way this proves the prime number theorem is by contradiction: Assume that there exists a $\lambda > 0$ such that for arbitrarily large values of $C$ 
\begin{align*}
	\exists x \geq C \text{ with } v(x) \geq \lambda x
\end{align*}
since $v(x)$ is non-decreasing and the integral converges, we must have that
\begin{align*}
	\lim_{x \to \infty}	\int_{x}^{\lambda x}\frac{v(t) - t}{t^{2}} dt = 0
\end{align*}
but it can be bounded by
\begin{align*}
	\int_{x}^{\lambda x}\frac{v(t) - t}{t^{2}} dt \geq \int_{x}^{\lambda x} \frac{v(x) - t}{t^{2}} dt \geq 
	\int_{x}^{\lambda x}\frac{\lambda x - t}{t^{2}} dt = \int_{1}^{\lambda} \frac{\lambda x - ux}{u^{2}x^{2}} x du > 0
\end{align*}
which can't possibly converge. By doing the same argument for $v(x) \leq \lambda x$ for $\lambda < 1$, we will have shown that $v(x) \sim x$.





