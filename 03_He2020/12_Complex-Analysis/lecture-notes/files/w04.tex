Let's say that we have a function given by a power series
\begin{align*}
	f(z) = a_0 + a_1 z + a_2 z^2 + \ldots + a_nz^{n-1} + \ldots
\end{align*}
then we can find out its derivative by term-wise calculation
\begin{align*}
	f'(z) &= a_1 + 2a_2z + \ldots + n a_n z^{n-1} + \ldots\\
	f''(z) &= 2a_2 + 2 \cdot 3 a_3 z + \ldots +(n-1)n a_n z^{n-2} + \ldots
	f^{(k)}(z) &= k!a_k + \frac{(k+1)!}{1!}a_{k+1}z + \frac{(k+2)!}{2!}a_{k+2}z^2 +  \ldots  
\end{align*}
In particular, if we are given some analytic function and want to detrmine its power series, we see that
\begin{align*}
	f^{(k)}(0) = k!a_k \quad\implies \quad a_k = \frac{1}{k!}f^{(k)}(0)
\end{align*}
which allows us to write us the \textbf{Taylor-Maclaurin series} for analytic functions
\begin{align*}
	f(z) = f(0) + \frac{1}{1!}f^{(1)}(z) + \frac{1}{2!}f^{(2)}z^2 + \ldots + \frac{1}{k!}f^{(k)}(0)z^k + \ldots
\end{align*}

\subsection{Exponential and trigonometric functions}
The exponential function and the trigonometric functions have the connection for $x \in \R$ that
\begin{align*}
	e^{ix} = \cos x + i \sin x
\end{align*}
Since the exponential function has so many nice properties which all follow from another, we have many options on how we want to define the fucntion. The version which we will take will be through it's property for its derivatives

\begin{definition}[Exponential functions]
The exponential function is the function $\exp: \C \to \C$ which is uniquely determined by the following differential equation
\begin{align*}
	\exp'(z) = \exp(z), \forall z \in \C, \quad \exp(0) = 1
\end{align*}
\end{definition}
Using the Taylor-Maclaurin series expansion, we can get
\begin{align*}
	f(z) &= a_0 + a_1 z + a_2z^2 + \ldots + a_nz^2 + \ldots\\
	f'(z) &= a_1 + 2a_2 z + 3a_3 z^2 + \ldots + na_n z^{n-1} + \ldots \end{align*}
From this we find that $a_n = \frac{1}{n!}$ so
\begin{align*}
	e^{z} := \exp(z) = 1 + \frac{z}{1!} + \frac{z^2}{2!} + \ldots + \frac{z^n}{n!} + \ldots
\end{align*}
Using Abel's convergence theorem, we see that (using Stirling's Formula) the power series has an inifite Radius of convergence, i.e. that the function is well-defined for all $z \in \C$.\\

We will see that the exponential function has the following properties

\begin{enumerate}
\item 	It has the additive property $e^{a+b} = e^{a} \cdot e^{b}$.
\item 	It respects the complex conjugate: $\exp(\overline{z}) = \overline{\exp(z)}$
\item 	For $x \in \R$ we have $\abs{e^{ix}} = 1$
\end{enumerate}

Proof:
\begin{enumerate}
\item Using the differential equation, it follows from the product rule that
	\begin{align*}
		D(e^{c-z}e^{z)} = D(e^{c-z}) e^{z} + e^{c-z}D(e^{z}) = -e^{c-z}e^{z} + e^{c-z}e^{z} = 0
	\end{align*}
\item Since all coefficients in the Taylor series are real, it follows immediately that $e^{\overline{z}} = \overline{e^{z}}$
\item From property (a) and (b) we find that
	\begin{align*}
		\abs{e^{ix}}^2 = e^{ix} \overline{e^{ix}} = e^{ix} e^{-ix} = e^{i(x-x)} = 1
	\end{align*}
\end{enumerate}

From (a) it follows immediately that the exponential is never zero, i.e. has a multiplicative inverse
\begin{align*}
	1 = e^{0} = e^{z-z} = e^{z}e^{-z}
\end{align*}
It also follows from property (c) that for $x,y \in \R$ we have
\begin{align*}
	\abs{e^{x + iy}} = \abs{e^{x}} \cdot \abs{e^{iy}} = e^{x} 
\end{align*}

As we said before, there are many ways to define the complex exponential and thus also many ways to define the trigonometric functions. We also could have started with the trigonometric functions and define the complex exponential from there, but here we do it like this:
\begin{definition}[Sine and Cosine]
	The complex Sine and Cosine functions are the functions $\sin, \cos: \C \to \C$ defined by
	\begin{align*}
		\sin(z) &:= \frac{e^{iz} - e^{-iz}}{2i}\\
		\cos(z) &:= \frac{e^{iz} + e^{-iz}}{2}
	\end{align*}
\end{definition}

We then can find the Taylor series of these functions to be

\begin{empheq}[box=\bluebase]{align*}
 \sin(z) &= z - \frac{z^3}{3!} + \frac{z^5}{5!} - \frac{z^7}{7!} + \frac{z^9}{9!} - \ldots\\
 \cos(z) &= 1 - \frac{z^2}{2!} + \frac{z^4}{4!} - \frac{z^6}{6!} + \frac{z^8}{8!} - \ldots
\end{empheq}

From these definitions, we wee the formula stated in the beginning
\begin{align*}
	e^{ix} = \cos(x) + i \sin(x)
\end{align*}
We also see that for $x \in \R$ we recover the triangular identity
\begin{align*}
	1 = \abs{e^{iy}}^2 = \abs{\cos(x) + i \sin(x)}^2 = \cos^2(x) + \sin^2(x)
\end{align*}
Aswell als the addition formula for $x_1,x_2 \in \R$:
\begin{align*}
	\cos(x_1 + x_2) + i \sin(x_1 + x_2) = e^{i(x_1 + x_2)} = e^{ix_1}e^{ix_2} = \left(\cos(x_1) + i \sin(x_1)\right) \left(\cos(x_2) + i \sin(x_2)\right)
\end{align*}

\begin{definition}[Periodicity]
We say that a function has a period $c$, if 
\begin{align*}
	f(z + c) = f(z), \forall z \in \C
\end{align*}
\end{definition}
For example, sine and cosine have Period $2\pi$ and the exponential function has period $2\pi i$.\\

If $z = e^w$, then we want to define the logarithm to be $\log(z) := w$\\
Since the exponential function is periodic, there are multiple solutions for $w$ also, since we know that $e^z$ is never equal to zero, there is no solution for $z = 0$\\
Using what we found for the complex exponentials we know that for $w = x + iy$ we have
\begin{align*}
	\abs{z} &= e^{x}, e^{iy} = \frac{z}{\abs{z}}\\
	\implies \log(z) = \log(x) + i \arg(z)
\end{align*}
where $\log(x)$ is the real logarithm and where the argument $\arg$ can be determined up to $\mod 2 \pi$.\\

But if we force to define the argument to be always in $[0, 2\pi)$, then the argument isn't a continuous function anymore.\\

\section{Topology}
For us to get some nice properties of the \emph{smoothness} of a funciton, we need to be able to \emph{zoom in} and talk about points being \emph{close} from another. \\
In order to have a notion of closeness we will need a \emph{distance} function, or more accurately a \emph{metric}.

\begin{definition}[Metric Space]
	Let $M$ be a set, a function $d: M \times M \to \R$ is called a \textbf{metric} if it satisifes a following properties:
	\begin{enumerate}
		\item 	Positive definiteness: $d(x,y) 0 0 \iff x = y$
		\item 	Symmetry: $d(x,y) = d(y,x)$
		\item 	Triangle inequality: $d(x,z) \leq d(x,y) + d(y,z)$
	\end{enumerate}
\end{definition}

It follows from these properties, that the metric must also be $\geq 0$.

We next define the open Ball centered around $x$ with radius $\delta$ to be the set
\begin{align*}
	B(x,\delta) := \{y \in M: d(x,y) < \delta\}
\end{align*}
We call a set $N$ a \textbf{neighborhood} of some $x \in M$ if
\begin{align*}
	\exists \delta > 0: B(x,\delta) \subseteq N
\end{align*}

We call a set $S$ \textbf{open}, if $S$ is a neighborhoods of all its points $x \in S$. We call it \textbf{closed}, if its complement $M \setminus S$ is open.\\

The \textbf{interior} of some set $X$ will be the \emph{largest} open set $S \subseteq X$
\begin{align*}
	\text{int} X := \bigcup_{\underset{S \text{ open}}{S \subseteq X}} S
\end{align*}
The \textbf{closure} is the \emph{smallest} closed set containing $X$:
\begin{align*}
	\overline{X} := \bigcup_{\underset{S \text{ closed}}{S \supseteq X}}S
\end{align*}

And the \textbf{boundary} is the difference between the two:
\begin{align*}
	\del X := \overline{X} \setminus \text{int} X
\end{align*}

When we talk about functions, that have some nice properties, we need to talk about when these properties make sense. Openness of a region is very important, as we then are able to ``zoom in'' enough to make our statements wihtin the region $X$. There are also some more properties we want to look at.

We call a set $X$ \textbf{connected}, if it cannot be written as the union of two disjoint, non-empty and open subsets. $X = U \sqcup V$. In other words, if there exist $U,V$ open subsets of $X$ such that
\begin{align*}
	S = U \cup V, \text{ and } U \cap V = \es  \implies U = \es \lor V = \es
\end{align*}

\begin{proposition}[]
	Intervals are connected subsets of $\R$.	
\end{proposition}
Proof: Let $I \subseteq \R$ be an interval and assume that $U, V \neq \es$ with $I = U \sqcup V$. Then take $u_1 \in U$ and $v_1 \in V$ and assume without loss of generality that $u_1 < v_1$. Take the midpoint $m_1 := \frac{u_1 + v_1}{2}$ which must either be in $U$ or $V$. If $m_1 \in U$, define $u_2 := m_1, v_2 := v_1$ and take the midpoint $m_2$ again. Since the sequences $\left(u_{n}\right)_{n \in \N}$ and $\left(v_{n}\right)_{n \in \N}$ are cauchy, the must converge, but also $U$ and $V$ are closed, so $u:= \lim_{n \to \infty} u_n \in U$ and $v:= \lim_{n \to \infty} v_n \in V$, and $\lim_{n \to \infty} \abs{a_n - b_n} \to 0 \implies u = v \lightning$.\\

\begin{theorem}[]
	In the plane, a non-empty open set is connected if and only if and only if every two points can be reached by a polygon line.
\end{theorem}
Totally rigorous definition and proof: Let $A$ be open and connected and for $a \in A$ let $A_1$ be the set of points that can be reached from $a$ by a polygon line and $A_2$ be thet set of points that can't.\\

These sets are obviously disjoint and have $A = A_1 \sqcup A_2$. Let $x \in A_1$. Since $A$ is open, there exists $\delta > 0$ such that $B(x,\delta) \subseteq A$. And for any $y \in B(x, \delta)$ there exists a line connecting $x$ and $y$. Therefore $B(x,\delta) \subseteq A_1$, which shows that $A_1$ is open. If on the other hand $x \in A_2$, then the Ball $B(x, \delta)$ must also be in $A_2$, or else we could extend the line $a-y$ to $a-y-x$.\\

Now assume that $A$ is such that every two points of $A$ can be reached by a polygon line and assume that $A = U \sqcup V$, of non-epty open sets $U,V$ and let $u \in U$, $v \in V$ be linked by a polygon line. Somewhere along the line, there must be a switch, from $U$ to $V$. And the polygon must locally look like an Interval. From the previous proposition, it follows that some point must exists that is both in $U$ and $V$.\\

The next property of sets we want is \emph{compactness}. We call a set $S$ \textbf{compact}, if for any open cover $S = \bigcup_{x \in X} O_x$, there exists a finite subcover $S = \bigcup_{y \in Y}O_y$ for $Y \subseteq X$ finite.\\

\begin{theorem}[]
	A subset $S \subseteq \R^n$ is compact if and only if $S$ is closed and bounded and if $S$ is compact, then every sequence in $S$ has an accumulation-point and a convergent subsequence.
\end{theorem}
Proof: Let's say $\left(x_{n}\right)_{n = 1}^{\infty}$ has no accumulation point, i.e. for all $y \in S$ there exists a $\delta_y > 0$ such that
\begin{align*}
	B(y,\delta_y) \cap \left(x_{n}\right)_{n = 1}^{\infty} \text{ is finite}
\end{align*}
Then we have an open cover using these intersections (which are open!). But since $S$ is compact, we must have a finite subcover of $S$, which would imply that $S$ itself only has finitely many elements. This can't be the case since we have an infinite sequence in $S$ without accumulation points.
