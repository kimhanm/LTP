% ==== 10.11.2020 ====
Recall that Cauchy's estimate allows us to show that if we have an analytic function on an open disk $\Delta$ then we can describe the $n$-th derivative through a path integral for closed path $\gamma$ in $\Delta$
\begin{align*}
	f^{(n)}(z) = \frac{n!}{2\pi i}\int_\gamma	 \frac{f(z')}{(z'-z)^{n+1}}dz
\end{align*}
Now, even if our function $f$ is not analytic on the entire open disk, Cauchy's integral formula still holds if $f$ has some singularities $\left(a_{j}\right)_{j \in J}$ and $f$ is analytic on $\Delta \setminus \left(a_{j}\right)_{j \in J}$ and such that
\begin{align*}
	\lim_{z \to a_j} (z - a_j)f(z) = 0
\end{align*}
as long as our path $\gamma$ doesn't go through these singularities. So we get the following theorem:

\begin{theorem}[Analytic Extension]
	If $f$ is analyitic in a region $\Omega'$ obtained from a region $\Omega$ by removing a point $a \in \Omega$ and if
	\begin{align*}
		\lim_{z \to a}(z-a)f(z) = 0
	\end{align*}
	then $f$ can be extended to an analytic in $\Omega$ be taking
	\begin{align*}
		f(a) = \frac{1}{2\pi i} \int_C \frac{f(z)}{z - a}d z
	\end{align*}
	where $C$ is a circle around $a$ contained in $\Omega$
\end{theorem}

If we use our theorem with the primitive
\begin{align*}
	F(z) = \frac{f(z) - f(a)}{z-a}
\end{align*}
then if $F$ is analytic in $\Omega \setminus \{a\}$ and if
\begin{align*}
	\lim_{z \to a}(z-a)F(z) = 0
\end{align*}
then the limit of $F$ at the point $a$ will be given by
\begin{align*}
	\lim_{z \to a}F(z) = f'(a)
\end{align*}
We then create a sequence of functions $f_i: \Omega \to \C$ given by
\begin{align*}
	f_1(z) := \left\{\begin{array}{ll}
			\frac{f(z) - f(a)}{z-a} & z \neq a\\
			f'(a) & z = a
	\end{array} \right.
\end{align*}
which is analytic and recursively define
\begin{align*}
	f_{n+1}(z) := \left\{\begin{array}{ll}
			\frac{f_n(z) - f_n(a)}{z-a} & z \neq a \\
			f_n'(a) & z = a
	\end{array} \right.
\end{align*}
which are all analytic. We then can write
\begin{align*}
	f(z) &= f(a) + (z-a)f_1(z)\\
	f_1(z) &= f_1(a) + (z-a)f_2(a)\\
				 &\vdots\\
	f_n(z) &= f_n(a) + (z-a) f_{n+1}(z)
\end{align*}
which, if we write this out, we obtain the following theorem.
\begin{theorem}[]
	If $f: \Omega \to \C$ is analytic in a region $\Omega$, then
	\begin{align*}
		f(z) = f(a) + (z-a)f_1(a) + (z-a)^2 f_2(a) + \ldots + (z-a)^{n} f_n(z)
	\end{align*}
	, where $f_n(z)$ is analytic in $\Omega$.
\end{theorem}

Now, if we take the $n-th$ derivative, and set $z = a$, then we get that
\begin{align*}
	f^{(n)}(a) = n!f_n(a) \implies f_n(a) = \frac{1}{n!}f^{(n)}(a)
\end{align*}
which just gives us the Taylor expansion 
\begin{align*}
	f(z) = f(a) + \frac{(z-a)}{1!} f'(a) + \frac{(z-a)^2}{2!}f''(a) + \ldots + \frac{(z-a)^n}{n!}f^{(n)}(a) + (z-a)^{n+1}f_{n+1}(z)
\end{align*}
where $f_{n+1}(z)$ is analytic in $\Omega$. 

Our next goal is to show that if $f(a) = 0$ and $f^{(n)}(a) = 0$ for all $n > 0$, then $f$ must be zero in the entire region.

If we take the Cauchy integral Formula for $f_n(z)$, we get that
\begin{align*}
	f_n(z) = \frac{1}{2\pi i}\int_C \frac{f_n(z')}{z' - z}dz'
\end{align*}
which, when plugged into the taylor expansion we get
\begin{align*}
	f_n(z) = \frac{1}{2\pi i} \int_C \frac{{\color{orange}\frac{f(z')}{(z'-a)^{n}}} - {\color{blue}\frac{f^{(1)}(a)}{1!(z'-a)^{n-1}} - \frac{f^{(2)}(a)}{2!(z'-a)^{n-2}} - \ldots - \frac{f^{(n-1)}(a)}{(n-1)!(z'-a)}}}{z'-z} dz'
\end{align*}
Now all the blue colored terms can be written up to a constant as
\begin{align*}
	F_k(a) = \int_C \frac{1}{(z'-a)^{k}(z'-z)}dz'
\end{align*}
and we want to show that $F_k(a) = 0$ for all $1 \leq k \leq n-1$. For $k = 1$ we have that
\begin{align*}
	F_1(a) = \int_C \frac{1}{(z'-a)(z'-z)}dz' = \frac{1}{a-z}\int_C \frac{1}{z' -a} - \frac{1}{z'-z}dz' = \frac{1}{a -z} \left[n(C,a) - n(C,z)\right]
\end{align*}
since $z$ and $a$ have the same winding number.
to show it for the other $k$, we just differentiate $F_(a)$ and we get 
\begin{align*}
	F_1'(a) = 2F_2(a) = 0, \quad \text{and} \quad F_1^{(k)}(a) = k! F_k(a) = 0
\end{align*}
so only the orange term remains and we get
\begin{empheq}[box=\bluebase]{align*}
	f_n(z) = \frac{1}{2\pi i}\int_C \frac{f(z')}{(z'-a)^{n}(z'-z)}dz'
\end{empheq}
For any $z$ inside of $C$. The reason this is useful is if we take some $a \in \Omega$ such that $f(a) = 0$ and $f^{(k)}(a) = 0$ for all $k$, then from the Taylor expansion, we have that for any $n > 0$ we must have that only the $f_n$ term remains, so $f(z) = (z-a)^{n}f_n(z)$. Then take the maximum of all values inside the circle $C$
\begin{align*}
	M := \max_{z \in C}\abs{f(z)} \implies
	\abs{f_n(z)} \leq \frac{1}{2\pi} 2\pi RM \frac{1}{R^{n}}\frac{1}{R - \abs{z-a}} = \frac{M}{R^{n-1}}\frac{1}{R - \abs{z-a}}
\end{align*}
And from the way we chose $R$ we have $\abs{z-a} \leq R$ , so
\begin{align*}
	\abs{f(z)} \leq \abs{z-a}^{n} \abs{f_n(z)} = \left(\frac{\abs{z-a}}{R}\right)^{n} \frac{RM}{R - \abs{z-a}}
\end{align*}
and for $n \to \infty$, then the term goes to zero. So $f(z)$ is zero inside the circle $C$. 

Therefore: If $f(a)$ and $f^{(n)}(a)$ are zero for all $n >0$, then this is also true in a neighborhood of $a$. And define the sets
\begin{align*}
	E_1 &:= \left\{z \big\vert f(z) = 0 \text{ and } f^{(n)}(z) \forall n > 0\right\}\\
	E_2 &:= \left\{z \big\vert f^{(n)}(z) \neq 0 \text{ for some }n\right\}
\end{align*}
These sets are disjoint and by continuity of $f^{(n)}(z)$ the sets are also open. So since $\Omega$ is connected, on of them must be empty! So $f$ is zero everywhere on $\Omega$.



Rephrased in another way, analytic functions that are not identically zeros, then their zeros are isolated. This in turn means that if two functions agree on some open region, then they must agree everywhere.



% ==== 13.11.20 ====
Take $f: \Omega \to \C$ analytic and not identical to zero and let $a$ be a zero of $f$. Then by the previous theorem there must exist some $n$ such that $f^{(n)}(a) \neq 0$ and without loss of generality, we can take the smallest one.
By the Taylor theorem we know that
\begin{align*}
	f(z) = f(a) + \frac{z-a}{1!}f'(a) + \frac{(z-a)^2}{2!}f^{(2)}(a) + \ldots f_n(z)(z- a)^{n}
\end{align*}
where $f_n$ is analytic. Since $n$ is the smallest number, where the derivative is non-zero, we have n particular
\begin{align*}
	f(z) = f_n(z) (z-a)^{n}
\end{align*}
So it must be that $f(a) \neq 0$, which means that there exists a $\delta > 0$ such that 
\begin{align*}
	f_n(z) \neq 0, \quad \forall z: \abs{a-z} < \delta
\end{align*}
Then take the punctured disk $\left\{z: 0 < \abs{a-z} < \delta\right\}$. Then
\begin{align*}
	f(z) = (z-a)^{n}f_n(z) \neq 0
\end{align*}
is non-zero for a punctured disk around $a$. This gives us the following theorem:
\begin{theorem}[]
If $f: \Omega \to \C$ is analytic and not identically zero, then the zeros of $f$ are isolated.
\end{theorem}
The contraposition of this theorem can be stated as follows:

Let $S \subseteq \C$ be a set with an accumulation point. Then if $f(z) = 0$ for all $z \in S$, then $f = 0$ in all of $\Omega$.


If we take two functions and look at their difference, we obtain the following theorem:
\begin{theorem}[]
If $f,g: \Omega \to \C$ are analytic and coincide in a set with an accumulation point, then $f = g$ in all of $\Omega$.
\end{theorem}

Now we want to study singularities at single points $a$, and how fast the $f$ explodes at $a$. If $f$ is analytic in a punctured neighborhood of $a$. If
\begin{align*}
	\lim_{z \to a} (z-a)f(z) = 0
\end{align*}
then the singularity at $a$ is removable: If $f$ blows up at $a$, i.e $\lim_{z \to a}f(z) = \infty$. This means that there exists a $\delta > 0$ such that
\begin{align*}
	0 \abs{z-a} < \delta \implies f(z) \neq 0
\end{align*}
if we look at its inverse $g = \frac{1}{f}$ on the puctured disk, $g$ is also analytic and
\begin{align*}
	\lim_{z \to a} (z-a) g(z) = \lim_{z \to a}\frac{z-a}{f(z)} = 0
\end{align*}
so $g$ has a removable singularity at $a$. Since $z$ is not identically zero there exists an $n$ such that
\begin{align*}
	g(z) = (z-a)^{n} g_n(z), \quad \text{for} \quad g_n(a) \neq 0
\end{align*}
then take the $\delta > 0$ such that $g_n(z)$ is non-zero in the punctured neighborhood $\left\{z \big\vert 0 <\abs{z-a} < \delta\right\}$. Then we know that $f_n(z) = \frac{1}{g_n(z)}$ is analytic and
\begin{align*}
	f(z) = (z-a)^{-n}f_n(z)
\end{align*}
which means that $a$ is a pole of order $n$ of $f$.

 
\begin{definition}[Meromorphic functions]
A function that is analytic except for poles is called \textbf{meromorphic} in $\Omega$.
\end{definition}
If $f$ and $g$ are anyltic in $\Omega$ and $g$ is not identical to zero, then $\frac{f}{g}$ is meromorphic.

Recall that when we talked about fractions of polynomials, we also defined orders of zeros and poles. The definition of order of poles and zeros of meromorphic functions is compatible with the definition for rational functions.

