\section{Complex Integration}

We define the integral of complex valued functions using the real-valued Integral. So let $f:[a,b] \to \C$ be a continuous function with $f(t) = u(t) + iv(t)$, then we define it's integral to be
\begin{align*}
	\int_{a}^{b}f(t) dt := \int_{a}^{b}u(t) dt + i \int_{a}^{b}v(t) dt
\end{align*}
The Integral will have the following properties:
\begin{enumerate}
	\item $\int_{a}^{b} c f(t) dt = c \int_{a}^{b}f(t) dt$
	\item Triangle Inequality:
		\begin{align*}
			\abs{\int_{a}^{b}f(t) dt}	\leq \int_{a}^{b} \abs{f(t)}dt
		\end{align*}
\end{enumerate}
Proof: If $\int_{a}^{b}f(t) dt = 0$, then there is nothing to show. If it is nonzero, then we can talk about its argument $\theta = \arg \int_{a}^{b}f(t) dt$. 
Since $\abs{z} = \text{Re}(\exp(-i \theta)z)$, we can write
\begin{align*}
	\abs{\int_{a}^{b}f(t) dt}	= \text{Re} \left[\exp(-i \theta) \int_{a}^{b}f(t) dt \right] = \int_{a}^{b} \text{Re} \left[\exp(-i \theta) f(t) \right] dt \leq \int_{a}^{b} \underbrace{\abs{\exp(-i \theta)}}_{= 1} \abs{f(t)}dt
\end{align*}
Where we used the fact that that the Real part is always smaller than the absolute value and tha the absolute value is multiplicative.

Now that we know how to integrate over Intervals $[a,b] \subseteq \R$, we want to know how to integrate over paths/arcs.\\

Let $\gamma$ be an arc that is parametrized by a differentiable function $z(t)$ for $a \leq t \leq b$.
Say $f$ is continuous on the arc $\gamma$. Then we define the Integral over the arc to be 
\begin{align*}
	\int_{\gamma} f(z) dz := \int_{a}^{b} (f \circ z)(t) \cdot \dot{z}(t) dt
\end{align*}
And if $\gamma$ is only piece-wise differentiable, then we define the integral piecewise, i.e. summing over all the differentiable parts.

Now this definition better be well defined, i.e. invariant under reparametrisations. If $t(\tau)[\alpha, \beta] \to [a,b]$ with $t(\alpha) = a, t(\beta) = b$ is a reparametrisation of $[a,b]$, then using the chain rule we have
\begin{align*}
 \int_{\tilde{\gamma}} f(z) dz &= \int_{\alpha}^{\beta}(f \circ z \circ t)(\tau) \cdot \frac{d}{d \tau}(z \circ t)(\tau) d \tau \\
															 &= \int_{\alpha}^{\beta} (f \circ (z \circ t))(\tau) \cdot \dot{z}(t(\tau)) \dot{t}(\tau) d \tau\\
															 &= \int_{a}^{b} (f \circ z)(t) \dot{z}(t) dt = \int_{\gamma}f(z) dt
\end{align*}

It is clear that if we traverse an arc the other way around with $w: [-b,-a] \to \C$ for $w(t) = z(-t)$, then 
\begin{align*}
	\int_{-\gamma} f(z) dz &= \int_{-b}^{-a}(f \circ w)(t) \dot{w}(t) dt = \int_{-b}^{-a} (f \circ z)(-t) (-\dot{z}(-t)) dt\\
												 &= \int_{a}^{b} (f \circ z)(t) \dot{z}(t) dt = - \int_{\gamma}f(z) dt
\end{align*}

Now if we have multiple arcs that intersect in their endpoints, we cann add the arcs and the integrals then
\begin{align*}
	\int_{\gamma_1 + \ldots + \gamma_n}^{\infty} f dz = \int_{\gamma_1}f dz + \ldots + \int_{\gamma_n}f dz
\end{align*}

Now what if a single harc has the same start end endpoints? It will create a loop and we can say that the Integral over closed curves is independent on the starting point. So if
$\gamma = \gamma_1 + \gamma_2$, then
\begin{align*}
	\int_{\gamma} = \int_{\gamma_1} + \int_{\gamma_2} = \int_{\gamma_2} + \int_{\gamma_1}
\end{align*}

\#\#\# Missing text.
\begin{align*}
\int_{\gamma} f(z) \overline{dz} := \overline{\int_{\gamma} \overline{f(z)} dz}\\
	\int_{\gamma}f dx := \frac{1}{2} \left(\int_{\gamma} f dz + \int_{\gamma} f \overline{dz}\right)\\
	\int_{\gamma}f dy := \frac{1}{2i} \left(\int_{\gamma} f dz - \int_{\gamma} f \overline{dz}\right)
\end{align*}
So if $f(z) = u(z) + iv(z)$ we can write with $dz = dx + dy$ that
\begin{align*}
	\int_{\gamma} f(z) dz = \int_{\gamma} u dx - v dy + i \int_{\gamma} v dx + u dy
\end{align*}
And we can also integrate with respect to length:
\begin{align*}
	\int_{\gamma} f ds := \int_{\gamma}f \abs{dz} = \int_{a}^{b} (f \circ z)(t) \cdot \abs{\dot{z}(t)} dt
\end{align*}
So in this case it doesn't matter what direction our arc is parametrized.\\


This allows us to define the line-integral as a function of the arcs. If $\Omega$ is a region (open and connected), and we wan to find out  $\int_{\gamma} p(x,y) dx + q(x,y) dy$, then we will sometimes see that this will only depend on the endpoints of $\gamma$ and not on the arc itself.
This might help us even integrate functions of the reals. We then can integrate along a different path over the complex numbers, such that the integration will be slightly easier to compute.\\


In the next chapters, we will focus on finding out when exactly the Integral only depends on the endpoints.

If $pdx + qdy$ are fixed, the the Integral depends only on the end points of $\gamma$ if and only if $\int_{\gamma} = 0$ for closed gamma.

Proof: $\implies:$ Becuse $\gamma$ and $-\gamma$ have the same endpoints we must have $\int_{\gamma} = \int_{-\gamma} = - \int_{\gamma} = 0$.\\
$\impliedby:$ If two curves have the same endpoints, then consider closed curve $\gamma_1 + (- \gamma_2)$. Whose integral must be zero, so then the integrals are the same.

In the real case, we often substituted the integral with the primive $F$ of a function $f$. Now we can write it like this:

\begin{theorem}[]
 The line integral $\int_{\gamma} pdx + qdy$ defined in a region $\Omega$ depends only on the end points of $\gamma \subseteq \Omega$ if and only if
 \begin{align*}
	 \exists U(x,y) \text{ on } \Omega \text{ such that } \frac{\del U}{\del x} = p, \frac{\del U}{\del y} = q 
 \end{align*}
 Since its derivatives are fixed, it is clear that the primitive $U$ is unique up to a constant.
\end{theorem}
Proof: If the primitive $U$ exists, then
\begin{align*}
	\int_{\gamma} pdx + qdy &= \int_{a}^{b} \frac{\del U}{\del x}\dot{x}(t) + \frac{\del U}{\del y} \dot{y}(t) dt\\
													&= \int_{a}^{b} \frac{d}{dt} U(x(t), y(t)) dt = U(x(b), y(b)) - U(x(a),y(a))
\end{align*}
If on the other hands the integrals only depend the endpoints, then chose any $(x_0,y_0) \in \Omega$. For any other point in $\Omega$, since it is connected, we have an arc going from our startpoint to that point. So we can just set
\begin{align*}
	U(x,y) = \int_{\gamma} pdx + qdy
\end{align*}
which is well defined, since the choice of $\gamma$ doesn't matter and we just have to check that it indeed is a primitve, but that is trivial:
\begin{align*}
	U(X;Y) = \text{const. } + \int_{\gamma} pdx + q dy \implies \frac{\del U}{\del x} = P
\end{align*}


So this gives us the central question, when is it that
\begin{align*}
	f(z) dz = f(z) dx + i f(z) dy
\end{align*}
is an exact differential? If we decompose our path into real and imaginary components, 
\begin{align*}
	z(t) = x(t) + iy(t)
\end{align*}
we would get 
\begin{align*}
	\int_{\gamma} f(z) dz &= \int_{a}^{b} (f \circ z)(t) \dot{z}(t) dt = \int_{a}^{b}f\left(x(t) + iy(t)\right) (x'(t) + iy'(t)) dt\\
												&= \int_{\gamma} f(x + iy) (dx + idy) 
\end{align*}
which can be viewed as a function from $\R^2$ to $\R^2$. So if we write $f(x,y) = u(x,y) + iv(x,y)$ we get
\begin{align*}
	\int_{\gamma}f(z) dz = \int_{\gamma}f(x,y) (dx + idy) = u(x,y) + iv(x,y) (dx + idy) = \int_{\gamma} u(x,y) dx - v(x,y) dy + i \int_{\gamma} u(x,y) dy + v(x,y) dx
\end{align*}
So we also could have defined the complex integral this way.\\

Now let's see when $f(z) dz$ is an exact differential. Does there exist an $F(x,y)$ such that $dF(z) = f(z) dz$. By decomposing we can write
\begin{align*}
	F(x,y) = U(x,y) + iV(x,y) = F(x+ iy)
\end{align*}
Then we can write the differential as
\begin{align*}
	dF(x,y) &= \frac{\del U}{\del x}(x,y)dx + \frac{\del U}{\del y}(x,y)dy + i \frac{\del V}{\del x}dx + i \frac{\del V}{\del y}dy\\
					&= \left(\frac{\del U}{\del x}(x,y) + i \frac{\del V}{\del x}(x,y)\right)dx + \left(\frac{\del U}{\del y}(x,y) + i \frac{\del V}{\del y}(x,y)\right)dy\\
					&= \frac{\del F}{\del x} dx + \frac{\del F}{\del x} dy
\end{align*}
so when is it equal to $f(z) dz$?. This is then the case if we have
\begin{align*}
	\frac{\del F}{\del x} = \left(\frac{\del U}{\del x}(x,y) + i \frac{\del V}{\del x}(x,y)\right) = \left(u(x,y) + i v(x,y)\right)\\
	\frac{\del F}{\del y} = \left(\frac{\del U}{\del y}(x,y) + i \frac{\del V}{\del y}(x,y)\right) = i\left(u(x,y) + i v(x,y)\right)
\end{align*}
So this is the case when
\begin{empheq}[box=\bluebase]{align*}
	\frac{\del F}{\del y} = i \frac{\del F}{\del x}	\iff \frac{\del U}{\del x} = \frac{\del V}{\del y} \quad \text{and} \quad \frac{\del U}{\del y} = -\frac{\del V}{\del x}
\end{empheq}
which means $F$ satisfies the Cauchy-riemann equations. So $F$ is analytic and we can $f$ in terms of the derivative.
\begin{align*}
	f(z) = \frac{\del F}{\del x}(z) = F'(z)
\end{align*}

So we end up with the following theorem.
\begin{theorem}[]
	Let $\Omega \subseteq C$ be a region, $f$ continous in $\Omega$, then $\int_{\gamma} f(z) dz$ depends only on the endpoints of $\gamma$ if and only if $f$ has a primitive $F: \Omega \to \C$ with $F'(z) = f(z)$
\end{theorem}

For example: for the function $f(z) = z$, is $f(z) dz$ an exact differential?  Homework!\\

Let's say we have a closed curve $\gamma \subseteq \C$ and let $n \geq 0$ and consider the function $f(z) = (z-a)^n$. What can we say about the integral $\int_{\gamma}f(z) dz = \int_{\gamma}(z-a)^n dz$?\\

Since $f$ is analytic and has primitive $\frac{(z-a)^{n+1}}{n+1}$, which is analytic in $\C$ the integral must be zero.

But if $n = -2$, then $(z-a)^{-2}$ is the derivative of $\frac{-1}{z-a}$, which is analytic except for $\C \setminus \{a\}$, but for $n = -1$, it has a primitive, but which isn't analytic. And then take the circle $C$ around $a$. Calculate as an exercise.
\begin{align*}
	\int_C \frac{1}{z-a} dz = ?
\end{align*}

