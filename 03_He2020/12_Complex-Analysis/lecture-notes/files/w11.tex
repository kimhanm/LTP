% Lecture 24.11.20

We saw that there is some topology going when when we asked on which regions $\Omega$ and closed curves $\gamma$ Cauchy's Theorem holds.

It is certainly not true for all open and connected regions. With the punctured disk $\Delta \setminus \{z_0\}$ being a counterexample.

In this course we won't give the real definition of topological terms\footnote{Because why the fuck not?}, but one that work for the $2$-dimensional case in.

\begin{definition}[]
We call a subset $\Omega \subseteq \C$ is simply connected if $\C_{\infty} \setminus \Omega$ is connected, where $\C_{\infty} = \C \cup \{\infty\}$ is the one-point compactification of $\C$.
\end{definition}
The addition of the point infinity is necessary, since for example the real line $\{z \big\vert \Image(z) = 0\}$ should be simply connected.

We have another equivalent definition
\begin{proposition}[]
	$\Omega \subseteq \C$ is simply connected if and only if for all closed curves $\gamma$ and any $a \in \C \setminus \Omega$ the winding number $n(\gamma,a) = 0$ is zero.
\end{proposition}

Proof: Let $\Omega \subseteq \C$ and $\gamma \subseteq \Omega$ be a closed curve. Then take the inside of the curve $\gamma^{\circ}$. Since $\Omega^{c}$ is connected, it must be inside on the of the regions delimited by $\gamma$. So $\gamma \cap \Omega^{c} = \es$.
Therefore for $z_0 \in \Omega^{c}$, it must be that $\Omega^{c}$ is contained in the unbounded region determined by $\gamma$. 
So $\forall a \in \Omega^{c}$, $a \notin \gamma \cup \gamma^{circ}$ and $n(\gamma,a) = 0$

Now assume that $\Omega^{c}$ is not connected and let $A$ and $B$ be to closed disjoint sets such that $A \cap B = \Omega$.
Without loss of generality $A$ must be the bounded one. Since they are disjoint, let $\delta$ be the distance between $A$ and $B$. Then fill $\Omega$ with a net of squares with side length $< \frac{1}{\sqrt{2}}\delta$
Then if we take all squares $Q_j$ such tha $Q_j \cap A \neq \es$, we see that the union of all of these is closed.
So there exists an $a \in \Omega^{c}$ and a curve, such that $n(a,\gamma) \neq 0$

\begin{definition}[]
	A cycle $\gamma \subseteq \Omega$ in a region $\Omega$ is \textbf{homologous} to zero (and we write $\gamma \sim 0 \mod \omega$), if for all $a \in \Omega^{c}: n(\gamma,a) = 0$.

	We write $\gamma_1 \sim \gamma_2$ if $\gamma_1 - \gamma_2 \sim 0 \mod \Omega$.
\end{definition}

\begin{ntheorem}[Cauchy's Theorem (General)]
	Let $\Omega \subseteq \Omega$ be a region, $f: \Omega \to \C$ be analytic. Then for all closed curves $\gamma \sim 0 \mod \Omega$
	\begin{align*}
		\int_{\gamma} f(z) dz = 0
	\end{align*}
\end{ntheorem}
As an immediate corollary we get the following

\begin{corollary}[]
Let $\Omega \subseteq \C$ be simply connected, then Cauchy's Theorem holds.\\

If $\Omega$ is simply connected, then $f: \Omega \to \C$ has a primitive.
\end{corollary}


Proof of Cauchy's Theorem: Let $\Omega \subseteq \C$ be a region, $\gamma \subseteq \Omega$ a closed curve and $f: \Omega \to \C$ analytic.

Since $\gamma$ is a closed curve, we can assume that $\Omega$ is bounded. Then take $\delta > 0$ and fill the space with squares $Q_j$ of side length $\delta >0$. Then the set
\begin{align*}
	J := \{j \big\vert Q_J \subseteq \Omega\}	
\end{align*}
is finite. We then approximate the boundary of $\Omega$ with the boundaries of the squares
\begin{align*}
	\Gamma_{\delta} := \sum_{j \in J} \del Q_J \quad \text{and} \quad \Omega_{\delta} := \text{int} \bigcup_{j \in J} Q_j
\end{align*}
Since $\gamma \subseteq \Omega$, there exists a $\delta > 0$ such that $\gamma \subseteq \Omega_{\delta}$.

Then take some $\xi \in \Omega \setminus \Omega_{\delta}$.Which belogs to a $Q_k$ for some $k \notin J$, i.e $Q_k \not\subseteq \Omega$. Therefore there exists some $\xi_0 \neq \Omega$ such that $\xi_0 \in Q_k$.

If we connect $\xi$ and $\xi_0$ with a line segment that doesn't intersect $\Omega_{\delta}$ and also not with $\gamma$ so we have
\begin{align*}
	n(\gamma,\xi) = n(\gamma,\xi) = 0
\end{align*}

So for all $\xi \in \Gamma_{\delta}: n(\gamma,\xi) = 0$. And for $z \in \text{int}(Q_{j_0})$ we can use Cauchy's integral formula in $\del Q_j$
\begin{align*}
	\frac{1}{2\pi i} \int_{\del Q_j} \frac{f(\xi)}{\xi - z}d \xi = \left\{\begin{array}{ll}
			f(z) & j = j_0\\
		 0 & 
	\end{array} \right.	
\end{align*}

If $\Omega$ were unbounded, it suffices to take a bounded $\Omega_R$ large enough such that $\gamma$ is contained.



% ==== lecture 27.11.20 ====


And what if $\Omega$ is not simply connected? If it has multiple connected regions, then we can say $\Omega^{c}$ has $n$ components.

Let's write $A_{1}, \ldots, A_{n}$ for those components. Then for some curve $\gamma$, the winding number stay contstant for each of those components, i.e
\begin{align*}
	a_1,a_2 \in A_k \implies n(\gamma,a_1) = n(\gamma,a_2)
\end{align*}
and if $a \in A_n$ is in the unbounded component we also have $n(\gamma,a) = 0$, since it is ``outside'' of the curve $\gamma$.
And for each $k$ there exists a curve $\gamma_k$ such that if $a \in A_k$ the winding number is $n(\gamma_k,a) = 1$ and zero in all the other $A_j$s.


Then 
\begin{align*}
	\gamma \sim c_1 \gamma_1 + c_2 \gamma_2 + \ldots + c_n \gamma_n \mod \Omega
\end{align*}

So for $f: \Omega \to \C$ we can write
\begin{align*}
	\int_{\gamma} f(z) dz = \sum_{k=1}^{n-1} c_k \int_{\gamma_k}f(z) dz
\end{align*}
where $A_n$ is the unbounded one which doesn't contribute anything to the integral


\subsection{Calculus of residues}
Let's say $f: \Omega' \to \C$ is anyltic in a simply connected region $\Omega$ except in isloated singularties $a_1, \ldots, a_n$.

Then we want to create a homology basis. Because $\Omega'$ is open
\begin{align*}
	\forall j \exists \delta_j \text{ such that } \overline{B(a_j,\delta_j)} \setminus \{a_j\} \subseteq \Omega'
\end{align*}
then define $C_j := \del B(a_j, \delta_j)$ and
\begin{align*}
	P_j := \int_{C_j} f(z) dz
\end{align*}
The goal is we want some way to compute the $P_j$s.

If $f(z) = \frac{1}{z - a_j}$, then
\begin{align*}
	\int_{C_j} \frac{1}{z- a_j}dz = 2\pi i	
\end{align*}
Then define $R_j := \frac{P_j}{2\pi i}$ then we know that
\begin{align*}
	\int_{C_j}f(z) - \frac{R_j}{z - a_j} dz = 0
\end{align*}

\begin{definition}[]
	Let $f: \Omega \setminus \{a_j\}_{j \in J} \to \C$ be holomorphic.\\
	The \textbf{residue} of $f$ at the isolated singularity $a_j$ is the complex number $R_j$ such that the function
	\begin{align*}
		g_j(z) := f(z) - \frac{R}{z - z_j}
	\end{align*} 
	is the derivative of a function that is analytic in $B(a_j,\epsilon)$ for some $\epsilon > 0$.
\end{definition}
Often, the residues will be really easy to compute and once we know the residues then we can compute the integrals.

We also write
\begin{align*}
	R_j = \text{Res}_{z = a_j}f(z)
\end{align*}
If $\gamma$ is a circle around $a_j$ with a radius small enough, then
\begin{align*}
	R_j = \frac{1}{2\pi i} \int_{\gamma}f(z) dz
\end{align*}




