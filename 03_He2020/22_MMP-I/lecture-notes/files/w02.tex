\begin{satz}[Fubini]
Sei $f$ eine messbare Funktion auf $E \times F$.
\begin{enumerate}
				\item Falls $f \geq 0$, so sind $\phi(y),\psi(x)$ messbar und es gilt
\begin{align*}
				\int_F\phi(y)dy = \int_F\int_Ef(x,y)dx dy = \int_E\int_F f(x,y) dy dx = \int_E\psi(x)dx \quad \in [0,\infty]
\end{align*}
	\item Ist $f$ komplexwertig und ist 
\begin{align*}
				\int_E \left(\int_F \abs{f(x,y)} dy\right) dx < \infty \quad \text{oder} \quad \int_F \left(\int_E \abs{f(x,y)} dx\right)dy < \infty
\end{align*}
so ist $f$ integrierbar und
\begin{align*}
	\int_{E \times F} \abs{f(x,y)}dx dy < \infty
\end{align*}

	\item Ist $f$ komplexwertig und über $E \times F$ integrierbar, so ist $f_x$, bzw. $f_y$ für fast alle $x$, bzw. $y$ integrierbar und $\phi$ und $\psi$ sind integrierbar und es gilt

\begin{align*}
	\int_F\phi(y)dy = \int_F\int_Ef(x,y)dx dy = \int_E\int_F f(x,y) dy dx = \int_E\psi(x)dx \quad \in [0,\infty]
\end{align*}
\end{enumerate}
\end{satz}

Bemerkung: Insbesondere ist z.B.
\begin{align*}
				\int_{R^n}f(x_1, \ldots, x_n) dx_1 \dots dx_n = \int_R \dots \left(\int_R f(x_1, \ldots, x_n) dx_1\right)\dots dx_n
\end{align*}
falls $f$ auf $\R^n$ integrierbar ist.


\subsection{$L^p$ Räume}

Wir möchten alle integrierbaren Funktionen nehmen und sie als Vektorraum betrachten.
\begin{definition}[]
	Zwei integrierbare Funktion $f,g: E \to \C$ mit $E \subseteq \R^n$ messbar, heissn \textbf{äquivalent} (schreibe $f \sim g$, falls für fast alle $x \in E$ gilt $f(x) = g(x)$.
\end{definition}

Dies ist eine Äquivalenzrelation und die Menge der Äquivalenzklassen heisst
\begin{align*}
				L^1(E) = \{f \text{ messbar}\big\vert \int_E \abs{f(x)} dx < \infty\}_{/\sim}
\end{align*}

Allgemeiner definiern wir für $p \geq 1$ die Menge
\begin{align*}
				L^p(E) := \{f \text{ messbar} \big\vert \int_E \abs{f(x)}^p dx < \infty\}_{/\sim}
\end{align*}


\begin{lemma}[]
				Für $f,g \in L^p(E), \lambda \in \C$ ist $f + \lambda g \in L^p(E)$. ALso bildet $L^p(E)$ einen Vektorraum.
\end{lemma}

Der Grund wieso man die äquivalenten Funktionen herausteilen möchte, ist damit man einen gescheiten Normbegriff haben. Dieser wird für die positive Definitheit gebraucht.
\begin{satz}[]
Der $L^p(E)$ Raum mit der Norm
\begin{align*}
				\|f\|_p := \left(\int_E\abs{f(x)}^p dx\right)^\frac{1}{p}
\end{align*}
ist ein normierter \textbf{Vektorraum} 
\end{satz}


Erinnerung Analysis:
\begin{itemize}
\item 	Eine Folge $(f_i)$ in einem normierten Vektorraum $V$ konvergiert gegen $f \in V$, falls $\lim_{i \to \infty}\|f_i - f\| = 0$.
\item 	Die Folge heisst \textbf{Cauchy-Folge} , falls $\forall \epsilon > 0 \exists N \in \N$ mit $\|f_i - f_j\| < \epsilon, \forall i,j > N$
\item	$V$ heisst \textbf{vollständig} oder \textbf{Banachraum} , falls jede Cauchyfolge konvergiert.
\end{itemize}

\begin{satz}[Fischer-Riesz]
Für alle $p \geq 1$ ist $L^p(E)$ ein Banachraum.
\end{satz}

In der Physik ist der Fall für $p = 2$ sehr wichtig, da $L^2(E)$ ein Hilbertraum bildet, und in der Physik sehr viele Dinge durch Hilberträume beschrieben werden.\\


Wir möchten die integrierbaren Funktion durch stetige Funktionen anhähern. Wir werden später noch zeigen, dass die stetigen Funktionen mit kompakten Träger dicht sind in $L^p(E)$.\\

Da in der Physik die meisten zu betrachtenden Mengen ``schön'' sind, werden wir der Menge $E$ einige Eigenschaften zuschreiben. Sei von nun $E \subseteq \R^n$ lokal kompakt, d.h. jeder Punkt von $E$ hat eine kompakte Umgebung, d.h. $x \in U \subseteq K \subseteq E$ für $U$ offen, $K$ kompakt.\\

Zum Beispiel sind endliche Durchschnitte und Vereinigungen von offenen oder Abgeschlossenen Teilmenge von $\R^n$ lokal kompakt. Aber zum Beispiel ist $\Q \subseteq \R$ nicht lokal kompakt.\\

\begin{definition}[]
	Der \textbf{Träger} (engl. \emph{suppport}) einer Funktion $f: E \to \C$ ist die Menge
\begin{align*}
				\supp(f) := \overline{\{x \in E \big\vert f(x) \neq 0\}}
\end{align*}
Wir schreiben $C_0(E)$ für die Menge aller stetigen Funktionen $f: E \to \C$ mit kompakten Träger.
\end{definition}


\begin{satz}[]
	Für lokal kompakte $E \subseteq \R^n$ (messbar) und $1 \leq p < \infty$ ist $C_0(E)$ dicht in $L^p(E)$, d.h. $\forall \epsilon > 0 \exists g \in  C_0(E)$ mit $\|f - g\| < \epsilon$
\end{satz}




% ==== Fourierreihen ====
\section{Fourierreihen}
\subsection{Definitionen, Darstellungssatz}

Sei $L > 0$ fix. \textbf{Fourierreihen} sind Reihen der Form
\begin{align*}
	 \sum_{n = - \infty}^{\infty} f_n e^{\frac{2\pi in}{L}x}, \quad f_n \in \C, x \in \R
\end{align*}
Konvergiert die Reihe für alle $x \in \R$, so ist die Summe $f(x)$ $L$-periodisch, also $f(x + L) = f(x), \forall x \in \R$.\\

Man kann sich sicher folgende Fragen stellen.
\begin{itemize}
\item Wann konvergiert diese Reihe?
\item Welche $L$-periodischen Funktionen lassen sich durch eine Fourierreihe darstellen und wie erhält man die Fourierkoeffiziente $f_n$?
\item Welche Anwendungen und welche Intuition hat man über die Fourierreihe?
\end{itemize}


\begin{satz}[]
				Sei $\{f_n\}_{n \in \Z}$ so, dass $\sum_{n \in \N} \abs{f_n} < \infty$. Dann konvergiert auch die Fourierreihe $\sum_{n = -\infty}^{\infty}f_n e^{\frac{2\pi in}{L}x}$ absolut und gleichmässig (in der $\|\cdot\|_\infty$-Norm) gegen eine $L$-periodische stetige Funktion $f$. Weiter gilt
\begin{empheq}[box=\bluebase]{align*}
				f_n = \frac{1}{L} \int_0^L e^{- \frac{2\pi in}{L}x}f(x) dx
\end{empheq}
\end{satz}

Beweis: \quad Die Absolute konvergenz folgt, da $\abs{e^{\frac{2\pi in}{L}x}} = 1$. Die Gleichmässige Konvergenz folgt aus
\begin{align*}
				\abs{f(x) - \sum_{\abs{n} \leq N} f_n e^{\frac{2\pi in}{L}x}} = \abs{\sum_{\abs{n} \geq N} f_n e^{\frac{2\pi in}{L}x}} \leq \sum_{\abs{n} \leq N}\abs{f_n} \to_{n \to \infty} 0
\end{align*}
gleichmässig in $x$. Da $f(x)$ der gleimässige Grenzwert von einer folge von stetigen Funktion $f_N(x)$ ist, ist $f$ auch selber stetig.\\
Zuletzt berechnen wir
\begin{align*}
				\frac{1}{L}\int_{0}^{L}e^{-\frac{2\pi in}{L}x} \lim_{N \to \infty} \sum_{\abs{k} \leq N} f_k e^{\frac{2\pi ik}{L}x}dx = (\ldots)
\end{align*}
Betrachten wir die einzelnen Glieder der folge in $N$, so sieht man, dass sie majorisiert wird durch $\sum_{k = -\infty}^{\infty}\abs{f_k} < \infty$. Also können wir den Limes und das Integral austauschen und erhalten 
\begin{align*}
		(\ldots) =	\lim_{N \to \infty} \sum_{\abs{k} \leq N}f_n \frac{1}{L} \int_{0}^{1}e^{\frac{2\pi i}{L}x(k-n)}dx = \lim_{N \to \infty}\sum_{\abs{k} \leq N}f_n \delta_{k,n} = f_n
\end{align*}


Beispiel: Für $\abs{z} < 1$ so ist die folgende Reihe absolut konvergent
\begin{align*}
\sum_{n = -\infty}^{\infty} z^{\abs{n}} e^{inx}
\end{align*}
Die Summe ist die geometrische Reihe und wir haben
\begin{align*}
				\sum_{n = -\infty}^\infty z^{\abs{n}}e^{inx} &= \sum_{n = 0}^{\infty} \left(ze^{ix}\right)^n + \sum_{n = - \infty}^{0} \left(\frac{1}{z}e^{ix}\right)^n -1 \\
	 &= \frac{1}{1 - ze^{ix}} + \frac{1}{1 - ze^{-ix}} - 1 = \frac{1 - z^2}{1 2z \cos x + z^2}
\end{align*}
Was uns den Koeffizient $f_n$ liefert:
\begin{align*}
				f_n = z^{\abs{n}} = \frac{1}{2\pi} \int_{0}^{2\pi} \frac{(1-z^2)e^{inx}}{1 - 2z \cos x + z^2}dx
\end{align*}



\subsection{Riemann-Lebesgue Lemma (allgemeine Version}

\begin{satz}[Riemann-Lebesgue Lemma]
	Sei $f \in L^1(\R)$. Dann gilt
	\begin{align*}
					\lim_{k \to \pm \infty} \int_{\R} f(x) e^{ikx}dx = 0
	\end{align*}
\end{satz}
Zum Beweis brauchen wir folgendes Lemma:
\begin{lemma}[]
				Sei $f \in L^1(\R)$. Dann gilt
\begin{align*}
				\lim_{t \to 0} \int_R \abs{f(x+t) - f(x)}dx = 0
\end{align*}
\end{lemma}
Sie besagt, dass die um $t$ verschobene Funktion gegen $f$ konvergiert (in der $\|\cdot\|_1$.\\
Beweis Lemma: \quad Wir machen das in zwei Schritten. Zuerst betrachten wir eine schwache version und nehmen an, $f$ sei stetig und mit kompakten Träger.\\
Sei nun $M = \max_{x \in \R} \abs{f(x)}$ (wohldefiniert da stetig, kompakter träger) und $\supp(f) \subseteq [-R,R]$. Dann gilt
\begin{itemize}
				\item $\abs{f(x+t) - f(x)} < 2M$
				\item $\abs{f(x+t) - f(x)} = 0$ für $\abs{x} > 2R$, falls $t < R$.
\end{itemize}
Dann wird also die Funktion majorisiert durch $2M \chi_{[-2R,2R]} \in L^1(\R)$ und nach dem Satz der majorisierten Konvergenz haben wir
\begin{align*}
				\lim_{t \to 0}\int_R \abs{f(x+t) - f(x)}dx = \int_{\R} \lim_{t \to 0} \abs{f(x + t) - f(x)}dx = 0
\end{align*}
da $f$ stetig ist.\\
Als nächstes zeigen wir es auch für beliebige Funktionen in $L^1(\R)$. Sei nun $f \in L^1(\R)$ und $ \epsilon > 0$ gegeben. Da die stetigen Funktionen mit kompakten Träger dicht sind in $L^1(\R)$, gibt es ein $g \in C_0(\R)$ sodass $\|f-g\|_1 < \epsilon$. Dann gilt
\begin{align*}
				\int_R \abs{f(x+t) - f(x)}dx = \|f_t - f\|_1 \leq \underbrace{\|f_t - g_t\|_1}_{= \|f - g\|_1 < \epsilon} + \underbrace{\|g_t - g\|_1}_{\text{Beweis für $C_0(\R)$}} + \|g - f\|_1 < 3 \epsilon 
\end{align*}

Nun kommen wir zum Beweis vom Riemann-Lebesgue Lemma. Es gilt
\begin{align*}
				\int_R f(x) e^{ikx} dx &= - \int_R f(x) e^{i(kx - \pi)} dx\\
															 &\stackrel{k \neq 0}{=} - \int_R f(x) e^{ik(x - \frac{\pi}{k}}dx\\
															 &= - \int_{\R} f(x + \frac{\pi}{k}) e^{ikx}dx
\end{align*}
Nun verwenden wir einmal die Form am Anfang und einmal die Form am schluss der Obigen gleichungen und erhalten
\begin{align*}
	\abs{2 \int_Rf(x) e^{ikx}dx} = \abs{\int_{\R}e^{ikx}\left(f(x) - f(x + \frac{\pi}{k}\right) dx}\leq \int_{\R} \abs{f(x) - f(x + \frac{\pi}{k})} dx \to 0
\end{align*}
nach dem Lemma.\\

Die Intuition des Beweises ist das wenn das $k$ gross wird, oszilliert der Integrand sehr schnell und man addiert und subtrahiert praktisch gleich viel.

