\section*{Organisation}
Chefassistent: Andrea Nützi, \href{mailto:andrea.nuetzi@math.ethz.ch}{andrea.nuetzi@math.ethz.ch}\\

Für gelöste Übungen gibt es einen Notenbonus von bis zu 0.25 Noten falls mehr als $40 \%$ sinnvoll bearbeitet sind.


In der MMP I werden vorallem Analytische Methoden der mathematischen Physik behandelt. Genauer werden folgende Themen besprochen:

\begin{itemize}
				\item Masstheorie, Lebesgue Integral (Skizzenhaft, d.h. viele Beweise werden ausgelassen)
	\item Fouriertheorie
	\item Hilberträume, Eigenwertprobleme + spezielle Funktionen
	\item	Distributionen 
	\item Randwertprobleme, Dirichletproblem
\end{itemize}

In der MMP II werden vorallem Algebraische Methoden betrachtet, das an Gruppen- und Darstellungstheorie anscheidet. Dort geht es darum, Symmetrien auszunutzen um nützliche Aussagen über Physikalische Probelme zu machen.

Immer wieder werden wir Beispiele aus der Physik anschauen.


Als Hautpreferenz ist der Skript von G. Felder


\section{Mass- und Integrationstheorie}
\subsection{Masstheorie}
Diese Thema befasst sicht damit, Teilmengen von einer Mengen ein ``Volumen''. Leider ist es nicht immer möglich, jeder Teilmenge ein gescheites Volumen zuzuordnen und man muss sich darauf beschränken, nur geweisse Teilmenge anzuschauen. Dazu dient die erste Definition

\begin{definition}[$\sigma$-Algebra]
				Sei $X$ eine Menge. Eine Teilmange $ \mathcal{A} \neq 0$ von $ \mathcal{P}(X)$ heisst \textbf{$\sigma$-Algebra}, falls gilt
			\begin{itemize}
			\item	$A \in \mathcal{A} \implies A^c := X \setminus A \in \mathcal{A}$
			\item Für jede Folge $(A_n)_{n=1}^{ \infty}$ in $ \mathcal{A}$ ist $\bigcup_{n=1}^{\infty} A_n \in \mathcal{A}$
			\end{itemize}
\end{definition}

Bemerkung: Damit gilt auch für beliebige $A, B \in \mathcal{A}$
\begin{itemize}
\item $A \cup B \in \mathcal{A}$, und $A \cap B = \left(A^c \cup B^c\right)^c \in \mathcal{A}$
\item	$A \setminus B = A \cap B^c \in \mathcal{A}$
\item $ \bigcap{n=1}^{\infty} A_n \in \mathcal{A}$, falls $A_n \in \mathcal{A}, \forall n$
\end{itemize}

\begin{definition}[Mass]
Ein Mass auf einer $\sigma$-Algebra $ \mathcal{A}$ ist eine Funktion $\mu: \mathcal{A} \to [0,\infty]$, sodass gilt
\begin{itemize}
				\item $\mu(\es) = 0$
				\item Für jede Folge $ \left(A_{n}\right)_{n = 1}^{\infty}$ mit $A_i \in \mathcal{A} \forall j$, und $A_i \cap A_j = \es, \forall i \neq j$ gilt
		\begin{align*}
						\mu( \bigcup_{i=1}^{\infty}A_i = \sum_{i = 1}^{\infty} \mu(A_i) 
		\end{align*}
\end{itemize}
\end{definition}


Beispiel: (Zählmass). Sei $ \mathcal{A} = \mathcal{P}(X) = \{A| A \subseteq X\}$. Dann ist $ \mathcal{A}$ eine $\sigma$-Algebra und
\begin{align*}
\mu: \mathcal{A} \to [0,1], \mu(A) = 
\begin{cases}
	\abs{A}, \text{falls} A \text{endlich}\\
	\infty, \text{sonst}
	\end{cases}
\end{align*}
Ist ein Mass.

\begin{lemma}[]
Sei $\mu$ ein Mass auf der $\sigma$-Algebra $ \mathcal{A}$. Dann gilt:
\begin{enumerate}
				\item $A, B \in \mathcal{A}, A \subseteq B \implies \mu(A) \leq \mu(B)$
				\item 	Sei $ \left(A_{i}\right)_{i \geq 1}$ eine Folge in $ \mathcal{A}$, dann ist $\mu\left( \bigcup_{i = 1}^{\infty} A_i\right) \leq \sum_{i = 1}^{\infty} \mu(A_i)$
\end{enumerate}
\end{lemma}

Für einen Quader $R = [a_1, b_1] \times \dots \times [a_n, b_n] \subseteq \R^n$ definieren wir das \textbf{Volumen}
\begin{align*}
	\abs{R} := (b_1 - a_1) \dots (b_n - a_n)
\end{align*}


\begin{definition}[]
	Sei $A \subseteq \R^n$ eine Teilmenge. Dann heist
	\begin{align*}
					\mu^*(A) := \inf \{ \sum_{i = 1}^{\infty} \abs{R_i} \big\vert R_i \subseteq \R^n \text{ Quader mit } \bigcup_{i=1}^{\infty}R_i \supseteq A\}
	\end{align*}
	das \textbf{Lebesgue'sche äussere Mass} von $A$.
\end{definition}
Achtung: $\mu^*$ ist kein Mass, sonder etwas schwächeres. Ein äussere Mass welches folgende Axiome erfüllt:
\begin{itemize}
				\item $\mu^*(\es) = 0$
				\item $A \subseteq B \implies \mu^*(A) \leq \mu^*(B)$
				\item $\mu^* \left( \bigcup_{i=1}^{\infty} A_i) \right) \leq \sum_{i = 1}^{\infty} \mu^*(A_i)$
\end{itemize}

Es ist möglich um von einem äusseren Mass ein Mass zu bekommen. Dazu definieren wir folgendes:
\begin{definition}[Lebesgue-Nullmenge]
				Eine Menge $A \subseteq \R^n$ heisst \textbf{(Lebesgue-)Nullmenge}, falls $\mu^*(A) = 0$.\\
				Wir sagen eine Eigenschaft $P(x)$ von Punkten $x \in \R^n$ gilt \textbf{fast überall}, (f.ü.), falls
				\begin{align*}
								\mu^* \left( \{ x \in \R^n \big\vert \lnot P(x) \} \right) = 0	
				\end{align*}
\end{definition}

\begin{definition}[Lebesgue-mesbar]
Eine Menge $E \subseteq \R^n$ heisst \textbf{(Lebesgue-)messbar}, falls für alle $A \subseteq \R^n$ gilt:
\begin{align*}
				\mu^*(A) = \mu^*(A \cap E) + \mu^*(A \cap E^c)
\end{align*}
\end{definition}

\begin{satz}[]
\begin{enumerate}
\item Die Lebesgue messbaren Menge bilden eine $\sigma$-Algebra $ \mathcal{A}$.
\item Die Einschränkung von $\mu^*$ auf $ \mathcal{A}$ ist ein Mass, das \textbf{Lebesgue'sche Mass} $\mu$
\item Alle offenen und alle abgeschlossenen Teilmenge von $\R^n$ sind Lebesgue-messbar und $\mu(R) = \abs{R}$ für alle Quader $R \subseteq \R^n$. 
\item Nullmengen sind messbar und haben Mass $0$.
\item Ist $E \subseteq \R^n$ messbar, so ist die Einschränkung von $\mu$ auf die messbaren Teilmenge von $E$ wieder ein Mass, das Lebesgue'sche Mass auf $E$.
\end{enumerate}
\end{satz}
Beweis Übung! 


\section{Das Lebesgue'sche Integral}
 
Sei $E \subseteq \R^n$ messbar und $\mu$ das Lebesgue'sche Mass auf $E$.


\begin{definition}[]
Wir sagen eine Funktion $f: E \to \R$ heisst \textbf{messbar}, falls für jedes Intervall $I \subseteq \R$ 
\begin{align*}
				f^{-1}(I) = \{x \in E \big\vert f(x) \in I\} \subseteq E
\end{align*}
messbar ist. Analog im komplexen heisst $f: E \to C$ messbar, falls $ \text{Re}(f)$ und $ \text{Im}(f)$ messbar sind.
\end{definition}

Wie beim Riemann integral möchten wir zuerst die Treppenfunktionen anschauen und diese verallgemeinern.

\begin{definition}[]
				Die \textbf{charakteristische Funktion} einer Menge $A \subseteq \R^n$  ist die Funktion $\chi_A: \R^n \to \{0,1\}$ gegeben durch
	\begin{align*}
		\chi_A(x) = \left\{\begin{array}{rcl}
						1, &\text{falls}& x \in A, \\
						0, &\text{sonst}&  
		\end{array} \right.
	\end{align*}
	Funktionen der Form $\phi(x) = \sum_{i = 1}^{m} \lambda_i \chi_{E_i}(x)$ mit $\lambda_i \in \R$ mit $E_i \subseteq \R^n$ \emph{mesbar} heissen \textbf{einfach}.
\end{definition}

\begin{definition}[]
				Sei $\phi(x) = \sum_{i = 1}^{m}\lambda_i \chi_{E_i}(x)$ eine einfache Funktion mit $\mu(E_i) < \infty, \forall i$, dann definieren wir mit $E :=  \bigcup_{i=1}^{m} E_i$ das \textbf{Integral} von $\phi$ als
	\begin{align*}
					\int_E \phi(x) dx := \sum_{i = 1}^{m} \lambda_i \mu(E_i)
	\end{align*}
	Falls alle $\lambda_i > 0$ sind und ein $\mu(E_i) = \infty$, dann setzen wir $\int_E \phi(x) dx := \infty$.
\end{definition}
Hier muss man aufpassen, dass diese Definition unabhängig von der Darstellung von $\phi$, bzw. der Zerlegung in die $E_i$ abhängig ist.\\
Es sei bemerkt, dass Treppenfunktionen im Sinne von Analysis I/II einfache Funktionen sind. Sei $\phi: [a,b] \to \R$ eine Treppenfeunktion, $\phi = \sum_{i = 1}^{m} \lambda_i \chi_{[a_i,b_i]}$. Dann ist
\begin{align*}
				\int_{[a,b]} \phi(x) dx = \sum_{i = 1}^{m}\lambda_i(b_i - a_i)
\end{align*}
also gleich dem Riemann'schen Integral von $\phi$.\\
Daraus folgt auch, dass alle Riemann-Integrierbare Funktionen auch Lebesgue-Integrierbar sind. Umgekehrt aber, gibt es aber Funktionen, die jetzt neu Lesbesgue-Integrierbar sind.\\

Ein Punkt ist ein Quader mit Seitenlänge $0$, also ist $\{x\} \subseteq \R$ messbar mit Mass $0$. Insbesondere ist jede abzählbare Teilmenge von $\R$ messbar.\\
Weil eben $\mu(\Q) = 0$ ist, haben wir, dass die Dirichelet-Funktion, welche gleich $\chi_{\Q}$ integrierbar ist (mit Integral gleich Null).
\begin{lemma}[]
				Seind $\phi_1, \phi_2 : E \to \R$ einfach mit $\phi_1(x) \leq \phi_2(x), \forall x \in E$, so ist
		\begin{align*}
			\int_E \phi_1(x) dx \leq \int_E \phi_2(x) dx
			\end{align*}
\end{lemma}



\begin{definition}[]
				Sei $f: E \to [0,\infty]$ messbar. Dann existiert eine Folge $\phi_1, \phi_2, \ldots$ von einfachen Funktionen mit $0 \geq \phi_1(x) \leq \phi_2(x) \leq \ldots, \forall x \in E$ mit $\lim{i \to \infty}(x) \phi_i(x) = f(x), \forall x \in E$.\\
Wegen der Monotonie des Integrals ist dann auch die Folge der Integrale monoton. In diesem Fall ist dieser Grenzwert 
	\begin{align*}
					\int_E f(x) dx := \lim{i \to \infty}\int_E \phi_i(x) dx \in [0,\infty]
	\end{align*}
	unabhängig von der Wahl der Folge $\phi_i$. Falls $\int_E f(x) dx < \infty$, so heisst $f$ \textbf{Lebesgue-integrierbar}, und wir nennen $\int_Ef(x)dx$ das \textbf{Lebesgue-integral} von $f$.
\end{definition}

Um auch Funktionen, die negative Werte annehmen, zu integrieren erweitern wir den Begriff wie folgt:

\begin{definition}[]
				Sei $f: E \to \R$ messbar und seien $f_{\pm} := \max\{0, \pm f(x)\}$, so dass $f = f_+ - f_-$. Dann heisst $f$ \textbf{Lebesgue-integrierbar}, falls $f_+$ und $f_-$ Lesbesgue integrierbar sind und wir schreieben für sein Lebesgue integral:
	\begin{align*}
					\int_E f(x) dx := \int_Ef_+(x) dx - \int_Ef_-(x)dx
	\end{align*}
	Komplex-wertige Funktionen $f: E \to \C$ heissen integirerbar, falls $\text{Re} f$ und $\text{Im}f$ integierbar sind und wir definieren wir das Integral als
	\begin{align*}
					\int_Ef(x)dx := \int_E \text{Re}f(x) dx + i \int_E \text{Im}f(x)dx
	\end{align*}
\end{definition}

\begin{lemma}[]
				Sei $f: E \to \C$ messbar. Dann ist $\abs{f}: E \to \R$ messbar und $f$ ist genau dann integrierbar, wenn $\abs{f}$ integrierbar ist, also
	\begin{align*}
					\int_E \abs{f(x)}dx < \infty
	\end{align*}
\end{lemma}
Es gibt wie immer verschiedene Arten das Integral zu schreiben. Hier sind verschiedene wege aufgelistet:

\begin{align*}
				\int_E f(x) dx = \int_E f dx = \int_E f = \int_E f(x_1, \ldots, x_n) d^nx = \int_e f(x_1, \ldots, x_n) d(x_1, \ldots, x_n)
\end{align*}

Die Rechenregeln des Lebesgue-integrals sind ähnlich wie beim Riemann-integral:

\begin{satz}[Rechenregeln Lesbesgue-Integral]
Seien $f,g: E \to \C$ integrierbar und $\alpha, \beta \in \C$. Dann gilt
\begin{enumerate}
\item $\alpha f + \beta g$ ist integrierbar und $\int_E(\alpha f + \beta g) = \alpha \int_Ef + \beta \int_E g$
\item $f \leq g \implies \int_E f \leq \int_E g$
\item $f(x) = g(x)$ fast überall $\implies \int_E f = \int_E g$
\item $\int_E \abs{f(x)} dx = 0 \Leftrightarrow f(x) = 0$ fast überall.
\item $\abs{\int_E f(x) dx} \leq \int_E \abs{f(x)}dx$
\item Ist $D \subseteq E$ messbar, so ist die Einschränkung $f|_D$ messbar und $\int_D f(x) dx = \int_E f(x) \chi_D(x) dx$
\item Ist $f$ Riemann-integrierbar auf einem kompakten Intervall $[a,b]$, so ist $f$ auch Lebesgue-integrierbar und ihre Integrale sind gleich. 
\item	Sei $x \mapsto Ax + b$ eine affin lineare Transformation des $\R^n$, mit $A \in \text{Gl}(n,\R)$. Dann ist $x \mapsto f(Ax + b)$ integierbar und $\int_{\R^n} f(x) dx = \abs{\det A} \int_{\R^n}f(Ax+b) dx$
\end{enumerate}
\end{satz}

\subsection{Konvergenzsätze}

\begin{satz}[der monotonen Konvergenz]
Sei $\left(f_{i}\right)_{i = 1}^{\infty}$ eine Folge von integrierbaren Funktionen auf $E$ mit $0 \leq f_1(x) \leq f_2 \leq \ldots$ und $f_i(x) \to f(x)$ konvergiert punktweise für alle $x \in E$. \\
Ist die Folge $\int_E f_i(x)dx$ beschränkt, so ist $f(x)$ integrierbar und es gilt
\begin{align*}
		\lim_{i \to \infty}\int_Ef_i(x)dx = \int_E \lim_{i \to \infty}f_i(x)dx = \int_Ef(x)dx
\end{align*}
\end{satz}

Der folgende Satz, im Englischen ``Dominated Convergence Theorem (DCT)'' genannt, wird Lebesgue zugeschrieben.
\begin{satz}[von der majorisierten Konvergenz]
Sei $\left(f_{i}\right)_{i = 1}^{\infty}$ eine Folge integrierbarer Funktionen auf $E$ mit $\lim_{i \to \infty}f_i(x) = f(x), \forall x  \in E$ und es existiere eine integierbare Funktion (\textbf{Majorante}  genannt) $g$ mit $\abs{f_i(x)} \leq g(x), \forall x \in E, i \in \N$.\\
	Dann ist $f$ integrierbar und es gilt
\begin{align*}
	\lim_{i \to \infty}\int_Ef_i(x)dx = \int_E \lim_{i \to \infty} f_i(x)dx = \int_Ef(x)dx
\end{align*}
\end{satz}


\textbf{Beispiel:} Berechne $\lim_{n \to \infty} \int_0^1 \frac{dx}{1 + \sqrt[n]{x}}$. Hier kann man die Majorante $g(x) = 1$ benutzen und somit den Limes in das Integral reinziehen:
\begin{align*}
	\lim_{i \to \infty} \int_0^1 \frac{dx}{1 + \sqrt[n]x} = \int_0^1 \lim_{i \to \infty} \frac{1}{1 + \sqrt[n]x}dx = \frac{1}{2}
\end{align*}

\subsection{Satz von Fubini}

Seien $E \subseteq \R^n, F \subseteq \R^m$ messbar.

\begin{lemma}[]
Sei $f$ eine messbare Funktion auf $E \times F$.
Dann sind die Funktionen $f_y: x \in E \mapsto f(x,y)$, bzw. $f_x: y \mapsto f(x,y)$ für fast alle $y$ bzw. $x$ messbar und
\end{lemma}


Wir können also, falls $f \geq 0$ die Integrale über $x$ bzw $y$ betrachten
\begin{align*}
	\phi(y) = \int_E f(x,y) dx \quad \text{und} \quad \psi(x) = \int_F f(x,y) dy
\end{align*}
zumindest für fast alle $y$ bzw. $x$.\\
