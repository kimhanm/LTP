
\subsection{Dirichlet Kern, erste Konvergenzsätze}

Sei nun $f$ eine Integrierbare Funktion auf $[0,L]$ Dann definieren Wir die Fourierkoeffizienten:
\begin{align*}
	f_n := \frac{1}{L}\int_{0}^{L} e^{-\frac{2\pi i n}{L}x}f(x) dx
\end{align*}


\begin{nsatz}[Variante des Riemann-Lebesgue Lemmas (Korollar)]
Es gilt für alle integrierbaren $f$ wie oben, dass 
\begin{align*}
f_n \to 0 \quad \text{für} \quad n \to \pm \infty
\end{align*}
\end{nsatz}
Beweis: Wende das Riemann-Lebesgue Lemma oben auf die erweiterte Funktion
\begin{align*}
	\tilde{f}(x) = \left\{\begin{array}{rcl}
			f(x), & x \in [0,L], \\
	    0 & \text{sonst}  
	\end{array} \right.
\end{align*}


\begin{korollar}[]
	Sei $f \in C^k([\R/L\Z)]$, d.h. $L$ periodisch und $k$-fach stetig diffbar.\\
	Dann gilt für die Fourierkoeffizienten
	\begin{align*}
	\abs{n}^k	\abs{f_n} \to 0 \quad \text{für} \quad n \to \pm \infty
	\end{align*}
\end{korollar}
Beweis: Durch $k$-fache partielle Integration, wobei bei $0$ und $L$ wegen der Periodizät ignoriert werden können. Für $n \geq 0$ haben wir
\begin{align*}
	f_n = \frac{1}{L} \int_{0}^{L}e^{-\frac{2\pi in }{L}x}f(x) dx = \frac{(-1)^k}{L} \left(- \frac{L}{2\pi i n}\right)^k \int_{0}^{L}e^{- \frac{2\pi i n}{L}x}f^{(k)}(x) dx = \left(\frac{L}{2 \pi i n}\right)^k g_n
\end{align*}
mit $g_n$ dem $n$-ten Fourierkoffizieten von $f^{(k)}$. Also haben wir nach dem Riemann Lebesgue Lemma, dass
\begin{align*}
	\abs{n}^k \abs{f_n} = \abs{\frac{L^k}{(2\pi)^k}} \abs{g_n} \to 0
\end{align*}

Wir betrachten nun die Konvergenz der Fourierreihe und dazu die Partialsummen
\begin{align*}
	(S_Nf)(x) = \sum_{n = -N}^{N}f_n e^{\frac{2\pi in}{L}x} = \frac{1}{L} \int_{0}^{L}f(y) \sum_{n = -N}^{N}e^{\frac{2\pi in}{L}(x-y)}dy
\end{align*}
Wir wollen du Summe im Integranden genauer betrachten:
\begin{definition}[Dirichletkern]
	Die Funktion $D_N(t) = \sum_{n = -N}^{N}e^{2\pi i n t}$ heisst \textbf{Dirichletkern}.
\end{definition}

\begin{lemma}[]
$D_n$ erfüllt
\begin{enumerate}
	\item $D_N(t + 1) = D_N(t) = D_N(-t)$
	\item $\int_{0}^{1}D_N(t) dt = 1$
	\item $D_N(t) = \left\{\begin{array}{ll}
		2N + 1 & t \in \Z \\
		\frac{\sin \left((2N + 1)\pi t\right)}{\sin(\pi t)} & t \notin \Z
	\end{array} \right.$
\end{enumerate}
\end{lemma}

Beweis, die ersten beiden Aussagen sind klar. Die dritte Aussage ist für $t \in \Z$ klar. Für $t \notin \Z$ hat man die geometrische Summe
\begin{align*}
	D_N &= e^{- 2\pi i N t}\sum_{n = 0}^{2N} e^{2\pi i n t}\\
			&=e^{2\pi i Nt} \frac{1 - \left(2^{2\pi i t}\right)^{2N + 1}}{1 - e^{2\pi i t}}\\
			&= \frac{e^{-2\pi iNt} - e^{2\pi i(N+1)t}}{1 - e^{2\pi i t}}\\
			&= \frac{e^{-\pi i t(2N+1)t} - e^{\pi i t(2N + 1)}}{e^{-\pi it}- e^{\pi it}}\\
			&= \frac{\sin((2N + 1)\pi t)}{\sin(\pi t)}
\end{align*}

Wir können uns den Dirichletken $D_N$ als oszillierende Funktion vorstellen, die im Limes $N \to  \infty$ gegen die Dirichlet delta distribution konvergiert.\\


\begin{satz}[]
	Sei $f \in C^1(\R/L\Z)$ unf $f_n$ die Fourierkoeffizienten von $f$. Dann konvergiert die Fourierreihe punktweise gegen die Funktion, d.h. für alle $x \in \R$ gilt
	\begin{align*}
	f(x) = \lim_{N \to \infty}\sum_{n = -N}^{N}f_n e^{\frac{2\pi i n}{L}x}
	\end{align*}
\end{satz}
Beweis: Wir schreiben
\begin{align*}
	(s_nf)(x) - f(x) &= \frac{1}{L} \int_{0}^{L}D_n \left(\frac{x-y}{L}\right)f(y)dy - f(x)\\
									 &= \frac{1}{L}\int_{0}^{L}D_n \left(\frac{x-y}{L}\right)(f(y) - f(x)) dy \\
									 &= \frac{1}{L}\int_{0}^{L} \sin \left((2N+1)\pi \frac{x-y}{L}\right) \frac{f(y) - f(x)}{\sin \left(\pi - \frac{x-y}{L}\right)}dy\\
									 &= \frac{1}{L}\int_{0}^{L} \frac{e^{(2N+1)\pi i \frac{x-y}{L}}- e^{-(2N+1)\pi i \frac{x-y}{L}}}{2i} \frac{f(y) - f(x)}{\sin \left(\pi - \frac{x-y}{L}\right)}dy
\end{align*}

Wir müssen noch prüfen ob $\sin \left(\pi - \frac{x-y}{L}\right)$ integrierbar ist. Sie kann man in den Stellen $\frac{x-y}{L} \in \Z$ zu einer stetigen Funktion fortsetzen. Also zeigen wir, dass der Grenzwer
\begin{align*}
	\lim_{y \to x - nL} \frac{f(y) - f(x)}{\sin \left(\pi - \frac{x-y}{L}\right)} &\stackrel{L.H}{=} \lim_{y \to x - nL} \frac{f'(y)}{\cos \left(\pi \frac{x-y}{L}\right)}	\\
																																								&= \frac{f'(y)}{\cos(n\pi)} \left(- \frac{L}{\pi}\right) = (-1)^{n+1} \frac{L}{\pi}f'(x) 
\end{align*}
exisitiert. Somit ist die Funktion stetig und insbesondere integrierbar. Also kann man das Riemann-Lebesgue Lemma verwenden und schliessen, dass $(s_Nf)(x) - f(x) \to 0$ für $N \to \infty$.\\


Bemerkung: Der gleiche Beweis zeigt sogar, dass die Fourierreihe einer stetigen Funktion $f$ gegen $f$ in allen Punkten, an denen $f$ diffbar ist konvergiert.\\

Wir zitieren ohne Beweis noch das folgende stärkere Resultat:\\
Eine Funktion $f: [a,b] \to \C$ heisst \textbf{von beschränkter Variation}  falls es eine Konstante $V$ gibt, sodass
\begin{align*}
	\sum_{i = 0}^{n-1} \abs{f(x_{i+1} - f(x_i)} < V
\end{align*}
für alle Einteilungen $a = x_0 < x_1 < \ldots < x_n = b$.\\
Zum Beispiel sind stückweise stetig differenzierbare Funktionen, die rechts und linksseitige Limites haben sind von beschränkter Variation mit
\begin{align*}
	V = \int_{a}^{b} \abs{f'(x)}dx + \sum_{x \in [a,b] \text{ Sprungstelle}} \text{ Sprungdistanz}
\end{align*}

Im Allgemeinen existieren für Funktionen von beschränkter Variation die einseitige Limites für alle $c \in [a,b]$
\begin{align*}
	f(c + 0) := \lim_{x \to a, x > c} f(x) \quad \text{und} \quad f(c-0) := \lim_{x \to c, x < c} f(x)
\end{align*}

\begin{satz}[]
	Sei $f$ $L$-periodisch und von beschränkter Variation auf $[a,b]$. Dann gilt 
	\begin{enumerate}
		\item $\lim_{N \to \infty} s_nf(x) = \frac{1}{2} \left(f(x+0) + f(x - 0)\right)$
		\item Die Konvergenz ist gleichmässig auf jedem abgeschlossenen Intervall $I \subseteq \R$, sofern $f$ in jedem $x \in I$ stetig ist.
	\end{enumerate}
\end{satz}

\begin{nlemma}[Gibs-Phänomen]
Achtung: In der Nähe von Sprungstellen ist die Konvergenz nicht gleichmässig.\\
Nach dem Gibbs-Phänomen nehmen die $s_N(f)$ für beliebig hohe $N$ nahe an der Sprungstelle einen Wert an, der sich c.a. $9\%$ vom einseitigen Grenzwert unterscheidet
\end{nlemma}

Als Beispiel betrachte man die Sägezahlfunktion
\begin{align*}
	f(x) = \frac{1}{2} - x \text{auf } [0,1) \text{ periodisch fortgesetzt}
\end{align*}
Dort sieht man sehr nahe bei der Sprungstelle immer diese Überschüsse immer ungefähr $9\%$ von der Sprunggrösse überschiessen.\\

Die Fourierkoeffiziente für die Sägezahlfunktion sind dann
\begin{align*}
f_0 = 0, \quad f_n = \frac{1}{2\pi in}, n \neq 0
\end{align*}
Wir berechnen
\begin{align*}
	g(x) = s_Nf(x) - f(x) = \sum_{n = -N, n \neq 0}^{N} \frac{1}{2\pi i n} e^{2\pi in x} - (\frac{1}{2} - x)
\end{align*}
Die Maxima finden wir durch ableiten:
\begin{align*}
	g'(x) = \sum_{n = -N, n \neq 0}^{N} e^{2\pi i n x} + 1 = D_n(x) = \frac{\sin((2N + 1) \pi x}{\sin(\pi x)}
\end{align*}
Die erste Nullstellle ist bei $x_1 = \frac{1}{2N + 1}$. Nach dem Hauptsatz gilt dann
\begin{align*}
	g(x_1) &= g(0) + \int_{0}^{x_1} g'(x) = - \frac{1}{2} + \int_{0}^{\frac{1}{2n + 1}} \frac{\sin((2N + 1) \pi x}{\sin(\pi x)} dx\\
				 &= -\frac{1}{2} + \int_{0}^{1} \frac{\sin(\pi x) \frac{\pi x}{2N + 1}}{\pi x \sin \left(\frac{\pi x}{2N + 1}\right)} dx \simeq 0.0895... + \mathcal{O}(\frac{1}{N})
\end{align*}


\subsection{Reellwertige Darstellung der Fourierriehen}
Sei $f \in C^1(\R/L\Z)$ reellwertig. Dann
\begin{align*}
	f_n = \overline{\frac{1}{L}\int_{0}^{L}f(x) e^{+ \frac{2 \pi in}{L}x}dx} = \overline{f}_{-n}
\end{align*}
Definiere $a_n, b_n \in \R$ durch
\begin{align*}
	f_n &= \frac{1}{2}(a_n - ib_n) \quad \text{für } n \geq 0\\
	\implies f_{-n} &= \frac{1}{2}(a_n + ib_n)
\end{align*}
Beachte $f_0 = \overline{f_0} \implies b_0 = 0$. Dann haben wir
\begin{align*}
	f(x) &= \sum_{n = -\infty}^{\infty}f_n e^{\frac{2\pi inx}{L}}\\
			 &=f_0 + \frac{1}{2} \sum_{n=1}^{\infty}(a_n - ib_n) \left(\cos \frac{2\pi n}{L}x + i \sin \frac{2\pi n}{L}x\right)	+ \frac{1}{2} \sum_{n = 1}^{\infty}(a_n + ib_n) \left(\cos \frac{2\pi n}{L}x - i \sin \frac{2\pi n}{L}x\right)\\
			 &= \frac{1}{2} a_0 + \sum_{n=1}^{\infty} \left(a_n \cos \left(\frac{2\pi n}{L}x\right) + b_n \sin \left(\frac{2\pi n}{L}x\right)\right)
\end{align*}
Es gilt ausserdem:
\begin{align*}
	a_n = 2 \text{Re} f_n = \frac{2}{L} \int_{0}^{L} f(x) \cos \left(\frac{2\pi n}{L}x\right)dx\\
	b_n = -2 \Image f_n = \frac{2}{L} \int_{0}^{L}f(x) \sin \left(\frac{2\pi n}{L}x\right)dx
\end{align*}

\subsection{Poissonsche Summationsformel}

Sei $f \in C^1(\R)$ sodass $\abs{f}, \abs{f'} \leq \frac{c}{1 + x^2}$ für ein $c > 0$ und sei
\begin{align*}
	g(x) = \sum_{k \in \Z} f(x + kL)
\end{align*}
Dann ist $g(x + L) = g(x)$ ($L$-periodisch). Ausserdem konvergiert die Reihe gleichmässig auf $[0,L]$, denn
\begin{align*}
	\abs{g(x) - \sum_{k \leq \N} f(x + kL)} \leq \sum_{k \in \N} \abs{f(x + kL} \leq \sum_{\abs{k} > N} \frac{c}{1 + (x + kL)^2} \leq \sum_{\abs{k} > N} \frac{c}{(\abs{x} -1 )^2 L^2} \to_{N \to \infty} 0
\end{align*}
Also ist $g$ stetig, da es ein gleichmässiger Limes stetiger Funktionen ist. Ebenso betrachtet man $f'$ und erhält, dass $g$ auch stetig differenzierbar ist. ($g \in C^1(\R/L\Z)$).\\

Betrachtet man die Fourierkoeffizienten, so erhält man
\begin{align*}
	g_n &= \frac{1}{L }\int_{0}^{L} \sum_{k \in \Z} f(x + kL) e^{- \frac{2\pi in }{L}x}dx \\
			&= \sum_{k \in \Z} \frac{1}{L} \int_{0}^{L}f(x + kL) e^{- \frac{2\pi in}{L}x}dx\\
			&= = \sum_{k \in \Z} \frac{1}{L} \int_{kL}^{(k+1)L}f(x) e^{- \frac{2 \pi in}{L}x}dx\\
			&= \frac{1}{L} \int_R f(x) e^{- \frac{2\pi i n}{L}x}dx
\end{align*}


\begin{definition}[Fourier-transformation]
	Die \textbf{Fouriertransformation}  von einer Funktion $f \in L^1(\R)$ ist 
	\begin{align*}
		\hat{f}(k) = \int_R f(x) e^{-ikx}dx
	\end{align*}
\end{definition}

Also gilt
\begin{align*}
	g(x) = \sum_{n \in \Z} f(x + nL) = \sum_{n \in \Z} g_n e^{\frac{2\pi in}{L}x} = \sum_{n \in \Z} \frac{1}{L} \hat{f} \left(\frac{2 \pi n}{L}\right) e^{\frac{2\pi in}{L}x} 
\end{align*}

Für $x = 0$ erhält man insbesondere die Poissonsche Summationsformel
\begin{align*}
	\sum_{n \in \Z} f(nL) = \frac{1}{L} \sum_{n \in \Z} \hat{f} \left(\frac{2\pi n}{L}\right)
\end{align*}


\subsection{Vertauschen von $\lim$ und $\int$ mit Ableitungen}
Sei $f_n: X \to \C$ Funktionen auf $X$. Wir sagen die Funktionen konvergieren gleichmässig gegen eine Funktion $f:X \to \C$, falls sie in der Supremumsnorm konvergiert. Also
\begin{align*}
	\|f_n - f\|_{\infty} := \sup_{x \in X} \abs{f_n(x) - f(x)} \to_{n \to \infty} 0
\end{align*}
Ein Beispiel von gleichmässiger Konvergenz sind konvergenze Reihen bei dem die Funktionen durch die Partialsummen gegeben sind:
\begin{align*}
	f_n = \sum_{k=1}^{n} h_k, \quad \text{und} \quad \sum_{k=1}^{\infty}\|h_k\|_{\infty} < \infty
\end{align*}
für $h_k: X \to \C$ Funktionen. 

\begin{satz}[]
	\begin{enumerate}
		\item Sei $(f_n: X \to \C)$ eine Folge stetiger Funktionen, die gleichmässig gegen eine Funktion $f: X \to \C$ konvergiert. Dan ist $f$ wieder stetig.
		\item Sei $X \subseteq R^{n}$ offen, und $(f_n: X \to \C)$ eine Folge stetig differenzierbarer Funktionen, die \emph{punktweise} gegen eine Funktion $f: X \to \C$ konvergiert. Konvergieren die partiellen ableitungen $\del_j f_n$ \emph{gleichmässig}, so darf man die Ableitung in den Limes ineinziehen, d.h.
		\begin{align*}
			\del_j \lim_{n \to \infty} f_n = \del_j f = \lim_{n \to \infty} \del_j f_n
		\end{align*}
	\end{enumerate}
\end{satz}
Beweis Analysis I/II.

\begin{satz}[]
	Sei $X$ ein metrischer Raum, $E \subseteq \R^n$ messbar und $f: X \times E \to \C$. Seien $x \mapsto f(x,y)$ stetig, $y \mapsto f(x,y)$ integrierbar und setze $F(x) = \int_E f(x,y) dy$. Dann gilt:
	\begin{enumerate}
	\item 	Existiert eine majorante $g \geq 0$ auf $E$, integerierbar mit 
		\begin{align*}
			\abs{f(x,y)} \leq g(y), \forall x,y
		\end{align*}
		so ist $F$ stetig.
	\item Sei $X \subseteq \R^n$ offen und die Funktionen $x \mapsto f(x,y)$ stetig differenzierbar für alle $y$. Gibt es eine majorante $g \geq 0$ integrierbar auf $E$, mit
		\begin{align*}
			\abs{\del_j f(x,y)} \leq g(y) \forall x, y, j
		\end{align*}
		Dann ist $F$ stetig differenzierbar und man kann die Ableitung in das Integral ziehen:
		\begin{align*}
			\del_j \int_{E} f(x,y) dy = \del_j F = \int_{E} \del_j f(x,y) dy
		\end{align*}
	\end{enumerate}
\end{satz}

Beweis: 
\begin{enumerate}
	\item Wir haben mit dem Satz der majorisierten Konvergenz, dass
	\begin{align*}
		\lim_{x \to x_0}F(x) = \lim_{x \to x_0} \int_{E} f(x,y) dy = \int_{E} \lim_{x \to x_0} f(x,y) dy = \int_{E}f(x_0,y) dy = F(x_0)
	\end{align*}
	\item Definiere die Funktion für ein $h$ glein genug.
		\begin{align*}
			\phi(x,y,h) := \frac{f(x + h e_j,y) - f(x,y)}{h}
		\end{align*}
		Der Grenzwert für $h \to 0$ ist die partielle Ableitung von $f$. Weil $\phi$ Integrierbar ist über $y$, und wegen dem Mittelwertsatzes gilt $\abs{\phi(x,y,h)} \leq g(y)$, so haben wir
		\begin{align*}
			\del_j F(x) = \lim_{h \to 0} \frac{F(x + h e_j) - F(x)}{h} = \lim_{h \to 0} \int_{E} \phi(x,y,h) dy \stackrel{DCT}{=} = \int_{E} \lim_{h \to 0} \phi(x,y,h) dy = \int_E \del_j f(x,y) dy
		\end{align*}
		Die Stetigkeit folgt aus dem ersten Teil des Satzes.
\end{enumerate}



\subsection{Wärmeleitungsgleichung auf einem Ring}
Wir stellen uns einen idealisierten eindimensionalen Ring mit Länge $L$ vor wo eine Temperaturverteilung drauf liegt und wir sind daran interessiert, wie sich die Temperaturverteilung mit der Zeit ändert.\\

Sie die Temperaturverteilung zur Zeit $t$ durch die Funktion $u(x,t)$ gegeben. Zunächst für $x \in [0,L]$, weil wir periodisch fortsetzen auf $x \in \R$.\\
Die Funktion erfüllt dann die Wärmeleitungsgleichung
\begin{align*}
	\frac{\del u}{\del t}(x,t) = D \frac{\del^2 u}{\del x^2}(x,t)
\end{align*}
wobei $D$ eine positive Konstante ist. Weiterhin sei die Anfangsbedingung $u(x,0) = f(x)$ vorgegen. Wir wollen dann eine Lösung für alle $t$ Lösen.\\

Durch geeignete re-skalierung der Koordinaten können wir immer davon Ausgehen, dass $L = 2 \pi$ und $D = 1$. Unser Anfangswertproblem wird dann zu
\begin{align*}
	\frac{\del u(x,t)}{\del t} = \frac{\del^2 u}{\del x^2}(x,t), \quad u(x,0) = f(x) 
\end{align*}
Wir nehmen zusätzlich an, dass $f \in C^{\infty}(R/2\pi\Z)$ und suchen Lösungen $u \in C^{\infty}(\R/2\pi\Z \times (0,\infty))$ und wollen sie noch stetig am Rand fortsetzen.\\

Sei $u(x,t)$ eine solche Lösung des Anfangswertproblems. Dann entwickle sie in Fourierreihen:
\begin{align*}
	u(x,t) = \sum_{n \in \N} f_n e^{inx}, \quad \text{mit} \quad u_n(t) = \frac{1}{2\pi} \int_{0}^{2\pi} u(x,t) e^{-inx}dx
\end{align*}
Zunächst bemerken wir, dass $u_n \in \C^{\infty}((0,\infty)) \cap C^0([0,\infty))$. Das folgt direkt da man die Ableitung in das Integral nehmen kann, weil eine Majorante existiert, da wir ein kompaktes Intervall hat und somit ein Maximum annimmt, also man nimmt 
\begin{align*}
	\sup_{(x,t) \in [0,2\pi] \times [a,b]} \abs{\del_t^p u(x,t)}
\end{align*}

Es gilt ausserdem, dass
\begin{align*}
	\del_t u_n(t) = \frac{1}{2\pi} \int_{0}^{2\pi} \del_t u(x,t) e^{-inx}dx = \frac{1}{2\pi} \del_t^2 u(x,t) e^{-inx}dx = \frac{1}{2\pi}\int_{0}^{2\pi} u(x,t) (-in)^2 e^{-inx}dx = -n^2u_n(t)
\end{align*}
Und $u_n(0) = f-N$. Die (eindeutige Lösung) dieser gewöhnlichen DGL ist
\begin{align*}
	u_n(t) = e^{-tn^2}f_n \implies u(x,t) = \sum_{n \in \Z} e^{-tn^2 + inx} f_n
\end{align*}
Umgekehrt müssen wir noch zeigen, dass $u$ wie oben wirklich eine Lösung der Wärmeleitungsgleichung ist.
\begin{enumerate}
	\item Die Funktion $u$ ist $C^{\infty}$ in $\R \times (0,\infty)$ und die partiellen Ableitungen sind
		\begin{align*}
			\del_t^j \del_x^k u = \sum_{n \in \Z} (-n^2)^j (in)^k e^{-tn^2 + inx}fn
		\end{align*}
		Damit wir den Satz benutzen dürfen, müssen wir zeigen, dass die partiellen Ableitungen gleichmässig konvergieren. Da stimmt, dass die Reihe oben normal konvergiert auf $\R \times [a,b]$ mit $a > 0$, denn es gilt
		\begin{align*}
		\sup_{\underset{t \in [a,b]}{x \in \R}} \abs{(-n^2)^j (in)^k e^{-tn^2 + inx}f_n} = n^{2j+k}\sup_{t \in [a,b]} \abs{e^{-tn^2}} = e^{-an^2} n^{2j+k} \abs{f_n} \to 0
		\end{align*}
		und die Reihe $\sum_{n \in \Z} e^{-an^2}n^{2j +k} \abs{f_n}$ konvergiert.
\end{enumerate}



