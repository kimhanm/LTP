\begin{ex}[]
  Consider the space of basis of $\R^{n}$ up to isometry.
  Since a basis of $\R^{n}$ consists of $n$ vectors, we can think of basis as a matrix $B \in \GL(n,\R)$.
  The equivalence relation is then given by 
 \begin{align*}
  B \sim A \iff \exists U \in O(n) \text{ such that } AU = B
 \end{align*}
 the resulting space is then $\faktor{\GL(n,\R)}{O(n)}$
\end{ex}

\begin{rem}[]
  The reason why homogenous spaces are called homogenous is that the space looks the same everywhere. (Think of $\R/\Z$).

  Since the multiplication $m: G \times G \to G$ is continous, multiplication with a fixed $a \in G$ is also continous since the maps
  \begin{align*}
    l_a: G \to  G, \quad g \mapsto ag
  \end{align*}
  can be written as the composition of $m$ and the map $\iota_a$ given by
  \begin{align*}
    \iota_a: G \to G \times G, \quad g \mapsto  (a,g)
  \end{align*}
  same is true for right multiplication with $a$.

  This has the consequence that if $U$ is a neighborhood of $e$, then $aU$ (or $Ua$) is a neighborhood of $a$.

  In particular, if $H \subseteq G$ is a subgroup, then for all $x,y \in G/H$ there exists a homeomorphism $f: G/H \to G/H$ such that $f(x) = y$.
  Such a homeomorphism is given by
  \begin{align*}
    f: \faktor{G}{H} \to \faktor{G}{H}, \quad gH \mapsto  ba^{-1}gH
  \end{align*}
  this mapping is indeed a homeomorphism since $f \circ \pi = \pi \circ l_{b^{-1}a}$ is continuous which shows continuity of $f$ (and similarly, of $f^{-1}$).
\end{rem}


\subsection{Orbit spaces}
We can think of groups as symmetries of spaces. Are there any special properties of such spaces?


We of course want our group actions to be continous.
\begin{dfn}[]
Let $G$ be a topological grouop and $X$ a topological space.

An \textbf{operation}/\textbf{continuous action} of $G$ on $X$ is a continuous group action of $G$ on $X$, i.e. a continuous map $\cdot: G \times X \to  X$ such that
\begin{enumerate}
  \item $1 x = x$ for all $x \in X$
  \item $g_1(g_2 x) = (g_1 g_2) x$ for all $g_1,g_2 \in G, x \in X$
\end{enumerate}
We call such a topological space $X$ a $G$-space.
\end{dfn}

\begin{dfn}[]
Let $X$ be a $G$-set and $x \in X$. The \textbf{orbit} of $x$ is the set
\begin{align*}
  G_x := \{gx \big\vert g \in G\}
\end{align*}
These orbits are equivalence classes of an equivalence relation $\sim$ on $X$ given by
\begin{align*}
  x \sim y \iff \exists g \in G: y = gx
\end{align*}
and as such, we call $X/G := X/\sim$ the \textbf{orbit space} of the group action.
\end{dfn}

\begin{ex}[]
  \phantom{a}
\begin{itemize}
  \item For $SO(n)_v = \{Av \big\vert A \in SO(n)\} = \{w \in \R^{n} \big\vert \abs{w} = \abs{n}\}$.
    We can visualize the orbits as $n-1$ spheres of any radius. 
    It is easy to see how the orbit space $\R^{n}/SO(n)$ is isomorphic to $[0,\infty)$.

  \item For $G = \R^{\times} = (\R \setminus \{0\}, \cdot)$ and $X = \R^{n+1} \setminus \{0\}$, the map
    \begin{align*}
      \R^{\times} \times X \to X, \quad (\lambda,x) \mapsto  \lambda x
    \end{align*}
    is a continuous action and the orbit space is called the $n$-dimensional \textbf{real projective space} $\R\IP^{n} = X/\R^{\ast}$.
    We can view this as consisting of straight lines through the origin of $\R^{n+1}$.
    One can also find an isomorphism $\faktor{\IS^{n}}{\{\pm 1\}} \iso \R\IP^{n}$.
\end{itemize}
\end{ex}



