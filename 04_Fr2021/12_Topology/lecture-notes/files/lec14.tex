\subsection{Functors}
If we look at categories where objects are Sets with structure and morphisms are strucutre preserving maps we might ask what the morphisms in the ``category of categories'' are.

The structure of a category would the morphisms, so we can ask what ``morphism-preserving'' morphisms are.

This gives rise to the definition of a functor.

\begin{dfn}[]
  Let $\textsf{C},\textsf{D}$ be categories. A \textbf{covariant Functor} $\mathcal{F}$ from $\textsf{C}$ to $\textsf{D}$ (written $\mathcal{F}: \textsf{C} \to  \textsf{D}$ does the following:

  \begin{itemize}
    \item It provides a map $F: \text{Ob}(\textsf{C}) \to  \text{Ob}(\textsf{D})$ that maps an object $X$ from $\textsf{C}$ to and object $F X$ in $\textsf{D}$.
    \item It maps morphisms $f \in \Hom(X,Y)$ to a morphism $F f \in \Hom(F X, F Y)$ such that
      \begin{enumerate}[{(}i{)}]
        \item $F (\id_X) = \id_{F X}$
        \item $F (g \circ f) = Fg \circ Ff$
      \end{enumerate}
      the second condition is equivalent to saying that for any morphism $f: X \to  Y$ in $\textsf{C}$, the following diagram commutes
\begin{center}
\begin{tikzcd}[] %\arrow[bend right,swap]{dr}{F}
  X \arrow[]{rr}{F} \arrow[]{d}{f} & & F X \arrow[]{d}{Ff}\\
  Y \arrow[]{rr}{F} & & F Y 
\end{tikzcd}
\end{center}
  \end{itemize}
\end{dfn}



\begin{ex}[]
  Quite a few concepts that we encountered througout Algebra or Topology are functors.
  \begin{enumerate}
    \item If we have a group $(G,\cdot,e)$, we can forget the group structure and only look at the underlying set $G$.
      Group homomorphisms turn into functions.
      We call this the \textbf{forgetful functor}.
      The same construction works for topological groups, where we forget the topology of a topological space.
    \item We can turn every field into a group, by removing the $0$. Group homomorphisms then become field homomorphisms.
    \item There is a functor $\Top \to \Htpy$ that preserves the objects (topolgocial spaces), but maps continuous maps to their homotopy equivalence classes.
    \item Consider the \textbf{endofunctor} $\textsf{F}: \textsf{Vect}_k \to  \textsf{Vect}_k$ which maps vector spaces to their dual and linear maps to their dual maps.
      So for vector spaces $V,W$ and a linear map $f \in \Hom(V,W)$ we define
      \begin{align*}
        F V = V^{\ast}, \quad \text{and} \quad Ff =: f^{\ast}: W^{\ast} \to  V^{\ast}, f^{\ast} \beta = \beta \circ f : V \to k
      \end{align*}
      This is an example of a \textbf{contravariant functor}, as the directon of $Ff$ is opposite to that of $f$.
  \end{enumerate}
\end{ex}


\begin{lem}[]\label{lem:func-iso}
Let $\textsf{F}: \textsf{C} \to  \textsf{D}$ be a functor.
If $X,Y \in \text{Ob}(\textsf{C})$ and $f \in \Hom(X,Y)$ is an isomorphism, then $Ff \in \Hom(FX,FY)$ (or $\Hom(FY,FX)$ for contravariant functors) is also an isomorphism.
\end{lem}
\begin{proof}
  Since $f$ is an isomorphism, there exists a morphism $g \in \Hom(Y,X)$ such that $g \circ f = \id_X$ and $f \circ g = \id_Y$.
  By the functor laws, $Ff$ and $Fg$ satisfy for covariant functors, 
  \begin{align*}
    Fg \circ Ff = F(g \circ f) = F \id_X = \id_{FX} \quad \text{and} \quad Ff \circ Fg = F(f \circ g) = F \id_Y = \id_{FY}
  \end{align*}
  and for contravariant functors:
  \begin{align*}
    Fg \circ Ff = F(f \circ g) = F \id_Y = \id_{F Y} \quad \text{and} \quad Ff \circ Fg = F(g \circ f) = F \id_X = \id_{F X}
  \end{align*}
\end{proof}
We can use this lemma to immediately prove that every homeomorphism (iso in $\Top$) is also a homotopy equivalence (iso in $\Htpy$).


\section{The fundamental group}
One central idea in category theory is the view that we can understand objects in a category by studying their Hom-sets to other objects.

Of course understanding \emph{all} Hom-sets to other objects is too much to reasonably do.
So what we can do is to instead look at ``nice'' spaces and study the Hom-sets to other spaces.

One such nice space is the circle space $\IS^{1}$.
Given a topological space $X$ and a basepoing $x_0 \in X$, we want to understand all possible maps $f \in \Hom(\IS^{1},X)$.
This corresponds to studying \textbf{loops} with basepoint $x_0$.

If we are given two loops $f,g$ that share the same basepoint $x_0$, then can move along one loop and then move along the next one to obtain a new loop with the same basepoint.
This gives us a binary operation on the Hom-Set $\Hom_{\Top^{\ast}}(\IS^{1},(X,x_0))$
It turns out that if we only consider continuous functions up to homotopy, then the Hom-Set $\Hom_{\Htpy^{\ast}}(\IS^{1},X)$ carries a group structure!


\subsection{Definition and examples}


\begin{dfn}[]
\begin{itemize}
  \item A path $\alpha: [0,1] \to  X$ is called a \textbf{loop} with basepoint $x_0$ if $\alpha(0) = \alpha(1) = x_0$.
  \item Two paths $\alpha,\beta: [0,1] \to  X$ are \textbf{homotopic rel. endpoints}, if there exists a homotopy preserving the endpoints. That is, if $\alpha\sim_h \beta$ with
    \begin{align*}
      h(0,t) = a(0) \quad \text{and} \quad h(1,t) = \alpha(1) \quad \text{for all} \quad t \in [0,1]
    \end{align*}
  \item If $\alpha$ is a path from $a$ to $b$ and $\beta$ is a path from $b$ to $c$, we define their \textbf{multiplication} as
    \begin{align*}
      \alpha \beta : [0,1] \to  X, t \mapsto  \left\{\begin{array}{ll}
          \alpha(2s) & s \leq \tfrac{1}{2}\\
          \beta(2s - 1) & s \geq \tfrac{1}{2}
      \end{array} \right.
    \end{align*}
\end{itemize} 
We write $\alpha \sim \beta$ rel endpoints and
\begin{align*}
  [\alpha] := \{\beta | \beta\text{ is a path and } \alpha \sim \beta \text{ rel endpoints}\}
\end{align*}
for the equivalence class.
\end{dfn}


\begin{dfn}[]
  Let $(X,x_0) \in \Top^{\ast}$.
  The \textbf{Fundamental Group} of $(X,x_0)$ is 
  \begin{align*}
    \pi_1(X,x_0) := \{[\alpha] \big\vert \alpha \text{ is a loop with basepoint $x_0$}\} = \Hom_{\Htpy^{\ast}}(\IS^{1},(X,x_0))
  \end{align*}
  with multiplication
  \begin{align*}
    ([\alpha],[\beta]) \mapsto [\alpha \beta]
  \end{align*}
\end{dfn}
We have to show that the multiplication is well-defined and that it does indeed define a group structure.
\begin{proof}
  The well-defined-ness of the multiplication is in the Exercise sheet 7, Problem 1.

  The other group axioms are easily proven.
  The identity of the group is the constant map $x_0$ and inverse is the reverse traversal of the loop.
\end{proof}

\begin{ex}[]
Let $K \subseteq \R^{n}$ be convex and $x_0 \in K$. Then the fundamental group $\pi_1(K,x_0)$ is the trivial group $\{e\}$.
\end{ex}

\begin{thm}[]
  \begin{enumerate}
    \item The fundamental group of the sphere is also the trivial group. More generally, for all $n \geq 2$, $\pi_1(\IS^{n},x_0) = \{e\}$ 
    \item The fundamental group of the circle is $\Z$ and representants of the group are the maps
      \begin{align*}
        \alpha_k:[0,1] \to \IS^{1}, \quad t \mapsto  e^{2 \pi i k t}
      \end{align*}
  \end{enumerate}
\end{thm}
\begin{proof}
\begin{enumerate}
  \item Let $n \geq 2$ and $\alpha$ a loop with basepoint $x_0$.
    We show that it is homotopic to the constant map $x_0$.
    To do this, we first show that there exists a loop $\beta \sim \alpha$ rel $x_0$ which is not surjective.
    If $\Image \alpha = \IS^{n}$, then take the hemisphere $D^{n} \subseteq \S^{n}$ with center $x_0$.
    Take the point $-x_0$ opposite to $x_0$ on the other hemisphere
    and let $(I_j)_{j \in J}$ be the maximal intervals such that $\Image(\alpha |_{I_J}) \subseteq D^{n}$.
    Because $D^{n}$ is contractible, we set
    \begin{align*}
      \beta(t) = \left\{\begin{array}{ll}
          \alpha(s) & s \notin \bigcup_{j \in J}I_j\\
          \beta_j(s) & s \in I_j
      \end{array} \right.
    \end{align*}
    such that $\beta_j(s): I_j \to \del D^{n}$ satisfies
    \begin{align*}
      \beta_j |_{\del I_j} = \alpha|_{\del I_j}
    \end{align*}

    From this, we take a point $x \in \IS^{n} \setminus \Image(B)$.
    Then we use the isomorphism $\phi: \IS^{n} \setminus x \to \R^{n}$.
    Because $\R^{n}$ is contractible it follows that $\alpha \sim \beta \sim x_0$.

  \item We only provide an informal proof. We will do it more rigorously later.
    For surjectvitiy of the map $\Phi: \Z \to  \pi_1(\IS^{1},1), k \mapsto  \alpha_k$, let $\alpha$ be a loop with basepoint $1$.

    The idea then is that we use the map 
    \begin{align*}
      \psi: \R \to \IS^{1}, \quad x \mapsto e^{2 \pi i x}
    \end{align*}
    that wraps $\R$ around the circle.
    We then show that there exists a path $\tilde{\alpha}: [0,1] \to  \R$ such that $\alpha = \psi \circ \tilde{\alpha}$.

    So if we set $k = \tilde{\alpha}(1) \in \Z$. then we can show that $\tilde{\alpha} \sim \tilde{a}_k$ rel endpoints.
    Therefore
    \begin{align*}
      \alpha = \pi \circ \tilde{\alpha} \sim \pi \circ \tilde{\alpha}_k = \alpha_k \text{ rel 1}
    \end{align*}
    For injectivity, we prove that $\Phi$ is group homomorphism with kernel $0$.
\end{enumerate}
\end{proof}


\begin{lem}[]
  The fundamental group defines a covariant functor $\pi_1: \Top^{\ast} \to \Grp$ that maps pointed topological spaces $(X,x_0)$ to their fundamental group $\pi_1(X,x_0)$ and that maps a basepoint preserving maps $f: (X,x_0) \to (Y,y_0)$ to group homomorphisms
\begin{align*}
  f_{\ast}: \pi_1(X,x_0) \to \pi_1(Y,y_0)
  , \quad
  [\alpha] \mapsto [f \circ \alpha]
\end{align*}
\end{lem}
\begin{proof}
We have to prove that 
\begin{enumerate}
  \item For any topological space $X$
    \begin{align*}
      (\id_X)_{\ast} = \id_{\pi_1(X,x_0)} \quad \text{ for all } x_0 \in X
    \end{align*}
  \item For continuous, basepoint preserving maps $f: (X,x_0) \to (Y,y_0)$, $g:(Y,y_0) \to (Z,z_0)$
    \begin{align*}
      (g \circ f)_{\ast} = g_{\ast} \circ f_{\ast}
    \end{align*}
\end{enumerate}
\end{proof}

\begin{cor}[]
  If $f: X \to  Y$ is a homeomorphism, then $f_{\ast}: \pi_1(X,x_0) \to  \pi_1(Y,y_0)$ is a group isomorphism.
\end{cor}
\begin{proof}
  This follows directly from Lemma \ref{lem:func-iso} because homeomorphisms are the isomorphisms in $\Top^{\ast}$ and group isommorphisms are the isomorphisms in $\Grp$.
\end{proof}


