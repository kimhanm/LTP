
\begin{ex}[]
For $\mathbb{D}^{2} = \{v \in \R^{2} \big\vert \abs{v} \leq 1\}$ we set $\sim$ on $\mathbb{D}^{2}$ to
\begin{align*}
  v \sim w \iff v = w \text{ or } \abs{v} = \abs{w} = 1
\end{align*}
Then resulting space is homeomorphic to the sphere $\mathbb{S}^{2}$

To prove this we give a mapping
\begin{align*}
  g: \mathbb{D}^{2}\to \mathbb{S}^{2}, \quad v \mapsto \left\{\begin{array}{ll}
      (0,0,-1) & \text{ if } v = (0,0)\\
      \left(
      \frac{\sqrt{1 - (2 \abs{v} - 1)^{2}}}{\abs{v}}v,
      2 \abs{v} - 1
      \right)& 
      \text{ otherwise}
  \end{array} \right.
\end{align*}
which is continuous. 
From the previous lemma, it follows that the induced map $f: \mathds{D}^{2}/\sim \to  \mathds{S}^{2}$.
And since it is also bijective and $\mathds{D}^{2}/\sim$ is compact, $f$ is a homeomorphism.
\end{ex}


While the previous lemma can be used to show that a function from the quotient space is continous, we can also look at functions into the quotient space and ask if they are continuous.
\begin{lem}[]
  Let $X$ and $Y$ be topological spaces, $\sigma$ an equivalence relation on $X$ and $\phi: Y \to X/\sim$ a map.

  If there exists a continuous map $\Phi: Y \to X$ such that $\phi = \pi \circ \Phi$, then $\phi$ is continuous.

\begin{center}
\begin{tikzcd}[] %\arrow[bend right,swap]{dr}{F}
  & X \arrow[]{d}{\pi}\\
  Y \arrow[dashed]{ur}{\Phi} \arrow[]{r}{\phi} & \faktor{X}{\sim}
\end{tikzcd}
\end{center}
\end{lem}
The proof is obvious since the composition of continous maps is continuous.

\begin{ex}[]
Let $n \in \N, n \geq 2$. Set $X = \R^{n}$ with the equivalence relation
\begin{align*}
  v \sim w \iff v_i = w_i \quad \forall i \leq n -1
\end{align*}
The quotient map can be thought of compressing the $n$-th dimension on $\R^{n}$ onto the remaining $n-1$ ones. 
From what we just showed, the mapping
\begin{align*}
  \phi: \R^{n-1} \to \R^{n}/\sim, \quad u \mapsto [(u,0)]
\end{align*}
is continuous and is a homeomorphism where the inverse map is given by
\begin{align*}
  \phi^{-1}: \R^{n}/\sim \to \R^{n-1}, \quad [v] \mapsto  (v_1,\ldots,v_{n-1})
\end{align*}
\end{ex}


\subsection{Properties of Quotient spaces}
It is trivial to see that if $X$ is a topological space with equivalence relation $\sim$, then
\begin{enumerate}
  \item $X$ is compact/connected/path $\implies$ $X/\sim$ is compact/connected/path connected.
    The converse is not always true.
  \item $X/\sim$ is $T_1$ (singletons are closed) if and only if the equivalence classes are closed in $X$.
\end{enumerate}

\begin{ex}[Cool examples]
  Consider $X = \R^{2}$ and two equivalence classes given by
  \begin{align*}
    [x]_1 := 
    \left\{\begin{array}{ll}
        \{(x,y) \big\vert y \in \R\} & \text{ for } \abs{x} \geq \frac{\pi}{2}\\
        \left\{\big(\arctan(y + \tan(x)),y\big) \big\vert y \in \R\right\}& 
        \text{ for } \abs{x} < \frac{\pi}{2}
    \end{array} \right.
  \end{align*}
  \begin{align*}
    [x]_2 := 
    \left\{\begin{array}{ll}
        \{(x,y) \big\vert y \in \R\} & \text{ for } \abs{x} \geq \frac{\pi}{2}\\
        \left\{\big(x, - (\tan(x))^{2} + y\big) \big\vert y \in \R\right\}& 
        \text{ for } \abs{x} < \frac{\pi}{2}
    \end{array} \right.
  \end{align*}
  We can show that $\R^{2}/\sim_1$ is homeomorphic to $\R$ and $\R^{2}/\sim_2$ is not $T_2$
\end{ex}


\subsection{Homogenous spaces}
Some of the important topological spaces such as $\R, \C,\Z, \mathbb{S}^{1}$ etc. carry a group structure.
Not only this, their group multiplication (or addition) is \emph{continuous} with respect to the product topology. 
Same goes for the inverse operation.
A generalisation is as follows
\begin{dfn}[]
  A topological space $G$ equipped with a group operation $\cdot$ is called a \textbf{topological group} if the multiplication and the inverse map
  \begin{align*}
    G \times G \to  G, \quad (a,b) \mapsto ab, \quad \text{and} \quad 
    G \to  G, \quad a \mapsto a^{-1}
  \end{align*}
  are continuous.
\end{dfn}
Note that the continuity condition has to be viewed in terms of the product topology on $G \times G$.

\begin{ex}[]
Since the product topology of discrete spaces is discrete, any group can be turned into a topological space with the discrete topology.

A special case of topological groups are \textbf{Lie Groups}, where instead of just requiring continuity, we also want the operations to be \emph{smooth}.

$\text{SO}(n,\K) \subseteq \text{SL}(n,\K) \subseteq \GL(n,\K)$ for $\K = \Q,\R,\C$ with the euclidean topology.
The reason that the group operations are continuous is that they are polynomial.
\end{ex}

\begin{dfn}[]
  Let $G$ be a topological group and $H \subseteq G$ a subgroup. 
  Then the set of equivalence classes $G/H$ with the quotient topology is called a \textbf{homogenous space}.
\end{dfn}
Note that since $\{e\} \subseteq G$ is also a subgroup, every topoogical space is also a homogenous space.

\begin{ex}[]
  Let $G = \C \setminus \{0\}$ be equipped with complex multiplication. Here, $H = \R_{>0}$ is a (normal) subgroup and the mapping
  \begin{align*}
    f: G/H \to \mathbb{S}^{1}, \quad z \mapsto \frac{z}{\abs{z}}
  \end{align*}
  is a homeomorphism and even a group isomorphism. An isomorphism in the category of topological groups.

  The reason is because $f \circ \pi: \C^{\ast} \to \mathbb{S}^{1}$ is continuous and because $H$ its kernel of it.
\end{ex}

But why study homogenous spaces?
It turns out that they have some nice properties:
\begin{lem}[]
Let $G$ be a topological group and $H \subseteq G$ a subgroup.
Then the quotient space $G/H$ is $T_2$ if and only if $H$ is closed in $G$
\end{lem}
\begin{cor}[]
Let $G$ be a topological group. Then $G$ is $T_2$ if and only if $\{e\} \subseteq G$ is closed.
\end{cor}
In particular, the distiction between $T_2$ and $T_1$ spaces drops.


In exercise sheet 4, we will prove the \textbf{$\bm{T_2}$ criterion}:
\begin{prop}[$T_2$ criterion]
  A topological space $X$ is $T_2$ if and only if the diagonal
  \begin{align*}
    \Delta_X = \{(x,x) \big\vert x \in X\} \subseteq X \times X
  \end{align*}
  is closed. 
  Moreover, if $\sim$ is an equivalence relation on $X$ such that the projection mapping $\pi: X \to X/\sim$ is open. 
  Then $X/\sim$ is $T_2$ if and only if the set
  \begin{align*}
    \left\{(x,y) \big\vert x \sim y\right\} \subseteq X \times X
  \end{align*}
  is closed.
\end{prop}

\begin{proof}[Proof Lemma]
  Since $T_2 \implies T_1$, $\{e\}$ is closed in $G/H$. 
  But by continuity of the quotient map, so is it's inverse image, which is $H \subseteq G$

  Ond the other hand, if $H$ is closed, then by the $T_2$ criterion, so is $R = \{(a,b) \big\vert a \sim b\} \subseteq G \times G$.
  But $R$ is just the inverse image $H$ of the continuous map 
  \begin{align*}
    m: G \times G \to G, \quad (a,b) \mapsto a^{-1}b
  \end{align*}
  and the proof follows.
\end{proof}
