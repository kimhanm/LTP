Da wir den Kreis $\mathbb{S}^1$ homeomorph ``rotieren'' können, nehme ich mal o.B.d.A an $z = 1 = e^{2\pi i \cdot 0} \in \mathbb{S}^1$

Dann ist das Urbild von dem $U = \mathbb{S}^1 \setminus \{z\}$ unter $\pi$ gerade
$$
\pi^{-1}(\mathbb{S}^1 \setminus \{1\}) = \R \setminus \mathbb{Z} = \bigsqcup_{k \in \Z} (k,k+1)
$$
Setzen wir $F = \Z$, und definieren die Abbildung
$$
\tilde{\pi}: U\times \Z
\to U, \quad (x,k) \mapsto x
$$
Also ist der Isomorphismus gegeben durch
$$
\Psi: U \times \Z \to \pi^{-1}(U) = \bigsqcup_{k \in \Z}(k,k+1)
$$
$$
(e^{2 \pi i x},k) \mapsto x + k
$$
(den inverse Isomorphismus anzugeben ist etwas aufwändiger, also habe ich eine andere Richtung als in deiner Frage gewählt)


Zur allgemeineren Frage: Per definition heissen die Objekte $(Y,\pi)$ und $(X \times F, \tilde{\pi})$ isomorph über $X$, 
falls es Homeomorphismen $\Phi,\Psi$ gibt, sodass folgendes Diagramm kommutiert.
\begin{center}
\begin{tikzcd}[column sep=0.8em] 
  X \times F 
  \arrow[]{dr}{\tilde{\pi}}  \arrow[]{rr}{\Psi}
  && 
  Y \arrow[]{dl}{\pi} \arrow[]{ll}{\Phi}
  \\
  & X
\end{tikzcd}
\end{center}
, wobei
$
\tilde{\pi}: X \times F \to X, (x,s) \mapsto x
$.


Fangen wir bei oben links bei $\tilde{\pi}^{-1}(x) = (x,F) \subseteq X \times F$ an, so muss gelten
$$
x = \tilde{\pi}(x,F) \stackrel{!}{=} \pi(\Psi((x,F))) \implies \Psi(x,F) \subseteq \pi^{-1}(x)
$$
Da $\Psi$ ein Isomorphismus ist, ist $F \cong (x,F)$ isomorph zu einem Unterraum von $\pi^{-1}(x)$.

Fangen wir oben rechts bei $\pi^{-1}(x) \subseteq Y$ an, so muss gelten
$$
x = \pi( \pi^{-1}(x)) \stackrel{!}{=} \tilde{\pi}(\Phi(\pi^{-1}(x))) \implies \Phi(\pi^{-1}(x)) = (x,S)
$$
für eine Teilmenge $S \subseteq F$. Da $\Phi$ ein Isomorphismus ist, ist $\pi^{-1}(x)$ isomorph zu einem Unterraum von $F$.


Das an sich zeigt nicht umbedingt, das $F \cong \pi^{-1}(x)$, denn analoge Aussagen gelten auch für $(0,1)$ und $[0,1]$.

Aber $F$ ist diskret, somit auch jeder Unterraum $S \subseteq F$.

Da die Kategorie der diskreten Topologischen räume ``isomorph'' ist zur Kategorie der Mengen, sind wir beim Satz von Kantor-Schröder Bernstein, und somit gibt es eine bijektive Abbildung zwischen $F$ und $\pi^{-1}(x)$.

Und weil eben beide diskret sind, ist diese bijetive abbildung auch stetig.



Bitte die obige Antwort auf die zweite Frage ignorieren. Es gibt einen viel einfacheren Beweis: Wir zeigen dass o.B.d.A. $X$ einelementig ist.

Angenommen, es gäbe eine triviale Faser $(Y,\pi) \cong (X \times F,\tilde{\pi})$, sodass $\pi^{-1}(x) \not\cong F$ für ein $x \in X$.

Dann betrachte man die triviale Faser $Y' := \pi^{-1}(x), \pi' := \pi|_{Y'}: Y' \to \{x\}$.

Wir haben dann den Isomorphismus über $\{x\}$: $(Y',\pi') \cong (\{x\}\times F,\tilde{\pi}')$

Und $Y' = \pi^{-1}(x) \not\cong F \cong \{x\} \times F$.



