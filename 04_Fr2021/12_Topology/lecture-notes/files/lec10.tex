\begin{dfn}[]
Let $X$ be a $G$-space, $x \in X$. 
We call $G_x := \text{Stab}_G(x) = \{g \in G \big\vert gx = x\} \subseteq G$ the \textbf{stabilizer} of $x$.
\end{dfn}
Clearly, the stabilizer is a subgroup.

\begin{ex}[]
  Consider the group action of $SO(n)$ with the point $x = e_1 \in \mathbb{S}^{n-1}$.

  Its stabilizer $\text{Stab}_{SO(n)}(x)$ is isomorphic to $SO(n-1)$, as a matrix $A \in SO(n)$ satisfying $Ae_1 = e_1$ must be of the form
  \begin{align*}
    A = \begin{pmatrix}
    1 & 0\\
    0 & B
    \end{pmatrix}
    , \quad \text{where} \quad A^{T}A = 1 \implies B^{T}B = 1 \implies B \in SO(n-1)
  \end{align*}
  It is also a homoemorphism as $S^{n-1}$ is Hausdorff and $SO(n)/\text{Stab}_G(x)$ is compact (closed and bounded).
\end{ex}

A simple, yet powerful theorem is the \textbf{topological orbit theorem}.
\begin{thm}[Topological Orbit theorem]
Let $X$ be a $G$ space and $x \in X$.
Then the map
\begin{align*}
  F: \faktor{G}{\text{Stab}_{G}(x)} \to \mathcal{O}_G(x), \quad g \text{Stab}_G(x) \mapsto gx
\end{align*}
is a continuous bijection.
\end{thm}
\begin{proof}
  This mapping is indeed well defined, as
  \begin{align*}
    a \text{Stab}_G(x) = b \text{Stab}_G(x) \implies \exists h \in \text{Stab}_G(x) \text{ with } a = bh \implies ax = (bh)x = b(hx) = bx
  \end{align*}
  By definition of the orbit, it's clearly surjective.
  For injectivity, we have
  \begin{align*}
    ax = bx \implies a^{-1}(bx) = a^{-1}(ax) = x \implies a^{-1}b \in \text{Stab}_G(x)
  \end{align*}
  and for continuity, we see that the mapping
  \begin{align*}
    f \circ \pi: G \to \mathcal{O}x, \quad g \mapsto  gx
  \end{align*}
  is continuous and can use Lemma \ref{lem:unipropquot} to show continuity of $f$. 
\end{proof}


\subsection{Collapsing of subspaces to a point}
\begin{dfn}[]
Let $X$ be a topological space and $A \subseteq X$ a non-empty subset.
We write $X/A := X/\sim$ for 
\begin{align*}
  x \sim y \iff (x = y) \text{ or } x,y \in A
\end{align*}
\end{dfn}
We have already seen this type of spaces when discussing quotient topologies.
For example for $X = \mathbb{D}^{2}$ and $A = \del X = \mathbb{S}^{1}$, we got $X/A \iso \mathbb{S}^{2}$


\begin{dfn}[]
Let $X$ be a topolgical space and $A_{1}, \ldots, A_{n} \subseteq X$ be non-empty, pairwise disjoint subsets. 
We write $X/\left(
  A_{1}, \ldots, A_{n}
\right) := X/\sim$, where
\begin{align*}
  x \sim y \iff x = y \text{ or } \exists i: x,y \in A_i
\end{align*}
\end{dfn}
Note that if $X$ is metrizable (or $T_2,T_4$ ) and the $A_i$ are closed, then the quotient space is $T_2$. (We will prove this later when discussing $T_4$ spaces).

\begin{ex}[The cone]
  Given a topological space $X$, the quotient space 
  \begin{align*}
    \text{Cone}(X) :=
    \faktor{X \times [0,1]}{X \times \{1\}}
  \end{align*}
  is called the \textbf{cone} over $X$.
  The name becomes clear if we look at the case $X = \mathbb{S}^{1}$, in which case it resembles an (ice-cream) cone.
\end{ex}

\begin{ex}[Suspension]
  For a topological space $X$, the \textbf{suspension} of $X$ is the space
  \begin{align*}
    \Sigma(X) := \faktor{X \times [-1,1]}{X \times \{-1\}, X \times \{1\}}
  \end{align*}
  which can also be obtained by taking two cones and ``glueing'' them together at $X \times \{0\}$. 
  (What glueing is will be defined later but it should make intuitive sense).
\end{ex}

\begin{ex}[]
For a subset $A \subseteq X$ of a topological space, the \textbf{cone over $A$} is the space
\begin{align*}
  C_A(X) := \faktor{X \times [0,1]}{A \times \{1\}}
\end{align*}
\end{ex}

\begin{dfn}[Wedge \& Smash]
For $X, Y$ toplogical spaces with basepoints $x_0 \in X, y_0 \in Y$, we define
\begin{align*}
  \text{The \textbf{wedge} } X \vee Y &:= X \times \{y_0\} \cup \{x_0\} \times Y \subseteq X \times Y\\
  \text{The \textbf{smash} } X \wedge Y &:= \faktor{X \times Y}{X \vee Y}
\end{align*}
\end{dfn}

\begin{ex}[]
  For $X = \mathbb{S}^{n}$ and $Y = \mathbb{S}^{m}$ for $n,m \geq 1$ we have that $X \wedge Y = \mathbb{S}^{n+m}$.

To prove this, we see $\mathbb{S}^{n} = \R^{n} \cup \{\infty\}$ as the one-point-compactification of $\R^{n}$ and define the mapping \begin{align*}
  g: \R^{n} \cup \{\infty\} \times \R^{m} \cup \{\infty\} \to \R^{n+m} \cup \{\infty\}
\end{align*}
given by
\begin{align*}
  g(x,y) =
  \left\{\begin{array}{ll}
      (x,y)  & \text{ if } x \in \R^{n}, y \in \R^{m}\\
     \infty& \text{ otherwise}
  \end{array} \right.
\end{align*}
and showing that $g^{-1}(\infty) = X \vee Y$.
\end{ex}

\subsection{Glueing of topological spaces}
What glueing means intuitively should be clear, let's see if we can define it in a topological setting.

\begin{dfn}[]
Let $X,Y$ be topological spaces, $X_0 \subseteq X$ and $\phi: X_0 \to  Y$ continuous.
For the equivalence relation $\sim$ on $X \sqcup Y$ generated by $x \sim \phi(x)$, we write
\begin{align*}
  Y \cup_{\phi} X := \faktor{X \sqcup Y}{\sim}
\end{align*}
for the \textbf{glueing} of $X$ onto $Y$ by $\phi$.

\end{dfn}


\begin{rem}[]
  Note that the embedding $X \to Y \cup_{\phi}X, x \mapsto [x]$
  is continuous as it is equal to $\pi \circ \iota_X$, for $\iota_X$ the inclusion and $\pi$ the projection mapping.

  The glueing is a generalisation of the collapsing of subspaces, as for $A \subseteq X$, we have a homeomorphism
  \begin{align*}
    f: X/A \to \{\ast\} \cup_{\phi} X, \quad [x] \mapsto  [x]
  \end{align*}
  for $\phi: A \to \{\ast\}$, because $x,y \in A \iff \phi(x) = \phi(y)$.

    The mapping $Y \to Y \cup_{\phi} X, \quad y \mapsto [y]$ is also inective and a homeomorphism to its image as the map is not only continuous, but open.
    (The same might not be true for $X$)
\end{rem}

\begin{ex}[Mapping torus]
  Let $\alpha: X \to  X$ be a homeomorphism.
  We call
  \begin{align*}
    \faktor{X \times [0,1]}{\alpha} := \faktor{X \times [0,1]}{\sim}, \quad \text{ with } \quad (x,0) \sim (\alpha(x),1)
  \end{align*}
  the \textbf{mapping torus} of $\alpha$.
  
  \begin{itemize}
    \item The mapping torus of $\alpha = \id_{\mathbb{S}^{1}}$ is the ordinary \textbf{Torus}.
    \item The \textbf{Moebius strip} $M$ is the mapping torus of $\alpha: [-,1] \to  [-1,1], x \mapsto -x$.
    \item The \textbf{Klein bottle} $K$ is the mapping torus of $\alpha: \mathbb{S}^{1} \to \mathbb{S}^{-1}, z \mapsto \overline{z}$.
  \end{itemize}
  If we glue two Moebius strips together with the mapping 
  \begin{align*}
  \phi: 
  \underbrace{\pi(\{-1,1\} \times[0,1])}_{= \del M} \to M
  , \quad 
  m \to  m
  \end{align*}
  we obtain the Klein bottle $K = M \cup_{\phi} M$.
  To show this, we use the mapping
  \begin{align*}
    g: [-1,1] \times [0,1] \sqcup [-1,1] \times [0,1] \to  \mathbb{S}^{1} \times [0,1]
    \\
    (x,t) \cup (y,s) \mapsto (e^{i\pi x /2},t), (-e^{-i \pi y/2},s)
  \end{align*}

  
\end{ex}




