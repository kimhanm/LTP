Recall the separation axioms $T_1,T_2,T_3,T_4$.
\begin{dfn}[]
  A topological space $X$ is called \textbf{normal}, if it satisfies the $T_4$ axiom, i.e. closed disjoint subsets can be separated by open subsets.
  i.e. For all such subsets $A,B$ there exist open subsets $U,V \subseteq X$ with $A \subseteq U, B \subseteq V$ and $V \cap U = \emptyset$.
  
\end{dfn}
\begin{rem}[]
  We do \textbf{not} call normal spaces ``$T_4$-spaces''. That is a definition on its own and should not be confused with the $T_4$ \texttt{axiom}.
\end{rem}
\begin{lem}[]
Metric spaces are normal
\end{lem}
\begin{proof}
Let $A,B \subseteq X$ closed and disjoint.
Since $B$ is closed, $\forall  a \in A: \exists \epsilon_a$ such that $B_{\epsilon_a}(a) \cap B = \emptyset$.
Then we take the union of all such balls, but with half their radii. And do the same for $A$.
\begin{align*}
  U = \bigcup_{a \in A}B_{\frac{\epsilon_a}{2}}(a)
  V = \bigcup_{b \in B}B_{\frac{\epsilon_b}{2}}(b)
\end{align*}
and clearly, $U \cap V = \emptyset$.
\end{proof}


\begin{dfn}[]
Let $X$ be Hausdorff. 
\begin{itemize}
  \item If $X$ satisfies the $T_3$ axiom, then $X$ is called a $T_3$ space.
  \item If $X$ satisffies the $T_4$ axiom, then $X$ is called a $T_4$ space.
\end{itemize}
\end{dfn}
\begin{rem}[]
  $X T_2$ and $T_2 \implies X T_2$ and $T_3 \implies X T_2 \implies T_1$. 
\end{rem}


\begin{thm}[Uhrysohn's lemma]
Let $X$ be a normal space. If $A,B \subseteq X$ are closed disjoint subsets, then there exists a continuous function
$f: X \to [0,1]$ such that $f|_A = 1, f|_B = 0$.
\end{thm}

We will prove it using the following lemma
\begin{lem}[Refinement lemma]
Let $X$ be normal and $M \subseteq N \subseteq X$ with $\overline{M} \subseteq \stackrel{o}{N}$.
Then there exists a subset $L \subseteq X$ with
\begin{align*}
  \overline{M} \subseteq \stackrel{o}{L} \subseteq \overline{L} \subseteq \stackrel{o}{N}
\end{align*}
In this case, we write $M < N \implies \exists L: M < L < N$
\end{lem}
\begin{proof}
  Note that the sets $\overline{M}$ and $X \setminus \stackrel{o}{N}$ are closed and disjoint. Since $X$ is normal there exist disjoint open subsets $U,V$ with $\overline{M} \subseteq U$ and $X \setminus \stackrel{o}{N} \subseteq V$.

  If we chose $L = U$, then $U \subseteq L \subseteq X \setminus V$ and since $X \setminus V$ is closed, we also have $U \subseteq X \setminus V$.
  \begin{align*}
    \overline{M} \subseteq L = \stackrel{o}{L} \subseteq \overline{L} \subseteq X \setminus V
  \end{align*}
\end{proof}

\begin{proof}[Proof theorem]
  We will construct a sequence of functions $f_n: X \to [0,1]$ that converges to a continuou function with $f(A) = \{1\}$ and $f(B) = \{0\}$.

  In the first step, we have $A< (X \setminus B)$ so by our lemma we get a set $L_{\frac{1}{2}}$ with $A < L_{\frac{1}{2}} < B$ and we can define a step function
  \begin{align*}
    f_1: X \to [0,1] \quad f_1|_A = 1, \quad f_1|_{L_{\frac{1}{2}} \setminus A } = \frac{1}{2}, \quad f_1|_{X \setminus L_{\frac{1}{2}}} = 0
  \end{align*}
  and inductively we keep refining the the sets to get differing levels.
  i.e. in the second step we create levels with values $\frac{k}{4}$
  \begin{align*}
    A = L_1 < L_{\frac{3}{4}} < L_{\frac{1}{2}} < L_{\frac{1}{4}} < L_0 = B  
  \end{align*}
  and in the general step 
  \begin{align*}
    A = L_1 < L_{\frac{2^{n}-1}{2^{n}}} < L_{\frac{2^{n-1} -1 }{2^{n-1}}} < \ldots < L_{\frac{1}{2^{n-1}}} < L_{\frac{1}{2^{n}}}<L_0 = B
  \end{align*}
  and define the step function as
  \begin{align*}
    f_n|_{L_{\frac{k}{2^{n}}}/L_{\frac{k+1}{2^{n}}}} = \frac{k}{2^{n}}, \quad k \in \{0,1,\ldots,2^{n-1}\}
  \end{align*}
  and by construction we still have $f|_A = 1, f|_B = 0$.

  Finally, we take the limit $f := \lim_{n \to \infty}f_n$ and show continuity:
  Let $\epsilon > 0, x \in X$.

  If $x \in A$ chose $n$ such that $\frac{1}{2^{n}} < \epsilon$ and set $U = L_{\frac{2^{n}-1}{2^{n}}} \supseteq A$ and clearly for all $y \in U$
  \begin{align*}
    \abs{f(x) - f(y)} = 1 - f(y) \leq 1 - f_n(y) \leq 1 - \frac{2^{n} - 1}{2^{n}} < \epsilon
  \end{align*}
  For $x \in B$ the argument is exactly the same, but for $x \notin A \cup B$ we take an $n in \N$ such that $\frac{1}{2^{n-1}} < \epsilon$.
  Then
  \begin{align*}
    X \setminus (A \cup B) = \bigcup_{k=1}^{2^{n}-1}
    \stackrel{o}{L_{\frac{k-1}{2^{n}}}} \setminus \overline{L_{\frac{k+1}{2^{n}}}}
  \end{align*}
  then there exists an index $k$ such that $x$ is in one of those levels
  \begin{align*}
    x \in \stackrel{o}{L_{\frac{k-1}{2^{n}}}} \setminus \overline{L_{\frac{k+1}{2^{n}}}} := U
  \end{align*}
  and so for all $y \in U$ we have $f(y) \in [\frac{k-1}{2^{n}}, \frac{k+1}{2^{n}}]$ by construction, which shows continuity.
\end{proof}


\subsection{Tiezsche Extension lemma}

\begin{thm}[Tiezsche Extension Lemma]
  Let $a < b \in \R$ and $X$ normal, $C \subseteq X$ closed and $f: C \to [a,b]$ continuous.
  Then $f$ can be extended to a continuous function $F:X \to [a,b]$ with $F|_C = f$.
\end{thm}
\begin{cor}[]
  The extension theorem also holds for arbitrary proudcts of closed intervals $[a_i,b_i]_{i \in I}$
\end{cor}

\begin{proof}[Proof corollary]
  Let $F: C \to \prod_{i \in I}[a_i,b_i]$ continuous.
  Then also $\pi_i \circ f: C \to [a_i,b_i]$ is continuous for all $i \in I$. 
  By the theorem these can be extended to functions $F_i: X \to  [a_i,b_i]$ and by the universal property of the product, there exists a continuous function
  \begin{align*}
    F: X \to \prod_{i \in I}[a_i,b_i] \text{ with }F|_C = f
  \end{align*}
\end{proof}


\begin{cor}[]
Tietzsche extension theorem can also be applied to functions $f: C \to \R$.
\end{cor}
If we take the homeomorphism $\phi: \R \to (-1,1) \subseteq [-1,1]$ then $\tilde{f} = \phi \circ f$ can be extended to an $\tilde{F}: X \to [-1,1]$.
To remove the points $\pm 1$ again, we 
setting $A = C$ and $B = \tilde{F}^{-1}(\pm 1)$.
Then by Uhrysohn's lemma there exists a function $\lambda: X \to [0,1]$ that is $1$ in $A$ and $0$ in $B$.

Then we can set
\begin{align*}
  \hat{F}: X \to  [-1,1], \quad x \mapsto  \tilde{F}(x) \lambda(x)
\end{align*}
then its image is only in $(-1,1)$, so the function $F = \phi^{-1}\circ\hat{F}$ is an extension of $f$.


\begin{proof}[Proof Theorem]
  Without loss of generality we can assume $[a,b] = [-1,1]$, so let $f: C \to [-1,1]$ continuous.

  What we do is create a sequence of continuous functions $F_n:X \to  [-1,1]$ such that
  \begin{enumerate}
    \item $\abs{F(c) - F_1(c) + F_2(c) + \ldots + F_n(c)} \leq (\frac{2}{3})^{n}$ for all $c \in \C$
    \item $\abs{F_n(x)} \leq \frac{1}{3} (\frac{2}{3})^{n-1}$ for all $x \in X$
  \end{enumerate}
  and then set $F : \sum_{n=1}^{\infty}F_n(X)$ which converges uniformly because it's the geometric series and thus is continuous.
  
  To construct the sequence we set
  \begin{align*}
    A_0 = f^{-1}([\tfrac{1}{3},1]), \quad B_0 := f^{-1}([-1,-\tfrac{1}{3}])
  \end{align*}
  which are closed disjoint subsets of $C$ (and thus also closed in $X$).
  By Uhryson's lemma there exists a function $F_1: X \to [-\tfrac{1}{3}, \tfrac{1}{3}]$.
  and we can set $f_1 := f - F_1|_C: [-\tfrac{2}{3},\tfrac{2}{3}]$.

  In the next step, we set
  \begin{align*}
    A_1 := f_1^{-1}([\frac{1}{3}\frac{2}{3}, \frac{2}{3}]), \quad B_1 := f_1^{-1}([- \frac{2}{3}, - \frac{1}{3} \frac{2}{3}])
  \end{align*}
  and again by Uhrysoh'ns lemma there exists a function 
  \begin{align*}
    F_2: X \to [- \frac{1}{3} \frac{2}{3}, \frac{1}{3} \frac{2}{3}] \quad \text{with} \quad F_2|_{A_1} = \frac{1}{3} \frac{2}{3}, \quad F_2(B_1) = -\frac{1}{3}\frac{2}{3}
  \end{align*}
  and just like before, we set 
  \begin{align*}
    f_2 := f_1 - F_2|_C : C \to [ - (\frac{2}{3})^{2}, (\frac{2}{3})^{2}]
  \end{align*}
  and so on. In the end we get a sequence of functions
  \begin{align*}
    f_n: C \to  [- (\frac{2}{3})^{n}, (\frac{2}{3})^{n}]
  \end{align*}
  which satisfies (a):
  \begin{align*}
    \underbrace{\underbrace{f(c) - F_1(c)}_{f_1(c)} - F_2(c)}_{= f_2(c)} - \ldots - F_n(c) = f_n(c) \in [-(\frac{2}{3})^{n}, (\frac{2}{3})^{n}]
  \end{align*}
  aswell as (b):
  \begin{align*}
    \abs{F_n(x)} \leq \frac{1}{3} (\frac{2}{3})^{n-1}
  \end{align*}
\end{proof}
