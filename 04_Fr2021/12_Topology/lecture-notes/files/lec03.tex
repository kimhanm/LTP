For sets $X,Y$ let $X \sqcup Y$ denote their disjoint union $X \times \{0\} \cup Y \times \{1\}$ and $X \times Y$ their cartesian product (as sets).
\begin{dfn}[]
  Let $(X,\tau_X)$ and $(Y,\tau_Y)$ be two topological spaces. 
  \begin{itemize}
    \item Their \textbf{coproduct} is the topological space $(X \sqcup Y, \tau_{X \sqcup Y}$, where
      \begin{align*}
        \tau_{X \sqcup Y} := \{U \sqcup V \big\vert U \in \tau_X, V \in \tau_Y\}
      \end{align*}
    \item Their (cartesian) \textbf{product} is the topological space $(X \times Y, \tau_{X \times Y})$, where
      \begin{align*}
        \tau_{X \times Y} := \{W \subseteq X \times Y \big\vert \forall (x,y) \in W \exists U \in \tau_X, \exists V \in \tau_Y: (x,y) \in U \times V \subseteq W \}
      \end{align*}
  \end{itemize}
\end{dfn}
Note that \emph{not} every open subset $W \subseteq X \times Y$ is of the form $W = U \times Y$, for $U \subseteq X$, $V \subseteq Y$ open. For example, the product topology on $\R^{2}$ contains open balls.


\subsection{Basis and Subbasis}
Consider $\R^{n}$ with the euclidean topology. 
We can intuitively see that every open set $U \subseteq \R^{n}$ can be written as the union of open balls.
So $U \subseteq \R^{n}$ is open if and only if there exists $(x_i,r_i)_{i \in I}$ such that $U = \bigcup_{i \in I}B_{r_i}(x_i)$.

\begin{dfn}[]
Let $X$ be a topological space and $\mathcal{B}$ a collection of open sets.
\begin{itemize}
  \item We call $\mathcal{B}$ a \textbf{basis} of the topology, if every open set can be written as a union of elements of $\mathcal{B}$.
  \item $\mathcal{B}$ is called a \textbf{subbasis} of the topology, if every open set can be written as a union of finite intersection of elements of $\mathcal{B}$.
\end{itemize}
\end{dfn}
\begin{rem}[]
  Every basis is a subbasis.
  Every collection $\mathcal{B}$ of open sets is the subbasis of a unique topology, the topology \textbf{generated} by $\mathcal{B}$, which is the smallest topology that contains $\mathcal{B}$.
\end{rem}


