\begin{rem}
  The following are pretty easy to prove
  \begin{enumerate}
    \item $X$ path connected $\implies$ $X$ connected. 
    \item The image of path connected spaces under continuous maps is path connected.
  \end{enumerate}
\end{rem}
\begin{proof}
\begin{enumerate}
  \item Let $X = U \sqcup V$ with $U,V \subseteq X$ non-empty and open. We can therefore chose $a \in U$ and $b \in V$. Since $X$ is path connected, there exists a continuous map $\gamma: [0,1] \to  X$ such that $\gamma(0) = a$ and $gamma(1) = (b)$.
    But since $[0,1]$ is connected, its image $\gamma([0,1])$ must also be connected.

    The decomposition $\gamma([0,1]) \cap U \sqcup \gamma([0,1]) \cap V$ is clearly disjoint, open and non-empty, but since $\gamma([0,1])$ is connected, their intersection is non-empty so
    \begin{align*}
      \exists c \in (\gamma([0,1]) \cap U) \cap (\gamma([0,1]) \cap V) \subseteq U \cap V
    \end{align*}
  \item It is trivial as the composition of contiuous maps is continuous. For $a,b \in X$ connected by $\gamma$, we can connect their images $f(a), f(b)$ using the composition $f \circ \gamma: [0,1] \to f(X)$.
\end{enumerate}
\end{proof}

\begin{ex}
  The converse, $X$ connected $\implies$ $X$ path connected is not always true.

  Take for example the closure of the \textbf{Topologists sine curve}:
  \begin{align*}
    X := \{0\} \times [-1,1] 
    \sqcup 
    \left\{\left(t, \sin \frac{1}{t}\right)
    \in \R^{2}\big\vert t \in (0,1]\right\} \subseteq \R^{2}
  \end{align*}
  where we write it as the disjoint union $X_0\sqcup X_1$.

  If we let $a = (1,\sin(1))$ and $b = (0,0)$ and assume that there exists a path $\gamma:[0,1] \to X$. Since the set $\{t \in [0,1] \big\vert \gamma_1(t) = 0\}$ is non-empty and closed it attains its minimum $s$. But then
  \begin{align*}
    \gamma_1([0,s]) \subseteq (0,1] \quad \lim_{t \to s} \gamma_1(t) = 0 \quad \gamma_1(0) = 1\\
    \gamma_1([0,s])) = (0,1] \implies \gamma([0,s) = X_1
  \end{align*}
  By the form of the sine curve, we can get a sequence of points $\left(s_{n}\right)_{n \in \N}$ whose image of the inverse sine function are its peaks:
  \begin{align*}
    \lim_{n \to \infty}s_n = s \quad \text{and} \quad \gamma_2(s_n) = 1
  \end{align*}
  but by the form of the sinus curve, we must get a sequcence $\left(t_{n}\right)_{n = 1}^{\infty}$ whose image is always the valles of the sine curve.
  \begin{align*}
    \lim_{n \to \infty} t_n = s \quad \text{and} \quad \gamma_2(t_n) = -1
  \end{align*}
  which contradicts continuity of $\gamma$.


  On the other hand, we can show that $X$ is connected. Assume we had a disjoint open nonempty partition $X = U \sqcup V$.
  Analogously to the reasoning above, we can assume without loss of gerality that $U = X_0$ and $V = X_1$.

  But since $U = X_0$ should be open in the subspace topology of $\R^{2}$, there must be an open set $\tilde{U} \subseteq \R^{2}$ such that $U = \tilde{U} \cap X_0 \ni (0,0)$. 
  It is clear however, that any neighborhood of $(0,0)$ has non-empty intersection with $X_1$.
\end{ex}


\begin{rem}
  \begin{itemize}
    \item The integers with the co-finite topology is connected but not path connected.
    \item For any finite topological space $X$ it is true that $X$ connected if and only if $X$ is path connected.
    \item For $X,Y$ non-empty topological spaces, then
      \begin{align*}
        X, Y \text{ (path) connected } &\iff X \times Y \text{ (path) connected}
      \end{align*}
    \item For subsets $A,B \subseteq X$ with $A \cap B \neq \emptyset$ we have
\begin{align*}
  A,B \text{ (path) connected } \implies A \cup B \text{ (path) connected}  
\end{align*}
    \item $X$ is \emph{not} path connected if and only if there exists a continuous map
      \begin{align*}
        f: X \to (\{0,1\}, \tau_{\text{disc.}})
      \end{align*}
      As a quick consequence, we get that $O_n(\R)$ is not connected. We can also show that its conneceted components are those with determinant $\pm 1$ each.
  \end{itemize}
\end{rem}
