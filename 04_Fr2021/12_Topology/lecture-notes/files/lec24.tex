\begin{lem}[]\label{lem:alt-covering}
There is an alternative definition of a covering space.
Let $\pi: Y \to X$ continuous and surjective. Then $\pi$ is a covering if and only if:

For all $x \in X$ there exists an open neighborhoood $U \subseteq X$ of $x$ and disjoint open subsets $(U_i)_{i\in I}$ of $Y$ such that
\begin{align*}
  \pi^{-1}(U) = \bigsqcup_{i \in I}U_i \quad \text{and} \quad \pi|_{U_i}^{U}: U_i \to U \text{ is a homeomorphism}
\end{align*}
\end{lem}
\begin{proof}
\begin{itemize}
  \item[$\implies$] Let $U \subseteq$ as in the definition. Set $U_i := U^{-1}(U \times \{i\})$ for all $i \in F$.
    By construction, they are open and disjoint and the projection $\pi|_{U_i}^{U}$ 
  \item[$\Leftarrow$] Let $U$ as in the lemma. Then set $F := \pi^{-1}(X) \subseteq Y$ with the subspace topology.
  Because $F \cap U_j = \{y_j\}$, we know that $F$ is discrete.
  Then for all $y_i \in \pi^{-1}(x)$ we define the map
  \begin{align*}
    \phi: \pi^{-1}(U) = \bigcup_{i \in I}U_i \to  U \times F, \quad y \mapsto (\pi(y),j)
  \end{align*}
  which is a homeomorphism.
\end{itemize}
\end{proof}

\begin{dfn}[]
A function $\pi: Y \to  X$ is called \textbf{local homeomorphism}, if
$\forall y \in Y$ there exists a neighborhood $V \subseteq Y$ of $y$ open and an open subset $U \subseteq X$ such that $\pi|_{V}^{U}: V \to  U$ is a homeomorphism.
\end{dfn}
Now that if $\abs{\pi^{-1}(x)} = 1$ for all $x \in X$. Then this is just a homeomorphism.

Local homeomorphisms are not the same as coverings!
\begin{ex}[]
  Set $X = \R$ and $Y = \R \times \{0\} \cup (0,\infty) \times \{1\}$ with the map
  \begin{align*}
    \pi: Y \to  X, (a,b) \mapsto  a
  \end{align*}
  is a local homeomorphism, but not a covering.
\end{ex}

\begin{dfn}[]
Let $\pi: Y \to  X$ be a covering.
\begin{itemize}
  \item The subset $U \subseteq X$ is open as in Lemma \ref{lem:alt-covering} is said to be \textbf{uniformly covered} by $\pi$ and the $U_i$ is called a \textbf{leaf} of $\pi$ over $U$.
  \item A covering $\pi$ is called a covering \textbf{of $n$ leaves}, if $\abs{\pi^{-1}(x)} = n$ for all $x \in X$.
\end{itemize}
\end{dfn}


\begin{ex}[]
Let $d \in \N$. The map
\begin{align*}
  \pi: \IS^{d} \to \R\P^{d} = \IS^{d}/\sim, \quad v \sim - v, \quad v \mapsto [v]
\end{align*}
is a covering of $2$ leaves.
\end{ex}


Our next goal is to show a correspondence between coverings of path connected spaces and subgroups of the fundamental group.
This correspondence is the reminiscent of the Galois correspondence we know from Algebra II. 

For example, if we know that the fundamental group of the circle is $\Z$.
Its subgroups are $\Z/n\Z$ for some $n\in \N$ which correspond to the maps
\begin{align*}
  \IS^{1} \to  \IS^{1}, \quad z \mapsto  e^{2 \pi i n z}, \quad n \in \N
\end{align*}

\section{Lifts}
Given a covering $\pi: Y \to X$ and a continuous map $f: Z \to  X$, we would like to ``lift'' $f$ to a function $\tilde{f}: Z \to  Y$. such that the following diagram commutes.
\begin{center}
\begin{tikzcd}[ ] %\arrow[bend right,swap]{dr}{F}
  & Y \arrow[]{d}{\pi}\\
  Z \arrow[]{r}{f}\arrow[dotted]{ur}{\tilde{f}} & X
\end{tikzcd}
\end{center}

\begin{dfn}[]
  Let $\pi: Y \to X$ a covering $a < b \in \R$ and $\alpha:[a,b] \to  X$ a path.
  A path $\tilde{\alpha}:[a,b] \to Y$ is called a \textbf{lift} of $\alpha$ to the starting point $y_0$ if $\tilde{\alpha}(a) = y_0$ and the following diagram commutes:
\begin{center}
\begin{tikzcd}[ ] %\arrow[bend right,swap]{dr}{F}
  & Y \arrow[]{d}{\pi}\\
  {[a,b]} \arrow[]{r}{\alpha}\arrow[dotted]{ur}{\tilde{\alpha}} & X
\end{tikzcd}
\end{center}
\end{dfn}
for example for the map
\begin{align*}
  \alpha_k: [0,1] \to \IS^{1}, \quad t \mapsto e^{2 \pi i k t}
\end{align*}
a lift of it is the path
\begin{align*}
  \tilde{\alpha}_k: [0,1] \to \R, t \mapsto kt
\end{align*}
where the covering is given by
\begin{align*}
  \pi: \R \to \IS^{1}, x \mapsto  e^{2 \pi i x}
\end{align*}

\begin{lem}[]
  Let $\pi: Y \to X$ a covering, $\alpha:[a,b] \to X$ a path and $y_0 \in \pi^{-1}(\alpha(a))$.
  
  Then there exists a unique lift of $\alpha$ to $y_0 \in Y$.
\end{lem}
\begin{proof}
  Without loss of generality we can assume $[a,b] = [0,1]$.
  First note that for any $U \subseteq X$ open and uniformly covered: 
  for any path $\beta:[0,1] \to U$, there exists a unique lift of $\beta$ to a $y_j \in \pi^{-1}(\beta(0))$.
  This is because the lift is uniquely determined by
  \begin{align*}
    \tilde{\beta}_j := \left(
      \pi|_{U_j}^{U}
    \right)^{-1} \circ \beta
  \end{align*}
  In particular, the images of $\tilde{\beta}_j$ and $\tilde{\beta}_k$ are disjoint for $j \neq k \in I$.
  \begin{itemize}
    \item For uniqueness of the lift, let $\tilde{\alpha}, \hat{\alpha}$ be two lifts of $\alpha$ to the point $y_0$.
      Then set
      \begin{align*}
        I := \{t \in [0,1] \big\vert \tilde{\alpha}(t) = \hat{\alpha}(t)\}
      \end{align*}
      Because $\tilde{\alpha}$ and $\hat{\alpha}$ start at the same point $y_0$, $I$ is non-empty.

      $I$ is also open, because for any $t_0 \in I$ chose a $U \subseteq X$ open and uniformly covered with $\alpha(t_0) \in U$.
      Then chose $a < b$ such that either
      \begin{align*}
        a < t_0 < b, \text{ or } a = 0 = t_0 < b \text{ or } a < t_0 = b = 1
      \end{align*}
      in any case, we noted that $\tilde{\alpha}$ and $\hat{\alpha}$ have to agree on $[a,b]$. 

      With the same argument, we can show that $I$ is also closed, because for some $t_0 \in [0,1] \setminus I$, we chose $U,a,b$ like before and show that $\tilde{\alpha}(t) \neq \hat{\alpha}(t)$ for all $t \in [a,b]$.
    \item For existence, set
      \begin{align*}
        I := \{t \in [0,1] \big\vert \alpha|_{[0,t]} \text{ has a lift to $y_0$}\}
      \end{align*}
      For $T := \sup_{t \in I}$ we either have $I = [0,T)$ or $I=[0,T]$.

      Chosing $U$ uniformly covered with $\alpha(T) \in U$ and $a<b$ as before, we can chose a lift $\tilde{\beta}:[a,b] \to  Y$ of $\alpha|_{[a,b]}$ with $\tilde{\beta}(a) = \tilde{\alpha}(a)$
      Then set set
      \begin{align*}
        \tilde{\alpha}:[0,b] \to  Y, \quad \tilde{\alpha}(t) := \left\{\begin{array}{ll}
            \tilde{\alpha}|_{[0,a](t)} & t \leq a\\
            \tilde{\beta}(t) &  t \in [a,b]
        \end{array} \right.
      \end{align*}
      then $\tilde{\alpha}$ is a lift of $\alpha|_{[0,b]}$.

      Which shows that if $T<1$, this is a contradiction that $T$ is the supremum, so we must be in the case, where $a < T = b = 1$.
  \end{itemize}
\end{proof}

Next, we want to lift homotopies.
\begin{lem}[]\label{lem:homotopy-lift}
  Let $\pi: Y \to X$ be a covering, $Z$ a topological space and $h: Z \times [0,1] \to  X$ a continuous function.

  If $\tilde{h}_0: Z \to Y$ is a lift of $h_0 = h(-,0): Z \to X$.

  Then there exists a unique lift $\tilde{h}:Z \times [0,1] \to Y$ such that the following diagram commutes
  \begin{center}
  \begin{tikzcd}[ ] %\arrow[bend right,swap]{dr}{F}
    Z \arrow[]{r}{\tilde{h}_0} \arrow[swap]{d}{\iota}& Y \arrow[]{d}{\pi}\\
    Z \times [0,1] \arrow[]{r}{h}\arrow[dotted]{ur}{\exists!\tilde{h}} & X
  \end{tikzcd}
  \end{center}
  in other words: if the outer square of the diagram commutes, then there exists a $\tilde{h}$ completing the diagram.

\end{lem}
\begin{rem}[]
  We say that the inclusion
  \begin{align*}
    \iota: Z \to Z \times [0,1], z \mapsto (z,0)
  \end{align*}
  has the \textbf{left lifting property} with respect to the morphism $\pi$.
\end{rem}
