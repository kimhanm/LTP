Two applications of the functoriality of the fundamental group are the Brower fixpoint theorem, aswell as the fundamental theorem of Algebra.


\begin{thm}[Brower Fixpoint theorem]
  Let $n \in \N$ and $f: \ID^{n} \to \ID^{n}$ continuous. 
  Then $f$ has a fixpoint, so $\exists x \in \ID^{n}$ with $f(x) = x$.
\end{thm}
We will only prove it for $n \leq 2$ by using the fundamental group. For higher $n$, we will have to use another invariant. 
\begin{proof}
  Assume that $f$ has no fixpoint.
  We then can show that there exists a retraction $\rho: \ID^{n} \to  \del \ID^{n} = \IS^{n-1}$.

  The reason this is true is because if $f(x) \neq x$, then there exists a unique line from $x$ to $f(x)$.
  We then define $\rho(x)$ to be the intersection of that line with $\del \ID^{n}$, so
  \begin{align*}
    \{\rho(x)\} = \del \IS^{n-1} \cap \{f(x) + t\left(
        x - f(x)
    \right) \big\vert t \in \R_{> 0}\}
  \end{align*}
  this function is continuous, because in a neighborhood $U$ of $x$, $f(U)$ is a neighborhood of $f(x)$, and geometrically $\rho(U)$ is a neighborhood of $\rho(x)$.

  Now, we show that there cannot exist such a retraction $\rho: \ID^{n} \to \IS^{n}-1$.

  We already saw this for $n = 1$, because $\ID^{1}$ is connected and $\{0,1\}$ isn't.

  For $n = 2$ assume there would exist such a retraction $\rho: \ID^{2} \to \IS^{1}$.
  This means
  \begin{align*}
   \id_{\IS^{1}} = \rho \circ \iota, \quad \text{where} \quad \iota: \IS^{1} \to \ID^{2}, s \mapsto s
  \end{align*}
  but by functoriality of the fundamental group we would have
  \begin{align*}
    \id_{\pi_1(\IS^{1},1)} = \left(
      \id_{\IS^{1}}
    \right)_\ast 
    =
    (\rho \circ \iota)_{\ast} = \rho_{\ast} \circ \iota_{\ast}
  \end{align*}
  where
  \begin{align*}
    \rho_{\ast}: \pi_1(\ID^{2},1) \to  \pi_1(\IS^{1},1) \quad \text{and} \quad \iota_{\ast}: \pi_1(\IS^{1},1) \to  \pi_1(\ID^{2},1)
  \end{align*}
  But this contradicty our previous calculation of their fundamental groups, where we found 
  $\pi_1(\ID^{2},1) = \{e\}$ and $\pi_{1}(\IS^{1},1) = \Z$.

  In a diagram, the argument looks like this
  \begin{center}
  \begin{tikzcd}[ ] %\arrow[bend right,swap]{dr}{F}
    \IS^{1} \arrow[]{r}{\iota} 
    & \ID^{2} \arrow[]{r}{\rho}
    & \IS^{1} \arrow[bend left]{d}{\pi_1}\\
    \Z \arrow[]{r}{\iota_{\ast}} 
    & \{e\} \arrow[]{r}{\rho_{\ast}}
    & \Z
  \end{tikzcd}
  \end{center}
\end{proof}
In the proof, we showed that if $\rho: X \to A$ is a retraction and $\iota: A \to  X$ is the embedding, then $\rho_{\ast}$ is surjective and $\iota_{\ast}$ is injective.

Moreover, if $\rho$ is a strong deformation retract, then $\rho_{\ast}$ and $\iota_{\ast}$ are inverse group homomorphisms.


Let's calculate some more fundamental groups.
\begin{ex}[]\label{ex:homotopy-of-spheres}
  Because $\IS^{1} \subseteq \C^{\ast}$ and
  $\IS^{n-1} \subseteq \R^{n} \setminus \{0\}$ for $n \geq 3$ 
  are strong deformation retracts, we have
\begin{align*}
  \pi_1(\R^{n} \setminus \{0\}) = \left\{\begin{array}{ll}
    \Z & n = 2\\
    \{e\} & n \geq 3
  \end{array} \right.
\end{align*}
\end{ex}


\begin{thm}[Fundamental theorem of Algebra]
  Let $p(z) = z^{d} + a_{d-1}z^{d-1} + \ldots + a_0 \in \C[z]$ be a non-constant polynomial.
  Then $\exists x_0 \in \C$ with $p(z_0) = 0$
\end{thm}
\begin{proof}
  Assume $p(z) \neq 0 \forall z \in \C$. 
  Then chose a radius $r$ large enough, (for example $r > \sum_{j=0}^{d-1}\abs{a_j}$) and construct he path
  \begin{align*}
    \alpha: [0,1] \to \C^{\ast}, \quad s \mapsto  \frac{p(re^{2 \pi i s})}{p(s)}
  \end{align*}
  Then, $\alpha$ is homotopic to the constant map $1$ with the homotopy
  \begin{align*}
    h(s,t) = \frac{p(tre^{2 \pi i s})}{p(tr)}
  \end{align*}
  If we move along the path $d$-times, we get the path $\alpha^{d}$, which is again homotopic to $\alpha$ by
  \begin{align*}
    h(s,t) = \frac{tp(re^{2 \pi i s}) + (1-t)(re^{2 \pi i s})^{d}}{tp(r) + (1-t) r^{d}} \in \C^{\ast}
  \end{align*}
  which is well-defined because
  \begin{align*}
    \abs{
      tp(re^{2 \pi i s}) + (1-t)(re^{2 \pi i s})^{d}
    }
    &\geq r^{d} - t \sum_{j=0}^{d -1}\abs{a_j}r^{j} \\
    &=
    r^{d-1}\left(r - t \sum_{j=0}^{d-1}\abs{a_j} r^{j - dt} \right)\\
    &\geq
    r^{d-1}(r - t \sum_{j=0}^{d-1}\abs{a_j}) \geq r^{d-1}\left(r - \sum_{j=0}^{d-1}\abs{a_j}\right) > 0
  \end{align*}
  but by the previous example with $\pi_1(\R^{2} \setminus \{0\}$ we would have
  \begin{align*}
    1 = [\alpha] = [\alpha_d] \neq 1 \quad \text{ in } \pi_1(\C^{\ast},1) \lightning
  \end{align*}
\end{proof}


The fundamental group by definition depends on the choice of the basepoint, but how much?
Can we predict when a change of the basepoint changes the fundamental group?
