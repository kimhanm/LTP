\section{Introduction}
Topology is the study of topological spaces and continous maps between them.

We will introduce topology through three aspects:
\begin{itemize}
  \item Topology as the part of Analysis which concerns itself about general properties of continuous functions.
    More concretely, we know that for a continuous function $f: \R \to \R$, that on a compact interval, $f$ attains its maximum and minimum. 
    We also know that the intermediate value theorem holds.

    In the language of topology, these theorems say that the image of \emph{compact} subsets is again \emph{compact}.
    It also says that the image of \emph{connected} subsets is again \emph{connected}.

    One generalisation of the intermediate theorem is Jordan's curve theorem: Let $\gamma$ be a closed simple curve in $\R^2$. Then $\R^2 \setminus \gamma$ has two connected components.
  \item Topology as the study, construction and classification of topological spaces.
    We can look at topology as the generalisation of metric spaces by replacing the notion of \emph{distance} with the notion of \emph{neighborhoods}.
    For example, we can try to use the square $Q = [0,1]^2 \subseteq \R$ to generate a new object by introduction an equivalence relation $\sim$, where $(1,y) \sim (0,y) \forall y \in [0,1]$. The \emph{quotient space} is obtained by \emph{glueing} two edges of the square which generates a cylindrical shape as a subset of $\R^3$. Using different kinds of equivalence relations, we can get the Moebius strip, a Torus or the Klein bottle.
    Another nice example is when we start with $\C^\ast = \C \setminus \{0\}$. Consider the quotient space
    \begin{align*}
      \faktor{\C^\ast}{\R_{>0}}: \quad \text{where} \quad z \sim w, \text{ if } \frac{z}{w} \in \R_{> 0}
    \end{align*}
    Which identifies two points, if they have the same argument. This space is \emph{homeomorphic} to the unit circle $\mathbb{S}^1 \subseteq \C$.
    Note that $(\C^\ast, \cdot)$ is a group, with $(\R_{>0}, \cdot)$ as  a normal divisor and $(\mathbb{S}^1,\cdot)$ is another subgroup of $(\C^\ast,\cdot)$.
    Another result is the classification of surfaces: Every \emph{orientable} compact surface without a boundary is homeomorphic to exactly one of the following surfaces:
  \begin{align*}
    \text{the unit ball} \mathbb{S}^2, \quad \text{higher tori of genus } n
  \end{align*}
  Other algebraic invariations that are used to classify topological spaces can be Numbers (\emph{euler characateristic}), Groups (\emph{Fundamental Groups}), fields, rings, vectorspaces etc.
  
  \item Construction and criteria for existence of continous maps.
    For example, let $U \subseteq \C^\ast$ be open and connected. Consider the exponential map $\exp: \C \to \C^\ast$, and the inclusion $\iota:U \to \C^\ast$ with the diagram
    \begin{tikzcd}[] %\arrow[bend right,swap]{dr}{F}
      & \C \arrow[]{d}{\exp}\\
      U \arrow[dashed]{r}{\iota} \arrow[]{ur}{\log}& \C^\ast
    \end{tikzcd}
    Such a map $\log: U \to \C$ exists if $U$ is simply connected.
\end{itemize}






