\begin{thm}[Tychonoff]
The product of compact spaces $\{X_i\}_{i \in I}$ is compact.
\end{thm}
If the index set $I$ is finite, then this seems quite clear, but to prove this for uncountable products, this theorem is quite strong.
It turns out that Tychonoff's theorem is equivalent to the axiom of choice.



\subsection{The role of the countability axioms}

\begin{dfn}[]
  Let $X,Y$ be topological spaces. A map $f: X \to  Y$ is called \textbf{sequentially continuous}, if for all sequences $\left(x_{n}\right)_{n \in \N})$ converging to $a$ in $X$, the sequence of their images $\left(f(x_{n})\right)_{n \in \N}$ converges to $f(a)$.
\end{dfn}
If $f$ is continuous, it is automatically sequentially continuous, as for a neighborhood $V$ of $f(a)$, there exists a neighborhood $U$ of $x$ such that $f(U) \subseteq V$.
And because the sequence converges
there exists an $N \in \N$ such that $n \geq N \implies x_n \in U \implies f(x_n) \in V$.

On the contrary, the opposite is not always true as we will see later.

\begin{lem}[]
Let $f: X \to  Y$ be function between continuous spaces and where $X$ satisfies the 1AA. Then
$f$ continuous $\iff$ $f$ sequentially continuous.
\end{lem}
\begin{proof}
We show contraposition:
Assume that $f$ is not continuous, so there exists a point $a \in X$ and a neighborhood $V$ of $f(a)$ such that all neighborhoods $U$ of $a$ do not satisfy $f(U) \subseteq V$.

Let $\mathcal{U} = \{U_1,U_2\ldots\}$ be countable neighborhood basis of $a$.
Then $\forall n$, chose an $x_n \in U_1 \cap \ldots \cap U_n$ with $f(x_n) \notin V$.
Clearly, this sequence converges to $a$ since for any neighborhood $U$ of $a$ there exists an $N \in \N$ such that $U$ is a neighborhood of $x_n$ which contains an element of $\mathcal{U}$.
Also,$f(x_n)$ clearly does not converge since $f(a) \notin V$.
\end{proof}



\begin{ex}[]
  Let $X = \Hom_{\Top}([0,1],[0,1])$ be the space of continuous functions with the subspace topology of $\Hom_{\Set}([0,1],[0,1]) = [0,1]^{[0,1]}$ with the product topology.

  We can show in the exercise classes that 
  \begin{align*}
    \lim_{n \to \infty} \phi_n = \phi \iff \lim_{n \to \infty}\phi_n(s) = \phi(s) \forall s \in [0,1]
  \end{align*}

  If chose the same space $X$, but with the topology $\tau_d$ induced by the $L$-metric 
  \begin{align*}
    d(\phi, \psi) = \int_0^{1} \abs{\phi(s) - \psi(s)}ds
  \end{align*}
  then the map
  \begin{align*}
    f: (X,\tau_{\text{prod}}) \to (Y,\tau_{d}), \quad \phi \mapsto  \phi
  \end{align*}
  is sequentially continuous but not continuous.

  By using the majorised converges theorem (and the majorant $1$), we get
  \begin{align*}
    \lim_{n \to \infty}\phi_n(s) = \phi(s) \implies \lim_{n \to \infty} \int_0^1 \abs{\phi_n(s) - \phi(s)} ds = 0
  \end{align*}

  The map $f$ is discontinuous at for example the constant function $\phi = 0$.

  Let $0 < \epsilon < 1$ and $V = B_{\epsilon}(f(\phi_0)) \subseteq (X,\tau_{d})$.

  We now show that all neighborhoods $U \subseteq (X,\tau_{\text{prod}})$ of $\phi$, $f(U) \nsubseteq V$.

  Let $U$ be such a neighborhood of $\phi$. Then there exists $s_1, \ldots, s_m \in [0,1]$ and $U_i \subseteq [0,1]$ open with $00in U_i$ such that
  \begin{align*}
    U 
    &\supseteq \pi_{s_1}^{-1}(U_1) \cap \ldots \cap \pi_{s_m}^{-1}(U_m)\\
    &= \{\psi: [0,1] \to [0,1] \big\vert \phi_i(s_i) \in U_i, \phi \text{ continuous}\}
  \end{align*}
  but because the condition only covers finitely any $i$ we can construct a function $\psi$ which is $0$ at each $s_i$, but almost everywhere else has value $1$.

  So $\psi \in U$ but clearly
  \begin{align*}
    d(\phi,\psi) = \int_0^1 \abs{\psi(s)} ds > \epsilon
  \end{align*}
\end{ex}


\begin{dfn}[]
A topological space $X$ is called \textbf{sequentially compact} if every sequence has a converging subsequence.
\end{dfn}
Generally speaking, compactness and sequential compactness do not imply eachother.
But some implications are true

\begin{lem}[]
  If $X$ satisfies the 1AA, then
\begin{itemize}
  \item $X$ compact $\implies$ $X$ sequentially compact
  \item If $X$ is a metric space, then $X$ sequentially compact $\iff X$ compact.
\end{itemize}
\end{lem}
\begin{proof}
Let $X$ be compact.
\begin{itemize}
  \item Let $\left(x_{n}\right)_{n \in \N}$ be a sequence in $X$.
    We want to find an $a \in X$ such that for all neighborhoods $U$ of $a$, $\forall n \in \N \exists m \geq n: x_m \in U$.

    Such an $a$ exists because if $\forall a \in X$ there exists an neighborhood $U_a$ and a number $N_a$ such that
    \begin{align*}
      n \geq N_a \implies x_n \notin U_a
    \end{align*}
    then by compactness we could cover $X$ with finitely many $U_a$ and take the maximum of the $N_a$. What this would give us is that $x_n \notin X$ for $n \geq \max \{N_{a_1}, N_{a_2}, \ldots, N_{a_m}\}$, which does not make sense.

    By 1AA, let $\mathcal{U} = \{U_1, \ldots\}$ be a countable neighborhood basis of $a$.
    Just like in the previous proof we inductivley chose $n_1 \leq n_2 \leq \ldots \leq n_l \leq \ldots$ such that
    \begin{align*}
      x_{n_l} \in U_1 \cap \ldots \cap U_l
    \end{align*}
    which means $\lim_{l \to \infty} x_{n_l} = a$.

  \item  See Analysis II
\end{itemize}
\end{proof}


\begin{ex}[Counterexamples]
  The space $[0,1]^{[0,1]}$ is compact, but not sequentially compact.

  The \textbf{long line} is 1AA, sequentially copmact, but not compact.
\end{ex}

To better understand why we need the 2AA, we will quickly cover manifolds.

\subsection{Manifolds}
\begin{dfn}[]
A subset $M \subseteq \R^{n}$ is called a $d$-dimensional \textbf{smooth submanifold} of $\R^{n}$ if for all $p \in M$ there exist open sets $U_p,V_p \subseteq \R^{n}$ and a diffeomorphism $\phi_p: U_p \to  V_p$ with
\begin{align*}
  \phi_p(U_p \cap M) = \{y \in V_p \big\vert y_i = 0, \forall i > d\} = V_p \cap (\R^{d} \times 0)
\end{align*}
\end{dfn}
Another analogous definition is that of a \textbf{topological submanifold}, where we replace smoothness with continuity.

\begin{ex}[]
  We say in Analysis II that if $F: \R^{n} \to  \R^{m}$ is smooth and has a regular value $v \in \R^{m}$ (i.e. $\text{rank} DF = \min\{n,m\}$, then $F^{-1}(v)$ is a $(n-m)$-dimensional submanifold (if $n \geq m$).

  From this, we see that the torus can be written as the preimage of hte smooth function
  \begin{align*}
    F(x,y,z) = (x^{2} + y^{2} + z^{2} + R^{2} - r^{2})^{2} - 4R^{2}(x^{2} + y^{2})
  \end{align*}
\end{ex}



\begin{dfn}[]
Let $X$ be a topological space.
We say that $X$ is a $d$-dimensional \textbf{topological manifold}, if 
\begin{itemize}
  \item $\forall  p \in M$ there exists a $U \subseteq X$ open with $p \in U$ and $U \iso \R^{4}$
  \item $X$ is Hausdorff
  \item $X$ satisfies the 2AA
\end{itemize}
\end{dfn}
But why do we want the two extra requirements? 
The \textbf{Embedding theorem} gives us an answer:
The defintions we just saw are in a certain sense equivalent:
\begin{thm}[Embedding theorem]
  \phantom{a}
  \begin{enumerate}
    \item All topological submanifolds of $\R^{n}$ are topological manifolds.
    \item If $X$ is a topological manifold, then there exists an $n \in \N$ and a topological submanifold $M \subseteq \R^{n}$ such that $X \iso M$.
  \end{enumerate}
\end{thm}
\begin{proof}
  The first one trivial: Since subspaces of $\R^{n}$ are ausdorff and fulfill the 2AA.

  And if $p \in M$, then there exists $U_p,V_p \subseteq \R^{n}$ open and a homeomorphism $\phi_p:U_p \to V_p$.
  What might happen is that the $U_p$ can intersect $M$ at multiple points and so we need to trim $U_p$.

  Chose $\epsilon > 0$ such that $B_{\epsilon}(\phi_p(p)) \subseteq V_p$. Then
  \begin{align*}
    U := \phi_p^{-1}(B_{\epsilon}(\phi_p(p))) \cap M \subseteq M
  \end{align*}
  is open and isomorhpic to $\R^{d}$.

  
  Showing (b) is more difficult and is not done in the lecture. See the corresponding section in Munkres for the proof.
\end{proof}


\subsection*{Classification of $d$-dimensional compact path connected manifolds}
\begin{itemize}
  \item[$d=1$] All are homemorphic to $S_1$
  \item[$d=2$] All are homeomorphic to exactly one of the following surfaces
    \begin{align*}
      \Sigma_0 = \IS^{2}, \quad \Sigma_1 = \IS^{1} \times \IS^{1} = \mathbb{T}, \quad \ldots \Sigma_n = \text{ genus $n$ Torus}, \ldots
    \end{align*}
    also called the orientable surfaces of genus $n \in \N$,
    and the non-orientable surfaces
    \begin{align*}
      \Sigma_1^{\text{non-or}} = \R\IP^{2}, \Sigma_2^{\text{non-or}} = \text{Klein Bottle}, \ldots,\Sigma_k^{\text{non-or}},\ldots
    \end{align*}
\end{itemize}
An important lemma to prove the classification theorem is to show that these surfaces are all distinct.
\begin{lem}[]
Let $n,m \in \N$. If $\Sigma_n \sim \Sigma_m$, then $n=m$.
\end{lem}
The proof can be done using the fundamental group.
