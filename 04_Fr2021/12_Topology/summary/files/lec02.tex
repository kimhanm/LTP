\subsection{The topological space}
The central idea behind the definition of a topological space is the notion of an \emph{open} set. 
In Analysis we know that a subset $U \subseteq \R$ is \emph{open}, if for every $x \in U: \exists \epsilon >0$ such that $(x - \epsilon, x + \epsilon) \subseteq U$.
\begin{enumerate}
  \item Looking at its properties we know that the union of open sets is open and that finite intersections of open sets is open. Moreover, the empty set $\emptyset$ and $\R$ itself are open.
  \item We called a set $A \subseteq \R$ \emph{closed}, if $\R \setminus A$ was open.
  \item We noted that the \emph{closure} of the open interval was the closed interval $[0,1]$.
  \item Sets can be neither open nor closed, or even both as the examples $[0,1)$ and $\R$ show.
\end{enumerate}

The following definition is a generalisation of openness on arbitrary sets. 
It turns out that just this alone is enough to define all the concepts described above!

\begin{dfn}[]
  Let $X$ be a set. A \textbf{topology} on $X$ is a collection of subsets $\tau \subseteq \mathcal{P}(X)$ such that
  \begin{itemize}
    \item The union of open sets is open: If $\{U_i\}_{i \in I}$ be a collection of open subsets $U_i \in \tau$, then $\bigcup_{i \in I} U_i \in \tau$ 
    \item Finite intersections of open subsets are open: $U,V \in \tau \implies U \cap V \in \tau$
    \item The empty set and $X$ itself are open: $\emptyset, X \in \tau$
  \end{itemize}
  Where subsetes $U \in \tau$ are called \emph{open} and $(X,\tau)$ is called a topological space.
\end{dfn}

\begin{ex}[]
  It is no surprise that $\R$ with the Analysis-open subsets is a topological space. We call this topology the \emph{euclidean} space. The proof is trivial.
\end{ex}




\begin{dfn}[]
  Let $X$ be a topological space
  \begin{itemize}
    \item A subset $A \subseteq X$ is called \textbf{closed}, if $A^{c} = X \setminus A$ is open.
    \item A subset $U \subseteq X$ is called a \textbf{neighborhood} of $x \in X$, if there exists an open set $V \subseteq X$ such that $x \in V \subseteq U$
  \end{itemize}
  Let $x \in X, B \subseteq X$. We call $x$
  \begin{itemize}
    \item an \textbf{inner point} of $B$ is $B$ is a neighborhood of $x$.
    \item an \textbf{exterior point} of $B$, if $B^{c}$ is a neighborhood of $x$.
    \item a \textbf{boundary point} of $B$ if neither $B$ nor $B^{c}$ are neighborhoods of $x$.
  \end{itemize}
  Analagously, define the
  \begin{itemize}
    \item \textbf{interior} $B^{\circ} := \{x \in X \big\vert x \text{ is an inner point of }B\}$
    \item \textbf{closure} $\overline{B}:= \{x \in X \big\vert x \text{ is \textbf{not} an exterior point of }B\}$
    \item \textbf{boundary} $\del B:= \{x \in X \big\vert x \text{ is a boundary point of }B\}$
  \end{itemize}
\end{dfn}
There are of course alternative ways to define a topology.
\begin{itemize}
  \item Instead of focusing on the open sets, we could just as well have started with the closed subsets, where we swap the finiteness condition for unions and intersections.
  \item Haussdorff's approach was to focus on neighborhoods instead of the open sets: A topological space is a tuple $(X,\mathcal{U})$ consisting of a set $X$ and a collection of families of subsets $\mathcal{U} = \{\mathcal{U}_x\}_{x \in X}$ with $\mathcal{U}_x \in \mathcal{P}(X)$ (the neighborhoods of $x$) such that
    \begin{enumerate}
      \item $x \in U_x$ and $X$ is a neighborhood of every point.
      \item If $V$ contains a neighborhood of $x$, then $V$ is also a neighborhood of $x$
      \item The intersection of two neighborhoodsof $x$ is again a neighborhood of $x$.
      \item Every neighborhood of $x$ contains a neighborhood of $x$ that contains all of its points.
    \end{enumerate}
  \item An approach we will take a look at in the exercise classes is using the \textbf{Hull axioms}: A topological space is a tuple $(X, \overline{\phantom{ }})$ consisting of a set $X$ and a map $\overline{\phantom{}}: \mathcal{P}(X) \to  \mathcal{P}(X)$ that satisfies
    \begin{enumerate}
      \item $\overline{\emptyset} = \emptyset$
      \item $A \subseteq \overline{\emptyset}$ for all $A \subseteq X$
      \item $\overline{A \cup B} = \overline{A} \cup \overline{B}$ for all $A,B \subseteq X$
    \end{enumerate}
\end{itemize}


\subsection{Metric spaces}

\begin{dfn}[]
  A \textbf{metric space} is a tuple $(X,d)$ consisting of a set $X$ and a \textbf{metric} $d: X \times X \to \R$ such that
  \begin{enumerate}
    \item $d$ is positive definite: $d(x,y) \geq 0, \forall x,y \in X$ and $d(x,y)= 0 \iff x = y$
    \item $d$ is symmetric: $d(x,y) = d(y,x), \forall x,y \in X$
    \item Triangle inequality: $d(x,z) \leq d(x,y) + d(y,z), \forall x,y,z \in X$
  \end{enumerate}
\end{dfn}
The euclidean metric on $\R^{n}$ given by $d(x,y) = \sqrt{\sum_{i=1}^{n}(x_i - y_i)^{2}}$ makes $\R^n$ a metric space.

We can turn any metric space into a topological space as follows
\begin{dfn}[]
  Let $(X,d)$ be a topological space. We call the collection
  \begin{align*}
    \tau_d := \{U \subseteq X \big\vert \forall x \in U \exists \epsilon > 0: B(x,\epsilon) \subseteq U\}
  \end{align*}
  the \textbf{induced topology} on $X$
\end{dfn}

\begin{ex}[]
  The euclidean metric is not the only valid metric. The \textbf{discrete metric} $d$ given by
  \begin{align*}
    d(x,y) = \left\{\begin{array}{ll}
      1 & x \neq y \\
      0 & x = 0
    \end{array} \right.
  \end{align*}
  and its induced topology is the \textbf{discrete topology} $\tau_{\text{disk}} = \mathcal{P}(X)$..
\end{ex}
We might ask: Is every topological space \textbf{metrisable}? That is, is there a metric $d$ on $X$ such that the induced topology $\tau_d$ on $X$ is the topology we started with?

The answer is No. Take for example the set $X = \{0,1\}$ and take the indiscrete topology $\tau = \{\emptyset, X\}$. The positive definiteness forbids this.

We also might ask if we lose some information by turning a metric space into the induced topological space. 
Can we always recover the metric from an induced topology? The answer again is No.
\begin{ex}[]
  Let $(X,d)$ be a metric space and define $\tilde{d}$ as
  \begin{align*}
    \tilde{d}(x,y) = \frac{d(x,y)}{1 + d(x,y)} < 1
  \end{align*}
  We can show that the two metrices induce the same topology on $X$. This quickly follows from the fact that
  \begin{align*}
    d(x,y) <\epsilon \iff \frac{d(x,y)}{1 + d(x,y)} < \frac{\epsilon}{1 + \epsilon}
  \end{align*}
  where we used the fact that the function $f(x) = \frac{x}{1 + x}$ is strictly monotonously increasing since $f'(x) > 0$ for $x > -1$.
\end{ex}

Even worse/better: All metrics on $\R^n$ that come form a norm induce the euclidean topology on $\R^n$. This follows form the equivalency of the norms in $\R^{n}$ that we know from Analysis II.

\subsection{Subspaces, Sums and Products}
Consider the unit sphere $\mathbb{S}^{2} \subseteq \R^3$ as a topological space. 
We can use the topology on $\R^3$ to give a topology to $\mathbb{S}^2$.

\begin{dfn}[]
  Let $(X,\tau)$ be a topological space and $Y \subseteq X$. Then
  \begin{align*}
    \tau_Y := \{U \cap Y \big\vert U \in  \tau\}
  \end{align*}
  defines a topology on $Y$ and is called the \textbf{subspace} topology on $Y$.
\end{dfn}
For example, for $Y = [0,1] \subseteq \R$, the ``half open'' interval $[0, \frac{1}{2})$ is open in $[0,1]$.

Note the following. For $B \subseteq X$ we have
\begin{itemize}
  \item $\del B = \overline{B} \setminus B^{\circ}$
  \item $\del \del B \subseteq \del B$
  \item If $B$ is closed, then $\del B = \del \del B$. This follows trivially from the fact that $\left(\del B\right)^{\circ}$ is empty.
\end{itemize}

