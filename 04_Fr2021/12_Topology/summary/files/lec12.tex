
\begin{ex}[]
$\IS^{n-1}$ and $\R^{n} \setminus \{0\}$ are homotopy equivalent, since we can use the inclusion map
$\iota: \IS^{n-1} \hookrightarrow \R^{n} \setminus \{0\},  \mapsto  x
$
aswell as the projection mapping
$
  \rho: \R^{n} \to \IS^{n-1}, v \mapsto \frac{x}{\abs{x}}
$
since their compositions are
\begin{align*}
  \rho \circ \iota = \id_{\IS^{n-1}}, \quad \iota \circ \rho \sim_h \id_{\R^{n} \setminus \{0\}}
  \quad \text{for} \quad 
  h(x,t) = tx + (1-t) \frac{x}{\abs{x}}
\end{align*}
\end{ex}

\begin{rem}[]
  It is easy to show that (see exercise sheet 6) if $X$ and $Y$ are homotopy equivalent, then
  \begin{enumerate}
    \item $X$ path connected $\iff Y$ path connected
    \item $X$ connected $\iff Y$ connected
    \item Homotopy equivalence forms an ``equivalence relation'' in the category of topological spaces:
      \begin{align*}
        X \sim Y, Y \sim Z \implies X \sim Z, \quad X \sim X, \quad X \sim Y \implies Y \sim X
      \end{align*}
  \end{enumerate}
\end{rem}

We would like to find some criteria when spaces can be homotopy equivalent to each other. The following definition will aide us in doing so.

\begin{dfn}[] \label{dfn:retraction}
Let $X$ be a topological space and $A \subseteq X$ with the inclusion map $\iota: A \hookrightarrow X$.
We say that $A$ is a \textbf{retract} of $X$, if there exists a continuous map $\rho: X \to A$ such that $\rho \circ \iota= \id_A$.
We call $\rho$ a \textbf{retraction} of $X$ unto $A$.
\begin{center}
\begin{tikzcd}[ ] %\arrow[bend right,swap]{dr}{F}
  A \arrow[hook]{r}{\iota} & X \arrow[]{r}{\rho} & A
\end{tikzcd}
\end{center}
\end{dfn}
\begin{ex}[]
\begin{itemize}
  \item The set $A = [0,1] \subseteq X = [0,1] \cup [2,3]$ is a rectract of $X$.
  \item $A = \{a,b\} \subseteq X = [0,1]$ is \emph{not} a retract.
  \item More generally, we can show that $\IS^{n-1} \subseteq \mathbb{D}^{n}$ is not a rectract, but for higher $n$ we need some algebraic topology.
\end{itemize}
\end{ex}



\begin{dfn}[]
Let $X$ be a topological space, $A \subseteq X$.
\begin{itemize}
  \item A retraction $\rho: X \to A$ is called a \textbf{deformation retraction} if additionally $\iota \circ \rho$ is homotopic to $\id_X$.
    \begin{center}
    \begin{tikzcd}[ ] %\arrow[bend right,swap]{dr}{F}
      X \arrow[]{r}{\rho} & A \arrow[hook]{r}{\iota} & X
    \end{tikzcd}
    \end{center}
    If the homotopy to the identity map can be chosen such that $h(t,a) = a$ for all $a \in A, t \in [0,1]$, the deformation retraction is called \textbf{strong}.
  \item $A$ is called a (strong) \textbf{deformation retract}, if such a (strong) deformation retraction exists.
\end{itemize}
\end{dfn}

\begin{ex}[]
The subset $A = \IS^{n-1} \subseteq X = \R^{n} \setminus \{0\}$ is a strong deformation retract of $X$ as we can use the same mapping $\rho$ from the example shown earlier.
\end{ex}

\begin{lem}[]
  Every space $X$ is homotopy equivalent to its deformation retract $A \subseteq X$
\end{lem}
\begin{proof}
If $\rho$ is the deformation retract and $\iota: A \to X$ is the inclusion mapping, then
\begin{align*}
  \iota \circ \rho \sim \id_X \quad \text{and} \quad \rho \circ \iota = \id_A
\end{align*}
\end{proof}

\begin{ex}[]
Let $\phi: \IS^{n-1} \to Y$ be continuous. Then for
\begin{align*}
  \iota(Y) \subseteq Y \cup_{\phi} (\mathbb{D}^{n} \setminus \{0\}) =: Q
\end{align*}
where $\iota:Y \to Q, y \mapsto [y]$ is the inclusion map, has as inverse a strong deformation retract.

Let $0 < k \leq n \in \N$. Then
\begin{align*}
  A := (\IS^{k-1} \times \IS^{n-k}) \subseteq \IS^{n} \setminus \left(
    \IS^{k-1} \times \{0\} \cup \{0\} \times \IS^{n-k}
  \right) =: X \subseteq \R^{k} \times \R^{n-k+1}
\end{align*}
is a strong deformation retract.

The mapping can be given by
\begin{align*}
  \rho: X \to  A, \quad (v,w) \mapsto  \left(
    \frac{v}{\abs{v}}, \frac{w}{\abs{w}}
  \right)
\end{align*}
where the homotopy is 
\begin{align*}
  h((v,w),t) = \left(
    \frac{tv + (1-t) \frac{v}{\abs{v}}}{\abs{tv + (1-t)\frac{v}{\abs{v}}}},
    \frac{tw + (1-t) \frac{w}{\abs{w}}}{\abs{tw + (1-t) \frac{w}{\abs{w}}}}
  \right)
\end{align*}
up to normalisation.
\end{ex}


\begin{lem}[]\label{lem:w5-2}
Let $X$ be a topological space. Then $X$ is contractible if and only if there exists a $x_0 \in X$ such that $\{x_0\} \subseteq X$ is a deformation retract.
\end{lem}
\begin{proof}
The proof is trivial and is left as an exercise to the reader.
\end{proof}


But why would we study Homotopy and Homotopy equivalence?
\begin{itemize}
  \item Many topological properties of topological spaces are the same for homotopy equivalent spaces.
  \item Many classical algebraic invariants are the same for homotpy equivalent spaces. 
\end{itemize}
One such example in this lecture the \textbf{Fundamental group}. $\pi_1$
The idea behind is to that to each topological space $X$ we associate a group $\pi_1(X)$ and to every continuous map $f: X \to Y$ we want to find a group homomorphism $\pi_1(X) \to \pi_2(X)$ that is a group isomorphism if $f$ is a homotopy equivalence.
In other words, we want to create a \textbf{functor} $\pi_1: \Top \to \Grp$.

