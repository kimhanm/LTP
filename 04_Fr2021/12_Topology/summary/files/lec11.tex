\section{Homotopy}
How do we measure ``holes'' in a topological space?
One idea is to look at a pre-defined object with a hole and see how this object maps to an arbitrary space by a continuous deformation.
We will have to define what continuous deformation means.
\subsection{Homotopy of maps}
\begin{dfn}[]
Let $f,g: X \to Y$ be continuous maps.
\begin{itemize}
  \item A continuous map $h: X \times [0,1] \to  Y$ such that
    \begin{align*}
      h(x,0) = f(x) \quad \text{ and } \quad h(x,1) = g(x) \quad\forall x \in X
    \end{align*}
    is called a \textbf{homotopy} between $f$ and $g$ and write $f \sim_h g$.
  \item We say that $f$ and $g$ are \textbf{homotopic} to each other, if such a homotopy exists and we write $f \sim g$.
\end{itemize}
\end{dfn}
The notation and naming are highly suggestive, and not without good reason.
\begin{rem}[]
  It is easy to prove that homotopy of maps forms an equivalence relation $\sim$ on $\Hom(X,Y)$:
  \begin{itemize}
    \item 
      Reflexivity is rather obvious, as
      \begin{align*}
        f \sim f \text{ via } h: X \times [0,1] \to Y, \quad (x,t) \mapsto f(x)
      \end{align*}
    \item For symmetry we reverse the homotpy. That is, if $f \sim_h g$ for some $h: X \times[0,1] \to Y$, then we get that for
      \begin{align*}
        \tilde{h}: X \times [0,1] \to  Y, \quad h(x,1-t)
      \end{align*}
      we have $g_{\tilde{h}}f$, as
      \begin{align*}
        \tilde{h}(x,0) = h(x,1) = g(x) \quad \text{and} \quad \tilde{h}(x,1) = h(x,0) = f(x) \quad \forall x \in X
      \end{align*}
    \item Transitivity is obtained by ``glueing'' together two homotopies.
      So if $f \sim_k g$ and $g \sim_l h$, then we can set
      \begin{align*}
        H: X \times [0,1] \to Y, \quad H(x,t) := \left\{\begin{array}{ll}
            k(x,2t) & t \leq \tfrac{1}{2}\\
            l(x,2t -1)& t \geq \tfrac{1}{2}
        \end{array} \right.
      \end{align*}
      this map is well defined as for $t = \tfrac{1}{2}$ we have
      \begin{align*}
        k(x,1) = g(x) = l(x,0)
      \end{align*}
      for continuity we can use the homeomorphism
      \begin{align*}
        \Phi: X \times [0,1] X \times[0,1] \cup_{\phi} X \times [1,2] =: Q
      \end{align*}
      given by
      \begin{align*}
        \phi: X \times \{1\} \to  X \{1\} (x,1) \mapsto (x,1) \quad \text{and} \quad \Phi: (x,t) \mapsto [(x,t)]
      \end{align*}
      and use the continuous map
      \begin{align*}
        r: X \times [0,1] \sqcup X \times [1,2] \to Y, r = h(x,t) \sqcup k(x,t-1)
      \end{align*}
      to define $H$ in another way. $H = R \circ \Phi$ and use Lemma \ref{lem:unipropquot} for
      \begin{align*}
        R: Q \to  Y, \quad [(x,t)] \mapsto r(x,t)
      \end{align*}
  \end{itemize}
\end{rem}

We denote the set of equivalence classes with respect to homotopy of maps as 
\begin{align*}
  [X,Y] := \faktor{\Hom(X,Y)}{\text{homotopy}}
\end{align*}

\begin{ex}[]
  Homotopy gives us another way of showing that certain special spaces are special.
  The space $\R^{n}$ is special in that for any space $X$, there is exactly one homotopy class on $\Hom(X,\R^{n})$. 
  In other words, every two continuous maps $f,g: X \to \R^{n}$ are homotopic, as a homotopy $f \sim_h g$ can be given by
  \begin{align*}
    h: X \times [0,1] \to \R^{n}, \quad (x,t) \mapsto  (1-t)f(x) + tg(x)
  \end{align*}
  The singleton space is special in that two maps $f,g: \{\ast\} \to Y$ are homotopic if and only if their images $f(\ast),g(\ast)$ are path connected in $Y$.
\end{ex}


\begin{lem}[]
  Homotopy not only defines an equivalence class structure inside a given $\Hom(X,Y)$, but that structure is also compatible with certain operations in the category of topological spaces such as composition and taking products.
  \begin{enumerate}
    \item Homotopies of maps is conserved under composition. 
      If $f\sim_h g$ in $\Hom(X,Y)$ and $f' \sim_{h'}g'$ in $\Hom(Y,Z)$, then $f' \circ f$ and $g' \circ g$ are homotopic in $\Hom(X,Z)$
      Such a homotopy $H$ can be given by
      \begin{align*}
        H: X \times [0,1] \to Z: (x,t) \mapsto  h'(h(x,t),t) 
      \end{align*}
    \item Homotopies of maps can be extended to products.
      If $f_i \sim_{h^{(i)}}g_i$ in $\Hom(X_i,Y_i)$ for $i \in I$, then 
      \begin{align*}
        H_t := \prod_{i \in I}h_t^{(i)}: \prod_{i \in I}X_i \to  \prod_{i \in I}Y_i
      \end{align*}
      defines a homotopy in $\Hom\left(\prod_{i \in I}X_i, \prod_{i \in I}Y_i\right)$.
  \end{enumerate}
\end{lem}


\subsection{Homotopy equivalence}
Now we have all the necessary tools to define what it means for two spaces to be continuously deformable into eachother.

\begin{dfn}[]
A continuous map $f: X \to  Y$ is a \textbf{homotopy equivalence} between $X$ and $Y$, if there exists an \textbf{homotopy inverse}. 
That is, a map $g: Y \to  X$ such that their compositions are homotoptic to the identity maps:
\begin{align*}
  g \circ f \sim \id_X \quad \text{and} \quad f \circ g \sim \id_Y
\end{align*}
The two spaces $X,Y$ are then called \textbf{homotopy equivalent} if such a pair $f,g$ exists and we write $X \sim Y$. (Not to be confused with $X \iso Y$).
\end{dfn}
Note that a homotopy equivalence is much weaker than a homeomorphism.
For example $\R^{n}$ is homotopy equivalent to the single ton space $\{\ast\}$
as all continuous maps $\R^{n} \to \R^{n}$ are homotopic.

More generally, a topological space $X$ is called \textbf{contractible}, if it is homotopy equivalent to $\{\ast\}$.

