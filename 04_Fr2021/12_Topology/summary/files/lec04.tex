\subsection{Continuous maps}
Recall the definition of continuity from Analysis. A function $f: \R \to \R$ is \emph{continuous}, if
\begin{align*}
  \forall x \in \R: \forall \epsilon > 0 \exists \delta > 0: \forall y \in \R: \abs{x - y} < \delta \implies \abs{f(x) - f(y)} < \epsilon
\end{align*}
and compare this with the topological definition of continuity:
\begin{dfn}[]
Let $X,Y$ be topological spaces. 
We say that a function $f: X \to Y$ is \textbf{continuous}, if the preimage of open subsets is open. So $\forall V \in \tau_Y: f^{-1}(V) \in \tau_X$

We say that $f$ is continuous at $x_0 \in X$, if for every neighborhood of $f(x_0)$ there exists a neighborhood $U$ of $x_0$ such that $f(U) \subseteq V$.
\end{dfn}

The following are pretty easy to prove:
\begin{rem}[]
  Let $X,Y$ be topological spaces and $f: X \to Y$
  \begin{enumerate}
    \item $f$ is continuous if and only if $f$ is continuous at $x \in X$ for every $x \in X$.
    \item The above notion of continuity is equivalent to the definition of the $\epsilon-\delta$ definition of continuity on a metric space..
    \item The identity $\id_X$ is continous the composition of continous maps is continuous.
    \item If $f$ is continuous, then the restriction $f|_A$ for a subset $A \subseteq X$ is continuous (in the subspace topology). In particular, the inclusion mapping is continuous.
    \item If $g: X \to Z$ is another function, then both $f$ and $g$ are continuous if and only if the product $(f,g): X \to Y \times Z$ is continuous.
    \item If $X$ is discrete, then any function $f: X \to  Y$ is continuous. 
      If $X$ is indiscrete, then only constant functions are continuous. If $Y$ is indiscrete, then any function into $Y$ is continuous.
    %\item If $Y$ is indiscrete, then any function $f: X \to Y$ is continuous. If $Y$ is discrete, then 
  \end{enumerate}
\end{rem}

Note that for the coproduct, the inclusions $\iota_X: X \to X \sqcup Y$ and $\iota_Y: Y \to X \sqcup Y$ are continuous.

For the product, the projection mappings $\pi_X: X \times Y \to X$ $\pi(x,y) = x$ are continuous with respect to the product topology.

We can also use this property to define the product spaces.

The product topology is the \emph{coarsest} topology on $X \times Y$, such that the projection mappings $\pi_X: X \times Y \to X$, $\pi_Y: X \times Y \to Y$ are continuous. 
This means that any other topology for which $\pi_X$ and $\pi_Y$ are continuous is bigger than $\tau_{X \times Y}$


Similar to the notion of Group isomorphisms, isomorphisms of vectorspaces, bijections of sets etc, we get the notion of isomorphism for topological spaces.
\begin{dfn}[]
  A bijective map $f: X \to Y$ such that its inverse $f^{-1}$ is continuous is called a \textbf{homeomorphism} and we write $f: X \stackrel{\iso}{\to} Y$ or $X \iso Y$.
\end{dfn}



\subsection{Connectedness}
We intuitively know what it means for a set to be connected. To put this in the language of topology, we obtain the following definition:
\begin{dfn}[]
A topological space is \textbf{connected}, if it can't be split into the disjoint union of two open, non-empty sets.
Alternatively: If $X = U \cup V$ for $U,V$ open, non-empty, then $U \cap V \neq \emptyset$.
\end{dfn}
\begin{lem}[]
The connected subsets in $\R$ are exactly the intervals
\end{lem}
\begin{proof}[]

Let $I \subseteq \R$ be connected. If $x,y \in I$ with $x \leq y$. Then $x \leq z \leq y \implies z \in I$ or else we could write $I$ as the disjoint union of the non-empty open sets.
\begin{align*}
  I = I \cap (-\infty,z) \sqcup I \cap (z,\infty)
\end{align*}
On the other hand, let $I \subseteq \R$ be an interval and $U,V$ oen and non-empty such that $I = U \cup V$. 
We then can show that $U \cap V \neq \emptyset$ by using the axiom of completeness for $\R$.
To do so, let $a \in U$ and $b \in V$. Then without loss of generality $a < b$. Set
\begin{align*}
  s := \sup \{x \in U \big\vert x < b\} \in I
\end{align*}
Since $I = U \cup V$, at least $s \in U$ or $s \in V$ has to be true.

If $s \in U$ then since $U$ is open, there exists an $\epsilon >0$ such that
\begin{align*}
  (s - \epsilon, s + \epsilon) \subseteq U \implies b \in U
\end{align*}
If $s \in V$ it follows analogously that $U \cap V \neq \emptyset$.
\end{proof}


The notion of connected sets gives us the generalisation of the intermediate value theorem.
\begin{thm}[]
  The image of connected sets under continuous functions is connected.
\end{thm}
\begin{proof}[]

Let $f: X \to Y$ be a continuous function and $A \subseteq X$ connected. Write $B = f(A) \subseteq Y$ and assume that $B = U \cup V$ for $U,V \subseteq Y$ open, non-empty. 
By continuity of $f$, their preimages $f^{-1}(U), f^{-1}(V)$ are open and non-empty. 
Moreover, since $B = f(A)$ we have that $f^{-1}(U) \cup f^{-1}(V) = A$. 
Since $A$ is connected, there exists an $x_0 \in f^{-1}(U) \cap f^{-1}(V)$, and therefore $f(x_0) \in U \cap V$.
\end{proof}


Usually, the definition of connectedness matches with our intuition, but there are some examples where that is not the case. 
A \emph{stronger} type of connectedness is that of path-connectedness

\begin{dfn}[]
  A \textbf{path} on a topological space $X$ is a continuous function $\gamma: [0,1] \to X$. 
  If we write $a = \gamma(0)$ and $b = \gamma(1)$, we
  We say that $\gamma$ is an $a-b$ path and that it \emph{connects} $a$ and $b$.

  A topological space $X$ is said to be \textbf{path connected}, if for any two points $a,b \in X$ there exists an $a-b$ path.
\end{dfn}





