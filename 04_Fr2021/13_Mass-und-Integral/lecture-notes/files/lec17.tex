\section{Integration}
Assume for this chapter, that $\mu$ is a Radon measure on $\R^{n}$ and $\Omega \subseteq \R^{n}$ is $\mu$-measurable.

\subsection{Definitions and Basic Properties}

\begin{dfn}[]
A function $g: \Omega \to \overline{\R}$ is called a \textbf{simple function}, if the image of $g$ is at most countable.
\end{dfn}

Because summing up series with both positive and negative coefficients can lead to some convergence issues (for example the sequence $a_n = (-1)^{n}$), we split a function into its positive and negative parts.
Define
\begin{align*}
  f^{+} := \max(f,0), \quad f^{-} := \max(-f,0)\\
  \implies f = f^{+} - f^{-}, \quad \abs{f} = f^{+} + f^{-}
\end{align*}


\begin{dfn}[] \label{dfn:integral}
\phantom{a}
  \begin{enumerate}
    \item 
    For $g: \Omega \to [0,\infty]$ is a \textbf{non-negative, simple, $\bm{\mu}$-measurable} function, we define
    \begin{empheq}[box=\bluebase]{align*}
      \int_{\Omega} g d \mu := \sum_{0 \leq y \leq \infty} y \cdot \mu\left(g^{-1}\{y\}\right)
    \end{empheq}
    where we use the convention $0 \cdot \infty = 0$. (We want the integral of the function $\bm{0}$ to be zero.)

  \item A simple, $\mu$-measurable function $g: \Omega \to [-\infty,+\infty]$ is called a $\bm{\mu}$\textbf{-integrable simple function}, if either $\int_{\Omega}g^{+}d \mu < \infty$ or $\int_{\Omega}g^{-} d \mu < \infty$.
    Then define
    \begin{align*}
      \int_{\Omega} g d \mu := \int_{\Omega}g^{+} d \mu - \int_{\Omega}g^{-} d \mu =  \sum_{-\infty \leq y \leq \infty} y \cdot \mu\left(g^{-1}\{y\}\right)
    \end{align*}
  %%%
  \item For any function $f: \Omega \to [-\infty,\infty]$ define the \textbf{upper integral} 
    \begin{empheq}[box=\bluebase]{align*}
      \upint_{\Omega} f d \mu :=
      \inf \left\{
        \int_{\Omega}g d \mu
        \big\vert g \geq f \mu\text{-a.e., $g$ is a $\mu$-integrable simple function}
      \right\}
    \end{empheq}
    aswell as the \textbf{lower integral}
    \begin{empheq}[box=\bluebase]{align*}
      \lowint_{\Omega} f d \mu :=
      \sup \left\{
        \int_{\Omega}e d \mu
        \big\vert e \leq f \mu\text{-a.e., $e$ is a $\mu$-integrable simple function}
      \right\}
    \end{empheq}
    %%%
    \item A $\mu$-measurable function $f: \Omega \to \overline{\R}$ is called \textbf{$\bm{\mu}$-integrable}, if the upper- and lower integral coincide, in which case we write
      \begin{align*}
        \int_{\Omega} f d \mu := \upint_{\Omega} f d \mu = \lowint_{\Omega}f d \mu
      \end{align*}
  \end{enumerate}
\end{dfn}
\textbf{Warning:} in some contexts, functions with integral $\pm \infty$ are called ``not integrable''. For our purposes, they are.

\begin{rem}[]
  It is easy to show that
  \begin{align*}
    \lowint_{\Omega} f d \mu \leq \upint_{\Omega} f d \mu
  \end{align*}

  In Exercise Sheet 09, we prove that the multiple definitions of an integral are ``consistent'', that is: for a $\mu$-integrable simple fucntion we have
  \begin{align*}
    \lowint_{\Omega} f d \mu = \upint_{\Omega} f d \mu = \int_{\Omega} f d \mu
  \end{align*}
  where the last integral is understood as the definition used in (b)
\end{rem}

\begin{prop}[] \label{prop:measurable-implies-integrable}
  Let $f: \Omega \to [0,\infty]$ be $\mu$-measurable. Then $f$ is $\mu$-integrable.
\end{prop}
\begin{proof}
  If $\lowint_{\Omega} f d \mu = \infty$, then it is trivial.
  Now, if $\lowint_{\Omega}f d \mu < \infty$, it means that $f(x) < \infty$ $\mu$-a.e.

  \begin{itemize}
    \item \textbf{Case $\mu(\Omega) < \infty$:}
      For all $\epsilon > 0$, set for $k \in \N$
      \begin{align*}
        A_k := \{x \in \Omega \big\vert k \epsilon \leq f(x) < (k+1)\epsilon\} = f^{-1}\left[k \epsilon, (k+1)\epsilon\right)
      \end{align*}
      Since $f(x) < \infty$ $\mu$-a.e. it means that for $\tilde{\Omega} := \bigcup_{k \in \N} A_k = f^{-1}[0,\infty)$, we have $\mu(\Omega \setminus \tilde{\Omega}) = 0$.

      To sandwich $f$ between $\mu$-integrable simple functions, we define
      \begin{align*}
        e(x) &:= \epsilon \sum_{k=0}^{\infty} k \chi_{A_k}(x)\\
        g(x) &:= \epsilon \sum_{k=0}^{\infty} (k+1) \chi_{A_k}(x)
      \end{align*}
      which gives us $e(x) \leq f(x) < g(x)$ $\mu$-a.e. and
      \begin{align*}
        \int_{\Omega}e d \mu 
        &\leq 
        \lowint_{\Omega} f d \mu 
        \leq \upint_{\Omega} f d \mu 
        \leq \int_{\Omega} g d \mu
        = \epsilon\sum_{k \in \N}(k+1) \mu(A_k)\\
        &= \epsilon \sum_{k\in \N} k \mu(A_k) + \epsilon \sum_{k \in \N} \mu(A_k)
        = \int_{\Omega} e d \mu + \epsilon \cdot \mu(\tilde{\Omega})
      \end{align*}
      where in the last step we used that the $A_k$ were mutually disjoint and were preimages of Borel sets and thus $\mu$-measurable.
      Because $\mu(\tilde{\Omega}) \leq \mu(\Omega) < \infty$, we can let $\epsilon \to 0$ and get the result.

    %%%
    \item \textbf{General Case:} Let $\Omega \subseteq \R^{n}$ be a $\mu$-measurable set. 
      Then take any countable covering of $\R^{n}$ with disjoint dyadic cubes $R^{n} = \bigcup_{l=1}^{\infty}Q_l$ and set $\Omega_l := \Omega \cap Q_l$.

      Since $\mu(\Omega_l) < \infty$, we are in the first case, so for all $\epsilon > 0 $ we can find $\mu$-integrable simple functions $e_l,g_l: \Omega_l \to [0,\infty]$ with $e_l \leq f \leq g_l$ $\mu$-a.e. and
      \begin{align*}
        \int_{\Omega_l} e_l d \mu 
        \leq 
        \int_{\Omega_l} g_l d \mu 
        \leq
        \int_{\Omega_l} e_l d \mu + \frac{\epsilon}{2^{l}}
      \end{align*}
      We then define
      \footnote{There is some minor abuse of notation but it's easy to extend the domain of $e_l,g_l$ to $\Omega$.}
      \begin{align*}
        e := \sum_{l=1}^{\infty} e_l \cdot \chi_{\Omega_l}, 
        \quad \text{and} \quad 
        g := \sum_{l=1}^{\infty} g_l \cdot \chi_{\Omega_l}
      \end{align*}
      which are again $\mu$-integrable simple functions satisfying $e \leq f \leq g$ for $\mu$-a.e. $x \in \Omega$ and
      \begin{align*}
        \int_{\Omega}e d \mu 
        \leq 
        \lowint_{\Omega}f d \mu 
        \leq 
        \upint_{\Omega}f d \mu
        \leq
        \int_{\Omega}g d \mu 
        = 
        \sum_{l=1}^{\infty} \int_{\Omega_l}g_l d \mu 
        \leq 
        \sum_{l=1}^{\infty} \int_{\Omega_l} e_l d \mu + \sum_{l=1}^{\infty} \frac{\epsilon}{2^{l}}
        = \int_{\Omega}e d \mu + \epsilon
      \end{align*}
      letting $\epsilon \to  0$, we get the result.
  \end{itemize}
\end{proof}




\begin{prop}[Monotonicity]\label{prop:monotonicity-integral}
  Let $f_1,f_2: \Omega \to \overline{\R}$ be $\mu$-integrable with $f_1 \leq f_2$ $\mu$-a.e.. Then
  \begin{align*}
    \int_{\Omega} f_1 d \mu \leq \int_{\Omega} f_2 d \mu
  \end{align*}
\end{prop}
\begin{proof}
  If a $\mu$-integrable simple function $g$ satisfies $g \geq f_2$ $\mu$-a.e., then it also satisfies $g \geq f_1$ $\mu$-a.e..

  Looking at the definition of $\mu$-integrable functions, 
  \begin{align*}
    \int f d \mu = 
      \inf \left\{
        \int_{\Omega}g d \mu
        \big\vert g \geq f \mu\text{-a.e., $g$ is a $\mu$-integrable simple function}
      \right\}
  \end{align*}
  then we are taking the infimum over a larger set for $f_1$ than for $f_2$, so 
  \begin{align*}
    \int_{\Omega}f_1 d \mu 
    =
    \upint_{\Omega}f_1 d \mu
    \leq
    \upint_{\Omega}f_2 d \mu
    = \int_{\Omega}f_2 d \mu
  \end{align*}
\end{proof}
As an immediate consequence, we have
\begin{cor}[]
  Let $f_1,f_2: \Omega \to \overline{\R}$ $\mu$-integrable with $f_1 = f_2$ $\mu$-a.e.
  \begin{align*}
    \int_{\Omega} f_1 d \mu = \int_{\Omega} f_2 d \mu
  \end{align*}
\end{cor}
\begin{dfn}[]
  Let $f: \Omega \to \overline{\R}$ be a function
  \begin{itemize}
    \item $f$ is called \textbf{$\bm{\mu}$-summable}, if $f$ is $\mu$-measurable and
      \begin{align*}
        \int_{\Omega} \abs{f} d \mu < \infty
      \end{align*}
    \item $f$ is called \textbf{locally $\bm{\mu}$-summable} in $\Omega$, if for all compact sets $K \subseteq \Omega$, $f|_K$ is $\mu$-summable.
  \end{itemize}
\end{dfn}


\begin{prop}
Let $f: \Omega \to \overline{\R}$.
\begin{itemize}
  \item If $f$ is $\mu$-summable, then it is $\mu$-integrable.
  \item If $f(x) = 0$ $\mu$-a.e., then $f$ is $\mu$-integrable and $\int_{\Omega} f d \mu = 0$.
\end{itemize}
\end{prop}
\begin{proof}
  \phantom{a}
  \begin{enumerate}
    \item We show that $\lowint_{\Omega}f d \mu = \upint_{\Omega} f d \mu$.
      Since $f$ is $\mu$-measurable, $f^{\pm} = \max(\pm f,0)$ are also $\mu$-measurable. Moreover, since $0 \leq f^{\pm} \leq \abs{f}$, by Proposition \ref{prop:measurable-implies-integrable}, they are $\mu$-integrable and satisfy $\int_{\Omega} f^{\pm} d \mu < \infty$.

  This means that for all $\epsilon > 0$, there exist $\mu$-integrable simple functions $e_{\pm} \leq f^{\pm} \leq g_{\pm}$ $\mu$-a.e. with
  \begin{align*}
    \int_{\Omega} e_{\pm} d \mu \leq \int_{\Omega}f^{\pm} d \mu \leq \int_{\Omega} g_{\pm} d \mu \leq \int_{\Omega} e_{\pm}  d \mu + \frac{\epsilon}{2}
  \end{align*}
  Setting $e := e_+ - g_-$ and $g := g_+ - e_-$, we see that again $e \leq f \leq g$ $\mu$-a.e. and
  \begin{align*}
    \int_{\Omega} e d \mu 
    &\leq 
    \lowint_{\Omega}f d \mu 
    \leq 
    \upint_{\Omega} f d \mu 
    \leq 
    \int_{\Omega} g d \mu 
    = 
    \int_{\Omega} g_+ d \mu- \int_{\Omega}e_- d \mu
    \\
    &\leq
    \int_{\Omega} e_+ d \mu + \frac{\epsilon}{2}+ \int_{\Omega} g_- d \mu + \frac{\epsilon}{2}
    = 
    \int_{\Omega} e d \mu + \epsilon
  \end{align*}
  Letting $\epsilon \to 0$ shows that $f$ is $\mu$-integrable.

  \item If $f(x) = 0$ $\mu$-a.e. we can set $e = g = \bm{0}$ and see that $e \leq f \leq g$ $\mu$-a.e., which shows
    \begin{align*}
      0 = \int_{\Omega}e d \mu \leq \lowint_{\Omega}f d \mu \leq \upint_{\Omega}f d \mu \leq \int_{\Omega} g d \mu = 0
    \end{align*}

  \end{enumerate}
\end{proof}

\begin{prop}
  Let $f: \Omega \to [0,\infty]$ be $\mu$-measurable.
  \begin{enumerate}
    \item $\int_{\Omega} f d \mu = 0 \implies f(x) = 0$ $\mu$-a.e.
    \item $\int_{\Omega} f d \mu < \infty \implies f(x) < \infty$ $\mu$-a.e.
  \end{enumerate}
\end{prop}
\begin{proof}
\begin{enumerate}
  \item Contraposition: Assume $f(x)$ is not $\mu$-a.e. zero. Then the $\mu$-measurable sets
    \begin{align*}
      A_k = \left\{x \in \Omega \big\vert f(x) \geq \frac{1}{k}\right\}, \quad k \geq 1
    \end{align*}
    form an increasing sequence and their union
    \begin{align*}
      A_{\infty} := \bigcup_{k=1}^{\infty} A_k = \{x \in \Omega \big\vert f(x) > 0\}
    \end{align*}
    has non-zero measure $0 < \mu(A_{\infty}) = \lim_{k \to \infty} \mu(A_k)$.

    This means that there exists some $K \geq 1$ such that $\mu(A_K) > 0$.
    Then the simple function 
    $s := \frac{1}{K} \chi_{A_K}$
    satisfies $s \leq f$, which implies
    \begin{align*}
      0 < \frac{1}{K} \mu(A_K) = \int_{\Omega} s d \mu \leq \int_{\Omega} f d \mu
    \end{align*}
  \item Contraposition: Assume there exists an $A \subseteq \Omega$ with
    $f(x) = \infty, \forall x \in A, \mu(A) > 0$
    then the simple function
    $s := \infty \cdot \chi_{A}$ satisfies $s \leq f$, and thus
    \begin{align*}
      \int_{\Omega} f d \mu \geq \int_{\Omega} s d \mu = \infty \cdot \mu(A) = \infty
    \end{align*}
\end{enumerate}
\end{proof}

\subsection{More Properties of the Integral}


\begin{thm}[Tchebyshev Inequality]\label{thm:tchebychev-inequality}
  Let $f: \Omega \to \overline{\R}$ be $\mu$-summable. Then for every $a > 0$:
  \begin{align*}
    \mu\left(
      \{x \in \Omega \big\vert \abs{f(x)} > a\}
      \leq \frac{1}{a} \int_{\Omega} \abs{f} d \mu
    \right)
  \end{align*}
\end{thm}
\begin{proof}
  Apply monotonicity (Proposition \ref{prop:monotonicity-integral}) on the functions
  \begin{align*}
      f_1 = a \cdot \chi_{\{x \in \Omega \big\vert \abs{f(x)} > a\}} \leq f_2 = \abs{f}
  \end{align*}
  here, $f_1$ takes on the value $a$, whenever $\abs{f(x)} > a$, and $0$ elsewhere.
\end{proof}

\begin{cor}[]
Let $f, f_k: \Omega \to \overline{\R}$ be $\mu$-integrable with
\begin{align*}
  \lim_{k \to \infty} \int_{\Omega} \abs{f_k - f} d \mu = 0
\end{align*}
Then $f_k \stackrel{\mu}{\to} f$ and there exists a subsequence $(f_{k_n})_{n \in \N}$ with $f_{k_n} \to f$ $\mu$-a.e.
\end{cor}
\begin{proof}
Applying Tchebyshev's inequality on the function $f_k - f$, it means that for all $\epsilon > 0$
\begin{align*}
  \mu\left(
    \{x \in \Omega \big\vert \abs{f_k - f} > \epsilon\}
  \right)
    \leq \frac{1}{\epsilon} \int_{\Omega} \abs{f_k - f} d \mu
\end{align*}
since the right hand side converges to $0$ as $k \to  \infty$, it follows that $f_k \stackrel{\mu}{\to} f$ (as in Definition \ref{dfn:convergence-in-measure}).

The second part follows from Theorem \ref{thm:subsequence-muae}.
\end{proof}
