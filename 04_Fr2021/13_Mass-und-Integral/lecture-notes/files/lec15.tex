
Recall that when defining the Riemann integral in Analysis I/II, we started by defining the Integral of \emph{step functions} ``by hand'', which were functions that were constant when decomposing them into intervals.

To define the Riemann integral of general types of functions, we defined the ``Ober- und Untersummen'', and if they coincided, took that as the value for the integral.

Our approach will be slightly more general

\begin{dfn}[]
Given a subset $A \subseteq \R^{n}$, we define the \textbf{characteristic function} of the set $A$ as
\begin{align*}
  \chi_A: \R^{n} \to \R, \quad \chi_A(x) = \left\{\begin{array}{ll}
      1 & \text{ if } x \in A\\
     0 &  \text{ otherwise}
  \end{array} \right.
\end{align*}
It's easy to see that $\chi_A$ is $\mu$-measurable if and only if $A$ is $\mu$-measurable.


A \textbf{simple function} is a function of the form
\begin{align*}
  f(x) = \sum_{i=1}^{\infty} c_i \chi_{A_i}(x), \quad c_i \in \R, A_i \subseteq \R^{n}, A_i \text{ mutually disjoint}
\end{align*}
And if the $A_i$ are $\mu$-measurable, then $f$ is called a \textbf{$\bm{\mu}$-measurable simple function}.

Equivalently, a function $f: \R^{n} \to  \R$ is a $\mu$-measurable simple function if and only if $f$ is $\mu$-measurable and the image of $f$ is a countable subset of $\R$.

\end{dfn}


The following theorem lets us decompose any non-negative $\mu$-measurable function into a simple function.
\begin{thm}[]
  Let $f: \Omega \to  [0,\infty]$ be $\mu$-measurable. Then there exist $\mu$-measurable sets $A_k \subseteq \Omega$ such that
  \begin{align*}
    f = \sum_{k=1}^{\infty} \frac{1}{k} \chi_{A_k}
  \end{align*}
\end{thm}
\begin{proof}
  We define the sets $A_k$ inductively, starting with
  \begin{align*}
    A_1 = \{x \in \Omega \big\vert f(x) \geq 1\} = f^{-1}[1, \infty]
  \end{align*}
  which is $\mu$-measurable. Then for all $k = 2,3, \ldots$, we define
  \begin{align*}
    A_k = \{x \in \Omega \big\vert \frac{1}{k} + \sum_{j=1}^{k-1} \frac{1}{j} \chi_{A_j}\}
  \end{align*}
  To show that this produces the function $f$, we show both inequalities in
  \begin{align*}
    f(x) = \sum_{k=1}^{\infty} \frac{1}{k} \chi_{A_k}(x)
  \end{align*}
  \begin{itemize}
    \item[$\geq$:] If $\sup \{k \big\vert x \in A_k\} = \infty$, then 
  \end{itemize}
  \texttt{Missing Rest}
\end{proof}

\begin{prop}[]
  Let $f: \Omega \to \R$ be continuous and $\mu$ a Borel measure.
  Then $f$ is $\mu$-measurable.
\end{prop}
\begin{proof}
  For any open set $U \subseteq \R^{n}$, $f^{-1}(U) = O \cap \Omega$ for some open set $O \subseteq \R^{n}$.

  Since $\mu$ is Borel, $f^{-1}(U)$ is $\mu$-measurable.
\end{proof}

From now on, we will say that a statement holds ``$\mu$-a.e.'' or ``\textbf{almost everywhere with respect to $\mu$}'' to mean that the set $A$, for which the statement does not hold, has $\mu(A) = 0$.


\subsection{Lusin's and Egoroff's Theorems}

Consider for example the sequence of ($\Leb^{1}$-measurable) functions $(f_k: [0,1] \to \R)_{k \in \N}$ given by
\begin{align*}
  f_k = \chi_{A_k}, \quad \text{for} \quad A_{k} = \left(1 - \frac{1}{2^{k}},1\right)
\end{align*}
this sequence converges pointwise to the constant function $\bm{0}$, but not uniformly.

However we can say that for any $\delta > 0$, it converges uniformly on the (compact) subset $A := [0,1 - \delta]$, which satisfies $\Leb^{1}([0,1] \setminus A) = \delta$.


\begin{dfn}[]
  Let $\Omega \subseteq \R^{n}$.
  We say that a sequence of functions $(f_k: \Omega \to \overline{\R})_{k \in \N}$ \textbf{converges $\mu$-almost uniformly} on $\Omega$ to a function $f: \Omega \to \overline{\R}$,
if for all $\delta > 0$ there exists a $\mu$-measurable subset $A \subseteq \Omega$ with $\mu(\Omega \setminus A) < \delta$ such that $(f_k)_{k \in \N}$ converges uniformly on $A$.
That is:
\begin{align*}
  \sup_{x \in A} \abs{f_k(x) - f(x)} \to  0 \text{ as } k \to  \infty
\end{align*}
\end{dfn}


\begin{thm}[Egoroff]
  Let $\Omega \subseteq \R^{n}$ be $\mu$-measurable with $\mu(\Omega) < \infty$,
  and let
  $f,(f_k)_{k \in \N}: \Omega \to \R$ $\mu$-measurable.
  \begin{enumerate}
    \item 
      If $f_k(x) \to  f(x)$ as $k \to  \infty$ for $\mu$-a.e. $x \in \Omega$, and $f(x)$ finite $\mu$-a.e., 
      then $(f_k)_{k \in \N}$ converges $\mu$-almost uniformally to $f$ on $\Omega$.
    \item 
      If additionally, $\mu$ is a Radon measure we can also assume that the set on which $(f_k)_{k \in \N}$ converges uniformly is compact.
      That is:

      $\forall \delta > 0\ \exists K \subseteq \Omega$ compact with $\mu(\Omega \setminus K) < \delta$ and
      \begin{align*}
        \sup_{x \in C} \abs{f_k(x) - f(x)} \to 0 \text{ as } k \to  \infty
      \end{align*}
      namely $(f_k)_{k \in \N}$ converges uniformly to $f$ on $K$.
  \end{enumerate}
\end{thm}

\begin{proof}
  \begin{enumerate}
    \item 
      Let $\delta > 0$ and for $i,j \in \N$ define the sets
      \begin{align*}
        C_{ij} := \bigcup_{k=j}^{\infty} \left\{x \in \Omega \big\vert \abs{f_k(x) - f(x)} > \frac{1}{2^{i}}\right\}
      \end{align*}
      which are $\mu$-measurable, since they are the pre-images of open subsets under a $\mu$-measurable function.
      They are also decreasing in $j$ (i.e. $C_{i,j + 1} \subseteq C_{i,j} \forall i,j$) and since $\mu(\Omega) < \infty$,
      we can use continuity from above. 
      Together with $f_k(x) \to f(x)$ for $\mu$-a.e. $x$, we get
      \begin{align*}
        \lim_{j \to \infty} \mu(C_{ij}) = \mu\left(
          \bigcap_{j=1}^{\infty}C_{ij}
        \right)
        = 0, \quad \forall i \in \N
      \end{align*}
      This means that for every $i$, there exists an $N(i) > 0$ such that
      \begin{align*}
        \mu\left(
          \bigcap_{j=1}^{N(i)}C_{i,j}
        \right)
        =
        \mu(C_{i,N(i)})  < \frac{\delta}{2^{i+1}}
      \end{align*}
      setting
      \begin{align*}
        A := \Omega \setminus \bigcup_{i=1}^{\infty}C_{i,N(i)}
      \end{align*}
      which is $\mu$-measurable, we get the estimate
      \begin{align*}
        \mu(\Omega \setminus A) = \mu \left(
          \bigcup_{i=1}^{\infty}C_{i,N(i)} 
        \right)
        < \sum_{i=1}^{\infty} \frac{\delta}{2^{i+1}} = \frac{\delta}{2}
      \end{align*}
      Moreover, for all $x \in A, i \in \N, k \geq N(i)$ we have
      \begin{align*}
        \abs{f_k(x) - f(x)} \leq \frac{1}{2^{i}}
      \end{align*}
      which shows uniform convergence of $(f_k) \to f$ on $A$.
    \item 
      By applying Theorem \ref{thm:radon-open-compact}, there exists a compact subset $K \subseteq A$ such that $\mu(A \setminus K) < \frac{\delta}{2}$, and so
      \begin{align*}
        \mu(\Omega \setminus K) \leq \mu(\Omega \setminus A) + \mu(A \setminus K) \leq \delta
      \end{align*}
  \end{enumerate}
\end{proof}

\begin{rem}[] \label{rem:egoroff-finite-necessary}
  The condition that $\mu(\Omega) < \infty$ is necessary.
Take for example the sequence of ``bump functions moving to the right'':
$f_k = \chi_{[k,k+1]}:\R \to \R$, which converges pointwise to $\bm{0}$.

It is clear that any set $A \subseteq \R$ on which the sequence converges uniformly must be bounded from the right, which means that $A$ cannot satisfy $\mu(\R \setminus A) < \delta$.
\end{rem}

In Analysis I/II, we proved that continous functions are Riemann integrable and that Riemann-integrable functions only have measure-zero points of discontinuity.

Because $\mu$ is assumed to be a Radon measure, all continuous functions are $\mu$-measurable.
We now show the generalisation of the other fact.

\begin{thm}[Lusin's Theorem]
  Let $\mu$ be a Radon measure on $\R^{n}$, $\Omega \subseteq \R^{n}$ be $\mu$-measurable 
  with $\mu(\Omega) < \infty$ and $f: \Omega \to \overline{\R}$ $\mu$-measurable and finite $\mu$.a.e..

  Then $\forall  \epsilon > 0\ \exists K \subseteq \Omega$ compact with $\mu(\Omega \setminus K) < \epsilon$ such that $f|_K$ is continuous.


\end{thm}
\begin{rem}[]
  \textbf{Warning:} The theorem states that the function $f|_K: K \to \overline{\R}$ is continous and \emph{not} that $f$ is continous at $x$ for all $x \in K$.

  If we drop the condition $\mu(\Omega) < \infty$, then we can still find such a set $C \subseteq \Omega$, which is closed, but not necessarily compact.
\end{rem}
\begin{proof}
  For each $i \in \N_{> 0}$, let $\{B_{ij}\}_{j \in \N}$ be a collection of disjoint Borel sets such that 
  \begin{align*}
    \R = \bigsqcup_{j = 1}^{\infty}B_{ij} \quad \text{and} \quad 
    \text{diam}(B_{ij}) := \sup \{\abs{x - y} \big\vert x,y \in B_{ij}\} < \frac{1}{i}
  \end{align*}
  Then define $A_{ij} := f^{-1}(B_{ij})$ which are $\mu$-measurable and let
  \begin{align*}
    \tilde{\Omega} := \bigcup_{j=1}^{\infty}A_{ij} \implies \Omega = \tilde{\Omega} \sqcup f^{-1}\{\pm \infty\}
  \end{align*}
  since $\mu$ is a Radon measure, there exist compact sets $K_{ij} \subseteq A_{ij}$ such that
  \begin{align*}
    \mu(A_{ij} \setminus K_{ij}) < \frac{\epsilon}{2^{i+j}}
  \end{align*}
  Then 
  \begin{align*}
    \mu\left(
      \tilde{\Omega} \setminus \bigcup_{j=1}^{\infty}K_{ij}
    \right)
    &=
    \mu\left(
      \bigcup_{j=1}^{\infty}A_{ij} 
      \setminus
      \bigcup_{j=1}^{\infty}K_{ij}
    \right)
    \\
    &\leq \mu\left(
      \bigcup_{j=1}^{\infty}(A_{ij} \setminus K_{ij})
    \right)
    \\
    &\leq \sum_{j=1}^{\infty} \mu(A_{ij} \setminus K_{ij})\\
    &< \sum_{j=1}^{\infty} \frac{\epsilon}{2^{i+j}} = \frac{\epsilon}{2^{i}}
  \end{align*}
  This means that for all $i$, there exists a $N(i)$ such that
  \begin{align*}
    \mu\left(
      \tilde{\Omega} \setminus \bigcup_{j=1}^{N(i)} K_{ij}  < \frac{\epsilon}{2^{i}}
    \right)
  \end{align*}

  Then define the compact sets
  \begin{align*}
    D_i := \bigcup_{j=1}^{N(i)} K_{ij}
    \quad \text{and} \quad
    K := \bigcap_{i=1}^{\infty}D_i
  \end{align*}
  For each $i,j$ chose some $b_{ij} \in B_{ij}$ and define $g_i: D_i \to \R, g_i(x) = b_{ij}$ if $x \in K_{ij}$ for all $j \leq N(i)$.

  Since the sets $\{K_{ij}\}_{j \in \N}$ are compact disjoint sets, this means that $g_i$ is locally constant and thus continuous.
  Moreover, by construction of $B_{ij}$, we have
  \begin{align*}
    \text{diam}(B_{ij}) < \frac{1}{i} \implies \abs{f(x) - g_i(x)} < \frac{1}{i}, \quad \forall x \in D_i
  \end{align*}
  which means that the sequence of continuous functions $(g_i|_K)_{i \in \N^{> 0}}:K \to \R$ converges uniformly to a continous function $f|_K$.

  The set $K$ also satisifes 
  \begin{align*}
    \mu(\tilde{\Omega} \setminus K) = \mu\left(
      \bigcup_{i=1}^{\infty} (\tilde{\Omega} \setminus D_i)
    \right)
    \leq \sum_{i=1}^{\infty} \mu(\tilde{\Omega \setminus D_i}) < \epsilon
  \end{align*}
  and since $f(x)$ is finite $\mu$-a.e., we have
  \begin{align*}
    \mu(\Omega \setminus K) \leq \mu(\tilde{\Omega} \setminus K) + \mu(f^{-1}\{\pm \infty\} \setminus K) \leq \epsilon + 0
  \end{align*}

  

\end{proof}



