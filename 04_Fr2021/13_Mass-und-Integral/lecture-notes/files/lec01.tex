\section{Measure spaces}

If we naively try to define a notion of measure that has some intuitive properties, we can run into some problems that give paradoxical results.
The \textbf{Riemann Integral} we saw in Analysis I/II also had some drawbacks of not being general enough. 
We can use measure theory to define a better definition of the integral.

\subsection{Algebras and $\sigma$-Algebras}
From now on, let $X$ denote a non-empty set.

\begin{dfn}[]\label{dfn:lim-set}
  For a sequence of subsets $\left(A_{n}\right)_{n = 1}^{\infty}$ in $\mathcal{P}(X)$. We define
\begin{align*}
  \limsup_{n \to \infty}A_n := \bigcap_{n=1}^{\infty} \bigcup_{m=n}^{\infty}A_m\\
  \liminf_{n \to \infty}A_n := \bigcup_{n=1}^{\infty} \bigcap_{m=n}^{\infty}A_m
\end{align*}
And if they are equal, we say that the sequence $\left(A_{n}\right)_{n = 1}^{\infty}$ converges to its limit $\lim_{n \to \infty} A_n$.
\end{dfn}

Informally, the $\limsup$ consists of elements of $X$ that occur in infinitely many $A_n$, whereas the $\liminf$ consists of elements that occur for all but finitely many $A_n$.


\begin{rem}[]
  \phantom{a}
  \begin{enumerate}
    \item $\liminf_{n \to \infty}A_n \subseteq \limsup_{n \to \infty}A_n$
    \item If $A_n \subseteq A_{n+1}$ for all $n \in \N$, then
      \begin{align*}
        \lim_{n \to \infty}A_n = \bigcup_{n=1}^{\infty}A_n
      \end{align*}
    \item If $A_n \supseteq A_{n+1}$ for all $n \in \N$, then
      \begin{align*}
        \lim_{n \to \infty}A_n = \bigcap_{n=1}^{\infty}A_n
      \end{align*}
  \end{enumerate}
  The similarity in names with the $\limsup$ and $\liminf$ from Analysis can be seen using the characteristic function 
  \begin{align*}
    \mathds{1}_A: X \to \{0,1\}\\
    \mathds{1}_A(x) = \left\{\begin{array}{ll}
      1 & x \in A \\
      0 & x \notin A
    \end{array} \right.
  \end{align*}
  It holds that
  \begin{align*}
    \limsup_{n \to \infty}A_n = A \iff \limsup_{n \to \infty} \mathds{1}_{A_n} = \mathds{1}_A\\
    \liminf_{n \to \infty}A_n = A \iff \liminf_{n \to \infty} \mathds{1}_{A_n} = \mathds{1}_A
  \end{align*}
  where the $\limsup$ and $\liminf$ on the left are as in Definition \ref{dfn:lim-set} and the ones on the right are the ones from Analysis.
\end{rem}

\begin{dfn}[Algebras of sets]
  A collection of subsets $\mathcal{A} \subseteq \mathcal{P}(X)$ is called an \textbf{algebra in $X$} if
  \begin{enumerate}
    \item $X \in \mathcal{A}$
    \item $A,B \in \mathcal{A} \implies A \cup B \in \mathcal{A}$
    \item $A\in \mathcal{A} \implies A^{c}\in \mathcal{A}$
  \end{enumerate}
  An algebra $\mathcal{E}$ is called a \textbf{$\sigma$-algebra}, if for any sequence $\left(A_{n}\right)_{n = 1}^{\infty}$ in $\mathcal{E}$ we have $\bigcup_{n=1}^{\infty}A_n \in \mathcal{E}$
\end{dfn}
Note that using the De Morgan's identity 
\begin{align*}
  \left(
    \bigcup_{n=1}^{\infty}A_n
  \right)^c
  =
  \bigcap_{n=1}^{\infty}A_n^c
\end{align*}
we can see that algebras ($\sigma$-algebras) are stable under finite (infinite) intersections aswell.




