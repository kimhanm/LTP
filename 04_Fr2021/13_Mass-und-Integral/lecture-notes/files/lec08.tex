\begin{dfn}[]
A measure $\mu$ on $\R^{n}$ is called \textbf{Borel}, if every Borel set is $\mu$-measurable.
\end{dfn}

The Lebesgue measure $\mathcal{L}^{n}$ on $\R^{n}$ is Borel and the $\sigma$-algebra of $\mathcal{L}^{n}$-measurable sets contains the Borel $\sigma$-algebra.


\begin{thm}[]
  For any subset $A \subseteq \R^{n}$ the following are equivalent
  \begin{enumerate}
    \item $A$ is $\mathcal{L}^{n}$-measurable..
    \item $\forall \epsilon > 0 \exists G \supseteq A$ with $\mathcal{L}^{n}(G \setminus A) <\epsilon$.
    \item $A$ it can be ``approximated'' from the inside and outside: $\forall \epsilon > 0 \exists F,G$ with $F \subseteq A \subseteq G$ such that
      \begin{align*}
        \LL^{n}(G \setminus A) + \LL^{n}(A \setminus F) \leq \epsilon
      \end{align*}
    \item $\forall \epsilon > 0 \exists F$ closed, $\exists G$ open, such that $\LL^{n}(G \setminus F) < \epsilon$.
  \end{enumerate}
\end{thm}




\subsection{Comparision between Lebesgue and Jordan Measure}
\begin{dfn}[]
A bounded subset $A \subseteq \R^{n}$ is \textbf{Jordan-measurable} if
\begin{align*}
  \underline{\mu}(A)
  :=
  \sup \{\vol(E)\big\vert E \subseteq A, E \text{ elementary set}\} 
  =
  \inf \{\vol(E)\big\vert A \subseteq E, E \text{ elementary set}\} 
  =:
  \overline{\mu}(A)
\end{align*}
\end{dfn}


As the following proof will show, The Lebesgue measure can measure more sets than the Jordan measure can.
\begin{thm}[]
Let $A \subseteq \R^{n}$ be bounded, then
\begin{enumerate}
  \item $\underline{\mu}(A) \leq\LL^{n}(A9 \leq \overline{\mu}(A)$
  \item If $A$ is Jordan-measurable, then $A$ is $\LL^{n}$-measurable and $\LL^{n}(A) = \mu(A)$.
\end{enumerate}
\end{thm}

One would naturally think that the volume of a cube should stay the same whether we move it by a fixed point or rotate it.
\begin{thm}[]
  The Lebegue measure is invariant under isometries $\Phi: \R^{n} \to \R^{n}, x \mapsto x_0 + Rx$, for $R \in O(n)$.
\end{thm}


\begin{dfn}[]
  A Borel measure $\mu$ on $\R^{n}$ is called \textbf{Borel regular}, if for every $A \subseteq \R^{n}$ there exists a borel set $B \supseteq A$ such that $\mu(A) = \mu(B)$.
\end{dfn}
The Lebesgue measure is Borel regular.


