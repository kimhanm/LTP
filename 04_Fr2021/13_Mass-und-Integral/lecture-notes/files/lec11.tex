\subsection{Hausdorff Measures}
The Lebesgue measure of a subset of $\R^{n}$ of dimension less than $n$ is always zero.
It also has the weakness of failing to properply measure fractal sets.
The Hausdorff measure tries to solve this.

We start by introducing an intermediate measure, where instead of covering a subset $A$ with quaders, we do this using open balls of radius smaller than some given $\delta >0 $.
\begin{dfn}[]
For $s \geq 0, \delta > 0$ and $A \subseteq \R^{n}$ non-empty, we set
\begin{align*}
  \mathcal{H}_{\delta}^{s}(A) := \inf \left\{
    \sum_{k \in I}r_k^{s}, A \subseteq \bigcup_{k \in I}B(x_k,r_k), 0< r_k \delta
  \right\}
\end{align*}
where the set $I$ is at most countable and we set $\mathcal{H}_{\delta}^{0}(\emptyset) = 0$.
\end{dfn}

\begin{rem}[]
  $\mathcal{H}_{\delta}^{s}$ defines a measure on $\R^{n}$ and for fixed $s,A$ it is inversely monotonous with respect to $\delta$, that is
  \begin{align*}
    \delta_2 \leq \delta_1 \implies \mathcal{H}_{\delta_1}^{s}(A) \leq \mathcal{H}_{\delta_2}^{s}(A)
  \end{align*}
  since every $\delta_2$ covering is also a $\delta_1$ covering.
\end{rem}
Since $\mathcal{H}_{\delta}^{s}$ is non-increasing in $\delta$, the limit
\begin{align*}
  \mathcal{H}^{s}(A) := \lim_{\delta \downarrow 0} \mathcal{H}_{\delta}^{s}(A) = \sup_{\delta > 0}\mathcal{H}_{\delta}^{s}(A)
\end{align*}
exists. We now use this as for our next definition.

\begin{dfn}[]
We call $\mathcal{H}^{s}$ the \textbf{$s$-dimensional Hausdorff measure} on $\R^{n}$
\end{dfn}
For $s = 0$, we call $\mathcal{H}^{0}$ the \textbf{counting measure}, which measure the cardinality of $A$.

Now that we call it a measure, we should of course prove some things


