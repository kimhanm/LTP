
\begin{rem}[] \label{rem:half-open-ring}
In the beginning of this section, we noted that the collection of bounded half-open sets 
$\mathcal{K} = \{[a,b) \big\vert a,b \in \R\}$
does not form an Algebra.
If we set $\tilde{\mathcal{K}}$ to be the 
collection of finite disjoint unions:
\begin{align*}
  \tilde{\mathcal{K}} = \left\{
    \bigsqcup_{k=1}^{m} [a_k,b_k) \big\vert m \geq 1 \in \N, a_k,b_k \in \R
  \right\}
\end{align*}
then the set $\tilde{\mathcal{K}}$ is stable under \emph{intersection} and \emph{difference}.
And we say that $\tilde{\mathcal{K}}$ forms a \textbf{ring}.

That is: $\emptyset \in \tilde{\mathcal{K}}$ and $A,B \in \tilde{\mathcal{K}} \implies A \cap B, A \setminus B \in \tilde{\mathcal{K}}$

The same is not true for the collection of open and closed intervals.
\end{rem}




\subsection{Hausdorff Measures}

Say we have the unit square $A := [0,1]^{2}$. It's $\Leb^{2}$-measure would of course be $1$.

If we however were to embed the square into $\R^{3}$ with
\begin{align*}
  \iota: \R^{2} \to \R^{3}, \quad (x,y) \mapsto  (x,y,0)
\end{align*}
we would find that $\Leb^{3}(\iota(A)) = 0$.
More generally, the Lebesgue measure $\Leb^{n}$ on subsets $A \subseteq \R^{n}$ that have ``dimension'' $< n$ is always going to be zero.

Moreover, the Lebesgue measure also has the weakness of failing to properply measure fractal sets (which can be though of as having non-integer ``dimension'')

It would be nice to define a collection of measures that are able to measure sets, regardless of whether they are embedded into a higher-dimensional space.
The Hausdorff measures try to solve this.

We start by introducing an intermediate measure, where instead of covering a subset $A$ with dyadic cubes (as was the case for the Lebesgue measure)
we do this using open balls of radius smaller than some $\delta >0 $.
\begin{dfn}[]
For $s \geq 0, \delta > 0$ and $A \subseteq \R^{n}$ non-empty, we set
\begin{align*}
  \mathcal{H}_{\delta}^{s}(A) := \inf \left\{
    \sum_{k \in I}r_k^{s}
    \bigg\vert
    I \text{ at most countable},
    A \subseteq \bigcup_{k \in I}B(x_k,r_k), 0< r_k < \delta
  \right\}
\end{align*}
where we set $\mathcal{H}_{\delta}^{0}(\emptyset) = 0$.
\end{dfn}


\begin{rem}[]
  $\mathcal{H}_{\delta}^{s}$ defines a measure on $\R^{n}$ and for fixed $s,A$, 
  the function $\delta \mapsto \mathcal{H}_{\delta}^{s}(A)$
  is non-increasing:
  \begin{align*}
    \delta_2 \leq \delta_1 \implies \mathcal{H}_{\delta_1}^{s}(A) \leq \mathcal{H}_{\delta_2}^{s}(A)
  \end{align*}
  since every $\delta_2$ covering is also a $\delta_1$ covering.
  Therefore, the limit
  \begin{empheq}[box=\bluebase]{align*}
    \mathcal{H}^{s}(A) := \lim_{\delta \downarrow 0} \mathcal{H}_{\delta}^{s}(A) = \sup_{\delta > 0}\mathcal{H}_{\delta}^{s}(A)
  \end{empheq}
exists. We now use this as for our next definition.
\end{rem}

\begin{dfn}[]
We call $\mathcal{H}^{s}$ the \textbf{$s$-dimensional Hausdorff measure} on $\R^{n}$
\end{dfn}
As hinted at earlier with the case of fractals, notice that $s$ may take on non-integer values.

To build some intuition, let's consider an example of a
``one-dimensional'' set $A \subseteq \R^{2}$.

\begin{ex}[]
  Let $A = \mathbb{S}^{1} = \{x \in \R^{2}, \|x\| = 1\}$.

  \begin{itemize}
    \item[$s = 0$:] We see that $\Hau_{\delta}^{0}(A)$ depends only on the number of balls covering $A$.

      If $\delta > 1$, we see that $A$ can be covered by the ball $B(0, 1 + \epsilon)$, for $\epsilon$ small enough.
      Therefore $\Hau_{\delta}^{0}(A) = 1$ for $\delta > 1$.

      On the other hand, if $\delta < 1$ then we have to cover $A$ by using multiple balls. 
      It should be clear that in the limit $\delta \to  0$, we have $\Hau^{s}(A) = \infty$.

    \item[$s = 1$:] Again, for $\delta > 1$, it's easy to see that the covering with the single ball $B(0,1 + \epsilon)$ give us $\Hau_{\delta}^{1}(A) \leq 1 + \epsilon$.

      But for arbitrary $\delta > 0$, let $n \in \N$ such that $\delta > \frac{\pi}{n}$.
      We then can place $n$ equally spaced balls along the circle each with radius $\frac{\pi}{n} + \epsilon$, for some $\epsilon> 0 $ small enough.
      This covers $A$ entirely and gives us an upper bound 
      \begin{align*}
        \Hau_{\delta}^{1}(A) \leq \sum_{i=1}^{n} \frac{\pi}{n} = \pi
      \end{align*}
      One can convince themself that there is no ``better'' covering strategy resulting in a lower upper bound, so $\Hau^{1}(A) = \pi$.

    \item[$s = 2$:] 
      The same covering strategy as decribed in the case $s = 1$ gives us the upper bound
      \begin{align*}
        \Hau_{\delta}^{s}(A) \leq \sum_{i=1}^{n} \left(\frac{\pi}{n}\right)^{2} = \frac{\pi^{2}}{n}
      \end{align*}
      But unlike for $s = 1$, chosing bigger and bigger $n$ means that we can make the $\Hau^{2}$-measure of $A$ arbitrarliy small.
      So $\Hau^{2}(A) = 0$.

      A similar argumentation also shows that $\Hau^{s}(A) = 0$ for all $s > 1$.
  \end{itemize}
\end{ex}

Before we prove that $\mathcal{H}^{s}$ is actually a measure, let's take a look at the case $s = 0$ more closely.

\begin{rem}[]
  $\Hau^{0}$ is the counting measure from \ref{ex:dirac-counting-indiscrete}.
\end{rem}
\begin{proof}
  If $A$ is finite and has $k$ elements $A = \{a_1,\ldots,a_k\}$, let $\delta > 0$ be the minimal distance between all elements.

  It easily follows from the triangle inequality, that $\Hau^{0}(A) \geq \Hau_{\frac{\delta}{2}}^{0} \geq k$.
  The other inequality is also trivial.

  If $A$ is infinite, then for any $k \in \N$ we can find a subset $A_k$ with at least $k$ elements. By monotonicity we have $k = \Hau^{0}(A_k) \leq \Hau^{0}(A)$ and in the limit $k \to  \infty$, the proof follows.

  And if $A$ is empty, by defintion we have $\Hau^{0}(\emptyset) = 0$.
\end{proof}
