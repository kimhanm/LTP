\section{Measurable Functions}

\subsection{Basic definitions}
For $X,Y$ nonempty sets and $f: X \to  Y$ with $A \subseteq Y$, the inverse image is defined as
\begin{align*}
  f^{-1}(A) = \{x \in X \big\vert f(x)\in A\}
\end{align*}
And it's easy to show that 
\begin{enumerate}
  \item $f^{-1}(A^{c}) = \left(f^{-1}(A)\right)^{c}$
  \item If $(A_k)_{k}$ is a sequence of subsets, then\footnote{The analogue for countable intersections follows easily from de Morgan's rule, (a) and (b)}
    \begin{align*}
      f^{-1} \left(
        \bigcup_{k = 1}^{\infty}A_k
      \right)
      = \bigcup_{k=1}^{\infty}f^{-1}(A_k)
    \end{align*}
 % \item 
 %   From de-Morgan's rule, (a) and (b) it also follows that
 %   \begin{align*}
 %     f^{-1}\left(
 %       \bigcap_{k=1}^{\infty} A_k
 %     \right)
 %     &= f^{-1}\left(
 %       \left(
 %         \bigcup_{k=1}^{\infty}A_k^{c}
 %       \right)^{c}
 %     \right)
 %     = f^{-1}\left(
 %       \bigcup_{k=1}^{\infty} A_k^{c}
 %     \right)^{c}\\
 %     &= 
 %     \left(
 %       \bigcup_{k=1}^{\infty}
 %       f^{-1}(A_k^{c})
 %     \right)^{c}
 %     = \bigcap_{k=1}^{\infty}
 %     (f^{-1}(A_k^{c}))^{c}
 %     = \bigcap_{k=1}^{\infty}
 %     f^{-1}(A_k)
 %   \end{align*}
\end{enumerate}
In particular, if $\mathcal{A} \subseteq \mathcal{P}(Y)$ is a $\sigma$-algebra, then 
\begin{align*}
  \Sigma := f^{-1}(\mathcal{A}) := \{f^{-1}(A) \big\vert A \in \mathcal{A}\}
\end{align*}
is a $\sigma$-algebra in $X$.






\begin{dfn}[]
  A function $f: \Omega \to [-\infty,\infty]$ is called \textbf{$\bm{\mu}$-measurable} if in the sense of definition \ref{dfn:mu-measurable}
  \begin{enumerate}
    \item $f^{-1} \{+ \infty\}, f^{-1}\{-\infty\}$ are $\mu$-measurable.
    \item$f^{-1}(U)$ for every $U \subseteq \R$ open is $\mu$-measurable.
  \end{enumerate}
\end{dfn}
The composition of $\mu$-measurable functions
