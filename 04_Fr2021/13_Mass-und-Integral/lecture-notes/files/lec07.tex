Recall that the Borel $\sigma$-algebra $\mathcal{B}(X)$ is the $\sigma$-algebra generated by open subsets of $X$.

\begin{dfn}[]
  A measure $\mu$ on $\R^{n}$ is called \textbf{Borel} (or a Borel measure), if every Borel set is $\mu$-measurable.
\end{dfn}

From Lemma \ref{lem:open-dyadic}, it follows that $\Leb^{n}$ is a Borel measure.

When we want to characterize $\Leb^{n}(A)$ for some subset $A \subseteq \R^{n}$, the definition used in the Carathéodory-Hahn extension where we consider all countable coverings using elementary sets is quite unwiedly.
The following theorem gives a nicer characterisation.

\begin{thm}[]
For every $A \subseteq \R^{n}$ it holds
\begin{empheq}[box=\bluebase]{align*}
  \Leb^{n}(A) = \inf_{A \subseteq U} \Leb^{n}(U), \quad U \text{ open}
\end{empheq}
\end{thm}
\begin{proof}
  By monotonicity, $\Leb^{n}(A) \leq \Leb^{n}(U)$ follows directly.

  For the other inequality, suppose that $\Leb^{n}(A) < \infty$ (or else the inequality is trivial).
  By definition, for any $\epsilon > 0$ we can find intervals $(I_k)_{k \in \N}$ with
  \begin{align*}
    A \subseteq \bigcup_{k=1}^{\infty}I_k, \quad \sum_{k=1}^{\infty}\vol(I_k) \leq \Leb^{n}(A) + \epsilon
  \end{align*}
  Since $\Leb^{n}(A) < \infty$, every interval $I_k$ must have finite volume and is thus bounded.
  So let $\tilde{I}_k \supseteq I_k$ be open bounded intervals with $\vol(\tilde{I}_k) \leq \vol(I_k) + \frac{\epsilon}{2^{k}}$.

  Setting $U := \bigcup_{k=1}^{\infty} \tilde{I}_k$, we see that $U$ is an open subset containing $A$ and it's volume is
  \begin{align*}
    \Leb^{n}(U) \leq \sum_{k=1}^{\infty} \vol(\tilde{I}_k) \leq \sum_{k=1}^{\infty} \vol(I_k) + \frac{\epsilon}{e^{k}} \leq \Leb^{n}(A) + 2 \epsilon
  \end{align*}
  since $\epsilon$ was arbitrary, the result follows.
\end{proof}

This alternative characterisation lets us find out what subsets $A \subseteq \R^{n}$ are $\Leb^{n}$-measurable.

\begin{thm}[]
  For any subset $A \subseteq \R^{n}$ the following are equivalent
  \begin{enumerate}
    \item $A$ is $\mathcal{L}^{n}$-measurable..
    \item $\forall \epsilon > 0\ \exists U \supseteq A$ with $\mathcal{L}^{n}(U \setminus A) <\epsilon$.
    \item $A$ it can be ``approximated'' from the inside and outside: $\forall \epsilon > 0\ \exists F$ closed, $U$ open with $F \subseteq A \subseteq U$ such that
      \begin{align*}
        \Leb^{n}(U \setminus A) + \Leb^{n}(A \setminus F) \leq \epsilon
      \end{align*}
    \item $\forall \epsilon > 0 \exists F$ closed, $\exists G$ open, such that $\Leb^{n}(G \setminus F) < \epsilon$.
  \end{enumerate}
\end{thm}




\subsection{Comparision between Lebesgue and Jordan Measure}
\begin{dfn}[]
A bounded subset $A \subseteq \R^{n}$ is \textbf{Jordan-measurable} if
\begin{align*}
  \underline{\mu}(A)
  :=
  \sup \{\vol(E)\big\vert E \subseteq A, E \text{ elementary set}\} 
  =
  \inf \{\vol(E)\big\vert A \subseteq E, E \text{ elementary set}\} 
  =:
  \overline{\mu}(A)
\end{align*}
\end{dfn}


As the following proof will show, The Lebesgue measure can measure more sets than the Jordan measure can.
\begin{thm}[]
Let $A \subseteq \R^{n}$ be bounded, then
\begin{enumerate}
  \item $\underline{\mu}(A) \leq\Leb^{n}(A9 \leq \overline{\mu}(A)$
  \item If $A$ is Jordan-measurable, then $A$ is $\Leb^{n}$-measurable and $\Leb^{n}(A) = \mu(A)$.
\end{enumerate}
\end{thm}

One would naturally think that the volume of a cube should stay the same whether we move it by a fixed point or rotate it.
\begin{thm}[]
  The Lebegue measure is invariant under isometries $\Phi: \R^{n} \to \R^{n}, x \mapsto x_0 + Rx$, for $R \in O(n)$.
\end{thm}


\begin{dfn}[]
  A Borel measure $\mu$ on $\R^{n}$ is called \textbf{Borel regular}, if for every $A \subseteq \R^{n}$ there exists a borel set $B \supseteq A$ such that $\mu(A) = \mu(B)$.
\end{dfn}
The Lebesgue measure is Borel regular.


