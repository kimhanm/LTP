
When we want to characterize $\Leb^{n}(A)$ for some subset $A \subseteq \R^{n}$, the definition used in the Carathéodory-Hahn extension where we consider all countable coverings using elementary sets is quite unwiedly.
The following theorem gives a nicer characterisation.

\begin{thm}[] \label{thm:leb-open}
For every $A \subseteq \R^{n}$ it holds
\begin{empheq}[box=\bluebase]{align*}
  \Leb^{n}(A) = \inf_{A \subseteq U} \Leb^{n}(U), \quad U \text{ open}
\end{empheq}
\end{thm}
\begin{proof}
  By monotonicity, $\Leb^{n}(A) \leq \Leb^{n}(U)$ follows directly.

  For the other inequality, suppose that $\Leb^{n}(A) < \infty$ (or else the inequality is trivial).
  By definition, for any $\epsilon > 0$ we can find intervals $(I_k)_{k \in \N}$ with
  \begin{align*}
    A \subseteq \bigcup_{k=1}^{\infty}I_k, \quad \sum_{k=1}^{\infty}\vol(I_k) \leq \Leb^{n}(A) + \epsilon
  \end{align*}
  Since $\Leb^{n}(A) < \infty$, every interval $I_k$ must have finite volume and is thus bounded.
  So let $\tilde{I}_k \supseteq I_k$ be open bounded intervals with $\vol(\tilde{I}_k) \leq \vol(I_k) + \frac{\epsilon}{2^{k}}$.

  Setting $U := \bigcup_{k=1}^{\infty} \tilde{I}_k$, we see that $U$ is an open subset containing $A$ and it's volume is
  \begin{align*}
    \Leb^{n}(U) \leq \sum_{k=1}^{\infty} \vol(\tilde{I}_k) \leq \sum_{k=1}^{\infty} \vol(I_k) + \frac{\epsilon}{e^{k}} \leq \Leb^{n}(A) + 2 \epsilon
  \end{align*}
  since $\epsilon$ was arbitrary, the result follows.
\end{proof}

This alternative characterisation lets us find out what subsets $A \subseteq \R^{n}$ are $\Leb^{n}$-measurable.

\begin{thm}[] \label{thm:leb-characterisation}
  For any subset $A \subseteq \R^{n}$ the following are equivalent
  \begin{enumerate}
    \item $A$ is $\mathcal{L}^{n}$-measurable.
    \item $\forall \epsilon > 0\ \exists U \supseteq A$ open with $\mathcal{L}^{n}(U \setminus A) <\epsilon$.
    \item $A$ it can be ``approximated'' from the inside and outside: $\forall \epsilon > 0\ \exists F$ closed, $U$ open with $F \subseteq A \subseteq U$ such that
      \begin{align*}
        \Leb^{n}(U \setminus A) + \Leb^{n}(A \setminus F) < \epsilon
      \end{align*}
    \item $\forall \epsilon > 0 \exists F$ closed, $\exists U$ open, such that $F \subseteq A \subseteq U$ and $\Leb^{n}(U \setminus F) < \epsilon$.
  \end{enumerate}
\end{thm}


\begin{proof}
  \phantom{a}
\begin{enumerate}
  \item[(a) $\implies$ (b):] Let $\epsilon >0$, $A$ be $\Leb^{n}$-measurable.

    \begin{itemize}
      \item If $\Leb^{n}(A) < \infty$, by the previous theorem, we can chose a $U \supseteq A$ open such that
        \begin{align*}
          \Leb^{n}(U) \leq \Leb^{n}(A) + \epsilon
        \end{align*}
        Because $A$ is $\Leb^{n}$-measurable we can use $U$ as a test set and get
        \begin{align*}
          \Leb^{n}(U) &= \Leb^{n}(U \cap A) + \Leb^{n}(U \setminus A)\\
                      &= \Leb^{n}(A) + \Leb^{n}(U \setminus A)
        \end{align*}
        which gives us
        \begin{align*}
          \Leb^{n}(U \setminus A) = \Leb^{n}(U) - \Leb^{n}(A) < \epsilon
        \end{align*}

      \item If $\Leb^{n}(A) = \infty$, we set
        \begin{align*}
          A_k = A \cap [-k,k]^{n} \implies A = \bigcup_{k=1}^{\infty}A_k
        \end{align*}
        since $\Leb^{n}(A_k) < \infty$, we are in the first case so we can find $U_k \supseteq A_k$ open with
        \begin{align*}
          \Leb^{n}(U_k \setminus A_k) < \frac{\epsilon}{2^{k}} \quad \forall k \in \N
        \end{align*}
        Then their union $U := \bigcup_{k=1}^{\infty}U_k$ is open and contains $A$.
        Moreover, we have
        \begin{align*}
          \Leb^{n}(U \setminus A) 
          &= \Leb^{n} \left(
            \bigcup_{k=1}^{\infty}(U_k \setminus A)
          \right)\\
          &\leq
          \Leb^{n} \left(
            \bigcup_{k=1}^{\infty}(U_k \setminus A_k)
          \right)\\
          &\leq \sum_{k=1}^{\infty} \Leb^{n}(U_k \setminus A_k) < \epsilon
        \end{align*}
    \end{itemize}



  \item[(b) $\implies$ (a):] Let $B \subseteq \R^{n}$. 
    For $\epsilon > 0$, chose $U \supseteq A$ open with $\Leb^{n}(U \setminus A) < \epsilon$.
    Then
    \begin{align*}
      B \setminus A \subseteq 
      (B \setminus U) \cup (U \setminus A)
    \end{align*}
    Since open subsets are $\Leb^{n}$-measurable, we have
    \begin{align*}
      \Leb^{n}(B) 
      &=
      \Leb^{n}(B \cap U) + \Leb^{n}(B \setminus U)\\
      &\geq
      \Leb^{n}(B \cap A) + \Leb^{n}(B \setminus A) - \Leb^{n}(U \setminus A)\\
      &\geq
      \Leb^{n}(B \cap A) + \Leb^{n}(B \setminus A) - \epsilon
    \end{align*}
    since $\epsilon$ was arbitrary, $\Leb^{n}$-measurability of $A$ follows.


  \item[(b) $\iff$ (c):] For $\epsilon > 0$ use (b) for $A^{c}$ to get an open set $V \supseteq A^{c}$ with
    $\Leb^{n}(V \setminus A^{c}) < \epsilon$.
    Then $F = V^{c} \subseteq A$ is closed and 
    \begin{align*}
      \Leb^{n}(A \setminus V^{c}) = \Leb^{n}(V \setminus A^{c}) < \epsilon
    \end{align*}
    The other implication is trivial.

  \item[(c) $\implies$ (d):] Using (c), we get $F \subseteq A$ closed and $U \supseteq A$ open. 
    Because $F \subseteq A \subseteq U$,
    \begin{align*}
      U \setminus F = (U \setminus A) \cup (A \setminus F) 
    \end{align*}
    it follows from subadditivity that 
    \begin{align*}
      \Leb^{n}(U \setminus F) \leq \Leb^{n}(U \setminus A) + \Leb^{n}(A \setminus F) < \epsilon
    \end{align*}

  \item[(d) $\implies$ (c):] For $\epsilon > 0$, use (d) to get $F \subseteq A$ closed, $U \supseteq A$ open with $\Leb^{n}(U \setminus F) < \epsilon$.
    Because $F \subseteq A \subseteq U$
    \begin{align*}
      U \setminus A \subseteq U \setminus F, \quad A \setminus F \subseteq U \setminus F
    \end{align*}
    so we get
    \begin{align*}
      \Leb^{n}(U \setminus A) + \Leb^{n}(A \setminus F) \leq 2 \Leb^{n}(U \setminus F) < 2 \epsilon
    \end{align*}
\end{enumerate}
\end{proof}



\subsection{Comparision between Lebesgue and Jordan Measure}

\begin{dfn}[]
  A bounded subset $A \subseteq \R^{n}$ is \textbf{Jordan-measurable} if $\underline{\mu}(A) = \overline{\mu}(A)$, where
\begin{align*}
  \underline{\mu}(A)
  &:=
  \underline{\int}_{\R^{n}} \chi_A d \mu :=
  \sup \{\vol(E)\big\vert E \subseteq A, E \text{ elementary set}\}\\
  \overline{\mu}(A)
  &:=
  \overline{\int}_{\R^n}\chi_A d \mu :=
  \inf \{\vol(E)\big\vert A \subseteq E, E \text{ elementary set}\} 
\end{align*}
If that is the case, denote the Jordan measure of $A$ with the common value $\mu(A)$.


We call $\underline{\mu}(A)$ the \textbf{Jordan inner measure} of $A$ and $\overline{\mu}(A)$ the \textbf{Jordan outer measure} of $A$.
\end{dfn}

\begin{ex}[]
For $f: I \to  \R$ continuous, $I \subseteq \R^{n}$ compact, its graph
\begin{align*}
  \Gamma = \{(x,f(x)) \big\vert x \in I\} \subseteq \R^{n+1}
\end{align*}
is a Jordan measurable set.

The area under a function
\begin{align*}
  G = \{(x,t) \in I \times \R \big\vert 0 \leq t \leq f(x)\}
\end{align*}
is also Jordan-measurable
\end{ex}



As the following theorem will show, 
the Lebesgue measure can measure more sets than the Jordan measure can.

\begin{thm}[]
Let $A \subseteq \R^{n}$ be bounded, then
\begin{enumerate}
  \item $\underline{\mu}(A) \leq\Leb^{n}(A) \leq \overline{\mu}(A)$
  \item If $A$ is Jordan-measurable, then $A$ is $\Leb^{n}$-measurable and $\Leb^{n}(A) = \mu(A)$.
\end{enumerate}
\end{thm}
\begin{proof}
\begin{enumerate}
  \item Because elementary sets are finite disjoint unions of intervals, we have
    \begin{align*}
      \Leb^{n}(A) 
      &= \inf \left\{
        \sum_{k=1}^{\infty}\vol(I_k) \big\vert A \subseteq \bigcup_{k=1}^{\infty}I_k, I_k \text{ intervals}
      \right\}
      \\
      &\leq
      \inf \left\{
        \sum_{k=1}^{m}\vol(I_k) \big\vert A \subseteq E = \bigsqcup_{k=1}^{m}I_k, I_k \text{ intervals}
      \right\}
      \\
      &= \overline{\mu}(A)
    \end{align*}
    For the other inequality, for every elementary set $E = \bigsqcup_{k=1}^{m}I_k \subseteq A$ we have
    \begin{align*}
      \vol(E) = \Leb^{n}(E) \leq \Leb^{n}(A)
    \end{align*}
    so when taking the sup over such $E$, we get
    \begin{align*}
      \underline{\mu}(A) \leq \Leb^{n}(A)
    \end{align*}


  \item If $A$ is Jordan measurable, then it follows from (i) that
    \begin{align*}
      \underline{\mu}(A) \leq \Leb^{n}(A) \leq \overline{\mu}(A) = \underline{\mu}(A)
    \end{align*}
    To show that $A$ is $\Leb^{n}$-measurable, we use characterisation (b) from Theorem \ref{thm:leb-characterisation}

    Because $A$ is bounded, $\Leb^{n}(A) < \infty$ and because it is Jordan-measurable, we can find for all $\epsilon > 0$ elementary sets $E_{\epsilon},E^{\epsilon}$ such that
    \begin{align*}
      E_{\epsilon} \subseteq A \subseteq E^{\epsilon}  \quad \text{and} \quad \vol(E^{\epsilon}) - \epsilon < \mu(A) < \vol(E_{\epsilon}) + \epsilon
    \end{align*}
    Because the volume doesn't depend on whether the intervals comprising the elementary set are open, half-open or closed, we can assume WLOG that $E^{\epsilon}$ is open, so
    \begin{align*}
      \Leb^{n}(E^{\epsilon} \setminus A)
      &\leq \Leb^{n}(E^{\epsilon} \setminus E_{\epsilon})
      =
      \vol(E^{\epsilon} \setminus E_{\epsilon})\\
      &=
      \vol(E^{\epsilon}) - \vol(E_{\epsilon}) < 2 \epsilon
    \end{align*}
    which shows the condition from the previous theorem.
\end{enumerate}
\end{proof}



