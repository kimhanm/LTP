\begin{dfn}[]
  For a collection of sets $\mathcal{K} \subseteq \mathcal{P}(X)$, the intersection of all $\sigma$-algebras containing $K$ forms again a $\sigma$-algebra.

  We call this the $\sigma$-algebra \textbf{generated by} $K$ and it its the smallest $\sigma$-algebra that contains $K$.

  The algebra generated by the open sets of a topology is called the \textbf{Borel $\sigma$-Algebra} of $X$, denoted $\mathcal{B}(X)$.
\end{dfn}



\subsection{Measures}
\begin{dfn}[]
  Let $\mathcal{A}$ be an Algebra on $X$ and $\mu: \mathcal{A} \to [0,\infty]$. We say that $\mu$ is
  \begin{itemize}
    \item \textbf{additive}, if for any \emph{finite} family of disjoint sets $A_{1}, \ldots, A_{n} \in \mathcal{A}$
      \begin{align*}
        \mu \left(
          \bigsqcup_{k=1}^{n}A_k
        \right)
        =
        \sum_{k=1}^{n}\mu(A_k)
      \end{align*}
    \item \textbf{$\sigma$-additive}, if for any \emph{countable} family of
      disjoint sets $\left(A_{n}\right)_{n \in \N} \subseteq \mathcal{A}$ such that $\bigsqcup_{k=1}^{\infty}A_k \in \mathcal{A}$
      \begin{align*}
        \mu \left(
          \bigsqcup_{k=1}^{\infty}A_k
        \right)
        =
        \sum_{k=1}^{\infty}\mu(A_k)
      \end{align*}
    \item A \textbf{pre-measure}, if it is $\sigma$-additive and satisfies $\mu(\emptyset) = 0$.
  \end{itemize}
\end{dfn}

\begin{rem}[]
  Let $\left(A_{n}\right)_{n \in \N}$ be a sequence of sets in $\mathcal{A}$ such that their union is again in $\mathcal{A}$.
\begin{enumerate}
  \item If $\mu$ is additive, then it is monotone with respect to incusion, i.e. $A \subseteq B \implies \mu(A) \leq \mu(B)$.
  \item If $\mu$ is additive and the sets $A_k$ are mutually disjoint, then
    \begin{align*}
        \mu \left(
          \bigsqcup_{k=1}^{\infty}A_k
        \right)
        \geq
        \sum_{k=1}^{\infty}\mu(A_k)
    \end{align*}
  \item If $\mu$ is $\sigma$-additive, then it is also \textbf{$\sigma$-subadditive}, which means that for any sequence $(A_n)_{n \in \N}$ in $\mathcal{A}$ with $\bigcup_{k=1}^{\infty}A_k \in \mathcal{A}$
    \begin{align*}
      \mu \left(
        \bigcup_{k=1}^{\infty} A_k
      \right)
      \leq \sum_{k=1}^{\infty}\mu(A_k)
    \end{align*}
    To see this, we can define the mutually disjoint sets
    \begin{align*}
      B_1 = A_1, \quad B_n = A_n \setminus \bigcup_{k=1}^{n-1}A_k \in \mathcal{A}
    \end{align*}
    Since $\bigsqcup_{k=1}^{\infty}B_k = \bigcup_{k=1}^{\infty}A_k$ and $\mu(B_k) \leq \mu(A_k)$ we have
    \begin{align*}
        \mu \left(
          \bigcup_{k=1}^{\infty}A_k
        \right)
        =
        \mu \left(
          \bigsqcup_{k=1}^{\infty}B_k
        \right)
        =
        \sum_{k=1}^{\infty}\mu(B_k)
        \leq
        \sum_{k=1}^{\infty}\mu(A_k)
    \end{align*}
\end{enumerate}
It follows immediately from (b) and (c) that 
\begin{empheq}[box=\bluebase]{align*}
  \mu \text{ is additive and $\sigma$-subadditive} \iff \mu \text{ is $\sigma$-additive}
\end{empheq}
\end{rem}
\begin{ex}[]
Not all additive functions are $\sigma$-additive. 
For $X = \N$ and
\begin{align*}
  \mathcal{A} = \{A \in \mathcal{P}(X) \big\vert A \text{ is finite or } A^{c} \text{ is finite}\}
\end{align*}
the function $\nu: \mathcal{A} \to [0,\infty]$ with $\nu(\emptyset) = 0$ and
\begin{align*}
  \nu(A) = \left\{\begin{array}{ll}
      \underset{n \in A}{\sum} \frac{1}{2^{n}} & \text{if $A$ is finite}\\
      \infty & \text{if $A^{c}$ is finite}
  \end{array} \right.
\end{align*}
is additive but not $\sigma$-additive because we can take the sequence
\begin{align*}
  A_1 = \{1\}, A_2 = \{2\}, A_3 = \{3\}, \ldots, A_n = \{n\}, \ldots
\end{align*}
which is a sequence of mutually disjoint sets satsfying
\begin{align*}
  &\nu(A_1) = \frac{1}{2}, \nu(A_2) = \frac{1}{4}, \ldots, \nu(A_n) = \frac{1}{2^{n}} \\
  \implies 
  &\nu \left(
    \bigsqcup_{n=1}^{\infty}A_n
  \right)
  = \nu(\N)
  = \infty \not\leq \sum_{n=1}^{\infty}\nu(A_n) = 1
\end{align*}
\end{ex}

\begin{dfn}[]
  A $\sigma$-additive function $\mu: \mathcal{A} \to [0,\infty]$ is called
    \begin{itemize}
      \item \textbf{finite}, if $\mu(X) < \infty$
      \item \textbf{$\sigma$-finite}, if there exists a sequence $\left(A_{n}\right)_{n \in \N} \subseteq \mathcal{A}$ such that
        \begin{align*}
          \bigcup_{n=1}^{\infty}A_n = X \quad \text{and} \quad \mu(A_n) < \infty \quad \forall n \in \N
        \end{align*}
    \end{itemize}
\end{dfn}
Clearly, $\mu$ finite $\implies$ $\mu$ $\sigma$-finite.

While pre-measures are only defined on algebras $\mathcal{A} \subseteq \mathcal{P}(X)$, we would like to extend the domain of such functions to $\mathcal{P}(X)$ without losing too many of its nice properties.
In particular, we want to keep monotonicity and $\sigma$-subadditivity:
\begin{dfn}[]
  A function $\mu: \mathcal{P}(X) \to [0,\infty]$ is called a \textbf{measure}\footnote{sometimes also called outer measure} on $X$, if
  \begin{enumerate}
    \item $\mu(\emptyset) = 0$
    \item $\mu$ is $\sigma$-subadditive:
    If $A \subseteq \bigcup_{k=1}^{\infty}A_k$, then $\mu(A) \leq \sum_{k=1}^{\infty}\mu(A_k)$
  \end{enumerate}
\end{dfn}
Note that subadditivity implies monotonicity with respect to inclusion, i.e. $A \subseteq B \implies \mu(A) \leq \mu(B)$.

\begin{dfn}[]
  Let $\mu$ be a measure on $X$ and $A \subseteq X$. We can \emph{restrict} $\mu$ to $A$ (written $\mu \mres A$) defined by
  \begin{align*}
    (\mu \mres A)(B) := \mu(A \cap B) \quad \forall B \subseteq X
  \end{align*}
\end{dfn}

\begin{dfn}[Carathéodory criterion]\label{dfn:mu-measurable}
A subset $A \subseteq X$ is called \textbf{$\mu$-measurable} if 
\begin{align*}
  \mu(B) = \mu(B \cap A) + \mu(B \setminus A), \quad \forall  B \subseteq X
\end{align*}
\end{dfn}

\begin{rem}[]
\begin{enumerate}
  \item By subadditivity of the measure, the definition is equivalent to
    \begin{empheq}[box=\bluebase]{align*}
      \mu(B) \geq \mu(B \cap A) + \mu(B \setminus A), \quad \forall B \subseteq X
    \end{empheq}
  \item If $\mu(A) = 0$, then $A$ is $\mu$-measurable.
\end{enumerate}
\end{rem}

