For a collection of sets $\mathcal{K} \subseteq \mathcal{P}(X)$, the intersection of all Algebras containing $K$ forms an algebra.
We call this the algebra \textbf{generated by} $K$.

The algebra generated by the open sets of a topology is called the \textbf{Borel $\sigma$-Algebra} of $X$.


\subsection{Measures}
\begin{dfn}[]
  Let $\mathcal{A}$ be an Algebra on $X$ and $\mu: \mathcal{A} \to [0,\infty]$ such that $\mu(\emptyset) = 0$. We say that $\mu$ is
  \begin{itemize}
    \item \textbf{additive}, if for any \emph{finite} family of disjoint sets $A_{1}, \ldots, A_{n} \in \mathcal{A}$
      \begin{align*}
        \mu \left(
          \bigsqcup_{k=1}^{n}A_k
        \right)
        =
        \sum_{k=1}^{n}\mu(A_k)
      \end{align*}
    \item \textbf{$\sigma$-additive}, if for any \emph{countable} family of
      disjoint sets $\left(A_{n}\right)_{n \in \N} \subseteq \mathcal{A}$ such that $\bigcup_{k=1}^{\infty}A_k \in \mathcal{A}$
      \begin{align*}
        \mu \left(
          \bigsqcup_{k=1}^{\infty}A_k
        \right)
        =
        \sum_{k=1}^{\infty}\mu(A_k)
      \end{align*}
  \end{itemize}
\end{dfn}

\begin{rem}[]
  Let $\left(A_{n}\right)_{n \in \N}$ be a sequence of sets in $\mathcal{A}$ such that their union is again in $\mathcal{A}$.
\begin{enumerate}
  \item If $\mu$ is additive, then it is monotone with respect to incusion, i.e. $A \subseteq B \implies \mu(A) \leq \mu(B)$.
  \item If $\mu$ is additive and the sets $A_k$ are mutually disjoint, then
    \begin{align*}
        \mu \left(
          \bigsqcup_{k=1}^{\infty}A_k
        \right)
        \geq
        \mu \left(
          \bigsqcup_{k=1}^{n}A_k
        \right)
        = \sum_{k=1}^{n}\mu(A_k) \quad \forall n \in \N
    \end{align*}
    so $\mu$ is \textbf{$\sigma$-superadditive}
    \begin{align*}
      \mu\left(
        \bigsqcup_{k=1}^{\infty}A_k
      \right)
      \geq
      \sum_{k=1}^{\infty}\mu(A_k)
    \end{align*}
  \item If $\mu$ is $\sigma$-additive, then it is also \textbf{$\sigma$-subadditive}. 
    To see this, we can define the mutually disjiont sets
    \begin{align*}
      B_1 = A_1, \quad B_n = A_n \setminus \bigcup_{k=1}^{n-1}A_k \in \mathcal{A}
    \end{align*}
    Since $\bigcup_{k=1}^{\infty}B_k = \bigcup_{k=1}^{\infty}A_k$ and $\mu(B_k) \leq \mu(A_k)$ we have
    \begin{align*}
        \mu \left(
          \bigcup_{k=1}^{\infty}A_k
        \right)
        =
        \mu \left(
          \bigsqcup_{k=1}^{\infty}B_k
        \right)
        =
        \sum_{k=1}^{\infty}\mu(B_k)
        \leq
        \sum_{k=1}^{\infty}\mu(A_k)
    \end{align*}
\end{enumerate}
It follows immediately from (b) and (c) that an additive function is $\sigma$-additive, if and only if it is $\sigma$-subadditive.
\end{rem}

\begin{dfn}[]
  A $\sigma$-additive function $\mu: \mathcal{A} \to [0,\infty]$ is called
    \begin{itemize}
      \item \textbf{finite}, if $\mu(X) < \infty$
      \item \textbf{$\sigma$-finite},if there exists a sequence $\left(A_{n}\right)_{n \in \N} \subseteq \mathcal{A}$ such that
        \begin{align*}
          \bigcup_{n=1}^{\infty}A_n = X \quad \text{and} \quad \mu(A_n) < \infty \quad \forall n \in \N
        \end{align*}
    \end{itemize}
\end{dfn}


\begin{dfn}[]
  A function $\mu: \mathcal{P}(X) \to [0,\infty]$ is called a \textbf{measure} on $X$, if
  \begin{enumerate}
    \item $\mu(\emptyset) = 0$
      For $A \subseteq \bigcup_{k=1}^{\infty}A_k$, we have $\mu(A) \leq \sum_{k=1}^{\infty}\mu(A_k)$
  \end{enumerate}
\end{dfn}
A measure is automatically monotonous and $\sigma$-subadditive.

\begin{dfn}[]
  Let $\mu$ be a measure on $X$ and $A \subseteq X$. We can \emph{restrict} $\mu$ to $A$ (written $\mu \mres A$) defined by
  \begin{align*}
    (\mu \mres A)(B) := \mu(A \cap B) \quad \forall B \subseteq X
  \end{align*}
\end{dfn}

\begin{dfn}[Carathéodory criterion]\label{dfn:mu-measurable}
A subset $A \subseteq X$ is called \textbf{$\mu$-measurable} if for all $B \subseteq X$
\begin{align*}
  \mu(B) = \mu(B \cap A) + \mu(B \setminus A)
\end{align*}
\end{dfn}
\begin{rem}[]
\begin{enumerate}
  \item By subadditivity of the measure, the definition is equivalent to
    \begin{empheq}[box=\bluebase]{align*}
      \mu(B) \geq \mu(B \cap A) + \mu(B \setminus A), \quad \forall B \subseteq X
    \end{empheq}
  \item If $\mu(A) = 0$, then $A$ is $\mu$-measurable.
\end{enumerate}
\end{rem}

