\subsection{Special-Examples of sets}

As we will see, not all subsets of $\R^{n}$ are $\Leb^{n}$-measurable.

To construct such a non-measurable set, we will use the Axiom of Choice, which states that for any family of non-empty disjoint sets $(A_i)_{i \in I}$, there exists a choice-function $f: I \to \bigcup_{i \in I} A_i$ such that $f(i) \in A_i$.

With this, we can construct the set $\{f(i) \big\vert i \in I\}$ that contains exactly one element from each set $A_i$.

For $x,y \in [0,1)$ we define $\oplus := (\mod 1 \circ +)$
\begin{align*}
  x \oplus y = \left\{\begin{array}{ll}
    x+y & \text{ if } x + 1 < 1\\
    x + y - 1 & \text{ if } x + y \geq 1
  \end{array} \right.
\end{align*}


So if we have a subset $E \subseteq [0,1)$, we can ``shift'' the set $E$ by $x$, with $E \oplus x \subseteq [0,1)$.

Where some part $E \cap  [0,1-x)$ moves naturally to the right and the set $E \cap [1-x,1)$ moves back to the left side. 
Set
\begin{align*}
  E_1 &:= E \cap [0,1-x) \oplus x\\
  E_2 &:= E \cap [1-x,1) \oplus x
\end{align*}
which are disjoint.

If $E$ is $\Leb^{1}$-meaurable, then the translated sets $E_1,E_2$ are also $\Leb^{1}$-measurable and 
\begin{align*}
  \Leb^{1}(E \oplus x) 
  &= \Leb^{1}(E_1) + \Leb^{1}(E_2)\\
  &= \Leb^{1}(E \cap [0,1-x)) + \Leb^{1}(E \cap [1-x,1))\\
  &= \Leb^{1}(E)
\end{align*}

Then we define the equivalence relation
\begin{align*}
  x,y \in [0,1) \quad x \sim y \iff x-y \in \Q
\end{align*}
by the axiom of choice, there exists a set $P \subseteq [0,1)$ that contains exactly one representative of each equivalence class.

By enumerating all rational points in $[0,1)$ by an index $Q \cap [0,1) = \{r_k\}_{k \in \N}$ with $r_0 = 0$ we define
\begin{align*}
  P_k := P \oplus r_k
\end{align*}

Then it is easy to see that
\begin{enumerate}
  \item The $P_j$ are disjoint and $[0,1) = \bigsqcup_{j=0}^{\infty}P_j$.

    Because if $x \in P_n \cap P_m$, then $x = p_n \oplus r_n = p_m \oplus r_m$. 
    Since $r_n,r_m \in \Q$ it follows that also $p_n-p_m \in \Q$ so they must be of the same equivalence class.

    It also covers $[0,1)$ because by construction, every $x \in [0,1)$ belongs to a unique equivalence class.

  \item If $P$ were $\Leb^{1}$-measurable, then so is $P_j = P \oplus r_j$ and $\Leb^{1}(P) = \Leb^{1}(P_j)$.

    We just showed this earlier.
\end{enumerate}
But $P$ cannot be $\Leb^{1}$-measurable, because by $\sigma$-additivity on $\Leb^{1}$-measurable subsets
\begin{align*}
  1 = \Leb^{1}([0,1)) = \sum_{i=0}^{\infty}\Leb^{1}(P_j) = \sum_{i=0}^{\infty} \Leb^{1}(P)
\end{align*}
and the right hand side is either $0$ or infinite.

So since $P$ is not $\Leb^{1}$-measurable there exists a set $B \subseteq \R$ with
\begin{align*}
  \Leb^{1}(B) < \Leb^{1}(B \cap P) + \Leb^{1}(B \setminus P)
\end{align*}

We also know that $\Leb^{1}(P)$ can't be zero, or else it would be $\Leb^{1}$-measurable.

Moreover, if $E \subseteq P$ is $\Leb^{1}$-measurable, then $\Leb^{1}(E) = 0$ because we can set
\begin{align*}
  E_i := E \oplus r_i \implies F := \bigsqcup_{i=0}^{\infty} E_i \subseteq [0,1) \text{ is $\Leb^{1}$-measurable}
\end{align*}
and we have
\begin{align*}
  1 = \Leb^{1}([0,1)) \geq \Leb^{1}(F) = \sum_{i=0}^{\infty}\Leb^{1}(E_i) = \sum_{i=0}^{\infty}\Leb^{1}(E)
\end{align*}
which can only be true if $\Leb^{1}(E) = 0$.


Not only does there exists a non-$\Leb^{1}$-measurable subset, we can construct more using $P$ as a ``template''.

\begin{prop}[]
  For every $A \subseteq \R$ with $\Leb^{1}(A) > 0$, there exists a subset $B \subseteq A$ that is not $\Leb^{1}$-measurable.
\end{prop}
\begin{proof}
  Because we can shift and scale $A$ or take subsets of $A$, we can assume without loss of generality that $A \subseteq (0,1)$.

  Then set $B_i = A \cap P_i$. Then $A = \bigsqcup_{i=0}^{\infty}B_i$

  As we showed earlier, if $B_i$ were $\Leb^{1}$-measuralble, then $\Leb^{1}(B_i) = 0$, which contradicts $\Leb^{1}(A) = \sum_{i=0}^{\infty} \Leb^{1}(B_i)$.
\end{proof}

\begin{prop}[]
  Every countable subset of $\R$ has Lebesgue measure zero.
\end{prop}
\begin{proof}
MISSING
\end{proof}




\subsubsection*{The Cantor tridadic set}

The real numbers can be defined as the set of Cauchy-sequences in $\Q$ up to limits.

The cantor set is the set of numbers whose triadic digits don't contain a $1$.
\begin{align*}
  C = \{x \in [0,1] \big\vert d_i(x) \in \{0,2\} \forall i\}
\end{align*}
Then $C$ is uncountable and $\Leb(C) = 0$.


