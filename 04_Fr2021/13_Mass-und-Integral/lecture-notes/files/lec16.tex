\subsection{Convergence in Measure}
For this section, let 
$\mu$ be an arbitrary measure on $\R^{n}$ and 
$\Omega \subseteq \R^{n}$ $\mu$-measurable and let 
$f,f_k: \Omega \to  \overline{\R}$ be $\mu$-measurable and 
$\abs{f(x)} < \infty$ $\mu$-a.e..


\begin{dfn}[]
  We say that the sequence $(f_k)_{k \in \N}$ converges \textbf{in measure $\mu$} to $f$ (written $f_k \stackrel{\mu}{\to} f$ as $k \to  \infty$)\footnote{In contrast to the lecturer, I will be using $f_k \stackrel{\mu}{\to} f$ as shorthand for ``$f_k \stackrel{\mu}{\to}f$ as $k \to \infty$'' as it should be clear from context.},
  if $\forall \epsilon > 0$
  \begin{empheq}[box=\bluebase]{align*}
    \lim_{k \to \infty} \mu\left(
      \left\{x \in \Omega \big\vert \abs{f(x) - f_k(x)} > \epsilon\right\}
    \right)
    = 0
  \end{empheq}
\end{dfn}
\begin{rem}[]
If the sequence converges uniformly, then also $f_k \stackrel{\mu}{\to} f$.
However pointwise convergence is not enough to show $f_k \stackrel{\mu}{\to} f$,
as the same Example as in \ref{rem:egoroff-finite-necessary} proves otherwise.

Also, the sequence $f_k = \chi_{\{0\}} \stackrel{\Leb^{1}}{\to} \bm{0}$ shows that pointwise convergence does not necessarily follow from in-measure convergence.
\end{rem}


\begin{thm}[]
  Let $\mu(\Omega) < \infty$. If $f_k \to f$ $\mu$-a.e. then $f_k \stackrel{\mu}{\to} f$.
\end{thm}
\begin{proof}
  By Egoroff's theorem, $(f_k)_{k \in \N}$ converges $\mu$-almost uniformally on $\Omega$.

  This means that for all $\epsilon > 0$ there exists a set $A$ with $\mu(\Omega \setminus A) < \delta$ such that for all $\epsilon > 0: \exists N \in \N$ with
    \begin{align*}
      \sup_{x \in A} \abs{f_n(x) - f(x)} < \epsilon \quad \forall n \geq N
    \end{align*}
    for such $n \geq N$ we have
    \begin{align*}
      \{x \in \Omega \big\vert \abs{f_n(x) - f(x)} > \epsilon\} \subseteq \Omega \setminus A
    \end{align*}
    Taking $\mu(\cdot)$ on both sides gives the result.
\end{proof}

\begin{rem}
The converse of this theorem does not hold, so the statement $f_k \to f$ $\mu$-a.e. is stronger than $f_k \stackrel{\mu}{\to} f$.

To see this, take $\Omega = [0,1)$ with the measure $\Leb^{1}$.
And set $f_k = \chi_{A_k}$, for
\begin{align*}
  A_1 = [0,1), \quad 
  A_2 = [0,\tfrac{1}{2}), A_3 = [\tfrac{1}{2},1), \quad 
  A_4 = [0,\tfrac{1}{4}), A_5 = [\tfrac{1}{4},\tfrac{2}{4}), \ldots, A_7 = [\tfrac{3}{4},1), \quad 
  A_8 = [0, \tfrac{1}{8}), \ldots
\end{align*}
and more generally for $k \geq 1$, chose $n$ such that $2^{n} \leq k < 2^{n+1}$ and set
\begin{align*}
  A_k = \left[\frac{k - 2^{n}}{2^{n}}, \frac{k- 2^{n}+1}{2^{n}}\right)
\end{align*}
Therefore
\begin{align*}
  \mu\left(
  \{\abs{f_k(x)} > 0\}
\right) = \mu(A_k) = \frac{1}{2^{n}} < \frac{2}{k} \implies f_k \stackrel{\mu}{\to} \bm{0}
\end{align*}
But the nowhere does the sequence converge pointwise to $0$.
\end{rem}

\begin{thm}[]\label{thm:subsequence-muae}
  Let $f_k \stackrel{\mu}{\to} f$. Then there exists a subsequence $(f_{k_n})_{n \in \N}$ that converges to $f$ $\mu$-a.e..
\end{thm}
\begin{proof}
  Since $f_k \stackrel{\mu}{\to} f$, for all $n \in \N$ there exists a $k_n \in \N$ such that
  \begin{align*}
    \mu\left(
      \left\{x \in \Omega \big\vert \abs{f_k(x) - f(x)}> 2^{-n}\right\} 
    \right) < 2^{-n}
    , \quad
    \forall k \geq k_n 
  \end{align*}
  Define for $h \geq 1$
  \begin{align*}
    A_n := \left\{x \in \Omega \big\vert \abs{f_{k_n}(x) - f(x)} > 2^{-n}\right\}
    \quad \text{and} \quad E_h := \bigcup_{n \geq h}A_n
  \end{align*}
  by subadditivity, we have
  \begin{align*}
    \mu(E_h) \leq \sum_{n=h}^{\infty}\mu(A_n) < 2^{-h + 1}
  \end{align*}
  For any $x \in \Omega \setminus E_h$ we have that 
  \begin{align*}
    \forall n \geq h:
    x \notin A_n, \implies \forall n \geq h: \abs{f_{k_n}(x) -f(x)} \leq 2^{-n}
  \end{align*}
  This means that for all $h \in \N$ the sequence $(f_{k_n})_{n \in \N}$ converges to $f(x)$ on $\Omega \setminus E_h$.

  Because $\mu(E_1) \leq \mu(\Omega) < \infty$ and the sequence $(E_h)_{h \in \N}$ is decreasing, we have by continuity from above:
  \begin{align*}
    E := \bigcap_{h=1}^{\infty}E_h
    \implies
    \mu(E) = \lim_{h \to \infty}\mu(E_h) = 0
  \end{align*}
  Since $(f_{k_n})_{n \in \N}$ converges to $f$ on $\Omega \setminus E_h$, the sequence $(f_{k_n}|_{E})_{n \in \N}$ converges to $f$ on $\Omega \setminus E$.

\end{proof}
