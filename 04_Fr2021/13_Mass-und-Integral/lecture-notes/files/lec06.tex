\subsection{Lebesgue Measure}
The Lebesgue measure is the Carethéodory-Hahn extension of the pre-measure of ``volumes'', that assigns products of intervals to their products of lengths.

We want to give a definition of what these ``volumes'' and what their measure is going to be.
\begin{dfn}[]
  For $a = (a_{1}, \ldots, a_{d}), b = (b_{1}, \ldots, b_{d}) \in \R^{d}$ we define the $d$-dimensional \textbf{interval}
  \begin{align*}
    (a,b)
    =
    \left\{\begin{array}{ll}
        \prod_{i=1}^{d}(a_i,b_i) & \text{ if } a_i < b_i \quad \forall i\\
       \emptyset & \text{ otherwise}
    \end{array} \right.
  \end{align*}
  in an analogous way, we define the closed and half-open boxes $[a,b], [a,b)$ or $(a,b]$.
  Like on the real line, we also allow the open ends to be $\pm \infty$.

  To each $d$-dimensional interval $I$ (whether open, closed or half-open) to be
  \begin{align*}
    \vol(I) :=
    \left\{\begin{array}{ll}
    \prod_{i=1}^{d}(b_i - a_i) \in [0,+\infty] 
       &
    \text{ if } a_i < b_i, \quad \forall i
       \\
      0 & \text{ otherwise}
    \end{array} \right.
  \end{align*}
  An \textbf{elementary set} is the finite disjiont union of intervals and we define its volume to be
  \begin{align*}
    \vol\left(
      \bigsqcup_{k=1}^{d}I_k 
    \right)
      :=
      \sum_{k=1}^{d}\vol(I_k)
  \end{align*}
\end{dfn}
\begin{rem}[]
  We can check easily that the volume function is well defined.
  For example, the decomposition $[0,2] = [0,1) \sqcup [1,2] = [0,1) \sqcup [1,1.5) \sqcup [1.5,2]$ should all give the same volume.

  More generally, if $I= \bigsqcup_{k=1}^{n}I_k = \bigsqcup_{j=1}^m J_j$ where $I_k,J_j$ are Intervals, then
  \begin{align*}
    \sum_{k=1}^{n}\vol(I_k) = \sum_{j=1}^{m} \vol(J_j)
  \end{align*}
  We of course have to show that our attempt to use the Carathéodory-Hahn Extension of $\vol$ on the elementary sets is well defined. 
  But it should be easy to see how the class of elmentary sets forms an algebra and that the $\vol$ function is a pre-measure on it.
\end{rem}
In our example above, we used half-open intervals of length $2^{-k}$ to decompose an interval.
A direct generalisation is to introduce the dyadic cubes to obtain a basic decomposition of sets in $\R^{d}$.
For every $k \in \N$, let $\mathcal{D}_k$ the collection of half open cubes
\begin{align*}
  \mathcal{D}_k := \left\{
    \prod_{i=1}^{d}\big[\frac{a_i}{2^{k}}, \frac{a_{i+1}}{2^{k}}\big) \big\vert a_i \in \Z
  \right\}
\end{align*}  
In particular, $\mathcal{D}_0$ is the collection of hpyercubes of edge length $1$ and vertices in $\Z^d$.




