\subsection{Lebesgue Measure}
The Lebesgue measure is the Carethéodory-Hahn extension of the pre-measure that corresponds to the ``physical'' notion of what a volume of simple objects such as $n$-dimensional hypercubes like $[0,1]^{n}$ is.

We want to give a precise definition of what these ``simple objects'' are and define the pre-measure.
\begin{dfn}[]
  For $a = (a_{1}, \ldots, a_{d}), b = (b_{1}, \ldots, b_{d}) \in \R^{d}$ we define the $d$-dimensional \textbf{interval}
  \begin{align*}
    (a,b)
    :=
    \left\{\begin{array}{ll}
        \prod_{i=1}^{d}(a_i,b_i) & \text{ if } a_i < b_i \quad \forall i\\
       \emptyset & \text{ otherwise}
    \end{array} \right.
    \subseteq \R^{d}
  \end{align*}
  in an analogous way, we define the closed and half-open boxes $[a,b], [a,b)$ or $(a,b]$.
  Like on the real line, we also allow the open ends to be $\pm \infty$.

  To each $d$-dimensional interval $I$ (whether open, closed or half-open), we define it's \textbf{volume} to be
  \begin{align*}
    \vol(I) :=
    \left\{\begin{array}{ll}
    \prod_{i=1}^{d}(b_i - a_i) \in [0,+\infty] 
       &
    \text{ if } a_i < b_i, \quad \forall i
       \\
      0 & \text{ otherwise}
    \end{array} \right.
  \end{align*}
  An \textbf{elementary set} is the finite disjiont union of intervals and we define its volume to be 
  \begin{align*}
    \vol\left(
      \bigsqcup_{k=1}^{d}I_k 
    \right)
      :=
      \sum_{k=1}^{d}\vol(I_k) \in [0,\infty]
  \end{align*}
\end{dfn}


\begin{rem}[]
  We can check easily that the volume function is well defined.
  For example, the decomposition $[0,2] = [0,1) \sqcup [1,2] = [0,1) \sqcup [1,1.5) \sqcup [1.5,2]$ should all give the same volume.

  More generally, if $I= \bigsqcup_{k=1}^{n}I_k = \bigsqcup_{j=1}^m J_j$ where $I_k,J_j$ are Intervals, then
  \begin{align*}
    \sum_{k=1}^{n}\vol(I_k) = \sum_{j=1}^{m} \vol(J_j)
  \end{align*}
  We of course have to show that our attempt to use the Carathéodory-Hahn Extension of $\vol$ on the elementary sets is well defined. 
  But it should be easy to see how the class of elmentary sets forms an algebra and that the $\vol$ function is a pre-measure on it.
\end{rem}
In our example above, we used half-open intervals of length $1, 2^{-1}$ to decompose the interval $[0,2] \subseteq \R$.

A direct generalisation for this in higher dimensions is to introduce finer and finer hypercubes that cover $\R^{d}$.
For $k \in \N$ let $\mathcal{D}_k$ the collection of half open cubes
\begin{align*}
  \mathcal{D}_k := \left\{
  \prod_{i=1}^{d}\big[\frac{a_i}{2^{k}}, \frac{a_{i}+1}{2^{k}}\big) \big\vert a_i \in \Z
  \right\}
\end{align*}  
In particular, $\mathcal{D}_0$ is the collection of hpyercubes of edge length $1$ and vertices in $\Z^d$.

We call the cubes of the collection
\begin{align*}
  \{Q \big\vert Q \in \mathcal{D}_k, k = 0,1,2, \ldots\}
\end{align*}
the \textbf{dyadic cubes}.

\begin{rem}
The dyadic cubes have the following properties:
\begin{enumerate}
  \item For all $k \in \N$, $\R^{n} = \bigsqcup_{Q \in \mathcal{D}_k}Q$.
  \item If $Q \in \mathcal{D}_k$ and $P \in \mathcal{D}_l$, with $l \leq k$, then either $Q \subseteq P$ or $P \cap Q = \emptyset$.
  \item Every $Q \in \mathcal{D}_k$ has volume $\vol(Q) = 2^{-kn}$.
\end{enumerate}
\end{rem}


\begin{dfn}[]
  The \textbf{Lebesgue measure} $\Leb^{n}$ is the Carathéodory Hahn extension of the volume defined on the algebra of elementary sets\footnote{Because elementary sets are finite disjoint unions of intervals, we can replace $E_k$ with intervals $I_k$}, i.e.
\begin{align*}
  \Leb^{n}(A) := \inf \left\{
    \sum_{k=1}^{\infty} \vol(E_k) \big\vert A \subseteq \bigcup_{k=1}^{\infty} E_k, E_k \text{ is an elementary set}
  \right\}
\end{align*}
\end{dfn}

If we want to measure open subsets $U \subseteq \R^{n}$ with the Lebesgue-measure, we want to ensure that a countable covering of $U$ with disjoint elementary sets $E_k$ is possible, or else taking the infimum makes it so that $U$ is not $\Leb^{n}$-measurable.

\begin{lem}[] \label{lem:open-dyadic}
  Every open set in $\R^{n}$ can be written as a countable union of disjoint dyadic cubes.
\end{lem}
\begin{proof}
Let $U \subseteq \R^{n}$ be a non-empty open subset.

Let $\mathcal{S}_0$ to be the collection of all cubes in $\mathcal{D}_0$ that lie entirely in $U$.
Let $\mathcal{S}_1$ to be the collection of all cubes in $\mathcal{D}_1$ that lie entirely in $U$, but are not subcubes of $\mathcal{S}_0$, etc.
Let $\mathcal{S}_k$ be the collection of cubes in $\mathcal{D}_k$ which are not subcubes of any cubes in $\mathcal{S}_0, \ldots, \mathcal{S}_{k-1}$.
Set $\mathcal{S} := \bigcup_{k \in \N}\mathcal{S}_k$.

Because each $\mathcal{D}_k$ is countable, $\mathcal{S}$ is countable.
By construction, the cubes in $\mathcal{S}$ are also disjoint.

Since $U$ is open and the cubes become arbitrarily small, every $x \in U$ will be covered by some $Q \in \mathcal{S}$, so $U = \bigsqcup_{Q \in \mathcal{S}}Q$.

\end{proof}

Recall that the Borel $\sigma$-algebra $\mathcal{B}(X)$ is the $\sigma$-algebra generated by open subsets of $X$.

\begin{dfn}[]
  A measure $\mu$ on $\R^{n}$ is called \textbf{Borel} (or a Borel measure), if every Borel set is $\mu$-measurable.
\end{dfn}

\begin{rem}[] \label{rem:leb-is-borel}
  From Lemma \ref{lem:open-dyadic}, it follows that $\Leb^{n}$ is a Borel measure.
\end{rem}
