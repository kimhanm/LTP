\begin{thm}[]
For $s \geq 0$, $\mathcal{H}^{s}$ is a Borel regular measure on $\R^{n}$
\end{thm}
\begin{proof}
  Let $s \geq 0$. 
  We first prove that it is a meausre.
  Clearly, $\mathcal{H}^{s}(\emptyset) = 0$.
  Let $(A_k)_{k \in \N}$ and $A \subseteq \bigcup_{k=1}^{\infty}A_k$.
  Since $\mathcal{H}_{\delta}^{s}$ is $\sigma$-subadditive for all $\delta > 0$, we get
  \begin{align*}
    \mathcal{H}_{\delta}^{s}(A)\leq \sum_{k}\mathcal{H}_{\delta}^{s}(A_k) \leq \sum_{k}\mathcal{H}^{s}(A_k)
  \end{align*}
  by taking the limit (in the definition of $\mathcal{H}^{s}$) we get the $\sigma$-subadditivity of $\mathcal{H}^{s}$.

  To show that it is Borel, we just show that it is metric.
  Let $A,B \subseteq \R^{n}$ such that $\delta_0 := \text{dist}(A,B) > 0$.
  We then take a covering $A \cup B$ of balls of size smaller than $\delta := \tfrac{\delta_0}{4}$ and claim that we can partition the covering into two non-overlapping coverings of $A$ and $B$ each.


  Since $\mathcal{H}_{\delta}^{s}$ takes the infimum over all such coverings, suppose that $A \cup B = \bigcup_{k}B(x_k,r_k)$ with $r_k < \delta$.
  Then we set
  \begin{align*}
    \mathcal{A} = \{B(x_k,r_k) \big\vert B(x_k,r_k) \cap A \neq \emptyset\}
    \mathcal{B} = \{B(x_k,r_k) \big\vert B(x_k,r_k) \cap B \neq \emptyset\}
  \end{align*}
  And it becomes obvious that 
  these are non-overlapping coverings of $A$ and $B$ each (by using the triangle inequality).

  Therefore, we get
  \begin{align*}
    \mathcal{H}_{\delta}^{s}(A) + \mathcal{H}_{\delta}^{s}(B) \leq \sum_{k}r_k^{s}
  \end{align*}
  and taking the infimum of coverings of $A \cup B$, this means
  \begin{align*}
    \mathcal{H}_{\delta}^{s}(A \cup B) \geq \mathcal{H}_{\delta}^{s}(A) + \mathcal{H}_{\delta}^{s}(B) 
  \end{align*}
  which, when taking the limit $\delta \to 0$ just states $\mathcal{H}^{s}(A \cup B) \geq \mathcal{H}^{s}(A) + \mathcal{H}^{s}(B)$.
  By ($\sigma$)-subadditivity of $\mathcal{H}^{s}$, the reverse inequality holds and so $\mathcal{H}^{s}(A \cup B) = \mathcal{H}^{s}(A) + \mathcal{H}^{s}(B)$ shows that $\mathcal{H}^{s}$ is metric and thus also Borel.

  For Borel regularity, let $A \subseteq \R^{n}$ and suppose $\mathcal{H}^{s}(A) < \infty$ (Otherwise, just take $B = \R^{n}$).
  By monotonicity of $\mathcal{H}_{\delta}^{s}$, this also means that $\mathcal{H}_{\delta}^{s}(A) < \infty$ for all $\delta > 0$.

  For $\delta = \tfrac{1}{l}$ 



\end{proof}

