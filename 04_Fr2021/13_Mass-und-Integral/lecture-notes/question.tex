From section 2.3 (Lusin's and Egoroff's Theorem) through section 3.3 (Convergence results), we sometimes started with the assumption that $\mu$ was a Radon measure.
As noted in the script, that assumption is not always necessary and I would like to know where we can discard it.

While reading the proofs through those sections, I found that we only used it section 2.3 for Lusin's theorem and as mentioned before in the thread, we can weaken Egoroff's theorem a bit and remove the assumption.
Can someone tell me if there was another proof in these sections that relied on the assumption that $\mu$ is Radon? I couldn't find any.

Also, would you accept this alternative Characterisation of Egoroff's theorem? The definition of $\mu$-almost uniform convergence is used in Wikipedia, for example.

\begin{bluebox}[Definition]
  Let $\Omega \subseteq \R^{n}$.
  We say that a sequence of functions $(f_k: \Omega \to \overline{\R})_{k \in \N}$ \textbf{converges $\mu$-almost uniformly} on $\Omega$ to a function $f: \Omega \to \overline{\R}$,
  if for all $\delta > 0$ there exists a $\mu$-measurable subset $A \subseteq \Omega$ with $\mu(\Omega \setminus A) < \delta$ such that $(f_k)_{k \in \N}$ converges uniformly on $A$.
  That is:
  \begin{align*}
    \sup_{x \in A} \abs{f_k(x) - f(x)} \to  0 \text{ as } k \to  \infty
  \end{align*}
\end{bluebox}


\begin{orangebox}[Egoroff's Theorem (alternative)]


  Let $\Omega \subseteq \R^{n}$ be $\mu$-measurable with $\mu(\Omega) < \infty$,
  and let
  $f,(f_k)_{k \in \N}: \Omega \to \overline{\R}$ $\mu$-measurable.
  \begin{enumerate}
    \item 
      If $f_k(x) \to  f(x)$ as $k \to  \infty$ for $\mu$-a.e. $x \in \Omega$, and $f(x)$ finite $\mu$-a.e., 
      then $(f_k)_{k \in \N}$ converges $\mu$-almost uniformally to $f$ on $\Omega$.
    \item 
      If additionally, $\mu$ is a Radon measure we can also assume that the set on which $(f_k)_{k \in \N}$ converges uniformly is compact.
      That is:

      $\forall \delta > 0\ \exists K \subseteq \Omega$ compact with $\mu(\Omega \setminus K) < \delta$ and
      \begin{align*}
        \sup_{x \in C} \abs{f_k(x) - f(x)} \to 0 \text{ as } k \to  \infty
      \end{align*}
  \end{enumerate}
\end{orangebox}

