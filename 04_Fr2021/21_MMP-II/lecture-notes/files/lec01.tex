\section*{Introduction}
The main topics for this course will the the study of Algebraic methods and symmetry. 
We will see later that we can describe symmetries with groups and representations.

Here are some examples of symmetry in physics
\begin{itemize}
  \item In mechanics, we could have coupled harmonic oscillators displaying mirror symmetry.
  \item The kepler problem will use $O(3)$-symmetry
  \item In solid state physics, we can look at crystal grids displaying translational $\Z^3$-symmetries
  \item In Quantum mechanics from last semester, we saw the hydrogen atom.
  \item In nuclear and particle physics, representation theory is central and gives us Gauge theory.
  \item In relativity, we see Lorentz symmetry.
\end{itemize}
We want too find out how to describe and use symmetry.

\section{Groups}

\subsection{Basic definitions and examples}
For completeness, we quickly repeat some definitions used.

\begin{dfn}[Basics]
A group is a set $G$ with a binary operation 
\begin{align*}
  \cdot: G \times G \to G, \quad (g,h) \mapsto g \cdot h = gh
\end{align*}
that satisfies the following
\begin{enumerate}
  \item (Associativity): $(gh)k = g(hk)$
  \item (Neutral element) $\exists 1 \in G$ such that $1 \cdot g = g \cdot 1 = g$ for all $g \in G$
  \item (Inverse elements) $\forall g \in G, \exists g^{-1} \in G$ such that $gg^{-1} = g^{-1}g = 1$
\end{enumerate}
It follows from the group axioms that the unit $1$ and the inverse elements are uniquely determined.

We define the \textbf{order} of $G$ to be its cardinality $\abs{G}$, where we just write $\infty$ if $G$ is infinite.

A group $G$ is \textbf{abelian}, if for all $g,h \in G$ we have $gh = hg$. 
For abelian groups, the additive notation ($g + h = h + g$) is often used, where we write the unit as $0$ and inverses as $-g$.

A \textbf{subgroup} of a group $G$ is a non-empty subset $H \subseteq G$ that is closed under the group multiplication, that is
\begin{enumerate}
  \item $h_1,h_2 \in H \implies h_1h_2 \in H$
  \item $h \in H \implies h^{-1} \in H$
\end{enumerate}

The \textbf{direct product} is the cartesian product $G_1 \times G_2$ with component-wise multiplication (i.e $(g_1,h_1) \cdot (g_2,h_2) = (g_1g_2,h_1h_2)$ with $(1_G,1_H)$ as unit)
\end{dfn}

\begin{ex}[]

\begin{itemize}
  \item $\Z,\R,\C$ or vector spaces are abelian groups under addition.
  \item The cyclic group $\Z/n\Z$ with addition is abelian and has order $n$
  \item The symmetric group $S_n$, consisting of bijective maps $\pi: \{1,\ldots,n\} \to \{1,\ldots,n\}$ called \emph{permutations}. It has order $\abs{S_n} = n!$
  \item The Dihedral group $D_{n}$ for $n \geq 2$, consisting of rotations $R \in O(2)$ that keep a regular $n$-gon invariant. The elements are
    \begin{align*}
      1, R, R^2, \ldots, R^{n-1}, S, RS, \ldots, R^{n-1}S
    \end{align*}
    where $R$ is multiplication with $e^{\frac{2 \pi i}{n}}$ and $S$ is complex conjucation, if one corner of the $n$-gon is situated at $1 \in \C$. In particular, $\abs{D_n} = 2n$.

    To prove this, we need to show that each element listed above is different and that they are all possible symmetries of the $n$-gon.
    If we let $v_0, \ldots v_{n-1}$ be the corners of the $n$-gon, then we see that every element above maps the two corners $(v_0,v_1)$ to different coner-pairs.
    Now let $X \in D_n$, then set $j$ such that $X v_0 = v_j$. Then it must be that $Xv_1$ is either $v_{j+1}$ or $v_{j-1}$.
    In the first case, we see that $R^{-j}Xv_0 = v_0$ and $R^{-j}Xv_1 = v_1$, so $X = R^{j}$.
    In the second case we see that $SR^{-j}Xv_0 = v_0$ and $SR^{-j}Xv_1 = v_1$, so $X = R^{j}S$.
    Additionally we can show that $SRS = R^{-1}$.
  \item The general linear groups $\GL(n,K)$ for $n \in \N$ and $K$ a field consisting of invertible $n \times n$ matrices.
  \item Similarly the orthogonal groups $O(n) \subseteq \GL(n,\R)$ consisting of orthogoal $n \times n$ matrices (i.e. $A^{T}A = 1$). Or equivalently, matrices such that $(Ax,Ay) = (x,y)$ with the standard dot product $(-,-)$.
    More generally we can define the symmetric bilinear form on $\R^{p+q}$ defined as
    \begin{align*}
      (x,y)_{p,q} = \sum_{i=1}^{p}x_iy_i - \sum_{j=p+1}^{p+q}x_i y_i
    \end{align*}
    we can use this to define the group $O(p,q) \subseteq \GL(n,\R)$ that preserve this bilinear form.
    Notice that this bilinear form is no longer positive/definite. Ofspecial interest is $(-,-)_{1,3}$ called the \emph{Minkowski}-``metric'' with the \textbf{Lorentz group} $O(1,3)$
  \item The \textbf{unitary group} $U(n) \subseteq \GL(n,\C)$ consisting of complex $n \times n$ such that $A^{\dagger}A = 1$, or that preserve the complex inner product $(-,-)$
    where we use the physics convention in the inner product
    \begin{align*}
      (x,y) = \sum_{j=1}^{n} \overline{x_j}y_j
    \end{align*}
  \item The \textbf{symplectic group}: Consider the antisymmetric bilinear form on $\R^{2n}$ given by
    \begin{align*}
      \omega(X,Y) = \sum_{i=1}^{n} (X_{2i - 1}Y_{2i} - X_{2i} Y_{2i-1}
    \end{align*}
    the symplectic group is then the subset $\text{Sp}(2n) \subseteq \GL(n,\R)$ that preserve this bilinear form: $\omega(X,Y) = \omega(AX,AY)$.
    This group can useful to describe hamiltonian systems, where the first $n$ coordinates represent space, and the other $n$ coordinates correspond to impulse.
  \item For any subgroup $G \subseteq \GL(n,\R)$ or $G \subseteq \GL(n,\C)$ we define the subgroup
    \begin{align*}
      SG := \{A \in G \big\vert \det A = 1\} \subseteq G
    \end{align*}
    Some special cases include the \textbf{special linear group} $\text{SL}(n,K)$, the special orthogonal/unitary group $SO(n)$ and $SU(n)$.
\end{itemize}
\end{ex}

  

