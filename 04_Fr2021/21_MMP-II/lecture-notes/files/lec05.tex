\subsection{Schur's Lemma}
How can we use the result from the previous section in our everyday use?

Let $(\rho_1,V_1)$ and $(\rho_2,V_2)$ be complex finite dimensional representations of a group $G$ and we want to study
\begin{align*}
  \Hom_G(V_1,V_2) = \{\phi: V_1 \to V_2 \text{ linear}\big\vert \phi \rho_1(g) = \rho_2(g) \rho, \forall g \in G\}
\end{align*}
\begin{thm}[Schur's Lemma]
  Let $(\rho_1,V_1)$ and $(\rho_2,V_2)$ be finite dimensional irreducible representations of a group $G$. Then
  \begin{enumerate}
    \item $\phi \in \Hom_G(V_1,V_2) \implies \phi = 0$ or $\phi$ is an isomorphism.
    \item $\phi \in \Hom_G(V_1,V_1) \implies \phi = \lambda \id_V$ for some $\lambda \in \C$
  \end{enumerate}
\end{thm}
\begin{proof}
\begin{enumerate}
  \item We can show that $\Ker \phi \subseteq V_1$ and $\Image \phi \subseteq V_2$ are invariant subspaces since
    \begin{align*}
      v \in \Ker \phi \implies \phi(\rho_1(g)v) = \rho_2(g)\phi(v) = 0(v) = 0 \implies \rho_1(g)v \in \Ker \phi\\
      \rho_2(g) \phi(v')= \phi(\rho_1(g)v') \implies \rho_2(g) \phi(v') \in \Image \phi
    \end{align*}
    But since they are both invariant subspaces and $V_1,V_2$ are irreducible it must that either
    \begin{align*}
      \Ker \phi = \{0\} \quad \text{or} \quad \Ker \phi = V_1\\
      \Image \phi = \{0\} \quad \text{or} \quad \Image \phi = V_2
    \end{align*}
    so either $\phi = 0$ or bijective. 
    (Notice how we didn't use that $V_i$ is complex.
  \item Let $\lambda \in \C$ be an eigenvalue of $\phi$\footnote{
      By the fundamental theorem of algebra, such an eigenvalue always exists, as it is the root of a polynomial with complex coefficients)
    }
    Consider $\phi - \lambda \id_{V} \in \Hom_G(V,V)$. It cannot have full rank, so using (a), it follows that $\phi = \lambda \id_{V}$.
\end{enumerate}
\end{proof}

\begin{cor}[]
  Every irreducible finite dimensional complex representation of an abelian group is one-dimensional.
\end{cor}
\begin{proof}
  Let $(\rho,V)$ irreducible. Since $G$ is abelian, $\rho(g) \in \Hom_G(V,V)$ for all $g \in G$. By Schur's lemma, it follows that $\rho(g) = \lambda(g) \id_V$, so every subspace is invariant. 
  But $V$ is irreducible, so that is only possible, if $V$ itself is one-dimensional.
\end{proof}



\section{Representation theory of finite groups}
In the following, let $G$ be a finite group and let all representations be finite dimensional.

\subsection{Orthogonality relation for matrix elements}

\begin{thm}[]
  Let $\rho: G \to \GL(V)$, $\rho': G \to \GL(V')$ irreducible and unitary representations of $G$ and let
  \begin{align*}
    (\rho_{ij}(g))_{ij} \quad \text{and} \quad (\rho_{ij}'g)_{ij}
  \end{align*}
  be the matrices of $\rho(g)$ and $\rho'(g)$ with respect to any orthonormal basis of $V,V'$.

  \begin{enumerate}
    \item If $\rho,\rho'$ are inequivalent, then
      \begin{align*}
        \frac{1}{\abs{G}} \sum_{g \in G}\overline{\rho_{ij}(g)}\rho_{kl}'(g) = 0 \quad \forall i,j,k,l
      \end{align*}
    \item If they are equivalent, then
      \begin{align*}
        \frac{1}{\abs{G}} \sum_{g \in G}\overline{\rho_{ij}(g)}\rho_{kl}(g) = \frac{1}{\dim V} \delta_{ik}\delta_{jl}
      \end{align*}
  \end{enumerate}
\end{thm}
\begin{proof}
Let $\Phi: V \to  V'$ be some linear map. Define
\begin{align*}
  \Psi_G = \frac{1}{\abs{G}} \sum_{g \in G} \rho'(g^{-1})\Phi \rho(g)
\end{align*}
Then we get with a change of variables $(g \mapsto  gh)$
\begin{align*}
  \rho'(h)\Phi_G 
  &= 
  \frac{1}{\abs{G}} \sum_{g \in G}\rho'((gh^{-1})^{-1}) \Phi \rho(g)\\
  &=
  \frac{1}{\abs{G}} \sum_{g \in G}\rho'(g^{-1})\Phi \rho(gh)\\
  &=
  \Phi_G \rho(h)
\end{align*}
which shows that $\Phi_G \in \Hom_G(V,V')$. 
By Schur's lemma, if $\rho,\rho'$ are inequivalent, then $\Phi_G =0$ and so
\begin{align*}
  (\Phi_G)_{jl}
  &=
  \frac{1}{\abs{G}} \sum_{g \in G} \sum_{i,k}
  \overline{\rho_{ij}'(g)} \Phi_{ik} \rho_{kl}(g)\\
  &=
  \sum_{i,k} \Phi_{ik} \left(
    \frac{1}{\abs{G}} \sum_{g \in G} \overline{\rho_{ij}'(g)} \rho_{kl}(g)
  \right)
  = 0
\end{align*}
since this holds for all $\Phi$, (a) follows.

On the other hand, if $\rho \sim\rho'$ are equivalent, then by Schur's Lemma $\Phi_G = c \id_V$ for some $c \in \C$. Then
\begin{align*}
  c 
  &= 
  \frac{1}{\dim V} A \trace \Phi_G\\
  \trace \Phi_G
  &=
  \frac{1}{\abs{G}}\sum_{g \in G}\trace \left(
    \rho(g^{-1})\Phi \rho(g)
  \right)
  \\
  &= \trace \Phi \cdot \frac{\abs{G}}{\abs{G}} = \sum_{i}\Phi_{ii}
\end{align*}
By comparing the coefficients of the expressions
\begin{align*}
  \sum_{i,k} \Phi_{ik} \left(
    \frac{1}{\abs{G}} \sum_{g \in G}\overline{\rho_{ij}(g)}\rho_{kl}(g)
  \right)
  =
  \delta_{jl} \frac{1}{\dim V} \sum_{i} \Phi_{ii}
\end{align*}
it follows that
\begin{align*}
  \frac{1}{\abs{G}} \sum_{g \in G}\overline{\rho_{ij}(g)} \rho_{kl}(g) = \delta_{jl}\delta_{kl} \frac{1}{\dim V\dim V}
\end{align*}
\end{proof}


\subsection{Characters}
\begin{dfn}[]
  The \textbf{character} of a representation $\rho: G \to \GL(V)$ is a function $\chi: G \to \C$ given by
  \begin{align*}
    \chi_{\rho}(g) = \trace (\rho(g)) = \sum_{i =1}^{\dim V}\rho_{ii}(g)
  \end{align*}
\end{dfn}


\begin{thm}[]
\begin{enumerate}
  \item $\chi_{\rho}(hgh^{-1}) = \chi_{\rho}(g)$ for all $g,h \in G$.
  \item If $\rho, \rho'$ are equivalent, then $\chi_{\rho} = \chi_{\rho'}$
\end{enumerate}
\end{thm}
\begin{proof}
  This follows directly from the fact that the trace is cyclic: $\trace(ABC) = \trace(C) = \trace(BCA)$.
\end{proof}


\begin{dfn}[]
The conjugation classes of $G$ are the sets of the form $\{hgh^{-1}\big\vert h \in G\}$, 
or equivalently, the orbits of the group action of $G$ on itself by conjugation
\begin{align*}
  G \to \text{Bij}(G), \quad h \mapsto ( g \mapsto  hgh^{-1})
\end{align*}
or equivalently, they are the equivalence classes of the equivalence relation $\sim$ given by
\begin{align*}
  g \sim g' \iff \exists h \in G: g' = hgh^{-1}
\end{align*}
\end{dfn}





































