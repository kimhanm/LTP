\begin{lem}[]
  The following are easy to show
\begin{enumerate}
  \item $\chi_{\rho}(1) = \dim V = \dim \rho$
  \item $\chi_{\rho \oplus \rho'}(g) = \chi_{\rho}(g) + \chi_{\rho'}(g)$
  \item $\chi_{\rho}(g^{-1}) = \overline{\chi_{\rho}(g)}$
\end{enumerate}
\end{lem}
\begin{proof}
  The first item is trivial. For (b), we look at the the direct sum as a block matrix
  \begin{align*}
    (\rho \oplus \rho)(g) = \begin{pmatrix}
      \rho(g) & 0\\
      0 & \rho'(g)
    \end{pmatrix}
  \end{align*}
  For (c) we use the fact that in an orthonormal basis, we can write the inverse matrix as $\rho_{ij}(g^{-1}) = \overline{\rho_{ij}(g)}$.
\end{proof}


\begin{ex}[Character of the regular representation]
  The regular representation on $V =\C(G)$ with the basis of delta functions $\delta_h$ is given by
  \begin{align*}
    \rho_{\text{reg}}(g) \delta_h := \delta_{gh}
  \end{align*}
  so to compute the character, need to see when $\rho_{\text{reg}}(g)$ maps a basis vector $\delta_h$ to itself. That is only the case when $gh = h \iff g= 1$, so
  \begin{align*}
    \chi_{\text{reg}}(g) = \trace(\rho_{\text{reg}}(g)) = \left\{\begin{array}{cl}
      \abs{G} & g = 1\\
      0 & g \neq 1
    \end{array} \right.
  \end{align*}
\end{ex}

\subsection{Orthogonality relations for characters}
We define the inner product on $V = \C(G)$ given by
\begin{align*}
  \scal{f_1,f_2} = \frac{1}{\abs{G}} \sum_{g \in G}\overline{f_1(g)}f_2(g)
\end{align*}
\begin{thm}[]
Let $\rho,\rho'$ be irreducible representations of $G$ with characters $\chi_{\rho}, \chi_{\rho'}$. Then
\begin{align*}
  \scal{\chi_{\rho},\chi_{\rho'}} = \left\{\begin{array}{ll}
    1 & \text{ if $\rho, \rho'$ are equivalent}\\
    0 & \text{ otherwise}
  \end{array} \right.
\end{align*}
\end{thm}
\begin{proof}
If they are not equivalent, then
\begin{align*}
  \scal{\chi_{\rho},\chi_{\rho'}} = \sum_{i,j}
  \underbrace{\scal{\rho_{ii},\rho_{jj}'}}_{=0} = 0
\end{align*}
and if they are equivalent, then they have the same character, so by the orthogonality relation for matrix elements, we have
\begin{align*}
  \scal{\chi_{\rho},\chi_{\rho'}} = \sum_{i,j} \scal{\rho_{ii},\rho_{jj}'} = \sum_{i,j}\delta_{ij}\delta_{ij} \frac{1}{\dim V} = \sum_{i}\frac{1}{\dim V} = 1
\end{align*}
\end{proof}
\begin{cor}[]
If $\rho = \rho_1 \oplus \ldots \oplus \rho_n$ is a decomposition into irreducible representation, then the number of $\rho_j$ that are equivalent to the irreducible representation $\sigma$ are given by
\begin{align*}
  \# \text{ of $j$, such that $\rho_j = \sigma$} =: n_{\sigma} = \scal{\chi_{\rho},\chi_{\sigma}} = \scal{\chi_{\sigma},\chi_{\rho}}
\end{align*}
\end{cor}
\begin{proof}
This follows trivially from the fact that
\begin{align*}
  \chi_{\rho} = \chi_{\rho_1} + \ldots + \chi_{\rho_n}
\end{align*}
so by taking the scalar product with $\chi_{\sigma}$ on both sides,  we can use the previous theorem to immediately get the result.
\end{proof}

\begin{cor}[]
  A representation $\rho$ is irreducible if and only if $\scal{\chi_{\rho},\chi_{\rho}} = 1$
\end{cor}
\begin{proof}
This also follows trivially from the theorem. 
If we had a decomposition $\rho = \rho_1 \oplus \ldots \oplus \rho_n$, then we would have
\begin{align*}
  \scal{\chi_{\rho},\chi_{\rho}} = \sum_{i,j}\scal{\chi_{\rho_i},\chi_{\rho_j}} \geq n
\end{align*}
\end{proof}

Now if we are given a group $G$, how many irreducible representation are there?
A trick is to look at the regular representation.
\subsection{Decomposition of the regular representation}
The regular representation contains a lot of information about other irreducible representations.
\begin{thm}[]
  Every irreducible representation $\sigma$ of $G$ appreas in $\C(G)$ with multiplicity $\dim \sigma$.
\end{thm}
\begin{proof}
  From the first corollary before, we can use the fact that $\chi_{\text{reg}}$ is $\abs{G}$ on the identity, so
  \begin{align*}
    n_{\sigma} = \scal{\chi_{\sigma},\chi_{\text{reg}}} = \abs{G} \chi_{\sigma}(1) \frac{1}{\abs{G}} = \chi_{\sigma}(1) = \dim \sigma
  \end{align*}
\end{proof}
\begin{cor}[]
  A finite group only has finitely many equivalence classes of irreducible representations.
  Moreover, if $\rho_1, \ldots, \rho_k$ is a list of equivalent irreducible representations,then
  \begin{align*}
    \abs{G} = \sum_{j=k}(\dim \rho_j)^{2}
  \end{align*}
\end{cor}
\begin{proof}
  It follows from $\dim(\C(G))= \abs{G}$ because every time we have a $\dim \sigma$-dimensional representation it occurs exactly $n_{\sigma} = \dim \sigma$ times.
\end{proof}

\begin{cor}[Peter Weyl theorem for finite groups]
Let $\rho_{1}, \ldots, \rho_{k}$ be a list of the irreducible inequivalent representations of $G$.

Then the matrix elements $\rho_{\alpha,ij}$ for $1 \leq \alpha \leq k$ with respect to an orthogonoal basis from an orthogonal basis of $\C(G)$
\end{cor}
\begin{proof}
  By the orthogonality relation for matrix elements, we konw that the $\rho_{\alpha,ij}$ already form an orthogonal family. 
  This family is also a basis, since there are exactly $\sum_{\alpha}(\dim \rho_{\alpha})^{2}= \abs{G}= \dim \C(G)$ many members in the family.
\end{proof}

\begin{dfn}[]
A function $f: G \to  \C$ is called a \textbf{class function}, if
\begin{align*}
  f(ghg^{-1}) = f(h) \quad \forall h,g \in G
\end{align*}
\end{dfn}
For example the characers of representations are class functions.
Also, the class functions are a subspace of $\C(G)$ and as such, a Hilbert space.


\begin{cor}[]
The characters $\chi_{1}, \ldots, \chi_{k}$ form an ONB of the Hilbert space of class functions.
\end{cor}
\begin{proof}
We just saw that the $\chi_{1}, \ldots, \chi_{k}$ form an orthogonal family and now we have to show that they also span the space of class functions.

Let $f: G \to \C$ be a class function, then we can write
\begin{align*}
  f(g) = \frac{1}{\abs{G}} \sum_{h \in G}f(hgh^{-1})
\end{align*}
but the matrix elements also form an orthonormal basis of all functions,so also of the space of class functions, so
\begin{align*}
  f = \sum_{\alpha,i,j} \lambda_{\alpha,ij} \rho_{\alpha,ij}
\end{align*}
which, when inserted into the formulation above, we get
\begin{align*}
  f(g) 
  &= 
  \sum_{\alpha,i,j} \lambda_{\alpha,ij} \frac{1}{\abs{G}} \sum_{h \in G}\rho_{\alpha,ij} (hgh^{-1})\\
  &=
  \sum_{\alpha,i,j}\lambda_{\alpha,ij}
\frac{1}{\abs{G}}\sum_{h \in G} \sum_{k,l} \rho_{\alpha,i,k}(h) \rho_{\alpha,k,l}(g) \overline{\rho_{\alpha,j,l}(h)}
  \\
  &=
  \sum_{\alpha,i,j}\lambda_{\alpha,ij} \frac{1}{\dim \rho_{\alpha}} \sum_{k,l} \rho_{\alpha,k,l}(g)  \delta_{ij}\delta_{kl}
\end{align*}
but the sum $\sum_{k,l} \delta_{kl}$ is exactly the trace or character $\chi_{\alpha}(g)$, so
\begin{align*}
  f(g)
  &=
  \sum_{\alpha,i,j}\lambda_{\alpha,ij} \chi_{\alpha}(g)\delta_{ij} \frac{1}{\dim \rho_{\alpha}}
  \\
  &=
  \sum_{\alpha,i}\lambda_{\alpha,ii}\chi_{\alpha}(g) \frac{1}{\dim \rho_{\alpha}}
\end{align*}
which is a linear combination of the $\chi_{\alpha}(g)$.
\end{proof}


\begin{cor}[]
The number of equivalence classes of irreducible representations is equal to the number of conjugacy classes in $G$.
\end{cor}
\begin{proof}
The number of conjugacy classes is the dimension of the space of class functions, which as we just proved is the number of characters $\chi_{\alpha}$ which equals the number of equivalency classes of irreducible representations.
\end{proof}

\subsection{The character table}
The character table is a table listing all irreducible representations and their characters on the conjugacy classes.


\begin{table}[h]
\centering
\begin{tabular}{C| C C C}
  G & [1]&  c & \ldots\\\hline
  \chi_1 & \ddots & & \ddots\\
  \chi_j & & \chi_j(c) &\\
  \chi_k & \ddots & & \ddots
\end{tabular}
\caption{The character table has in the left column all irreducible representations and on the top row the conjugacy classes of the group. 
The entries is the character evaluated at a (representant) of the conjucacy class.}
\end{table}

\begin{ex}[]
  The group $G = S_3$ has $6$ elements, which can be written in terms of the cycle $\tau =(123)$ and the permutation $\sigma = (12)$
\begin{align*}
  G = \{1,\sigma,\tau,\tau \sigma \tau^{-1}, \tau^{2}st^{-2},sts^{-1}\}
\end{align*}
This group has three conjugacy classses:
\begin{align*}
  [1], [s] = \{s,tst^{-1},t^{2}st^{-2}\}, [t] = \{t,sts^{-1-1}\}
\end{align*}
the irreducible representations are
\begin{align*}
  \rho_{1} \text{ trivial}, \rho_2 \text{ on }\C^{2}, \quad \rho_{\epsilon} \text{ on } V = \C, \rho_{\epsilon}(\sigma) = \sgn(\sigma)
\end{align*}
the character table is then
\begin{table}[h]
\centering
\begin{tabular}{C|CCC}
  6 S_3 & [1] & 3[s] & 2[t]\\\hline
  \chi_1 &  1 & 1 & 1\\
  \chi_2 & 2 & a & b \\
  \chi_{\epsilon} & 1 & -1 & +1
\end{tabular}
\caption{The character table of $S_3$. The numbers show the number of elements in the conjugacy class. Non-trivial entries can be found using the orthogonalit condition}
\end{table}
To fill out the character table, we can use weighted orthogonality of the caracter. 
\begin{align*}
  \scal{\chi_1,\chi_2} = \frac{1}{\abs{G}} \sum_{g \in G}\overline{\chi_1(g)}\chi_2(g) \stackrel{!}{=} 0
\end{align*}
which means that we need
\begin{align*}
  \scal{\chi_1,\chi_2} = 0 \implies 2 + 3a + 2b = 0, \quad \text{and} \quad \scal{\chi_2,\chi_{\epsilon}} = 0 \implies 2 - 3 a + 2 b = 0
\end{align*}
which has the solution $a = 0, b = -1$.

Note that if a matrix has orthonormal rows, then its transpose also has orthonormal rows, which are exactly the columns of the matrix.
\begin{align*}
  \left(
    \sqrt{\frac{\abs{C_j}}{\abs{G}}}
    \chi_i(C_j)
  \right)_{ij} \in O(n)
\end{align*}
This shows that the orthonormality of the characters (with respect to the weighted inner product) corresponds to the orthonormality (with respect to the ``unweighted'' inner product) of the conjugacy classes, so
\begin{align*}
  \sum_{j=1}^{k} \overline{\chi_j(C_{\alpha})} \chi_j(C_{\beta}) = \frac{\abs{G}}{\abs{C_{\alpha}}} \delta_{\alpha,\beta}
\end{align*}
\end{ex}

The orthonormality gives us many tricks to compute the character table.
\begin{itemize}
  \item  $\abs{G} = \sum_{j=1}^{k} \left(\dim \rho_j\right)^{2}$
    so if the order of $G$ is rather small, then this gives us some restrictions on the dimensions of the representation.
  \item Orthogonality of rows and columns (with weights) can be used to find missing entries
  \item The existence of trivial representations (and if possible, the signum representation) lets us fill out the initial rows.
    Also note that the first column contains the dimension of the representations.
\end{itemize}

\subsection{The canonical decomposition of a representation}
Let $\rho_i: G \to \GL(V_i)$ be a list of inequivalent irreducible representations of $G$.

Now let $\rho$ be some other representation of $G$ on $V$.
If we decompose $V = U_{1}\oplus \ldots \oplus U_{n}$ into irreducible representations, then define $W_i$ as the direct sum of all $U_j$ that are equivalent to $\rho_i$ (which means that $\rho|U_j$ is equivalent to $\rho_i$).

Then we write $V = W_{1}\oplus + \ldots \oplus W_{k}$.
\begin{thm}[]
The decomposition $V = W_{1}\oplus \ldots \oplus W_{k}$ is invariant under the choice of the decomposition $V = U_{1}\oplus \ldots \oplus U_{n}$ and the projection
\begin{align*}
  p_i: V \to  W_i, \quad w_1 \oplus  \ldots \oplus w_k \mapsto  w_i
\end{align*}
is given by
\begin{align*}
  p_i(v) = \frac{\dim V_i}{\abs{G}} \sum_{g \in G}\overline{\chi_i(g)} \rho(g) v
\end{align*}
\end{thm}
\begin{proof}
Since $W_i = \Image p_i$ and the $p_i$are independent of the decomposition into the $U_i$, we only need to show the second statement.
We can show that $p_i \in \Hom_G(V,V)$ commutes with the group action, because
\begin{align*}
  p_i(\rho(h)v) 
  &= \frac{\dim V_i}{\abs{G}} \sum_{g \in G}\overline{\chi_i(g)}\rho(gh) v\\
  &=
  \frac{\dim V_i}{\abs{G}} \sum_{g \in G}\overline{\chi_i(g)} \rho(hg)v\\
  &= \rho(h) (p_i(v))
\end{align*}
For$v \in U_j$ we have $\rho(g)v \in U_j$, so $p_iv \in U_j$.
Now we can use Schur's Lemma for $p_i|_{U_j}: U_j \to  U_j$. 
The lemma then says that
\begin{align*}
  p_i|_{U_j} = c_{ij}1_{U_j}
  \implies
  \trace(p_i|_{U_j}) = c_{ij} \dim U_j
\end{align*}
and so
\begin{align*}
  \frac{\dim V_i}{\abs{G}} \sum_{g \in G} \overline{\chi_{i}(g)} \underbrace{\trace((\rho(g)|_{U_j})}_{= \chi_{\rho|_{U_j}}(g) =: \chi(g)} = \dim V_i \scal{\chi_i,\chi} = \left\{\begin{array}{ll}
    1 & \text{ if } \rho|_{U_j}, \rho_i \text{ are equivalent}\\
    0 &  \text{ otherwise}
  \end{array} \right.
\end{align*}
and so, we can find the coefficient
\begin{align*}
  c_{ij} = \left\{\begin{array}{ll}
    1 & \text{ if } \rho|_{U_j}, \rho_i \text{ are equivalent}\\
     0&  \text{ otherwise}
  \end{array} \right.
\end{align*}
so the mapping
\begin{align*}
  p_i(v) = \left\{\begin{array}{ll}
    v & \text{ if }v\in W_i\\
     0 & v \in W_j, j \neq i
  \end{array} \right.
\end{align*}
does indeed define the projector.

\end{proof}




