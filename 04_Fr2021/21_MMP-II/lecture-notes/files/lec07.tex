We call the decomposition $V = W_{1}\oplus \ldots \oplus W_{k}$ the \textbf{canonical decomposition} and the $W_i$ the \textbf{isotypical compents}.

So given a representation $\chi$, we can write $V$ as the direct sum
\begin{align*}
  V = \bigoplus_{i=1}^{k}V_i \otimes \C^{n_i}, \quad \text{for} \quad n_i = \scal{\chi_i,\chi}
\end{align*}

\subsection{The dihedral group $D_n$}

Recall that the dihedral group was generated by rotations and reflections:
\begin{align*}
  D_n = \{R^{a}S^{b} \big\vert a = 0, \ldots, n-1, b=0,1\}
\end{align*}
Every representation is then uniquely determined by $\rho(R) =: \overline{R}$ and $\rho(S)=: \overline{S}\in \GL(V)$.
These obviously cannot be chosen freely, but must follow the following relations
\begin{align*}
  R^{n} = S^{2} = 1, \quad SR = R^{-1}S \implies \overline{R}^{n} = \overline{S}^{2} = 1, \quad \overline{S}\overline{R} = \overline{R}^{-1}\overline{S}
\end{align*}
and it turns out that any other relation can be written using the above relations because
\begin{align*}
  R^{a}S^{b}R^{a'}S^{b'} = R^{a - a'-ba'}S^{b+b'}
\end{align*}
so every choice of $\overline{R},\overline{S}$ defines a well-defined representation of $D_n$.

Now if we want to find all irreducible representations of a group, we start with the one-dimensional ones.

For $V = \C$, the elements $\overline{R},\overline{S}$ can be viewed as elements in $\C \setminus 0$.
We immediately get that
\begin{align*}
  \overline{S}^{2} = 1 \implies \overline{S} = \pm 1 \implies \overline{S}\overline{R} = \overline{R}^{-1}\overline{S} \iff \overline{R}^{2} = 1
\end{align*}
which gives us two cases. For $n$ odd, we must have $\overline{R}^{n} = 1 \implies \overline{R} = +1$ and for $n$ even, both $\overline{R} = \pm 1$ are possible.

So for $n$ odd, tere are two irreducible $1$-dimensional representations$\rho_{\pm}$, and for $n$ even, there are four $1$-dimensional irreducible representations $\rho_{\pm,\pm}$.


Now to find two-dimensional representations $(V = \C^{2})$, let $v \in V$ be an eigenvalue of $\overline{R}$ to the eigenvalue $\epsilon$ ($\overline{R}v = \epsilon v$.
The relatios described above now give us
\begin{align*}
  \epsilon \overline{S}v = \overline{S}\overline{R} v = \overline{R}^{-1}\overline{S}v \iff \overline{R}\overline{S}v = \frac{1}{\epsilon} \overline{v}
\end{align*}
which means $\overline{S}v$ is an eigenvector of $\overline{R}$ to the eigenvalue $\tfrac{1}{\epsilon}$.

Now, $v, \overline{S}v$ would be linearly dependent, then $\text{span}(v)$ would be an invariant subspace, so they must be independent.
That means that they form a basis of $\C^{2}$ and we can write our matrices with respect to this basis:
\begin{align*}
  \overline{R} = \begin{pmatrix}
  \epsilon & 0\\
  0 & \tfrac{1}{\epsilon}
  \end{pmatrix},
  \quad \text{and} \quad \overline{S} = \begin{pmatrix}
  0 & 1\\
  1 & 0
  \end{pmatrix}
\end{align*}
Now, the condition $\overline{R}^{n} = 1$ can only be satisfied if
\begin{align*}
  \epsilon = e^{\frac{2 \pi i}{n}j}, \quad \text{for} \quad j \in \Z
\end{align*}
which gives us representations $\rho_j$ for $j = 1,2, \ldots, \lfloor \tfrac{n-1}{2}\rfloor$ and we claim that they are all possible irreducible representations up to equivalence.

We prove this by looking at the characters
\begin{align*}
  \chi_j(g) &= \trace(\rho_j(g)) \implies \chi_j(R^{a)} = \epsilon^{a} + \epsilon^{-a} = e^{\frac{2 \pi i}{n}j} + e^{-\frac{2 \pi i}{n}j}= 2 \cos \left(
    \frac{2 \pi j}{n}a
  \right)\\
    \chi_j(R^{s}S) &= 0
\end{align*}
and then use the orthogonality conditions to find the other ones:
\begin{align*}
  \scal{\chi_i,\chi_j} 
  &= \frac{1}{\abs{G}} \sum_{a=0}^{n-1}\overline{
    \left(
      \epsilon_i^{a}+\epsilon_i^{-a}
  \right)}
    \left(
    \epsilon_j^{a} + \epsilon_j^{-a}
  \right)
  \\
  &= \frac{1}{2n} \sum_{a=0}^{n-}1
  \left(
    \left(
      \epsilon_i \epsilon_j
    \right)^{a}
    + \left(
      \epsilon_i \epsilon_j^{-1}
    \right)^{a}
    + \left(
      \epsilon_i^{-1} \epsilon_j
    \right)^{a}
    + \left(
      \epsilon_i^{-1}\epsilon_j^{-1}
    \right)^{a}
 \right)
\end{align*}
so for $\xi$ the $n$-th root of unity, we get
\begin{align*}
  \sum_{a=0}^{n-1} \xi^{a} = \left\{\begin{array}{ll}
    \frac{1 - \xi^{n}}{1 - \chi} = 0 & \xi \neq 0\\
     n &  \xi = 1
  \end{array} \right.
\end{align*}
which means that since the first term $\left(\epsilon_i \epsilon_j\right)^{a}$ is not $1$, it drops out of the sum, aswell as the last part 
$\left(\epsilon_i^{-1}\epsilon_j^{-1}\right)^{a}$.
So the only remaining terms are
\begin{align*}
  \scal{\chi_i,\chi_j} 
  &=
  \frac{1}{2n} \left(
    n \delta_{ij} + n \delta_{ij}
  \right) = \delta_ij
\end{align*}
which shows that the representations $\rho_i,\rho_j$ are irreducible and not equivalent for $i \neq j$.

Now we need to shwow that they form all possible irreducible representations up to equivalency.
To do this, we use the formula 
\begin{align*}
  \abs{G} = \sum_{j=1}^{k} \left(
    \dim \rho_j
  \right)^{2}
\end{align*}
to check that there aren't any more. 
Indeed, for $n$ even we have
\begin{align*}
  1^{2} + 1^{2} + 1^{2} + 1^{2} + 2^{2} \frac{n-2}{2} = 2n = \abs{G}
\end{align*}
and for $n$ odd:
\begin{align*}
  1^{2} + 1^{2} + 2^{2} \frac{n-1}{2} = 2n = \abs{G}
\end{align*}
which checks out.
  

\subsection{Compact groups}
We can generalize some results for finite groups to when the group is compact.
The representation theory for finite groups was made possible by taking averages of the group by dividing by $\frac{1}{\abs{G}}$ when summing over the entire group.

It turns out what we can generalize the integral over $\R$ to an integral over compact (or even locally compact groups)
\begin{align*}
  \int_G f(g) d \mu \quad \text{where} \quad d \mu \text{ is the \textbf{Haar measure}}
\end{align*}
which is normed $\int_G 1 d \mu = 1$ and has the invariance property:
\begin{align*}
  \int_G f(gh) d \mu = \int_G f(g) d \mu, \quad \forall h \in G
\end{align*}
the construction of the Haar measure takes some measure theory and is out of the scope of the lecture.
Applying the results of the previous chapters for compact groups, we the following results:

\textbf{Orthogonality} of matrix elements and characters given by the inner product on $L^{2}(G)$
\begin{align*}
  \scal{f_1,f_2} := \int_G \overline{f_1(g)}f_2(g) d \mu
\end{align*}
One property that we lose is that we can't use $\C(G)$ for the regular representation, but have to use $L^{2}(G)$.
The decomposition of the regular distribution gives us the (actual) Peter-Weyl theorem.

One example of an infinite, but compact topological group is $SO(n)$ where we use the term $\frac{1}{\text{vol}(SO(n))} \int_{SO(n)} d \mu$ with the standard integral over $SO(n)\subset \R^{n^{2}}$ all the time.






