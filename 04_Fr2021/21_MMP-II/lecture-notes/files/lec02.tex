\begin{dfn}[]
  A \textbf{group action} of a group $G$ on a set $M$, is a map
  \begin{align*}
    G \times M \to  M, \quad (g,x) \mapsto  gx
  \end{align*}
  that is compatible with the group operation, i.e. $g(hx) = (gh)x$

  Equivalently, a group action is a group homomorphism $G \to \text{Bij}(M)$.
\end{dfn}
Every group $G$ can act on $G$ itself in three ways:
\begin{enumerate}
  \item Left-action: $(g,x) \mapsto gx$
  \item right-action: $(g,x) \mapsto xg^{1}$
  \item conjugation: $(g,x) \mapsto gxg^{-1}$
\end{enumerate}

\begin{dfn}[]
A \textbf{group homomorphism} between two groups $G,S$ is a map $\phi: G \to S$ that is compatible with the group operations.
\begin{align*}
  \phi(gh) = \phi(g) \phi(h) \forall g,h \in G
\end{align*}
If $\phi$ is bijective, we call $\phi$ an isomorphism.
\end{dfn}
We can show easily that if $\phi: G \to S$ is a group homomorhpism, then
\begin{enumerate}
  \item $\phi(1) = 1$, $\phi(g)^{-1} = \phi(g^{-1})$
  \item $\Ker(\phi) \subseteq G$ and $\Image(\phi) \subseteq S$ are subgroups
\item $\phi$ is injective if and only if $\Ker(\phi) = \{1\}$
  \item The composition of group homomorphisms is again a group homomorphism.
\end{enumerate}

\begin{dfn}[]
  Let $H \subseteq G$ be a subgroup. We define the (Left-)\textbf{cosets} $G/H$ the set of equivalence classes with respect to the equivalence relation $\sim$ on $G$:
  \begin{align*}
    g_1 \sim g_2 \iff \exists h \in H: g_1h = g_2
  \end{align*}
  In general, $G/H$ does not have a group structure that is compatible with multiplication in $G$.
  That is possible if $ghg^{-1} \in H$ for all $g \in G, h \in H$.

  If that is the case, we call $H$ a \textbf{normal divisor} of $G$. And the mapping 
  \begin{align*}
    G/H \times G/H \to  G/H, \quad (g_1H,g_2H) \mapsto (g_1g_2)H
  \end{align*}
  is a well defined group operation. We then call $G/H$ the \textbf{factor group} of $G$ modulo $H$.
  We can see this because
  \begin{align*}
    g_1h_1g_2h_2 = g_1g_2(\underbrace{g_2^{-1}h_1g_2}_{\in H})h_2 = (g_1g_2)hh_2
  \end{align*}
\end{dfn}
\begin{thm}[]
Let $G$ be a group and $H \subseteq G$ a subgroup. Then the following are equivalent
\begin{itemize}
  \item $H$ is a normal divisor of $G$
  \item There exists a group $S$ and a group homomorphism $\phi: G \to S$ such that $H = \Ker \phi$
  \item $(xH)(yH) = (xy)H$ for all $x,y \in G$
  \item $G/H$ has a group structure such that the projection $\pi: G \to  G/H$, $\pi(g) = gH$ is a group homomorphism.
\end{itemize}
\end{thm}
\begin{ex}[]
\begin{itemize}
  \item $n\Z \subseteq \Z$ is a normal divisor.
  \item Quotient vector spaces with addition.
  \item The kernel of a group homomorphism and quotient is isomorphic to its image.
  \item $\{+1,-1\} \subseteq \SL(n,\C)$ is a normal divisor and the factor group is called the \textbf{projective special linear group} $PSL(2,\C) = \faktor{SL(2,\C)}{\{\pm 1\}}$ which is isomorphic to the group of Moebious transformations of the Riemann-sphere.
    \begin{align*}
      z \mapsto \frac{az + b}{cz + d}
    \end{align*}
    where for a matrix $A = \begin{pmatrix}
    a & b\\
    c & d
    \end{pmatrix}
    $, both $A$ an $-A$ determine the same Moebius transformation.
\end{itemize}
\end{ex}

\begin{dfn}[]
  Let $G,H$ be groups and $\rho: G \to  \Aut(H)$ a group homomorphism. If we write $\rho_g = \rho(g) \in \Aut(H)$, then $G \times H$ with multiplication
  \begin{align*}
    (g_1,h_1) \cdot (g_2,h_2) = (g_1,g_2, h_1 \rho_{g_1}(h_2))
  \end{align*}
  is called the \textbf{semi-direct} product $G \ltimes_{\rho} H$. Note that the unit is simply $(1,1)$ and the inverse is
  \begin{align*}
    (g,h)^{-1} = (g^{-1},\rho_{g^{-1}}(h^{-1}))
  \end{align*}
\end{dfn}
\begin{ex}[]
The affine transformation
  \begin{align*}
    x \mapsto Ax + b, \quad \text{for} \quad A \in O(3)
  \end{align*}
  defines a semidirect product $IO(3) := O(3) \ltimes \R^3$ with multiplication
  \begin{align*}
    (A_1,b_1) \cdot (A_2,b_2) = (A_1A_2, b_1 + A_1 b_2)
  \end{align*}
  where $\rho_A(b) = Ab$
  A special case is the \textbf{Poincare} group $IO(1,3) = O(1,3) \ltimes \R^4$
\end{ex}

\subsection{Lie Groups}
\begin{dfn}[]
  Named after Norwegian mathematician Sophus Lie, a \textbf{Lie Group} is a smooth manifold such that the multiplication and inverse maps are smooth. 
  Note that the multiplication must be smooth with respect to the product manifold $G \times G$, not $G$ itself.
\end{dfn}
Examples are $\GL(n,\R), \GL(n,\C), O(p,q), (\R^n,+), U(p,q)$.

We can easily see that for example, $\GL(n,\R) \subseteq \R^{n^2}$ is open (the determinant is polynomial and smooth and $\R \setminus \{0\}$ is open)
The multiplication is polynomial in the coefficients and the inverse $A^{-1} = \frac{A^{\#}}{\det A}$ is a rational function without poles. So they are both smooth.

For $O(n)$, note that $O(n)$ is the zero locus of the smooth operation $A \mapsto A^{T}A - 1$. We know from Analysis II that such sets define a smooth manifold.



\begin{dfn}[]
  A \textbf{path} in a metric space is a continuous map $\gamma: [0,1] \to  X$.
  We say that $\gamma$ \emph{connects} $\gamma(0)$ and $\gamma(1)$.
  If for all points $x,y \in G$ there exists such a path connecting $x$ and $y$, we say that $X$ is \textbf{path connected}.
  The (path)\textbf{connected components} of $X$ are the equivalence classes under the equivalence relation
  \begin{align*}
    x \sim y \iff \exists \text{ path } \gamma: \gamma(0) = x, \gamma(1) = y
  \end{align*}
\end{dfn}

\begin{thm}[]
Let $G$ be a group and $G_0 \subseteq G$ be path connected component containing the identity. 
Then $G_0$ is a normal divisor of $G$ and $\faktor{G}{G_0}$ is isomorhpic to the group of path connected components.
\end{thm}
Proof: Exercise.



































