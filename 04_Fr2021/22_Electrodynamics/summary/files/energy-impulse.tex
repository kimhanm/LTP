\subsection{Energy Impulse Tensor}
Given charges $q_n$ at positions $\vec{r}_n(t)$ with energies $E_n(t)$ and momentum $p_n(t)$ the charge and current density is given by
\begin{align*}
  \rho(\vec{x},t) = \sum_{n} q_n \delta(\vec{x} - \vec{r}_n(t)), \quad \vec{j}(\vec{x},t) = \sum_{n}q_n \frac{d \vec{r}_n(t)}{d t} \delta(\vec{x} - \vec{r}_n(t))
\end{align*}
So the energy and energy current densities are
\begin{align*}
  \sum_{n}E_n(t) \delta(\vec{x} - \vec{r}_n), \quad \sum_{n}E_n(t) \frac{d \vec{r}_n(t)}{d t} \delta(\vec{x} - \vec{r}_n(t))
\end{align*}
aswell as the impulse and impulse current densities
\begin{align*}
  \sum_{n} p_n^{i}(t) \delta(\vec{x} - \vec{r}_n(t), \quad \sum_{n}p_n^{i}(t) \frac{d \vec{r}_n(t)}{d t} \delta(\vec{x} - \vec{r}_n(t))
\end{align*}
respectively.


Using tensor notation, we combine charge and current densities to a four vector
\begin{empheq}[box=\bluebase]{align*}
  j^{\mu} := (c \rho, \vec{j})
\end{empheq}
define the four-vector impulse $p^{\mu} = (E,\vec{p})$ and combine energy and momentum into the \textbf{Energy Momentum Tensor}
\begin{align*}
  T^{\mu \nu} := \sum_{n} p_n^{\mu}(t) p_n^{\mu} \frac{d r_n^{\nu}}{d t} \delta(\vec{x} - \vec{r}_n(t))
\end{align*}
which corresponds to a $4 \times 4$ matrix of the Layout
\begin{align*}
  T^{\mu \nu} = \begin{pmatrix}
  \text{Energy density} & \text{Energy current density}\\
  \text{Impulse density} & \text{Impulse current density}
  \end{pmatrix}
\end{align*}
where the imuplse current density corresponds to a $3 \times 3$ submatrix.

If we set $c = 1$, then energy and impulse are the same so we see that $T^{\mu \nu}$ is symmetric, giving us the form
\begin{align*}
  T^{\mu \nu} = \sum_{n} \frac{p_n^{\mu} p_n^{\nu}}{E_n}\delta(\vec{x} - \vec{r}_n(t))
\end{align*}


\subsection{Energy-impulse tensor in electromagnetic field}

If we have charges $q_n$ in an electromagnetic field, then the energy and impulse are \emph{not} conserved but instead go into the electromagnetic field.


The energy impulse tensor in an electromagnetic field is 
\begin{empheq}[box=\bluebase]{align*}
  T^{\mu \nu}_{\text{em}} := \tensor{F}{^\mu_\rho} F^{\rho \nu} + \frac{1}{4} g^{\mu \nu} F_{\rho \sigma}F^{\rho \sigma}
\end{empheq}
which is symmetric and gauge invariant. Its components are
\begin{align*}
  w := T^{00}_{\text{em}} = \frac{\vec{E}^{2} + \vec{B}^{2}}{2} \quad \text{and} \quad T^{0i}_{\text{em}} = T^{i0}_{\text{em}} = (\vec{E} \times \vec{B})_I
\end{align*}
where $w$ is the energy density of the electromagnetic field. 
Defining the sum
\begin{align*}
  \Theta^{\mu \nu} := T^{\mu \nu} + T^{\mu \nu}_{\text{em}}
\end{align*}
we see that energy and momentum is conserved
\begin{align*}
  \del_{\nu} T^{\mu \nu}_{\text{em}} = - F^{\mu \nu}j_{\nu}
  \implies 
  \del_{\nu} \Theta^{\mu \nu} = 0
\end{align*}
Taking the sum of the four-momentum of the charges
\begin{align*}
  P^{\mu}_{\text{charges}} = \sum_{n} p_n^{\mu}
\end{align*}
and the momentum of the electromangentic field
\begin{align*}
  P^{\mu}_{\text{em}} = \int d^{3} \vec{x} T^{\mu 0}
\end{align*}
the total momentum is conserved.
\begin{align*}
  P^{\mu} := \int d^{3} \vec{x} \Theta^{\mu 0} = P^{\mu}_{\text{charges}} + P^{\mu}_{\text{em}} = \text{const}
\end{align*}

We also define the \textbf{Poynting vector} as
\begin{empheq}[box=\bluebase]{align*}
  S^{i} = T^{0i}_{\text{em}} = (\vec{E} \times \vec{B})^{i}
\end{empheq}
so the equation
\begin{align*}
  \del_0 T^{00} + \del_i T^{i0}  = - F^{0i}j_i
\end{align*}
turns into the well-known formula
\begin{align*}
  \frac{\del w}{\del t} + \vec{\nabla} \cdot \vec{S} = - \vec{E} \cdot \vec{j}
\end{align*}

