\section{Magnetostatics}
In Magnetostatics, we consider systems where the current is steady. This means in particular that
\begin{align*}
  \vec{J} = \const, \implies \rho = \const, \quad \vec{E} = \const, \quad \vec{B} = \const
\end{align*}

Under the \textbf{Coulomb-Eichung}
\begin{table}[h]
\centering
\begin{tabular}{L|L}
  \text{Electrostatics} & \text{Magnetostatics}\\
  \Phi(\vec{x}) = \frac{1}{4 \pi \epsilon_0} \int d^{3}\vec{y} \frac{\rho(\vec{y})}{\abs{\vec{x}-\vec{y}}}
  &
  A(\vec{x}) = \frac{1}{4 \pi \epsilon_0c^{2}} \int d^{3} \vec{y} \frac{\vec{J}(\vec{y})}{\abs{\vec{x}- \vec{y}}}
\\
  \vec{E} = - \nabla \Phi
  &
  \vec{B} = \vec{\nabla} \times \vec{A}
\\
  \vec{\nabla} \cdot \vec{E} = \frac{\rho}{\epsilon_0}
  &
  \vec{\nabla} \times \vec{B} = \frac{\vec{J}}{\epsilon_0c^{2}}
\\
  \vec{\nabla} \times \vec{E} = 0
  &
  \vec{\nabla} \cdot \vec{B} = 0
\\
  \int_{\del V} d \vec{S} \cdot \vec{E} = \frac{Q_{\text{inside}}}{\epsilon_0}
  &
  \oint_{\del S}\vec{B}\cdot d \vec{\ell} = \frac{I_{\text{inside}}}{\epsilon_0 c^{2}}
\end{tabular}
\caption{Analogies between Electrostatics and Magnetostatics}
\end{table}
