\section{Lagrange Formalism of Electrodynamics}
A charged particle in an electromagnetic field experiences the Lorentz force
\begin{align*}
  \vec{F} = \frac{d \vec{p}}{d t} = q(\vec{E} + \vec{v} \times \vec{B})
\end{align*}
In special relativity, the momentum is $m \vec{v} \gamma = \frac{m \vec{v}}{\sqrt{1 - \vec{v}^{2}}}$.

And the Lagrangian is
\begin{align*}
  \mathcal{L}(\vec{x},\vec{v},t) = - \sqrt{1 - \vec{v}^{2}} - e(\Phi - \vec{v} \cdot \vec{A})
\end{align*}

Recall that the Euler-Lagrange equations were
\begin{align*}
  \frac{d }{d t} \frac{\del L}{\del v_i} = \frac{\del L}{\del x_i}
\end{align*}

\begin{align*}
  \mathcal{L}(A^{\mu}, \del^{\nu}A^{\mu}) = - \frac{1}{4} F_{\mu \nu} F^{\mu \nu} - j_{\mu}A^{\mu}
\end{align*}

