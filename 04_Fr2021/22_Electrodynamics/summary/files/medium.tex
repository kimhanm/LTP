\section{Electrodynamics in a medium}

We define ``taking the average'' of a function $F(\vec{x})$ as its convolution
\begin{empheq}[box=\bluebase]{align*}
  \scal{F(\vec{x})} := \int d^{3} \vec{y}f(\vec{y}) F(\vec{x} - \vec{y}) = (F \ast f)(\vec{x})
\end{empheq}
for some smooth non-negative \emph{weighing kernel} $f$.


\subsection{Charge and current density}

By defining the \textbf{effective charge density} $\scal{\rho_{\text{eff}}}$
\begin{empheq}[box=\bluebase]{align*}
  \scal{\rho_{\text{eff}}} &:= 
  \scal{
    \sum_{i \in \text{free}}q_i \delta(\vec{x} - \vec{x}_i)
  }
  +
  \scal{
    \sum_{n \in \text{molec}}
    q_{n,\text{tot}}
    \delta(\vec{x} - \vec{x}_n)
  }
  \\
  &= \scal{\rho_{\text{free}}} + \scal{\rho_{\text{atomic}}}
\end{empheq}
aswell as the \textbf{polarised charge density}
\begin{empheq}[box=\bluebase]{align*}
  \scal{\rho_{\text{pol}}}
  = - \vec{\nabla} \cdot \vec{P}
  \quad \text{for} \quad 
  \vec{P} = \scal{\sum_{n \in \text{molec}} \vec{p}_n \delta(\vec{x} - \vec{x}_n)}
\end{empheq}
where $\vec{P}$ is called the \textbf{polarisation} of the medium, this lets us decompose the total charge density into the effective and polarised terms:

\begin{empheq}[box=\bluebase]{align*}
  \scal{\rho} 
  &\approx \rho_{\text{eff}} + \rho_{\text{pol}} \\
  &=
  \scal{\rho_{\text{free}}} + \scal{\rho_{\text{atomic}}}- \vec{\nabla} \cdot \vec{P}
\end{empheq}
where the approximation is up to first order with respect to $\vec{x}_{n,k}$.


For the current density, we define the \textbf{effective current density}
\begin{empheq}[box=\bluebase]{align*}
  \vec{j}_{\text{eff}} 
  &:=
  \scal{\sum_{i \in \text{free}}
  q_i \vec{v}_i \delta(\vec{x} - \vec{x}_i)
}
  + \scal{
    \sum_{n \in \text{molec}}
    q_{n,\text{tot}}\vec{v}_n \delta(\vec{x} - \vec{x}_n)
  }\\
  &= \scal{\vec{j}_{\text{free}}}
  +
  \scal{\vec{j}_{\text{atomic}}}
\end{empheq}
, the \textbf{polarised current density}
\begin{empheq}[box=\bluebase]{align*}
  \scal{\vec{j}_{\text{pol}}} := \frac{\del \vec{P}}{\del t} \quad \text{for} \quad \vec{P} = \scal{\sum_{n \in \text{molec}}
    \vec{p}_n \delta(\vec{x} - \vec{x}_n)
  }
\end{empheq}
aswell as the \textbf{magnetized current density}
\begin{empheq}[box=\bluebase]{align*}
  \scal{\vec{j}_{\text{mag}}} := \vec{\nabla} \times \vec{M} \quad \text{for} \quad \vec{M} := \scal{\sum_{n \in \text{molec}}\vec{m}_n \delta(\vec{x} - \vec{x}_n)}
\end{empheq}
, where $\vec{m}_n$ is the \textbf{magnetic moment} of the molecule given by
\begin{empheq}[box=\bluebase]{align*}
  \vec{m}_n := \sum_{k=1}^{r_n} \frac{q_k}{2}(\vec{x}_k \times \vec{v}_k)
\end{empheq}

In the end, we are left with the nice compact formula for the total current density averaged over atomic distances:
\begin{empheq}[box=\bluebase]{align*}
  \scal{\vec{j}} 
  &\approx
  \scal{\vec{j}_{\text{eff}}}
  + \scal{\vec{j}_{\text{pol}}}
  + \scal{\vec{j}_{\text{mag}}}
  \\
  &=
  \scal{\vec{j}_{\text{free}}} + \scal{\vec{j}_{\text{atomic}}}
  + \frac{\del \vec{P}}{\del t}
  + \vec{\nabla}\times \vec{M}
\end{empheq}
with the approximation being good when the $\vec{v}_{n,k}$ and $\vec{x}_{n,k}$ are small.


\subsection{Maxwell Equations}
We define the \textbf{dielectric field} $\vec{D}$ and the b
\begin{empheq}[box=\bluebase]{align*}
  \vec{D} &:= \epsilon_0 \scal{\vec{E}} + \vec{P}\\
  \vec{H} &:= \scal{\vec{B}} - \frac{\vec{M}}{c^{2} \epsilon_0}
\end{empheq}
which give us the inhomogenuous maxwell equations
\begin{align*}
  \vec{\nabla} \cdot \vec{D} &= \scal{\rho_{\text{eff}}}\\
  \vec{\nabla} \times \vec{H} = \frac{\vec{J}}{c^{2} \epsilon_0} + \frac{1}{\epsilon_0 c^{2}} \frac{\del \vec{D}}{\del t}
\end{align*}
and the homogenous maxwell equations stay the same.


\subsection{Dielelectric materials}
We consider materials, where $\vec{M} = 0$.

Then
\begin{align*} 
  \vec{D} = \epsilon_0 \vec{E} + \vec{P} = \epsilon \vec{E}
\end{align*}

\begin{ex}[]
  Consider two metal sheets with charges $\pm Q$ distanced $d$ apart with a dielectric medium inbetween with $\epsilon = (1 + \chi) > \epsilon_0$.


  Using $\vec{\nabla} \cdot \vec{D} = \rho$, it follows from Gauss' Law for a Box around the condensator:
  \begin{align*}
    \int_{\text{Box}} \vec{\nabla} \cdot D =  \int_{\del \text{Box}} dA \vec{D} \cdot \vec{n} \implies D = \frac{Q}{2A}
  \end{align*}
  The energy decreases.
\end{ex}

\subsection{Continuity conditions}

We have two adjacents fields
\begin{align*}
  (\epsilon,\vec{E},\vec{D}), \quad (\epsilon', \vec{E}',\vec{D}')
\end{align*}
Then $E_{\parallel},H_{\parallel},D_{\bot},B_{\bot}$ are continuos, but $D_{\parallel},B_{\parallel},E_{\bot},H_{\bot}$ are not.


\subsection{Waves in dielectric medium}


