\subsection{Rayleigh scattering}
Instead of looking at charges which are free to move and accelerate, we look at charges which are bound into some system.

For example, we look at an electron bound by an atom and when the electron is hit by an electromagnetic wave, then it wants to accelerate. 
But the atom (which is basically a harmonic oscillator) will react to this acceleration and ``pull it back in''.

Since the electric field at $x=0$ is of the form
\begin{align*}
  \vec{E}(t)|_{x = 0} = E e^{i \omega t} \hat{e}
\end{align*}
the equation for a charge bound on a harmonic oscillator is
\begin{align*}
  \frac{q E(t)}{m} = \ddot{x} + \gamma \dot{x} + \omega_0^{2} x
\end{align*}
which has the general solution
\begin{align*}
  \vec{x}(t)  = \frac{\frac{q \vec{E}(t)}{m}}{\omega_0^{2} - \omega^{2} + i \gamma \omega}
\end{align*}
so the acceleration is
\begin{align*}
  \ddot{\vec{x}}(t) = \frac{q \vec{E}(t)}{m} \frac{- \omega^{2}}{\omega_0^{2} - \omega^{2} + i \gamma \omega}
\end{align*}
Comparing it with the Thomson scattering, 
we see that the acceleration is now frequency dependent and we can express it in terms of the Thomson acceleration.
\begin{align*}
  \ddot{\vec{x}}_{\text{Thomson}}(t) = \frac{q \vec{E}(t)}{m}\cos(\omega t), \implies \ddot{\vec{x}}_{\text{Rayleigh}}(t) = \ddot{\vec{x}}_{\text{Thomson}}(t) \frac{- \omega^{2}}{\omega_0^{2} - \omega^{2} + i \gamma \omega}
\end{align*}
After calculating the cross section, we find
\begin{align*}
  \sigma_{\text{Thomson}} = \frac{8 \pi}{3} r_q^{2} \implies
  \sigma_{\text{Rayleigh}} = \sigma_{\text{Thomson}} \frac{\omega^{4}}{(\omega^{2}- \omega_0^{2})^{2} + \gamma^{2} \omega^{2}}
\end{align*}
If $\omega \ll \omega_0$ we get a $\omega^{4}$ dependence
\begin{align*}
  \sigma_{\text{Rayleigh}} \cong \sigma_{\text{Thomson}} \frac{\omega^{4}}{\omega_0^{4}}
\end{align*}
What this is tellig us is that higher frequences are scatter more than lower frequencies.

This explains why the sky is blue during the day and becomes yellow/red at sunrise/sunset.


\section{Macroscopic Maxwell equations}
We would like to be able to describe the effects of matter on the electromagnetic fields.

The problem is that we are often only able to look at macroscoptic measurements of currents and the electromagnetic field and not the ``theoretical'' effects of matter in a small scale.

So for example, it could very well be that our description of how electrons move through a wire are completeley wrong in a microscopic scale.
But by relying on the fact that on a larger scale, many of the  the small effects cancel out, we are left with a theory that can describe the average movement of the charges and the electromagnetic field.


We define what it means to take the average over a field $\vec{F}(\vec{x},t)$ a distance via
\begin{empheq}[box=\bluebase]{align*}
  \scal{F(\vec{x},t)} \int d^{3}\vec{y} f(\vec{y}) F(\vec{x}- \vec{y},t)
\end{empheq}
where $f(\vec{y})$ is some \emph{weighing factor}.

