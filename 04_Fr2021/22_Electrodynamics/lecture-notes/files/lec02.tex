
We also saw that the Maxwell equations are consistent in different references frames under translationan and rotation, but not with Gallileian transformation.

Last lecture, we defined rotations as matrices $R$ such that $R^{T}R = 1$.
Physically, we additionally require that rotations should have determinant $1$, because the identity matrix should be included.
Under rotation, we also know that the inner product is invariant, i.e.
\begin{align*}
  (\vec{A},\vec{B}) \to (\vec{A}',\vec{B}') = \vec{A}R^{T}R \vec{B} = (\vec{A},\vec{B})
\end{align*}
Also, the outer product transforms in the following manner
\begin{align*}
  \vec{A}\times \vec{B} \to  \vec{A}' \times \vec{B}' = R \vec{A} \times \vec{R} 
\end{align*}
Notice how the maxwell equations where we see the inner product is \emph{invariant} and the maxwell equations where we see the outer product, they are \emph{covariant} under rotations. Because the left hand side and the right hand side also transform in the same way, the maxwell equations stay consistent.

%Starting from Coulomb's law
%\begin{align*}
  %\vec{F}_{yx} = \frac{q}{4\pi \epsilon_0} \frac{q_1}{\abs{\vec{x}- \vec{y}}^{3}}(\vec{x} - \vec{y})
%\end{align*}
%We can derive Gauss's law. By defining the scalar potential $\Phi(\vec{x})$ given by
%\begin{align*}
  %\Phi(\vec{x}) = \frac{1}{4 \pi \epsilon_0}\int d^{3}\vec{y} \frac{\rho(\vec{y}}{\abs{\vec{x}- \vec{y}}}
%\end{align*}
\subsection{Electrostatics}
We will study the Maxwell equations where we have no electrical currents with steady densities of charges ($\phi(\vec{x},t) = \rho(\vec{x})$

In particular, we will have no magnetic field ($\vec{B} = 0$) and we want to ask if we can calculate the electricfield if we know the charge density.

The maxwell equations will then be
\begin{align*}
  \vec{\nabla}\cdot \vec{E} = \frac{\rho}{\epsilon_0} \quad \text{and} \quad \vec{\nabla}\times \vec{E} = 0
\end{align*}
A naive attempt would be to just use Gauss's law to solve
\begin{align*}
  \int_V d^{3} \vec{x} \vec{\nabla} \cdot \vec{E} = \int_V d^{3} \vec{x} \frac{\rho(\vec{x})}{\epsilon_0}\\
  \implies \int_{\del V} d \vec{S} \cdot \vec{E} = \int_{V} d^{3} \vec{x} \frac{\rho(x)}{\epsilon_0} = \frac{Q}{\epsilon_0}
\end{align*}
This is saying that the total flux through the surface bounding the volume quals the total charge inside the volume.

Also notice how the equation is linear for $\vec{E}$, so if we can decompose $\rho = \rho_1 + \rho_2$ we can solve them independently and put the solutions $\vec{E} = \vec{E}_1 + \vec{E}_2$ together.

This makes solving complex systems much easier as just solving for isolated charges lets us calculate a variety of problems.

Consider a single charge $q$ at position $\vec{y}$ and draw a sphere of radius $R$ around it. The spherical symmetry of the system means that it is sufficient to solve the electric field as a function of the radius $\vec{E}(\vec{x}) = \vec{E}(R)$.
With $\vec{r}$ the radial vector perpendicular to the sphere, we see that $d \vec{S}= d S \vec{r}$. 
The electric field ist just
\begin{align*}
  \vec{E} = E \vec{r} = E \frac{\vec{x} - \vec{y}}{\abs{\vec{x} - \vec{x} - \vec{y}}} = E \vec{r} \frac{1}{R}
\end{align*}
Because the surface area of the sphere is $4 \pi R^{2}$ we get that
\begin{align*}
  \int_{\del V}\vec{E} d \vec{S} = q \implies E(R) = \frac{1}{4 \pi \epsilon_0} q \frac{1}{R^{2}} \hat{r}
\end{align*}
where $\hat{r} = \frac{\vec{r}}{\abs{\vec{r}}}$ is the unit vector in radial direction.

Using the linearity described earlier, we can easily generalize this for any finite amount of isolated charges $q_i$.
And we get a solution $\vec{E}(\vec{x})$ that solves the Maxwell equations for $\vec{x} \neq \vec{y}_i$.

This covers the discrete charge distributions, but not continuous ones. Then we have
\begin{align*}
  q dV = dq
\end{align*}
instead of taking a sum, we take the integral
\begin{align*}
  \vec{E}(\vec{x}) = \frac{1}{4 \pi \epsilon_0} \int d^{3}\vec{y} \frac{\rho(\vec{y})}{\abs{\vec{x} - \vec{y}}^{3}} (\vec{x} - \vec{y}))
\end{align*}

Notice that the integrand diverges as $\vec{y}$ approaches $\vec{x}$, so we need to assume that we can separate the small distance effects from the large distance effects.

Let's consider the special case where the charge distribution is not quite discrete but is localized via a guassian distribution
\begin{align*}
  \rho(\vec{y}) =N \exp\left(
    - a (\vec{y} - \vec{y}_0)^2
  \right)
\end{align*}
We expect that from far away, the charge distribution can be approximated to a discrete charge $q = \int d^{3}\vec{y} \rho(\vec{y})$
because main contributors to the charge is confined in a very small space around $\vec{y}_0$.

So when we solve for the electric field, we assume that how the charge $q$ is distributed in a small region does not matter much. So
\begin{align*}
  <F> = \int d^{3} \vec{y} F(\vec{y}) \rho(\vec{y}) \simeq \int d^{3}\vec{y} F(\vec{y}_0) \rho(\vec{y})
\end{align*}
Theoretical physicist Paul Dirac came up with the \textbf{Dirac $\delta$-function}
\begin{align*}
  \delta(y - y_0) = \left\{\begin{array}{ll}
    \infty & y = y_0\\
    0 & y \neq y_0 
  \end{array} \right.
\end{align*}
Which has the property that when integrating it with a function
\begin{align*}
  \int_{-\infty}^{\infty}dy \delta(y - y_0) F(y) = F(y_0)
\end{align*}
we obtain the function value at the point $y_0$.
We can therefore use the dirac delta function to describe the charge density as 
\begin{align*}
  \rho(y) = q \delta(y - y_0)
\end{align*}

We now can verify that the maxwell equations are fulfilled.
\begin{align*}
  \vec{E} = - \vec{\nabla} \Phi\\
  \implies (\vec{\nabla}\times \vec{E})_i = - (\vec{\nabla} \times \vec{\nabla} \Phi)_i = - \epsilon_{ijk} \frac{\del }{\del x_j} \frac{\del \Phi}{\del x_k}\\
  &= - \epsilon_{ijk} \frac{\del^2 \Phi}{\del x_j \del x_k} = 0
\end{align*}
where $\epsilon_{ijk}$ is the total antisymmetric tensor.

The summation is zero as we are multiplying a symmetric tensor $\frac{\del^2}{\del x_j \del x_k}$ with an antisymmetric one $\epsilon_{ijk}$.

