\section{Magnetostatics}

In magentostatics, we have that the charges do not move and that the $\vec{E}$ and $\vec{B}$ field do not change over time
\begin{align*}
  \frac{\del \rho}{\del t} = \frac{\del \vec{J}}{\del t} = \frac{\del E}{\del t} = \frac{\del \vec{B}}{\del t} = 0
\end{align*}
so the Maxwell equations
\begin{align*}
  \vec{\nabla}\cdot \vec{E} = \frac{\rho}{\epsilon_0}, \quad \vec{\nabla} \times \vec{E} = - \frac{\del \vec{B}}{\del t}, \quad \vec{\nabla} \cdot \vec{B} = 0, \quad \vec{\nabla} \times \vec{B} = \frac{\vec{J}}{\epsilon_0c^{2}} + \frac{1}{c^{2}} \frac{\del \vec{E}}{\del t}
\end{align*}
turn into the decoupled form
\begin{align*}
  \vec{\nabla} \cdot \vec{B} = 0, \quad \vec{\nabla} \times \vec{B} = \frac{\vec{J}}{\epsilon_0 c^{2}}, \quad \vec{\nabla} \cdot \vec{E} = \frac{\rho}{\epsilon_0}, \quad \vec{\nabla}\times \vec{E} = 0
\end{align*}
where the $\vec{E}$ and $\vec{B}$ field do not have an interdependence.


The main goals of this chapter will be the following
\begin{itemize}
  \item What exactly is the current density $\vec{J}$?
  \item Look at the integral form of the above equations.
  \item Introduce the \textbf{vector potential} $\vec{A}$, leading to \textbf{gauge symmetry}
  \item Study the universality of magnetostatics and its applications
  \item Understand the duality of $\vec{E}$ and $\vec{B}$. In particular, we will see that magntic fields turn into electric fields when changing the reference frame and vice versa.
\end{itemize}



To give a definition of a current density $\vec{J}$, we consider a stream of charges moving in some direction $\vec{v}$.
We then place a surface $d \vec{S}$ perpendicular to the direction and measure how much charge moves trough the surface

\begin{empheq}[box=\bluebase]{align*}
  \frac{\dd Q}{\dd t} =: \vec{J} \cdot d\vec{S}
\end{empheq}

Now assume we have a volume $V$ and we we measure how many charges are escaping the volume.
Well, that measurement should match what we measure for the current trough its surface $\del V$, which is exactly $\vec{J} \cdot d\vec{S}$.
Using the divergence theorem, we get
\begin{align*}
  \int_V d^{3}x \frac{\del \rho}{\del t } = \int_{\del V} \vec{J} \cdot d\vec{S} = - \int_{V} dV \vec{\nabla} \cdot \vec{J}
\end{align*}
therefore, we get the \textbf{continuity equation}
\begin{empheq}[box=\bluebase]{align*}
  \frac{\del \rho}{\del t } + \vec{\nabla} \cdot \vec{J} = 0
\end{empheq}
This gives us the relation 
\begin{empheq}[box=\bluebase]{align*}
  \vec{J} = \rho \vec{v}
\end{empheq}
that suffices as an alternate definition of the charge density

The integral form of the maxwell equation
\begin{align*}
  c^{2} \vec{\nabla} \times \vec{B} = \frac{\vec{J}}{\epsilon_0}
\end{align*}
is the following
\begin{align*}
  c^{2} \int_{\del V}d \vec{S} \cdot \left(
    \vec{\nabla} \times \vec{B}
  \right)
  =
  \int_{\del} ...
\end{align*}

Since the magnetic field has no divergence
\begin{align*}
 \vec{\nabla} \cdot \vec{B} = 0
\end{align*}
there exists a \textbf{vector potential} $\vec{A}$ such that
\begin{align*}
  \vec{B} = \vec{\nabla} \times \vec{A}
\end{align*}
which satisfies
\begin{align*}
  \vec{\nabla} \cdot \left(
    \vec{\nabla} \times \vec{A}
  \right)
  =
  \frac{\del }{\del x_i} (\vec{\nabla} \times \vec{A})_i
  =
  \epsilon_{ijk} \frac{\del^{2}}{\del_i \del_j} A_k
  = 0
\end{align*}
where $\epsilon_{ijk}$ is the total antisymmetric \textbf{Levi-Civita} tensor and we have used the fact the the contraction of an antisymmetric with a symmetric tensor is always zero.

Using hte Maxwell equations, we also end up with
\begin{align*}
  \vec{\nabla} \times \vec{B} = \frac{\vec{J}}{\epsilon_0 c^{2}} \implies \vec{\nabla}(\vec{\nabla} \cdot \vec{A}) = \frac{\vec{J}}{\epsilon_0 c^{2}}
\end{align*}

Let's see how the vector potential transforms under
\begin{align*}
  \vec{A}(\vec{x}) \to \vec{A}'(\vec{x}') := \vec{A}(\vec{x}) + \vec{\nabla} f(\vec{x})
\end{align*}
Since $\vec{F} = q \vec{v} \times \vec{B}$ stays invariant, we have that
\begin{align*}
  \vec{B}' := \vec{\nabla} \times \vec{A}' = \vec{\nabla} \times \vec{A} + \underbrace{\vec{\nabla} \times \vec{\nabla}f}_{=0} = \vec{B}
\end{align*}
so the vector potential is only unique up to the gradient of a potential.
This symmetry is called \textbf{gauge symmetry} and is important in the full standard model of particle physics.

This means that if we don't like how our $\vec{A}$ looks we can chose $f$ such that $\vec{A}'$ is nice.

We can for example require $\vec{\nabla} \cdot \vec{A}' = 0$, we we can chose $f$ such that
\begin{align*}
  \vec{\nabla}^{2} f = - \vec{\nabla}\cdot \vec{A} = \rho(x) 
\end{align*}
which is just a poission equation and is solvable.
This then gives us the solution for $\vec{A}$:
\begin{align*}
  \vec{A}(\vec{x}) = \frac{1}{4 \pi \epsilon_0 c^{2}} \int d^{3}\vec{y} \frac{\vec{J}(\vec{y})}{\abs{\vec{x} - \vec{y}}}
\end{align*}

The vector potential can be used to describe a magnetic dipole.

\begin{align*}
  \vec{A}(\vec{r}) = \frac{1}{4 \pi \epsilon_0 c^{2}}  
  \left[
    \frac{1}{\abs{\vec{r}}}
    \int d^{3}\vec{x}\vec{J}(\vec{x}) +
    \frac{1}{\abs{\vec{r}}^{3}} \int d^{3} \vec{x} \vec{J}(\vec{x}) (\vec{x} \cdot \vec{r}) + \ldots
  \right]
\end{align*}
but in magnetostatics, there is no total current and so the first integral vanishes.
This is also called \textbf{dipole aproximation}, so we then have
\begin{align*}
  \vec{A}(\vec{r}) = \frac{1}{4 \pi \epsilon_0 c^{2}}
  \left(
  \int d^{3}\vec{x} \frac{1}{2} \vec{x} \times \vec{J}(\vec{x})
  \right)
  \times \frac{\vec{r}}{\abs{\vec{r}}^{3}}
\end{align*}
If we define the \textbf{magnetic moment} $\vec{\mu}$ to be
\begin{empheq}[box=\bluebase]{align*}
  \vec{\mu} := \frac{1}{2} \int d^{3}\vec{x} \left(
    \vec{x} \times \vec{J}(\vec{x})
  \right)
\end{empheq}
then we can simply find out the vector potential as
\begin{empheq}[box=\bluebase]{align*}
  \vec{A}(\vec{r}) = \frac{1}{4 \pi \epsilon_0}\vec{\mu} \times \frac{\vec{r}}{r^{3}}
\end{empheq}
With some calculation, one can also derive the formula
\begin{align*}
  \vec{B} = (\vec{\mu} \cdot \vec{\nabla}) \vec{\nabla}\frac{1}{r}
\end{align*}
and up to first order, the force acting on an object $\Omega$ with current density $\vec{J}$ can be described as
\begin{empheq}[box=\bluebase]{align*}
  \vec{F} = \vec{\nabla}_x (\vec{\mu} \cdot \vec{B})|_{x = 0}
\end{empheq}

\subsection{Relativity of $\vec{E}$ and $\vec{B}$ fields}
Consider a wire in which electrons move with velocity $\vec{v}$ and assume that the wire is electrically neutral ($\rho_{-} = -\rho_{+}$).

Then we put an observer with charge $q$ at some distance $r$ from the wire that also moves with the same velocity $\vec{v}$ along the wire.

From the outside, we see current generating a magnetic field
and an charge $q$ moving inside the magnetic field.
So we see the Lorentz force
\begin{align*}
  F = q v B = \frac{1}{2 \pi \epsilon_0} \frac{qv}{c^{2}} \frac{I}{r} = \frac{qS}{2 \pi \epsilon_0} \frac{\rho_{-}}{r}\frac{v^{2}}{c^{2}}
\end{align*}
where $\rho_{-}$ is the charge density of the moving electrons and $S$ is the cross section of the wire.

Now we change the reference frame to that of the observing particle $q$.
It doesn't see any moving charges so no magnetic fields. 
But then it would no longer experience the Lorentz force.
How can this happen?

What happens is that the positive charges, which in the static reference frame cancel out the electrons now are moving in the opposite direction.

So because of the Lorentz contraction from special relativity, the charge density of the electrons $\rho_{-}'$ and the positive ones $\rho_{+}'$ no longer cancel each other out.

This means that the particle $q$ sees that the wire is charged.
Let's calculate what charge the particle $q$ expects.

Given a section of the wire of volume $\Delta V$ we have that in the resting reference frame:
\begin{align*}
 Nq = Q = \rho_{\text{rest}} \Delta V
\end{align*}
but in the moving reference frame, we have
\begin{align*}
  Q' = \rho_{\text{moving}} \Delta V_{\text{moving}}
\end{align*}
Because total charge is invariant under Lorentz transformation, we have $Q = Q'$, so by the Lorentz contraction factor $\gamma$:
\begin{align*}
  \rho_{\text{moving}} = \rho_{\text{rest}} \frac{\Delta V_{\text{rest}}}{\Delta V_{\text{moving}}} = \gamma \rho_{\text{rest}}
\end{align*}
So because the electrons were resting and are moving in the frame of $q$ and the positive charges were moving and are now resting, we have
\begin{align*}
  \rho' = \rho_{+}' + \rho_{-}' = \gamma \rho_{+} + \frac{1}{\gamma} \rho_{-}
  = \frac{\rho_{-}}{\sqrt{1 - \frac{v^{2}}{c^{2}}}} \frac{v^{2}}{c^{2}}
\end{align*}
So the new force that acts on the charge $q$ is now an electric force, instead of a magnetic one. 
And so we can explain the same physical effect with an electric field or a magnetic field.

This shows how the electric and magnetic field are really dual to each other under Lorentz transformation.
