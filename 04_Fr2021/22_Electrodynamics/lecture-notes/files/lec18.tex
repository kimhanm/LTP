

\subsection{The Energy-Momentum Tensor}
Consider a collection of particles $\{n\}$ at positions $\vec{r}_n(t)$ and momentum
\begin{align*}
  p_n^{\mu} = m_n \gamma \frac{d x^{\mu}}{d t} = (m_n \gamma c, m_n \gamma \vec{v}_n)
\end{align*}
then the energy density would be
\begin{align*}
  \sum_{n}p_n^{0} \delta(\vec{x} - \vec{r}_n(t))
\end{align*}
and the energy current density is
\begin{align*}
  \sum_{n}p_n^{0} \frac{d \vec{r}_n}{d t}\delta(\vec{x}- \vec{r}_n(t))
\end{align*}
and just like before, we can combine the energy and energy current density into a single four object
\begin{align*}
  T^{\mu \nu}
  :=
  \sum_{n}p_n^{\mu} \frac{d r_n^{\nu}}{d t} \delta(\vec{x} - \vec{r}_n(t))
\end{align*}
called the \textbf{energy-momentum tensor}.
Or, by integrating through the time variable, we get the form
\begin{empheq}[box=\bluebase]{align*}
  T^{\mu \nu} = \sum_{n} \int d \tau p_n^{\mu} \frac{d x_n^{\nu}}{d \tau} \delta(x^{\rho} - r_n^{\rho}(\tau)
\end{empheq}
And it transforms as a rank $2$ contravariant tensor, i.e.
\begin{align*}
  T^{\mu \nu} \mapsto \tilde{T}^{\mu \nu} = \tensor{\Lambda}{^{\mu}_{\rho}}\tensor{\Lambda}{^{\nu}_{\sigma}} T^{\rho \sigma}
\end{align*}
One can also show that it is symmetric ($T^{\mu \nu} = T^{\nu \mu}$), because
\begin{align*}
  p_n^{\nu} = m_n \frac{d r_n^{\nu}}{d \tau} = m_n \gamma \frac{d r_n^{\nu}}{d t}
\end{align*}
from which we find the symmetric form
\begin{align*}
  T^{\mu \nu} = \sum_{n} \frac{p_n^{\mu} p_n^{\nu}}{E_n} \delta(\vec{x} - \vec{r}_n(t))
\end{align*}


In classical mechanics, we know that unless an external force $\vec{F}_{\text{ext}}$ is exerted on the system, the total energy should be invariant.

So taking the covariant derivative of the energy-momentum tensor we get with the chain rule
\begin{align*}
  \del_i T^{\mu i}
  &=
  \sum_{n} p_n^{\mu} \frac{d r_n^{i}}{d t}\frac{\del }{\del x^{i}} \delta(\vec{x} - \vec{r}_n)\\
  &=
  \sum_{n} p_n^{\mu} \frac{d r_n^{i}}{d t}(-\frac{\del }{\del r_n^{i}}) \delta(\vec{x} - \vec{r}_n)\\
  &=
  -\sum_{n} p_n^{\mu} \frac{\del }{\del t} \delta(\vec{x} - \vec{r}_n)\\
  &=
  -\frac{\del }{\del t}\sum_{n} p_n^{\mu} \delta(\vec{x} - \vec{r}_n)
  +
  \sum_{n} \frac{\del p_n^{\mu}}{\del t} \delta(\vec{x} - \vec{r}_n)
  \\
  &=
  - \frac{\del }{\del t}T^{\mu 0} + \sum_{n} \frac{\del p_n^{\mu}}{\del t} \delta(\vec{x} - \vec{r}_n)\\
  &= - \frac{\del }{\del t}T^{\mu 0} + \sum_{n} \frac{\del p_n^{\mu}}{\del t} \delta(\vec{x} - \vec{r}_n)
\end{align*}
which gives us
\begin{align*}
  \delta_{\nu} T^{\mu \nu} = \sum_{n} \frac{\del p_n^{\mu}}{\del t} \delta(\vec{x} - \vec{r}_n(t))
\end{align*}
%let's first understand what $T^{\mu 0}$ is. We will get
%\begin{align*}
  %T^{\mu 0} = \sum_{n} p_n^{\mu} \frac{\del t}{\del t} \delta(\vec{x} - \vec{r}_n(t)) = \sum_{n}p_n^{\mu} \delta(\vec{x} - \vec{r}_n(t))
%\end{align*} 

Let's define the right and side in the as
\begin{empheq}[box=\bluebase]{align*}
  G^{\mu} := \sum_{n} \frac{\del p_n^{\mu}}{\del t} \delta(\vec{x} - \vec{r}_n(t)) = \sum_{n} \frac{\del \tau}{\del t}f_n^{\mu}(t) \delta(\vec{x} - \vec{r}_n)
\end{empheq}
can be though of as being the \textbf{density of force}. 
Which gives us
\begin{align*}
  \delta_{\nu} T^{\mu \nu} = G^{\mu}
\end{align*}
Now if there are no external forces such that the particles interact only when they collide with each other, then the force density is
\begin{align*}
  G^{\mu} = \sum_{n} \frac{\del p_n^{\mu}}{\del t} \delta(\vec{x} - \vec{r}_n) = \sum_{\text{collision points } x_{\text{coll}}} \delta(\vec{x} - \vec{x}_{\text{coll}}(t)) \frac{d }{d t} \sum_{n \in \text{coll}} p_n^{\mu}(t)
\end{align*}
And at each collision point, the particles conserve momentum, and therefore
\begin{align*}
  \delta_{\nu} T^{\mu \nu} = 0
\end{align*}
Just like the continuity equation for charge $\delta_{\mu}j^{\mu} = 0$ proves that the total charge in the universe is constant
\begin{align*}
  \frac{\del }{\del \tau} \int d^{3} \vec{x} \rho = 0 \implies Q_{\text{total}} = \text{const}
\end{align*}
we have a similar invariant that arises from the condition $G^{\mu} = 0$, namely
\begin{align*}
  \delta_{\nu} T^{\mu \nu} = 0 \implies \delta_0 \int d^{3} \vec{x} T^{\mu 0} + \int d^{3} \vec{x} \del_i T^{\mu i}
\end{align*}
and by the divergence theorem, the second part vanishes and we get
\begin{align*}
  \del_0 \int d^{3 \vec{x}} T^{\mu 0} = 0 
\end{align*}
which implies that the \textbf{total momentum} $P^{\mu}$
\begin{empheq}[box=\bluebase]{align*}
  P^{\mu} := \int d^{3} \vec{x} T^{\mu 0} = \sum_{n} p_n^{\mu}
\end{empheq}
is constant.

\section{Electrodynamics as a relativistic theory}
The goal of this chapter is to formulate electrodynamics in such a way that it is invariant under changes of references.

We will make life a bit easier and set 
\begin{align*}
  c = 1, \quad \epsilon_0 = 1
\end{align*}
and not worry about the units. This will make the relations a bit more transparent.
The maxwell equations are
\begin{align*}
  \vec{\nabla} \cdot \vec{E} =\rho, \quad 
  \vec{\nabla} \times \vec{E} + \frac{\del \vec{B}}{\del t} = 0\\
  \vec{\nabla} \cdot \vec{B} = 0, \quad
  \vec{\nabla} \times \vec{B} - \frac{\del \vec{E}}{\del t} = \vec{j}\\
\end{align*}

We introduce the antisymmetric \textbf{electromagnetic field tensor} $F^{\mu \nu}$ sing the components of the $\vec{E}=(E^{1},E^{2},E^{3})$ and $\vec{B}=(B^{1},B^{2},B^{3})$ field as follows:
\begin{empheq}[box=\bluebase]{align*}
  F^{\mu \nu} := \begin{pmatrix}
  0 & -E^{1} & -E^{2}  & -E^{3}\\
  E^{1} &0  &-B^{3}  & B^{2}\\
  E^{2} &B^{3}  &  0& -B^{1}\\
  E^{3} &-B^{2}  &  B^{1}& 0
  \end{pmatrix}
\end{empheq}
Clearly, we can recover the components of the electric and magnetic field via
\begin{align*}
  E^{i} = F^{i0}, \quad B^{i} = - \frac{1}{2}\epsilon_{ijk} F^{jk}, \quad \text{or} \quad F^{ij} = -\epsilon_{ijk} B^{k}
\end{align*}

What this allows us to do is to express the Maxwell equations in one simple equation
\begin{empheq}[box=\bluebase]{align*}
  \del_{\mu} F^{\mu \nu} = j^{\nu}
\end{empheq}
indeed, for $\nu=0$we have
\begin{align*}
  \del_0 F^{00} + \del_i F^{i0} = \del_i E^{i} = \vec{\nabla} \cdot \vec{E} = \rho = j^{0}
\end{align*}
and for $\nu=i=1,2,3$ we get
\begin{align*}
  \del_0 F^{0i} + \del_j F^{ji} = - \del_0 E^{i} + \del_j (- \epsilon_{jik} B^{k}) = - \frac{\del E^{i}}{\del t} + \epsilon_{ijk}\del_j B^{k} = - \frac{\del E^{i}}{\del t } + \vec{\nabla} \times \vec{B})^{i} = j^{i}
\end{align*}

The remaining maxwell equations tell us that the electric and magnetic field can be obtained by the scalar and vector potential
\begin{align*}
  \vec{E} = - \vec{\nabla} \Phi - \frac{\del \vec{A}}{\del t}, \quad \text{and} \quad \vec{B} = \vec{\nabla} \times \vec{A}
\end{align*}
by combining the potentials into a four-vector
\begin{empheq}[box=\bluebase]{align*}
  A^{\mu} := (\Phi, \vec{A}) = (\Phi,A^{1},A^{2},A^{3})
\end{empheq}
the equations can be written using contra-variant derivative
\begin{empheq}[box=\bluebase]{align*}
  F^{\mu \nu} = \del^{\mu} A^{\nu} - \del^{\nu} A^{\mu}
\end{empheq}
and we can again verify this. For $\mu = 0, \nu=i=1,2,3$ we have
\begin{align*}
  F^{0i} = -E^{i} = \frac{\del A^{i}}{\del t} + \frac{\del \phi}{\del x^{i}} = \del^{0}A^{i} - \del^{i}A^{0}
\end{align*}
and for $\mu=i, \nu = j = 1,2,3$:
\begin{align*}
  F^{ij}
  = 
  - \epsilon_{ijk}B^{k} = - \epsilon_{ijk} (\vec{\nabla} \times \vec{A})^{k} 
  = - \underbrace{\epsilon_{ijk} \epsilon_{k l m}}_{= \delta_{il} \delta_{jm} - \delta_{im} \delta_{jl}} \frac{\del }{\del x^{l}} A^{m} 
  = \del_j A^{i} - \del_i A^{j} 
  = \del^{i} A^{j} - \del^{j}A^{i}
\end{align*}

In this new relativistic notation, the gauge transformations also become simple.
A gauge transformation is any transformations which satisfies
\begin{align*}
  A_{\mu} \mapsto  \tilde{A}_{\mu} = A_{\mu} + \del_{\mu} \chi
\end{align*}
where $\chi$ is some scalar function.
What we previously knew als the gauge-fixing condition
\begin{align*}
  \vec{\nabla} \cdot \vec{A} + \frac{\del \phi}{\del t} = 0
\end{align*}
can be written as $\del_{\mu}A^{\mu} = 0$. In the Lorentz gauge, the maxwell equations become
\begin{align*}
  \del^{2}A^{\mu} = j^{\mu}
\end{align*}



