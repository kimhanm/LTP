
Just like in classical mechanics, where the study of invariants lets us solve many problems, we are interested in invariants of Lorentz transformations.

To find such invariants, we will use the \textbf{scalar product}
\begin{empheq}[box=\bluebase]{align*}
  A \cdot B := A_{\mu} B^{\mu} = g_{\mu\nu}A^{\nu}B^{\mu}= g^{\mu \nu}A_{\mu}B_{\nu}
\end{empheq}
What is nice about the scalar product is that the result is always going to be invariant under a Lorentz transformation, as
\begin{align*}
  A \cdot B \mapsto \tilde{A} \cdot \tilde{B} = \tilde{A}_{\mu}\tilde{B}^{\mu} = A_{\nu}\tensor{\Lambda}{_{\mu}^{\nu}}\tensor{\Lambda}{^{\mu}_{\rho}}B^{\rho} = A_{\nu}\tensor{\delta}{^{\nu}_{\rho}}B^{\rho} = A_{\nu}B^{\nu} = A \cdot B
\end{align*}

We can use this to show that the d'Alembert operator is an invariant.
Using the tensor notation for covariant and contravariant derivatives
\begin{empheq}[box=\bluebase]{align*}
  \del_{\mu} := \frac{\del }{\del x^{\mu}} = \big(\frac{1}{c}\frac{\del }{\del t}, \vec{\nabla}\big)
  , \quad \text{and} \quad 
  \del^{\mu} := \frac{\del }{\del x_{\mu}} = g^{\mu \nu}\del_{\nu} = \big(\frac{1}{c}\frac{\del }{\del t}, - \vec{\nabla}\big)
\end{empheq}
we can write the d'Alembert operator as the scalar product
\begin{empheq}[box=\bluebase]{align*}
  \square = \del^{2} := \del \cdot \del = \del_{\mu}\del^{\mu} = \frac{1}{c^{2}} \frac{\del^{2}}{\del t^{2}} - \vec{\nabla}^{2}
\end{empheq}

\subsection{Currents and densities}


Consider a collection of charges $\{q_n\}$ at positions $\vec{r}_n(t)$.
Then the charge and current densities are given by
\begin{align*}
  \rho(\vec{x},t) = \sum_{i}q_n \delta(\vec{x} - \vec{r}_n(t))
\end{align*}
\begin{align*}
  \vec{j}(\vec{x},t) = \sum_{i}q_n \frac{d \vec{x}}{d t} \delta(\vec{x} - \vec{r}_n(t))
\end{align*}

Using four-vector notation, we can combine the charge and current densities into one object
\begin{empheq}[box=\bluebase]{align*}
  j^{\mu} := (c \rho, \vec{j})
\end{empheq}
and write the equations in the compact form
\begin{align*}
  j^{\mu}(\vec{x},t) = \sum_{i}q_n \frac{d x^{\mu}}{d t}\delta(\vec{x}- \vec{r}_n(t))
\end{align*}
The notation suggests that it is contravariant.
To prove this, we will use the four vector $\delta$ function 
\begin{align*}
  \delta(x^{\mu} - y^{\mu}) = \delta(x^{0} - y^{0})\delta(\vec{x} - \vec{y})
\end{align*}
and see that under a Lorentz transformation
\begin{align*}
  \delta(T^{\mu}) \mapsto  \delta(\tilde{T}^{\mu}) = \delta(\tensor{\Lambda}{^\mu_\nu}T^{\nu}) = \frac{\delta(T^{\nu}}{\abs{\det \Lambda}} = \delta(T^{\nu}))
\end{align*}
this allows us to write the current-density four vector using the integral
\begin{align*}
  j^{\mu}(\vec{x},t) = c \sum_{n}q_n \int d t' \frac{d x^{\mu}}{d t'} \delta(x^{\mu} - r_n^{\mu}(t)
\end{align*}
where
\begin{align*}
  x^{\mu} = (ct', \vec{x}), \quad r_n^{\mu}(t) = (ct, \vec{r}_n(t)
\end{align*}
Under a change of variables
\begin{align*}
  t' \mapsto \tau \implies dt' = d \tau \frac{\gamma}{c}
\end{align*}
we get
\begin{empheq}[box=\bluebase]{align*}
  j^{\mu}(\vec{x},t) = c \sum_{n} q_n \int d \tau \frac{d x^{\mu}}{d \tau} \delta(x^{\mu} - r_n^{\mu}(\tau))
\end{empheq}
so the current-density tensor $j^{\mu}$ transforms exactly as $\tfrac{d x^{\mu}}{d \tau}$ and as such is contravariant.

The continuity equation we know in the form
\begin{align*}
  \frac{\del \rho}{\del t} + \vec{\nabla} \cdot \vec{j} = 0
\end{align*}
can now be written in the relativistic elegant form
\begin{align*}
  \del_{\mu}j^{\mu}= 0
\end{align*}


\subsection{The Energy-Momentum Tensor}
Consider a collection of particles $\{n\}$ at positions $\vec{r}_n(t)$ and momentum
\begin{align*}
  p_n^{\mu} = m_n \gamma \frac{d x^{\mu}}{d t} = (m_n \gamma c, m_n \gamma \vec{v}_n)
\end{align*}
then the energy density would be
\begin{align*}
  \sum_{n}p_n^{0} \delta(\vec{x} - \vec{r}_n(t))
\end{align*}
and the energy current density is
\begin{align*}
  \sum_{n}p_n^{0} \frac{d \vec{r}_n}{d t}\delta(\vec{x}- \vec{r}_n(t))
\end{align*}
and just like before, we can combine the energy and energy current density into a single four object
\begin{align*}
  T^{\mu \nu}
  :=
  \sum_{n}p_n^{\mu} \frac{d r_n^{\nu}}{d t} \delta(\vec{x} - \vec{r}_n(t))
\end{align*}
called the \textbf{energy-momentum tensor}.
Or, by integrating through the time variable, we get the form
\begin{empheq}[box=\bluebase]{align*}
  T^{\mu \nu} = \sum_{n} \int d \tau p_n^{\mu} \frac{d x_n^{\nu}}{d \tau} \delta(x^{\rho} - r_n^{\rho}(\tau)
\end{empheq}
And it transforms as a rank $2$ contravariant tensor, i.e.
\begin{align*}
  T^{\mu \nu} \mapsto \tilde{T}^{\mu \nu} = \tensor{\Lambda}{^{\mu}_{\rho}}\tensor{\Lambda}{^{\nu}_{\sigma}} T^{\rho \sigma}
\end{align*}
One can also show that it is symmetric ($T^{\mu \nu} = T^{\nu \mu}$), because
\begin{align*}
  p_n^{\nu} = m_n \frac{d r_n^{\nu}}{d \tau} = m_n \gamma \frac{d r_n^{\nu}}{d t}
\end{align*}
from which we find the symmetric form
\begin{align*}
  T^{\mu \nu} = \sum_{n} \frac{p_n^{\mu} p_n^{\nu}}{E_n} \delta(\vec{x} - \vec{r}_n(t))
\end{align*}


In classical mechanics, we know that unless an external force $\vec{F}_{\text{ext}}$ is exerted on the system, the total energy should be invariant.

