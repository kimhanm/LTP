
The power radiated per solid angle $d \Omega$ is
\begin{align*}
  \frac{d P_{\text{rad}}}{d \Omega} = \frac{q^{2}}{16 \pi^{2}} \abs{\vec{a}}^{2} \sin^{2} \Theta
\end{align*}
So the total power radiated over all angles is
\begin{empheq}[box=\bluebase]{align*}
  P_{\text{rad}} = \int d \Omega \frac{d P_{\text{rad}}}{d \Omega}
  = \frac{q^{2}}{4 \pi} \frac{2}{3} \abs{\vec{a}}^{2} \left(
    \frac{1}{\epsilon_0 c}
  \right)
\end{empheq}
also known as \textbf{Larmor's formula}.
Writing it in terms of the momentum, we get
\begin{align*}
  P_{\text{rad}} = \frac{q^{2}}{4 \pi \epsilon_0 c} \frac{2}{3} \left(
    \frac{d \vec{p}}{dt}
  \right)^{2} \frac{1}{m^{2}}
\end{align*} 

Now, that we calculated it in the reference frame of the charge ($\vec{v} = 0$), we can use a Lorentz transformation to get it for arbitrary $\vec{v}$.

Since radiation power is Energy/Time, which both transform contravariantly, we see that it is an invariant under Lorentz transformations.

If we take the scalar product of the force at rest
\begin{align*}
  \frac{d p^{\mu}}{d \tau} = (0, \frac{d \vec{p}}{d t}) \implies \frac{d p^{\mu}}{d \tau} \frac{d p_{\mu}}{d \tau} = - \left(
    \frac{d \vec{p}}{d t}
  \right)^{2}
\end{align*}
we recover the term in the momentum formula for the radiated power, so
\begin{align*}
  P_{\text{rad}} = - \frac{q^{2}}{4 \pi} \frac{2}{3m^{2}} \frac{d p^{\mu}}{d \tau} \frac{d p_{\mu}}{d \tau}
\end{align*}
and since the scalar product is invariant under Lorentz transformations, the above formula is valid in all refernece frames!

Doing some further manipulation gives us the result
\begin{empheq}[box=\bluebase]{align*}
  P_{\text{rad}} = \frac{q^{2}}{4 \pi} \frac{2}{3} \gamma^{6} \left[
    \abs{\vec{a}}^{2} - \abs{\vec{v} \times \vec{a}}^{2}
  \right]
\end{empheq}
Notice that no matter how fast a particle is moving, if it is not accelerating ($\vec{a} = 0$), then it will not radiate any energy.

Looking at the $\gamma^{6}$ term, we see
\begin{align*}
  \gamma^{6} = \frac{1}{\left(
    1 - \frac{v^{2}}{c^{2}}
  \right)^{3}}
\end{align*}
that for $v \sim c$, the term becomes really big.
What this means is that if we wanted to accelerate particels (like what the LHC is doing), 
the particle wants to radiate so much energy that it is almost impossible to acelerate it further.

It would be nice to find the most efficient want to accelerate a particle.

We could accelerate it in circular motion or along a straight line.


Looking at the $- \abs{\vec{v} \times \vec{a}}^{2}$ term, we see that in linear acceleration, the cross product vanishes.
\begin{align*}
  \vec{v} \parallel \vec{a} \implies P_{\text{rad}} = \frac{q^{2}}{4 \pi} \frac{2}{3} \gamma^{6} \abs{\vec{a}}^{2}
\end{align*}

However, in circular acceleration, we have $\vec{v} \bot \vec{a}$, maximising the cross product.
This lets us reduce the $\gamma^{6}$ factor to just $\gamma^{4}$
\begin{align*}
\vec{v} \bot \vec{a} \implies
  P_{\text{rad}} 
  &= \frac{q^{2}}{4 \pi} \frac{2}{3} \gamma^{6} \underbrace{\left[
  \abs{\vec{a}}^{2} - \abs{\vec{v}}^{2} \abs{\vec{a}}^{2}\right]}_{= \abs{\vec{a}}^{2} \gamma^{-2}} = \frac{q^{2}}{4 \pi} \frac{2}{3} \gamma^{4} \abs{\vec{a}}^{2}
\end{align*}


We now know about the total radiated power, but let's study the angular distribution of the radiation for arbitrary $\vec{v}$.

After some calcuations, we can show that we would get
\begin{align*}
  \frac{d P_{\text{rad}}}{d \Omega} = \frac{q^{2}}{16 \pi^{2}} \abs{\vec{a}}^{2} \frac{\sin^{2} \Theta}{(1 - v \cos \Theta)^{6}}
\end{align*}
And we will find that the result is absurd because the radiation for a relativistic particle at angle $\Theta= 0$ will become almost infinite.

To resolve this, we will need quantum mechanics.


\section{Scattering}
When we have a charge $q$ that gets hit by an electromagnetic wave, it will then get accelerated and the emit some radiation.

We will consider two case
\begin{itemize}
  \item The charge $q$ is free (Thomson scattering)
  \item The charge $q$ is in a \textbf{bound state} (Rayleigh scattering)
\end{itemize}
Their qualitative differences is how the indicdent elctromagnetic wave accelerates the charge.

\subsection{Thomson scattering}
We have a free charge $q$ in rest frame $(\vec{v} = 0),\vec{x}= 0j$ that gets hit by a planar electromagnetic wave with direction $\hat{e}$, magnitude $E$ and angular velocity $\omega$.
\begin{align*}
  \vec{E}(\vec{x}) = \hat{e} E \text{Re} e^{i(\omega t - \vec{k}\vec{x})}
  = \hat{e} E \cos(\omega t - \vec{k} \cdot \vec{x})
\end{align*}
We know that the force acting on the charge is
\begin{align*}
  \vec{F} = q \vec{E}(0) = \hat{e} qE \cos(\omega t)
\end{align*}
and it will experience an acceleration
\begin{align*}
  \vec{a} = \frac{\vec{F}}{m} = \frac{q}{m}E \cos(\omega t) \hat{e}
\end{align*}
Acceleration means radiation, and its angular distribution of the radiation is given by Larmor's formul
\begin{align*}
  \frac{d P_{\text{rad}}}{d \Omega} = \frac{q^{2}}{16 \pi^{2}} \abs{\vec{a}}^{2} \sin^{2} \Theta = \frac{q^{4}}{16 \pi^{2} m^{2}} \sin^{2} \Theta E^{2} \cos^{2}(\omega t)
\end{align*}
where $\Theta$ is the angle between $\hat{e}$ and the direction of radiation $\hat{n}$.

Averaing over time (i.e. integrating $\cos(\omega t)$ over $t$) we get 
\begin{align*}
  \scal{\frac{d P_{\text{rad}}}{d \Omega}} = \frac{1}{T} \int_0^{T} \frac{d P_{\text{rad}}}{d \Omega}(t) dt = \frac{q^{4}}{32 \pi^{2}m^{2}}E^{2} \sin^{2}\Theta
\end{align*}

We can validate the formula with an experiment where hit a charge $q$ with a laser and see how much radiation it gives off as a function of the angle $\Theta$.

The time average of the Poynting vector is
\begin{align*}
  \scal{S} = \scal{E^{2} \cos^{2}(\omega t} = \frac{E^{2}}{2}
\end{align*}
which lets us calculate the differential cross section
\begin{align*}
  \frac{d \sigma}{d \Omega} := \frac{\scal{\frac{d P_{\text{rad}}}{d \Omega}}}{\scal{S}} = \frac{q^{4}}{16 \pi^{2} m^{2}} \sin^{2} \Theta
\end{align*}
With spherical coordinates $(\theta,\phi,\psi)$ we can see that with unpolarised litght, the direction $\hat{e}$ takes on a random angle $\psi$.
But what can always be controlled is the direction of the light $\vec{k}$, so we want to measure $\theta$.
Using
\begin{align*}
  \sin^{2} \Theta = 1 - \sin^{2} \theta \cos^{2}(\phi - \psi)
\end{align*}
and by taking the average over $\psi$,
\begin{align*}
  \frac{1}{2 \pi}\int_0^{2 \pi} d \psi \cos^{2}(\phi - \psi) = \frac{1}{2} 
\end{align*}
which means that the differential cross section for unpolarised light is
\begin{align*}
  \frac{d \sigma}{d \Omega} = \frac{q^{4}}{16 \pi^{2} m^{2}} \frac{1 + \cos^{2} \theta}{2}
\end{align*}
so the total cross section becimes
\begin{align*}
  \sigma = \int d \Omega \frac{d \sigma}{d \Omega} = \frac{q^{4}}{16 \pi^{2}m^{2}} \frac{8 \pi}{3} = \frac{8 \pi}{3} r_q^{2}, \quad \text{where} \quad r_q := \frac{q^{2}}{4 \pi \epsilon_0 m c^{2}}
\end{align*}
is the \textbf{Thomson radius}




