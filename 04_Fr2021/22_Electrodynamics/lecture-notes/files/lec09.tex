Now that we have got the angular part down, 
we now want to solve for the radial part $R(r)$.
Using the Ansatz
\begin{align*}
  \Psi(r,\theta,\phi) 
  = 
  \sum_{l,m}
  \frac{R_{lm}(r)}{r} Y_{lm}(\theta,\phi)
\end{align*}
And checking the differential equation $\nabla^{2} \Psi = \lambda \Psi$, we get that $R_{lm}$ has to satisfy
\begin{align*}
  \frac{\dd^{2}}{\dd{}r^{2}} \frac{R_{lm}(r)}{r} - \frac{l(l+1)}{r^{3}}
  =
  \lambda \frac{R_{lm}(r)}{r}
\end{align*}

In the special case of a zero eigenvalue $\lambda = 0$, the differential equation becomes
\begin{align*}
  \frac{1}{R}\frac{\dd^{2} R}{\dd r^{2}} = \frac{l(l+1)}{r^{2}}
\end{align*}
and with the ansatz $R(r) = r^{\alpha}$, we obtain two solutions
\begin{align*}
  R_l(r) = A_l r^{-l} + B_l r^{l+1}
\end{align*}
which means the the general solution of the \textbf{Laplace} differential equation 
\begin{align*}
  \nabla^{2} \Psi(r,\theta,\phi) = 0
\end{align*}
is given by
\begin{empheq}[box=\bluebase]{align*}
  \Psi(r,\theta,\phi) 
  = 
  \sum_{l=0}^{\infty} \sum_{m= -l}^{l} 
    \left(
      a_{l,m}r^{-l - 1} + b_{lm}r^{l}
    \right)
    Y_{lm}(\theta,\phi)
\end{empheq}

where we can solve for the coefficients $a_{lm}, b_{lm}$ with boundary conditions. (See MMP-I).

\subsection{Multipole expansion}
When we're given a charge distribution $\rho(\vec{x})$ contained to a small volume $V$, we want to find out the potential outside of the region $V'$.

Inutitively, the charge distribution approximately ``looks like'' a point-charge from a point $\vec{r}$ far away from $V'$.

Starting with the potential
\begin{align*}
  \Phi(\vec{r}) = \frac{1}{4 \pi \epsilon_0} \int_{V'} d^{3} \vec{x} \frac{\rho(\vec{x})}{\abs{\vec{x} - \vec{r}}}
\end{align*}
we expand the inverse distance $\tfrac{1}{\abs{\vec{x} - \vec{r}}}$ in terms of the angle $\gamma$ and distance $\frac{x}{r}$ to get
\begin{align*}
  \Phi(\vec{r}) = \frac{1}{4 \pi \epsilon_0} \sum_{l=0}^{\infty}\frac{1}{r^{l+1}} \int_{V'} d^{3} \vec{x}\rho(\vec{x}) x^{l} P_l(\cos \gamma)
\end{align*}
Writing the vectors $\vec{r},\vec{x}$ in spherical coordinates
\begin{align*}
  \vec{r} =: (r, \theta,\phi), \quad \vec{x} =: (x, \theta_x,\phi_x)
\end{align*}
we can show that
\begin{align*}
  \cos \gamma = \frac{\vec{x} \cdot \vec{r}}{xr} = \cos \theta \cos \theta_x + \sin \theta \sin \theta_x \cos(\phi - \phi_x)
\end{align*}
to do the integration, we use the identity
\begin{align*}
  P_l(\cos \gamma) = \frac{4 \pi}{1 + 2l} \sum_{m=-l}^{l} Y_{lm}^{\ast}(\theta_x,\phi_x) Y_{lm}(\theta,\phi)
\end{align*}
so the Potential can be expressed as
\begin{align*}
  \Phi(\vec{r}) = \frac{1}{\epsilon_0} \sum_{l=0}^{\infty} \frac{1}{1 + 2l} \frac{1}{r^{l+1}} \sum_{m=-l}^{l} q_{lm}Y_{lm}(\theta,\phi) 
\end{align*}
, where we define the \textbf{multipole moments} $q_{lm}$ as
\begin{empheq}[box=\bluebase]{align*}
  q_{lm} := \int_{V'} d^{3}\vec{x} Y_{lm}^{\ast}(\theta_x, \phi_x) \rho(\vec{x}) x^{l}
\end{empheq}
which characterisze the geometry of the charge distribution. For example
\begin{align*}
  q_{00} = \int_{V'} d^{3} \vec{x} \rho(\vec{x}) Y_{00}(\theta,\phi) = \frac{1}{\sqrt{4 \pi}}\int_{V'}d^{3}\vec{x}\rho(\vec{x})
  =
  \frac{Q}{\sqrt{4 \pi}}
\end{align*}
is proportional to the total charge. 
For $l = 1$, we have 
\begin{align*}
  q_{11} 
  = - \sqrt{\frac{3}{8 \pi}}(p_1 - ip_2)
  , \quad q_{10} 
  = \sqrt{\frac{3}{4 \pi}}p_3, \quad 
  q_{1-1} 
  = \sqrt{\frac{3}{8 \pi}}(p_1 + ip_2)
\end{align*}
where
\begin{align*}
  \vec{p} = (p_1,p_2,p_3) = \int d^{3}\vec{x} \vec{x} \rho(\vec{x})
\end{align*}
is the \textbf{dipole moment}.

So the third order aproximation of the Potential is then given by
\begin{align*}
  \Phi(\vec{x}) = \frac{1}{4 \pi \epsilon_0} \left[
    \frac{Q}{r} + \frac{\vec{p} \cdot \vec{x}}{r^{3}} + \ldots
  \right]
\end{align*}

For $l = 2$ we will see the \textbf{quadrupole tensor}
\begin{align*}
  Q_{ij} = \int d^{3} \vec{x} (x_ix_j - x^{2}\delta_{ij})\rho(\vec{x})
\end{align*}
to express $q_{20},q_{21},q_{22}$.
For higher $l$ the $q_{lm}$ have higher powers of $1/r$ in them and therefore contribute less and less to the potential.






