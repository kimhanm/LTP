\subsection{Energy and Momentum}

Two very important invariants in classical mechanics are energy and momentum.
Recall that momentum is defined as
\begin{align*}
  \vec{p} = m \frac{d \vec{x}}{d t}
\end{align*}
the relativistic four-vector analogue is
\begin{empheq}[box=\bluebase]{align*}
  p^{\mu} := mc \frac{d x^{\mu}}{d \tau}
\end{empheq}
and under a Lorentz transformation, the momentum transforms as follows
\begin{align*}
  \tilde{p}^{\mu} =\tensor{\Lambda}{^{\mu}_{\nu}} p^{\nu}
\end{align*}

Since the time interval can be written as
\begin{align*}
  d \tau = \sqrt{c^{2} dt^{2} - d \vec{x}^{2}} = cdt \sqrt{1 - \frac{\vec{v}^{2}}{c^{2}}} = \frac{cdt}{\gamma}
\end{align*}
we find that the time component is given by
\begin{align*}
  p^{0}
  =
  mc \frac{d x^{0}}{d \tau} 
  =
  \frac{mc^{2} dt}{\frac{c\ dt}{\gamma}}
  = m \gamma c
\end{align*}
and the space components are
\begin{align*}
  p^{i} = mc \frac{d x^{i}}{d \tau} = m \gamma \frac{d x^{i}}{d t} = m \gamma v^{i}
\end{align*}
so for small velocities $(\gamma \sim 1)$ we recover the classical definition of momentum.

In the second order approximation of $\gamma$
\begin{align*}
  \gamma = 1 + \frac{1}{2}\frac{v^{2}}{c^{2}} + \mathcal{O}(\frac{v^{4}}{c^{4}})
\end{align*}
, the time component becomes
\begin{align*}
  p^{0} = mc + \frac{1}{2c}mv^{2} + \ldots
\end{align*} 
Notice that we see the kinetic energy $\tfrac{1}{2}mv^{2}$ in the second order term.

Using this, we define the \textbf{relativistic energy} of a particle with
\begin{empheq}[box=\bluebase]{align*}
  E := cp^{0} = m \gamma c^{2} = mc^{2} + \frac{1}{2}mv^{2} + \ldots
\end{empheq}
where the term $mc^{2}$ is also called \textbf{rest energy} of the particle.

We are implicitly making a bet that this quantity does behave like we expect energy to behave, so we need to experimentally verify that this is the case.

Substituting the relation
\begin{align*}
  \vec{p}^{2} = m^{2} \gamma^{2} \vec{v}^{2}
\end{align*}
into the equation $E = cp^{0}$, we obtain the \textbf{energy-momentum relation} of special relativity:
\begin{empheq}[box=\bluebase]{align*}
  E^{2} = c^{2} \vec{p}^{2} + m^{2}c^{4}
\end{empheq}
from which we can derive all sorts of useful relations.


\subsection{The inverse of a Lorentz transformation}
Recall the definition of the metric tensor
\begin{align*}
  g_{\mu\nu} = \text{diag}(1,-1,-1,-1)
\end{align*}
Note that it is its own inverse. But instead of simply writing $g_{\mu\nu}^{-1} = g_{\mu\nu}$, we will instead introduce a new piece of notation, where we raise the indices to define its inverse
\begin{align*}
  g^{\mu \nu} := g_{\mu\nu}^{-1} =  \text{diag}(1,-1,-1,-1) \quad (= g_{\mu\nu})
\end{align*}

this might look like it unnecessary complicates things because we have two different symbols ($g_{\mu\nu}, g^{\mu\nu}$) for the exact same matrix, but it turns out that the concept of raising and lowering indices is very useful.

Before moving on, note that the fact that $g^{\mu \nu}$ is the inverse of $g_{\mu\nu}$ can be compactly written by saying that the two matrices satisfy the Kronecker relation
\begin{align*}
  g^{\mu \nu}g_{\nu \rho} = \tensor{\delta}{^\mu_\nu}
\end{align*}
where $\tensor{\delta}{^\mu_\nu}$ is the Kronecker delta.

If we wish to find the inverse of a Lorentz transformation associated to a boost $\vec{v}$, we would physically expect that the inverse is the transformation associated to the inverse boost $-\vec{v}$, so
\begin{align*}
  (\Lambda_{\nu}^{\mu})^{-1}(\vec{v}) = \Lambda_{\nu}^{\mu}(-\vec{v})
\end{align*}
We can easily prove this, but as before, we will denote the inverse of the matrix with $\tensor{\Lambda}{_\mu^\nu} := \tensor{(\Lambda^{-1})}{^\mu_\nu}$.

It is important to regocnize that the horizontal positions of the indices is important.
From now on, we will pay attention to index ordering.

One can verify that the inverse is given by
\begin{align*}
  \tensor{\Lambda}{_{\mu}^{\nu}}
  = 
  g_{\mu \rho}g^{\nu \sigma} \tensor{\Lambda}{^{\rho}_{\sigma}}
\end{align*}
Indeed, this does satisfy the Kronecker relation since
\begin{align*}
  \tensor{\Lambda}{^{\mu}_{\lambda}}
  \tensor{\Lambda}{_{\mu}^{\nu}}
  = 
  g_{\mu \rho}g^{\nu \sigma} 
  \tensor{\Lambda}{^{\rho}_{\sigma}}
  \tensor{\Lambda}{^{\mu}_{\lambda}}
  = 
  g_{\sigma \lambda} g^{\nu \sigma} = 
  \tensor{\delta}{^{\lambda}_{\nu}}
\end{align*}

Looking at the components of the inverse, we see
\begin{align*}
  \tensor{\Lambda}{_0^0}
  (\vec{v}) = \gamma, 
  \quad
  \tensor{\Lambda}{_i^0}(\vec{v}) = \tensor{\Lambda}{_0^{i}}(\vec{v}) = - \gamma \frac{v^{i}}{c},\\
  \tensor{\Lambda}{_i^{j}}(\vec{v}) = \delta_i^{j} + \frac{v^{i}v^{j}}{v^{2}}(\gamma - 1)
\end{align*}
which does match our physical prediction $\tensor{\Lambda}{_{\mu}^{\nu}}(\vec{v}) = \tensor{\Lambda}{^\mu_\nu}(-\vec{v})$.


\subsection{Tensors}

Recall that in a change of reference
\begin{align*}
  x^{\mu} \mapsto \tilde{x}^{\mu} = \tensor{\Lambda}{^{\mu}_{\nu}} x^{\nu} + \rho^{\mu}
\end{align*}
many of the quantities we defined such as $f^{\mu},p^{\mu},dx^{\mu}$ etc. transformed in the same way, namely
\begin{align*}
  f^{\mu} &\mapsto  \tilde{f}^{\mu} = \tensor{\Lambda}{^{\mu}_{\nu}} f^{\nu}\\
  p^{\mu} &\mapsto  \tilde{p}^{\mu} = \tensor{\Lambda}{^{\mu}_{\nu}} p^{\nu}\\
  dx^{\mu} &\mapsto d \tilde{x}^{\mu} = \tensor{\Lambda}{^{\mu}_{\nu}} dx^{\nu}
\end{align*}
We will use this as our definition of what a tensor is.

A \textbf{contra-variant tensor} is any set of four components that transforms according to the rule
\begin{align*}
  T^{\mu} \mapsto  \tilde{T}^{\mu} = \tensor{\Lambda}{^{\mu}_{\nu}} T^{\nu}
\end{align*}
Not everything is such a contavariant tensor, as some things, (called scalars) don't transform at all!

There is another important class of objects that transform not with $\tensor{\Lambda}{^{\mu}_{\nu}}$, but rather with the inverse $\tensor{\Lambda}{_{\mu}^{\nu}}$.

One example is the (covariant) derivative $\frac{\del }{\del x^{\mu}}$ which transforms as
\begin{align*}
  \frac{\del }{\del x^{\mu}} \mapsto  \frac{\del }{\del \tilde{x}^{\mu}} = \frac{\del x^{\rho}}{\del \tilde{x}^{\mu}}\frac{\del }{\del x^{\rho}} = \Lambda_{\mu}^{\rho} \frac{\del }{\del x^{\rho}}
\end{align*}

so, a \textbf{covariant tensor} is any set four components that transform to the rule
\begin{align*}
  T_{\mu} = \Lambda_{\mu}^{\nu} T_{\nu} = g_{\mu \rho}g^{\nu \sigma} \Lambda_{\sigma}^{\rho} T_{\nu}
\end{align*}
and we will use lower indices to denote covariant objects.
The contra- and covariant tensors play the role of a vector space and the dual space from linear algebra. 

Using the metric tensor, we can transform a contravariant tensor into a covariant one and vice versa.
We do this by \textbf{raising/lowering} indices:
\begin{empheq}[box=\bluebase]{align*}
  T^{\mu} \mapsto T_{\mu} = g_{\mu\nu}T^{\nu} \quad \text{and} \quad T_{\mu} \mapsto T^{\mu} = g^{\mu \nu}T_{\nu}
\end{empheq}
the above transformation is an isomorphism for finite dimensional vector spaces, where we naturally identify the double dual with the vector space itself:
\begin{center}
\begin{tikzcd}[row sep=0.8em] %\arrow[bend right,swap]{dr}{F}
  W \arrow[]{r}{} & W^{\ast} \arrow[]{r}{} & (W^{\ast})^{\ast} = W\\
  U^{\nu} \arrow[]{r}{g_{\mu \nu}} & U_{\mu} \arrow[]{r}{g^{\mu \nu}}& U^{\nu}
\end{tikzcd}
\end{center}

Indeed,the dual of a contravariant tensor $T^{\mu}$ is contravariant as under a Lorentz transformation, we have
\begin{align*}
  T_{\mu}\mapsto  \tilde{T}_{\mu} = g_{\mu \nu} \tilde{T}^{\nu} = g_{\mu\nu} \tensor{\Lambda}{^{\nu}_{\rho}} T^{\rho} = g_{\mu\nu} \tensor{\Lambda}{^{\nu}_{\rho}}g^{\rho \sigma} \tensor{T}{_{\sigma}} = \tensor{\Lambda}{_{\mu}^{\sigma}}T_{\sigma}
\end{align*}


More generally, we can define \textbf{higher rank} tensors, where we can have multiple ``upstairs/downstairs'' indices depending on how the tensor transforms.

\begin{align*}
  T_{\nu_1 \nu_2 \ldots \nu_n}^{\mu_1 \mu_2 \ldots \mu_m} \mapsto 
  \tilde{T}_{\nu_1 \nu_2 \ldots \nu_n}^{\mu_1 \mu_2 \ldots \mu_m} 
  =
  \tensor{\Lambda}{^{\mu_1}_{\sigma_1}} \ldots \tensor{\Lambda}{^{\mu_m}_{\sigma_m}}
  \tensor{\Lambda}{_{\nu_1}^{\rho_1}}
  \ldots
  \tensor{\Lambda}{_{\nu_n}^{\rho_n}}
  T_{\nu_1 \nu_2 \ldots \nu_n}^{\mu_1 \mu_2 \ldots \mu_m} 
\end{align*}


\begin{ex}[]
  In each of the two cases, describe what is happening using usual (non-tensor) language.
  \begin{enumerate}
    \item A tensor $\tensor{x}{^\mu}$ transforms in the following way:
      \begin{align*}
        x^{\mu} \mapsto \tilde{x}^{\mu} = \tensor{\Lambda}{^\mu_\nu}x^{\nu}
      \end{align*}
    \item Describe the transformation
      \begin{align*}
        \tensor{A}{^\mu_\nu} \mapsto \tensor{\tilde{A}}{^{\mu}_\nu} = \tensor{\Lambda}{^{\mu}_\sigma}\tensor{A}{^{\sigma}_{\nu}}
      \end{align*}
    \item Describe the transformation
      \begin{align*}
        \tensor{A}{^\mu_\nu} \mapsto \tensor{\tilde{A}}{^{\mu}_\nu} = \tensor{\Lambda}{^{\sigma}_\nu}\tensor{A}{^{\mu}_\sigma}
      \end{align*}
      Hint: Write down the definition of Matrix multiplication for two matrices $A,B$: $(AB)_{ij} = \sum_{k=1}^{n}\ldots$
    \item Describe the transformation
      \begin{align*}
        \tensor{A}{^\mu_\nu} \mapsto \tensor{\tilde{A}}{^\mu_\nu} = \tensor{\Lambda}{^{\mu}_{\rho}} \tensor{\Lambda}{^{\sigma}_{\nu}}\tensor{A}{^\rho_{\sigma}}
      \end{align*}
      Hint: Break down the transformation into two steps.
    \item The (covariant rank $2$) tensor $T^{\mu \nu}$ transforms as follows
      \begin{align*}
        T^{\mu \nu} \mapsto \tilde{T}^{\mu \nu} = \tensor{\Lambda}{^{\mu}_{\rho}}\tensor{\Lambda}{^{\nu}_{\sigma}}T^{\rho \sigma}
      \end{align*}
      Hint: use $(A^{T})_{ij} = A_{ji}$.
  \end{enumerate}
  \textbf{Solution:}
  \begin{enumerate}
    \item This is just matrix-vector multiplication: $(Ax)_{i} = a_{ij}x_j$. The transformation is therefore $x \mapsto Ax$.
    \item For matrices $A,B$ matrix multiplication is defined as $(AB)_{ij} = \sum_{k=1}^{n}a_{ik}b_{kj}$.
      The common index $\sigma$ over which the summation occurs is the first index of $A$ and the second index of $\Lambda$. 
      This therefore describes left-multiplication with $\Lambda$: $A \mapsto \Lambda A$.
    \item Notice that since multiplication of scalars is commutative, we have $\tensor{\Lambda}{^\sigma_{\nu}} \tensor{A}{^{\mu}_\sigma} = \tensor{A}{^{\mu}_\sigma} \tensor{\Lambda}{^{\sigma_{\nu}}}$.
      The summation occurs over the second index of $A$ and the first index of $\Lambda$, which describes \emph{right}-multiplication with $\Lambda$.

    \item Here, we combine the transformation of (b) and (c), i.e. the transformation
      \begin{align*}
        A \mapsto \Lambda A \Lambda
      \end{align*}
      Check this by computing the triple product
      \begin{align*}
        (ABC)_{\mu \nu} = (AB)_{\mu \sigma} C_{\nu} = A_{\mu \rho} B_{\rho \sigma} C_{\sigma \nu}
      \end{align*}
    \item As $\tensor{(\Lambda^{T})}{^\nu_\sigma} =  \tensor{\Lambda}{^{\sigma}_{\nu}}$, we get the same as in (d), except we transpose the right multiplication matrix:
      \begin{align*}
        A \mapsto \Lambda A \Lambda^{T}
      \end{align*}
  \end{enumerate}

  We will soon see some important covariant rank $2$ tensors such as the \textbf{Energy momentum tensor tensor} $\tensor{T}{^\mu_\nu}$ and the \textbf{Electromagnetic field tensor} $\tensor{F}{^\mu_\nu}$.

  Knowing the matrix-representation for their transformations can be helpful.
\end{ex}
