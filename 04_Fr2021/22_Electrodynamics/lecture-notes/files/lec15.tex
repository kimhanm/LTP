\subsection{Time dilation}
Consider a clock which is at rest for an observer in reference frame $O$.
Between two clock ticks, the clock will have moved in the space-timeinterval 
\begin{align*}
  d \vec{x} = 0, \quad dt = \Delta t
\end{align*}
and its proper time interval is
\begin{align*}
  d\tau^{2} = (c^{2} dt^{2} - d \vec{x}^{2}) = c^{2} (\Delta t)^{2}
\end{align*}
A second observer $O'$ sees the clock moving with velocity $\vec{v}$. Then two ticks of the clock will be separated by a space-time interval
\begin{align*}
  d \vec{\tilde{x}} = \vec{v} d \tilde{t}, \quad d \tilde{t} = \Delta \tilde{t}
\end{align*}
the proper-time interval in the new reference frame is then
\begin{align*}
  d \tilde{s}^{2} = c^{2} d \tilde{t}^{2} - d \vec{x}^{2} = c^{2} (\Delta \tilde{t})^{2} \left(
    1 - \frac{v^{2}}{c^{2}}
  \right)
\end{align*}
and since the proper time is the same in both reference frames, we get
\begin{empheq}[box=\bluebase]{align*}
  \Delta \tilde{t} = \frac{1}{\sqrt{1 - \frac{v^{2}}{c^{2}}}} = \gamma \Delta t
\end{empheq}

Now, imagine that the clock emits light with frequency
\begin{align*}
  \omega = \frac{2 \pi}{\Delta t}
\end{align*}
The observer $O'$ will see the time interval as $\Delta t' = \gamma \Delta t$, and in the same time period, the distance to the light sources increases by $v_r \Delta t'$, where $v_r$ is the component of the velocity in the direction of he light.

Observer $O'$ measures the time between two wave fronts $\Delta t_0$ as
\begin{align*}
  c \Delta t_0 = c \Delta t' + v_r \Delta t' \implies \Delta t_0 = \gamma(1 + \frac{v_r}{c})\Delta t
\end{align*}
so the perceived frequencey for $O'$ is
\begin{align*}
  \omega_0 = \frac{2 \pi}{\Delta t_0} = \omega \frac{\sqrt{1 - \tfrac{v^{2}}{c^{2}}}}{1 + \frac{v_r}{c}}
\end{align*}
and in the special case where the light source is moving directly away from $O'$, $v_r = v$ and so
\begin{align*}
  \omega_0 = \sqrt{\frac{1 - \tfrac{v}{c}}{1 + \tfrac{v}{c}}} \omega
\end{align*}

Consider the case, where we know that a star has a lot of hydrogen. 
Looking at the frequency spectrum of the emitted light, we would expect there to be a peak where the characteristic frequency for hydrogen is.

Then we compare the theoretical peak with the measurements and see how much the hydrogen peak has moved.
This let's us calculate how fast the star is moving away/towards us.

\subsection{Relativistic force}
The classical formula for a force exerted on a particle with mass $m$ is the well-known $\vec{F} = m \frac{d^{2} \vec{x}}{d t^{2}}$.

For the relativistic analogue, we define the \textbf{relativistic force} acting on a particle as
\begin{empheq}[box=\bluebase]{align*}
  f^{\mu} = mc^{2} \frac{d^{2} x^{\mu}}{d\tau^{2}} = g_{\rho \sigma} \frac{d^{2} x^{\mu}}{dx^{\rho}dx^{\sigma}}
\end{empheq}

and check if we can recover Newton's equations when in a stationary reference frame.
Indeed, if the particle is at rest, then the proper time interval only has it's temporal part and we get
\begin{align*}
  d \tau = c d t \implies f_{\text{rest}}^{i} = m  \frac{d^{2} x^{i}}{d t^{2}} = F_{\text{Newton}}^{i}
\end{align*}

Now in a different frame of reference, where the particle is moving with veloctiy $\vec{v}$, we get
\begin{align*}
  f^{\mu} = \Lambda_{\nu}^{\mu}f^{\nu}_{\text{rest}}
\end{align*}
so in particular, the time component is
\begin{align*}
  f^{0} = \Lambda_{\nu}^{0}f_{\text{rest}}^{\nu} = 0 + \Lambda_{i}^{0}f_{\text{rest}}^{i} = \gamma \frac{\vec{v} \cdot \vec{F}_{\text{Newton}}}{c}
\end{align*}
and the spatial components are
\begin{align*}
  f^{i} = \Lambda_{\nu}^{i}f_{\text{rest}}^{\nu} = \Lambda_{j}^{i} f_{\text{rest}}^{j}
\end{align*}

So as a whole, we get
\begin{align*}
  \vec{f} = \vec{F}_{\text{Newton}} + (\gamma - 1)\vec{v}  \frac{\vec{v}\cdot \vec{F}_{\text{Newton}}}{v^{2}}, \quad f^{0} = \frac{\vec{v}}{c} \cdot \vec{f}
\end{align*}


If Newtonian mechanics, we can calculate the trajectory $\vec{x}(t)$ as a function of time by solving
\begin{align*}
  \frac{d^{2} \vec{x}}{d t^{2}} = \frac{1}{m} \vec{F}(\vec{x},t)
\end{align*}
But in special relativity, if we want to find the trajectory as a function of proper time $x^{\mu}(\tau)$ we cannot simply solve
\begin{align*}
  \frac{d^{2} x^{\mu}}{d \tau^{2}} = \frac{1}{m} f^{\mu}
\end{align*}
because we would need to know what $f^{0}$ is, but for that we need to already know one of the other three equations.

We resolve this by defining
\begin{align*}
  \Omega := g_{\mu\nu} \frac{d x^{\mu}}{d \tau} \frac{d x^{\nu}}{d \tau}
\end{align*}
and requiring that this be an invariant under change of references.
Missing 15 mins.
