If we combine the vector and the scalar potential into a $4$-vector, we can write even more compactly

\begin{align*}
  \begin{pmatrix}
    \Phi(\vec{x},t)\\
    \vec{A}(\vec{x},t)
  \end{pmatrix}
  =
  \frac{1}{4 \pi \epsilon_0}
  \int d^{3} \vec{y}
  \frac{1}{\abs{\vec{x} - \vec{y}}}
  \begin{pmatrix}
    \rho(\vec{y}, t_{\text{ret}}\\
    \frac{1}{c}\vec{J}(\vec{y}, t_{\text{ret}})
  \end{pmatrix}
\end{align*}

By using the identity
\begin{align*}
  2 \delta(a^{2} - \abs{b}^{2}) \Theta(a) 
  = 
  2 \delta((a - \abs{b})(a + \abs{b})) \Theta(a)
  =
  \frac{\delta(a- \abs{b}}{a}\Theta(a)
  =
  \frac{\delta(a - \abs{b}}{\abs{b}}
\end{align*}
we can re-write the Green's function as
\begin{empheq}[box=\bluebase]{align*}
  G(\vec{x} - \vec{y}, t - t') = \frac{1}{2 \pi} \delta\left(
    (t - t')^{2} - \frac{\abs{\vec{x} - \vec{y}}^{2}}{c^{2}}  
  \right) \Theta(t - t')
\end{empheq}

So when we calculate $\Phi$ using the new formula for the Green's function, we get
\begin{align*}
  \Phi(\vec{x},t) = \frac{1}{4 \pi \epsilon_0 c} 
  \int d^{3} \vec{y} dt' \delta\left((t-t')^{2} - \frac{\abs{\vec{x} - \vec{y}}^{2}}{c^{2}} \right)
  \rho(\vec{y},t)\Theta(t-t')
\end{align*}



Let's use these equations in an example
\subsection{Potential of a moving charge with constant velocity}
We consider a charge $q$ moving with constant velocity $\vec{v}$
and want to calculate $\Phi(\vec{x},t)$.
Assume that at time $t' = 0$, it is at positoin $\vec{y} = 0$.
The charge density is then given by
\begin{align*}
  \rho(\vec{y},t') = q \delta(\vec{y} - \vec{v} t')
\end{align*}
so substituting into the equation for the potential, we get
\begin{align*}
  \Phi(\vec{x},t) 
  &= 
  \frac{1}{2 \pi \epsilon_0 c}
  \int d^{3}\vec{y} dt'
  \delta\left(
    (t - t')^{2}
    -
    \frac{\abs{\vec{x}-\vec{y}}^{2}}{c^{2}}
  \right)
  q\delta(\vec{y} - \vec{v} t')
  \Theta(t - t')\\
  &=
  \frac{q}{2 \pi \epsilon_0 c}
  \int dt' \Theta(t-t')
  \delta\left(
  (t -t')^{2} - \frac{\abs{\vec{x} - \vec{v}t}^{2}}{c^{2}}
\right)
\end{align*}
So we now need to find the zeros of the argument in the delta function.
By decomposing $\vec{x}$ into it's paralell $\vec{x}_{\parallel}$ and perpendicular $\vec{x}_{\bot}$ components with respect to the velocity $\vec{v}$, we can find the solutions to
\begin{align*}
  0 = (t -t')^{2} - \frac{\abs{\vec{x} - \vec{v}t'}^{2}}{c^{2}} =
  t'^{2} - 2t't + t^{2} - \frac{(x_{\parallel} - v t')^{2} + x_{\bot}^{2}}{c^{2}}
\end{align*}
by defining the \textbf{boosted variables}
\begin{align*}
  x_b = \gamma(x_{\parallel} - vt), \quad t_b = \gamma \left(
    t - \frac{x_{\parallel}v}{c^{2}} 
  \right)
    \quad \text{for} \quad \gamma = \frac{1}{\sqrt{1 - \tfrac{v^{2}}{c^{2}}}}
\end{align*}
and introducing the quantity
\begin{align*}
  \tau^{2} = c^{2} t^{2} - (x_{\parallel}^{2} + x_{\bot}^{2}) = c^{2} t_b^{2} - (x_b^{2} + x_{\bot}^{2})
\end{align*}
we can re-write the equation for $t'$ from before to 
\begin{align*}
  \frac{t'^{2}}{\gamma^{2}} - 2 \frac{t'}{\gamma} + \frac{\tau^{2}}{c^{2}} = 0
\end{align*}
which is just a quadratic. The discriminant is
\begin{align*}
  \Delta =  \frac{4(t_b^{2} - \tau^{2}/c^{2})}{\gamma^{2}} = \frac{4 r_b^{2}}{\gamma^{2}c^{2}}, \quad \text{for} \quad r_b^{2} = x_b^{2} + x_{\bot^{2}}
\end{align*}
so the solutions for $t'$ are
\begin{align*}
  t_{\pm} = \gamma \left(
    t_b \pm \frac{r_b}{c}
  \right)
\end{align*}
so we can now do the $t'$ for $\Phi$ to get
\begin{align*}
  \Phi(\vec{x},t) = \frac{q}{4 \pi \epsilon_0} \frac{\gamma}{r_br_b}
\end{align*}
similarly, for the vector potential, we obtain
\begin{align*}
  \vec{A}(\vec{x},t) = \frac{\vec{v}}{c^{2}} \Phi(\vec{x},t)
\end{align*}
Comparing this to the reference frame where the charge is at rest, we get
\begin{align*}
  \Phi(\vec{x},t)|_{\text{rest}} = \frac{q}{4 \pi \epsilon_0} \frac{1}{\left[x_{\parallel}^{2} + x_{\bot}^{2}\right]^{1/2}} , \quad \vec{A}(\vec{x},t)|_{\text{rest}} = 0
\end{align*}
so we see that coordinates tranform according to the Lorentz transformations we know form special relativity:
\begin{align*}
x_{\parallel} \mapsto \tilde{x}_{\parallel} = \gamma(x_{\parallel} - vt), \quad x_{\bot} \mapsto \tilde{x}_{\bot} = x_{\bot}, \quad t \mapsto \tilde{t} = \gamma t
\end{align*}

The equations become more concise to write if we combine time and space, and scalar and vector potential to \textbf{4-vectors}.
\begin{align*}
  A^{\mu}
  :=
  \begin{pmatrix}
  \Phi\\
  \vec{A}
  \end{pmatrix}, \quad
  x^{\mu} = \begin{pmatrix}
  ct\\
  \vec{x}
  \end{pmatrix}
\end{align*}


