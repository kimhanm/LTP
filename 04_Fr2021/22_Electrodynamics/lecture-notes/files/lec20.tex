\section{Radiation}

We wish to study the electromagnetic field generated by a moving charge $q$ at $\vec{r}(t)$.

If the accleration $\vec{a} = \frac{d^{2} r(t)}{dt^{2}}$ is non-zero, then energy will be radiated outwards, capable of reaching infinite distances.

We will first compute the potential $A^{\mu}$ and use it to calculate the field tensor $F^{\mu \nu}$.

Performing the $d^{4}y$ integration, we get
\begin{align*}
  A^{\mu}(x) 
  &= \int d^{4}y G_{\text{ret}}(x - y)j^{\mu}(y)\\
  &= \frac{q}{2 \pi} \int d \tau v^{\mu}(\tau) \delta\left(
    (d - r(\tau))^{2}
  \right)
  \Theta(x^{0} > r^{0}(\tau))
\end{align*}
To do the $d \tau$ integration, we need to solve for the root in the expression in the delta function, adding the constraint by the $\Theta$ function:
\begin{align*}
  f(\tau) = (x - r(\tau))^{2} = 0 \quad \text{and} \quad x^{0} > r^{0}(\tau)
\end{align*}
solving for this gives us
\begin{align*}
  x^{0} - r^{0}(\tau) = \abs{\vec{x} - \vec{r}(\tau)}\tag{$\ast$}
\end{align*}
If we know the roots $\tau_i$ we can use the fact that
\begin{align*}
  \delta(f(\tau)) = \sum_{i} \frac{\delta(\tau - \tau_i)}{\abs{\frac{d f}{d \tau}}|_{\tau = \tau_i}}
\end{align*}
If however the velocity is $v < c$, then there can only be one solution $\tau_{\text{ret}}$.
Since the derivative is
\begin{align*}
  \frac{d f(\tau)}{d \tau} 
  = ...
  = - 2 \scal{v, x - r(\tau)}
\end{align*}
so the vector potential becomes
\begin{align*}
  A^{\mu}(x) = \frac{q}{4 \pi} \frac{v^{\mu}(\tau_{\text{ret}}}{\scal{v(\tau_{\text{ret}},x - r(\tau_{\text{ret}})}}
\end{align*}
known as the Lienard-Wiechert potential.
Note that in the rest frame of the particle, we have $v^{\mu} = (1,0,0,0)$, so the equation just becomes the Coulomb potential
\begin{align*}
  A^{\mu}(x) 
  = \frac{q}{4 \pi} \frac{(1,\vec{0})}{\abs{x^{0} - r^{0}}}
  \stackrel{\ast}{=} \frac{q}{4 \pi} \frac{(1,\vec{0})}{\abs{\vec{x} - \vec{r}}}
\end{align*}

As another sanity check, we can solve for the potential of a particle moving with constant velocity.


If we wanted to get the $\vec{E},\vec{B}$ field from our four potential $A^{\mu}$, we could calculate the electromagnetic field tensor $F^{\mu \nu}$ with
\begin{align*}
  F^{\mu \nu} = \del^{\mu}A^{\nu} - \del^{\nu} A^{\mu}
\end{align*}
but this process is rather laborious.
What we can do is to instead define the four vector
\begin{empheq}[box=\bluebase]{align*}
  R^{\mu} := x^{\mu} - r^{\mu}(\tau_{\text{ret}}) =: \abs{\vec{R}} (1, \hat{n})
\end{empheq}
which is the distance vector of the observer and the charge at the retaded time $\tau_{\text{ret}}$.

The condition $\ast$ from earlier simply says
\begin{align*}
  R^{2} = R_{\mu}R^{\mu} = 0 \implies R^{0} = \abs{\vec{R}}
\end{align*}

With this, we can show that the electric and magnetic field are orthogonal. 
Indeed, because
\begin{align*}
  E^{i} = F^{i0} = \frac{1}{4 \pi (1 - \hat{n} \cdot \vec{v})^{3}} \left[
    \frac{(1 - \vec{v}^{2})}{\abs{\vec{R}}^{2}}
    (\hat{n} - \vec{v})
    +
    \frac{1}{\abs{\vec{R}}}
    \hat{n} \times \left(
      (\hat{n} - \vec{n}) \times \vec{a}
    \right)
  \right]
\end{align*}
and the B field we get
\begin{center}
missing
\end{center}
So in the end we have found
\begin{align*}
  \vec{B} = \hat{n} \times \vec{E}
\end{align*}

\subsection{Larmor's Formula}

Consider a particle that is accelerated, but is momentarily at rest.


The Poynting vector is given by
\begin{align*}
  \vec{S} = \vec{E} \times \vec{B} = \vec{E} \times (\hat{n} \times \vec{E}) = \abs{\vec{E}}^{2} \hat{n} - (\vec{E} \cdot \hat{n}) \vec{E}
\end{align*}
and by expanding it in $\abs{\vec{R}}^{-1}$, we get
\begin{align*}
  \vec{S} = \hat{n} \frac{q^{2}}{16 \pi^{2} \abs{\vec{R}}^{2}}
  \abs{\hat{n} \times (\hat{n} \times \vec{a})}^{2}
  + \mathcal{O}(\abs{\vec{R}}^{3})
\end{align*}
or if we let $\theta$ to be the angle between $\hat{n}$ and $\vec{a}$:
\begin{align*}
  \vec{S} = \hat{n}
  \frac{q^{2}}{16 \pi^{2} \abs{\vec{R}}^{2}}
  \abs{\vec{a}}^{2} 
  \sin^{2} \theta + 
  + \mathcal{O}(\abs{\vec{R}}^{3})
\end{align*}

The power emitted through a segment $d \vec{A}$ of a closed surface $\del V$ around the retared position of the charge $q$ is
\begin{align*}
  dP := \frac{d W}{d t} = d \vec{A} \cdot \vec{S}
\end{align*}
which when it is
[...]


And the power radiated per solid angle $d \Omega$ is
\begin{align*}
  \frac{d P_{\text{rad}}}{d \Omega} = \frac{q^{2}}{16 \pi^{2}} \abs{\vec{a}}^{2} \sin^{2} \theta
\end{align*}
So the total power radiated over all angles is
\begin{align*}
  P_{\text{rad}} = \int d \Omega \frac{d P_{\text{rad}}}{d \Omega}
  = \frac{q^{2}}{4 \pi} \frac{2}{3} \abs{\vec{a}}^{2}
\end{align*}

