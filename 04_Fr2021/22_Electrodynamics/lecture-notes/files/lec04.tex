\subsection{Static distributions and Energy}
We found in the last lecture that depending on whether we used the discrete or continuous formula for the Energy, we got an additional \emph{self-energy} term.

By introducing the \textbf{energy density} $\omega$ given by
\begin{align*}
  W &= \frac{\epsilon_0}{2} \int_V d^{3}\vec{x} \vec{E}^2(\vec{x}) = \int_V d^{3} \vec{x} \omega(\vec{x}) 
  \implies
  \omega := \frac{\epsilon_0}{2}\vec{E}^{2}
\end{align*}

If instead of just calculating the total energy, we can just calculate the \emph{change} in energy between to states. 
Then the self-energy terms will drop out and we get a coherent result between the continuous and discrete measurements.


\begin{ex}[Pressure on a conductor]
  By definition, a \textbf{conductor} is a piece of material, where electrons (or charges) can move around freely.
In reality, ideal conductors don't exist as moving charges radiate energy and lose energy there will be some friction.

Conductors exhibit the property, that if we introduce some exess charge on a conductor, then the electrons will move around until they don't want to anymore. 
This state will be reached when the electron distribution minimizes their total energy.
We can formalize this by saying that when the electrons don't move around anymore, the electric field on the inside of the conductor is zero.
\begin{align*}
  \vec{E}_{\text{inside}} = 0
\end{align*}
or else it would exert a force on a charge, moving it around.

Since $\vec{E} = - \vec{\nabla} \Phi$, it follows that $\Phi_{\text{inside}} = $constant.

Another property is that on the surface, the electric field must be \emph{perpendicular to the surface}.
If it had some horizontal component, it would mean that the electrons on the surface would move around.
Also, using Gauss's law on the surface, if the conductor has non-zero exess charge, the perpendicular component is non-zero.
\begin{align*}
  \vec{E}_{\parallel} = 0, \vec{E}_{\bot} \neq 0
\end{align*}
moreover, if we let $\sigma$ be the charge density on the surface, then
\begin{align*}
  \Delta S \cdot E &= \frac{\sigma}{\epsilon_0} \cdot \Delta S\\
  \implies \vec{E} &= \frac{\sigma}{\epsilon_0} \vec{n}
\end{align*}
, where $\vec{n}$ is the surface normal.
This means that the energy density $\omega$ is given by
\begin{align*}
  \omega = \frac{\epsilon_0}{2} \vec{E}^{2} = \frac{\sigma^{2}}{2 \epsilon_0}
\end{align*}

With this, we can calculate the pressure, or energy difference.

Now image that we take our conductor deforms slightly such that it has a part where the sufrace bulges by a small amount with additional volume $\Delta S \cdot \Delta x$.

This will change the energy as follows

\begin{align*}
  E_{\text{deformed}} &= E_{\text{original}} + \int_{\Delta S \cdot \Delta x}d^{3} \vec{x} \omega\\
  &= E_{\text{original}} + \Delta S \cdot \Delta x \cdot \frac{\sigma^{2}}{2 \epsilon_0}\\
  \implies \frac{E_{\text{deformed}} - E_{\text{original}}}{\Delta x} &= \Delta S \frac{\sigma^{2}}{2 \epsilon_0}
\end{align*}
where we assume that the integrand is very small, so the integrand $\omega$ is practically constant.

So the difference in the energy is the work done by pushing theelectrons around. The work is force times distance, so 
\begin{align*}
  \frac{\text{Force} \cdot \Delta x}{\Delta x} &= \Delta S \frac{\sigma^{2}}{2 \epsilon_0}\\
  \text{Pressure} &= \frac{\text{Force}}{\Delta S}  = \frac{\sigma^{2}}{2 \epsilon_0}
\end{align*}
So the pressure rises quadratically with the charge.
\end{ex}

Up to now, we were able to solve the Maxwell equations in integral form
\begin{align*}
\Phi = \frac{1}{4 \pi \epsilon_0} \int d^{3} y \frac{\rho(\vec{y})}{\abs{\vec{x} - \vec{y}}}
\end{align*}
where the integral goes over all of space. 

The problem is that this requires knowlege of all charges throughout all of space.
What we want to do is to find an alternative way to solve the problem which only requires local information.
What is the minimal amount of information that we need on the boundary $\del V$ in order to compute $\vec{E}$ inside $V$.

The main tool will be the poisson equation
\begin{align*}
  \vec{\nabla}^{2} \Phi = - \frac{\rho}{\epsilon_0}
\end{align*}
First note that unless we have a boundary condition, the solution is not unique.
Assume we have two solutions $\Phi_1, \Phi_2$ inside the volume. 
Then their difference $\Phi_1 - \Phi_2$ satisfies $ \Delta(\Phi_1 - \Phi_2) = 0$.
In particular, we can write
\begin{align*}
  (\Phi_1 - \Phi_2)\Delta (\Phi_1 - \Phi_2) &= 0\\
  \implies \abs{
    \nabla(\Phi_1 - \Phi_2) 
  }
  - \vec{\nabla}\left(
    (\Phi_1 - \Phi_2) \vec{\nabla}(\Phi_1 - \Phi_2)
\right) &= 0
\end{align*}
So using $\vec{E} = - \vec{\nabla}\Phi$ we get that
\begin{align*}
  (\vec{E}_1 - \vec{E}_2)^{2} + \vec{\nabla} \left(
    (\Phi_1 - \Phi_2) (\vec{E}_1 - \vec{E}_2)
  \right)
  = 0
\end{align*}
so by integrating over a volume $V$, we get using Gauss's Law
\begin{align*}
  \int_V d^{3} \vec{x} (\vec{E}_1 - \vec{E}_2)^{2}
  &=
  - \int_V d^{3} \vec{x} \vec{\nabla} \cdot \left(
    (\Phi_1 - \Phi_2) (\vec{E}_1 - \vec{E}_2)
  \right)\\
  &= - \int_{\del V} d \vec{S} \cdot (\vec{E}_1 - \vec{E}_2)(\Phi_1 - \Phi_2)
\end{align*}


Assume that we know the potential everywhere on the boundary.
If the boundary conditions give us a unique solutions for $\Phi$, i.e. $\Phi_1 - \Phi_2 = 0$, then the right hand side becomes zero.

The only way that the positive definite integral on the left hand side is zero, if its integrand is zero, so
\begin{align*}
  \int_V d^{3} \vec{x} \underbrace{(\vec{E}_1 - \vec{E}_2)^{2}}_{\geq 0} = 0 \implies \vec{E}_1(\vec{x}) = \vec{E}_2(\vec{x}) \quad \forall \vec{x} \in V
\end{align*}
If $\Phi$ is known uniquely on the boundary, then $\vec{E}$ is uniquely determined inside the volume.
Similarly, if $\vec{E}$ is known uniquely on the boundary, then $\vec{E}$ is uniquely determined inside the volume.

\subsection{Solving the poisson equation}
Given boundary conitions, we want to solve $\vec{\nabla}^{2} \Phi = - \frac{\rho}{\epsilon_0}$ using the \textbf{method of Green's functions}.

A \textbf{Green's function} is a function, whose laplacian is a delta function:
\begin{align*}
  \vec{\nabla}_{\vec{x}}^{2} G(\vec{x},\vec{y}) = - 4 \pi \delta(\vec{x} - \vec{y})
\end{align*}
one such example is the inverse distance
\begin{align*}
  G(\vec{x},\vec{y}) = \frac{1}{\abs{\vec{x} - \vec{y}}}
\end{align*}
or more generally, we can add to it any harmonic function $F$. (i.e. $\vec{\nabla}^{2} F(\vec{x},\vec{y}) = 0$.

Consider a function $\vec{F}$ given by
\begin{align*}
  \vec{F} = \Psi \vec{\nabla} \Phi - \Phi \vec{\nabla} \Psi
\end{align*}
where $\Phi$ is a potential and $\Psi = G(\vec{x},\vec{y})$ is a Green's function.
\begin{align*}
  \vec{\nabla}^{2} \Phi = - \frac{\rho}{\epsilon_0}, \quad \vec{\nabla}_{\vec{x}}^{2} \Psi(\vec{x},\vec{y}) = - 4 \pi \delta(\vec{x} - \vec{y})
\end{align*}

Taking the divergence of $\vec{F}$ we are left with
\begin{align*}
  \vec{\nabla}\cdot \vec{F} &= \Phi \vec{\nabla}^{2} \Phi - \Phi \vec{\nabla}^{2}\Psi\\
                            &= \Psi(\vec{x},\vec{y}) \left(
                              - \frac{\rho(x)}{\epsilon_0}
                            \right)
                            -
                            \Phi(\vec{x}) (- 4 \pi \Phi(\vec{x}) \delta(\vec{x} - \vec{y})
\end{align*}
by integrating over a Volume $V$ we obtain
\begin{align*}
  \int_V d^{3} \vec{x} \vec{\nabla} \cdot \vec{F} 
  &= 
  4 \pi \int_V d^{3} \vec{x} \Phi(\vec{x}) \delta(\vec{x} - \vec{y})
  - \frac{1}{\epsilon_0} \int_V d^{3} \vec{x} \rho(\vec{x}) \Psi(\vec{x},\vec{y})\\
  &=
  4 \pi \Phi(\vec{y})
  -
  \frac{1}{\epsilon_0} \int_V d^{3}\vec{x} \rho(\vec{x}) \Psi(\vec{x},\vec{y})
\end{align*}
We can solve this for $\Phi(\vec{y})$ and use Gauss's Law for the divergence $\vec{\nabla} \cdot \vec{F}$.
\begin{align*}
  \Phi(\vec{y}) &= \frac{1}{4 \pi \epsilon_0} \int_V d^{3} \rho(\vec{x}) \Psi(\vec{x},\vec{y}) + \frac{1}{4 \pi} \int_{\del V} d \vec{S} \cdot \vec{F}
\end{align*}
Recall our definition of $\vec{F}$. This gives us
\begin{align*}
  \int_{\del V} d \vec{S} \cdot \vec{F} 
  &=
  \int_{\del V}d \vec{S} \cdot\left(
    \Psi \vec{\nabla} \Phi - \Phi \vec{\nabla}\Psi
  \right)\\
  &= \int_{\del V}d \vec{S} \left(
    G(\vec{x}, \vec{y}) (- \vec{E}(\vec{x})
    -
    \Phi(\vec{x}) \vec{\nabla}_{\vec{x}}G(\vec{x},\vec{y})
  \right)\\
  &=
  - \int_{\del V} d \vec{S} \cdot \left(
    \vec{E}(\vec{x})G(\vec{x}, \vec{y}) + \Phi(\vec{x}) \vec{\nabla}_{\vec{x}} G(\vec{x},\vec{y})
  \right)
\end{align*}
So if $\vec{E}$ and $\Phi$ are given on the boundary $\del V$ and construct the Green function$G$, then we can solve for the the field inside the volume $V$.

But notice that our current technique requires knowledge of both $\Phi$ and $\vec{E}$ on the bondary.
We want to only need one of them.
Let's suppose that we only know $\Phi(\vec{x})$ on the boundary and chose our Green's function
\begin{align*}
  G(\vec{x},\vec{y}) = \frac{1}{\abs{\vec{x} - \vec{y}}} + F(\vec{x},\vec{y})
\end{align*}
for a harmonic $F$ such that the green's function vanishes on boundary.
\begin{align*}
  G_D(\vec{x}, \vec{y}) = 0, \forall \vec{x} \in \del V, \vec{y} \in V
\end{align*}
We call this the \textbf{Dirichelet Green's function}. 
Since it vanishes on the surface $\del V$, we can simplify the previous equation to
\begin{align*}
  \Phi(\vec{y}) &= \frac{1}{4 \pi \epsilon_0} \int_V d^{3}\vec{x} \rho(\vec{x}) G_D(\vec{x},\vec{y})
  -
  \frac{1}{4 \pi} \int_{\del V} d \vec{S} \Phi(\vec{x}) \vec{\nabla}_{\vec{x}}G(\vec{x},\vec{y})
\end{align*}
to solve for $\vec{E}$ and $\Phi$ everywhere inside the volume $V$.

