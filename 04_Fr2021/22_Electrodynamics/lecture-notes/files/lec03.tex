By defining the \textbf{scalar potential}
\begin{align*}
  \Phi(\vec{x}) =\frac{1}{4 \pi \epsilon_0} \int d^{3} \vec{y} \frac{\rho(\vec{y})}{\abs{\vec{x} - \vec{y}}}
\end{align*}
we can compute the divergence of the electric field
\begin{align*}
  \vec{\nabla} \cdot \vec{E} = - \nabla^{2} \Phi = \frac{1}{4 \pi \epsilon_0} \int d^{3} \vec{y} \rho(\vec{y}) \left(
    - \nabla_{\vec{x}}^{2} \frac{1}{\abs{\vec{x} - \vec{y}}}
  \right)
\end{align*}
we have seen in MMP I that
\begin{align*}
  \nabla_{\vec{x}}^{2} \frac{1}{\abs{\vec{x}- \vec{y}}} &= - 4 \pi \delta(\vec{x}- \vec{y})
\end{align*}
so the divergence will be 
\begin{align*}
  \vec{\nabla} \cdot \vec{E} = - \nabla^{2} \Phi = \frac{\rho}{\epsilon_0}
\end{align*}
Which is the differential form of Gauss's Law.


What is the physical meaning of the scalar potential?
Because the force acting on a particle with charge $q$ is
\begin{align*}
  F = q \vec{E} = - q \vec{\nabla}\Phi
\end{align*}
we can see that the Work done when moving from point $A$ to point $B$ is given by
\begin{align*}
  \abs{W_{A \to  B}} = \int_{\vec{x}_A}^{\vec{x}_B} F \det = \abs{\Phi(\vec{x}_B) - \Phi(\vec{x}_A)}
\end{align*}

Let's say that we have $N$ charges $q_i$ at positions $\vec{y}_i$ and we want to know how much energy they have?

What we can do is to think of all charges starting infinitely far away. Since they are each infinitely far apart, we cansay that they have zero potential energy.

What we then do is to bring the charges into their position, one-by-one and measure the work needed to do that.

For the first charge, we won't need any work as the eletric field is zero everywhere. 
This then generates the first potential
\begin{align*}
  \Phi_1(\vec{x}) = \frac{1}{4 \pi \epsilon_0} \frac{q_1}{\abs{\vec{x} - \vec{y}_1}}
\end{align*}
For the second charge, the energy needed is going to be
\begin{align*}
  W_2 = q_2 (\Phi_1(\vec{y}_2) = \frac{1}{4 \pi \epsilon_0} \frac{q_1 q_2}{\abs{\vec{y}_2 - \vec{y}_1}}
\end{align*}
and this generates a potential $\Phi_2$, which is just the superposition of the first potential and the potential generated by the second isolated particle.

More generally, after $n$-charges are set into postion, bringing the $n+1$-th particle from infinity will take
\begin{align*}
  W_{n+1} = \frac{1}{4 \pi \epsilon_0} q_{n+1} \sum_{i=1}^{n}\frac{q_i}{\abs{\vec{y}_{n+1}}- \vec{y}_i}
\end{align*}
so the total energy will be
\begin{align*}
  W_{\text{tot}} &= W_1 + W_2 + \ldots + W_N\\
  &= \sum_{n=2}^{N}\sum_{1 \leq i < n} \frac{q_i q_n}{\abs{\vec{y}_n} - \vec{y}_i}\\
  &= \frac{1}{8 \pi \epsilon_0} \sum_{\underset{i \neq j}{1 \leq i,j \leq N}} \frac{q_iq_j}{\abs{\vec{y}_i - \vec{y}_j}}
\end{align*}
where we use the symmetry to sum over all $1 \leq i \neq j \leq N$ and divide by two to account for double counting.



In the continous limit we assume that $q_i = dV \rho(\vec{y}_i)$, so 
\begin{align*}
  W &= \int \frac{d^{3} \vec{x} d^{3}\vec{y}}{8 \pi \epsilon_0} \frac{\rho(\vec{x}) \rho(\vec{y})}{\abs{\vec{x} - \vec{y}}}\\
    &= \int d^{3}\vec{x} \rho(\vec{x}) \Phi(\vec{x})\\
    &= - \frac{\epsilon_0}{2} \int d^{3} \vec{x} (\nabla^{2} \Phi) \Phi(\vec{x})\\
    &= - \frac{\epsilon_0}{2} \int d^{3} \vec{x} \left[
      \vec{\nabla}(\Phi \vec{\nabla} \Phi) - (\vec{\nabla}\Phi)(\vec{\nabla}\Phi)
    \right] 
\end{align*}
where we can have rewritten the charge density in terms of the poisson equation.
This is nice because we only need to know the scalar potential in the last reformulation.

Notice that the integral goes from the origin to infinity. But there, the surface term $\nabla^{2}\Phi$ vanishes at infinity, so we can write
\begin{align*}
  W = \frac{\epsilon_0}{2} \int d^{3}\vec{x} (\vec{\nabla}\Phi)^{2}\\
  &= \frac{\epsilon_0}{2} \int d^{3}\vec{x} \vec{E}^{2}(\vec{x})
\end{align*}
which is always a positive quantitiy.


Let's compare the discrete sum with the continuous limit in the case of a single charge $q$ at position $\vec{y}_0$.
Its charge density can be described using a delta function.
\begin{align*}
  \rho(\vec{x}) = q \delta(\vec{x} - \vec{y}_0)
\end{align*}
Notice how the continuous expression will give us an answer that we don't expect:
\begin{align*}
  W &= \int d^{3}\vec{x}d^{3}\vec{y} \frac{q^{2}}{8 \pi \epsilon_0} \delta(\vec{x} - \vec{y}_0) \delta(\vec{y} - \vec{y}_0) \frac{1}{\abs{\vec{x} - \vec{y}}}\\
  &= \frac{q^{2}}{8 \pi \epsilon_0} \frac{1}{\abs{\vec{y}_0 - \vec{y}_0}} = \infty
\end{align*}
But recall that bringing a single charge from infinity to an empty field should take no energy.
We call this infinity \textbf{self-energy}.

One might argue that this formula is non-sense, but it nontheless insightful.
The problem orginiates from our attempt model what a charge at a point position looks like.
We can't simply put a charge into an infinitesimal point and claim that the physics works as usual.

So instead of of assuming that the charge distribution is a simple delta function, let's write spread it out a little bit
\begin{align*}
  \rho_{\delta}(\vec{x}) = \frac{q}{\pi^{3}} \frac{\delta^{3}}{(x_1^{2} + \delta^{2})(x_2^{2}+\delta^{2})(x_3^{2} + \delta^{2})}
\end{align*}
where $\delta \ll 1$ is a small parameter.
Notice that as $\delta \to 0$, the $\rho_{\delta}(\vec{x})$ approaches the delta function.

So from far away, the charge density looks like the delta function and we can show that if we integrate over the charge density, we obtain the total charge:
\begin{align*}
  \int d^{3} \vec{x} \rho_{\delta}(\vec{x}) = q
\end{align*}
We can also show that for finite $\delta$, the integral
\begin{align*}
  \int d^{3} \vec{x} d^{3} \vec{y} \frac{\rho_{\delta}(\vec{x}) \rho_{\delta}(\vec{x})}{\abs{\vec{x} - \vec{y}}} =: g(\delta)
\end{align*}
is finite, and we can associate to this integral the \emph{finite} self-energy $E_{\text{self}}^{(\delta)}$


At infinity, the potential energy goes to zero, but the self-energy does not vanish.
So the thing that stays consistent between the discrete and continuous formulation is not $W$ itself, but
\begin{align*}
  \Delta W := W - W_{\text{self}}
\end{align*}
In quantum field theory, the analogue of self-energy is when a chage exhibits self-interaction by way of emitting a photon and absorbing it again.
The charge-photon-photon interaction turn out to give us Coulomb's Law.


So the self-energy emerges when we count self-interactions of particles.
