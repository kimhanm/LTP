\section{Maxwell Equations in a medium}

After combining the averaged charge and current density $\scal{\rho}$ and $\scal{\vec{j}}$ into the averaged $4$ vector $\scal{j^{\mu}}$,
we can find out the electromagnetic field tensur (and thus the averaged $\vec{E}$ and $\vec{B}$ fields with the equation

\begin{align*}
  \delta_{\mu} \scal{F^{\mu \nu}} = \scal{j^{\nu}}
\end{align*}

in particular, we get

\begin{align*}
  \vec{\nabla} \cdot \scal{\vec{B}} 
  &= 0
  \\
  \vec{\nabla} \times \scal{\vec{E}} 
  &=
  - \frac{\del \scal{\vec{B}}}{\del }
  \\
  \vec{\nabla} \cdot \left(
    \scal{\vec{E}} + \frac{\vec{P}}{\epsilon_0}
  \right)
  &=
  \frac{\scal{\rho_{\text{eff}}}}{\epsilon_0}
  \\
  \vec{\nabla} \times \left(
    \scal{\vec{B}} - \frac{\vec{M}}{c^{2}\epsilon_0}
  \right)
  &=
  \frac{\scal{\vec{J}_{\text{eff}}}}{c^{2}\epsilon_0} + \frac{1}{c^{2}}
  \frac{\del }{\del t} \left(
    \vec{E} + \frac{\vec{P}}{\epsilon_0}
  \right)
\end{align*}

To simplify the maxwell equations, we define the $\vec{D}$ and $\vec{H}$ field
\begin{empheq}[box=\bluebase]{align*}
  \vec{D} &:= \epsilon_0 \scal{\vec{E}} + \vec{P}\\
  \vec{H} &:= \scal{\vec{B}} - \frac{\vec{M}}{c^{2}\epsilon_0}
\end{empheq}
where from now on we drop the $\scal{}$ symbol and always implicitly assume that we are take averages.

The Maxwell equations then take the form
\begin{empheq}[box=\bluebase]{align*}
  \vec{\nabla}\cdot \vec{D} 
  &= \rho\\
  \vec{\nabla} \times \vec{E} 
  &=
  - \frac{\del \vec{B}}{\del t}
  \\
  \vec{\nabla} \cdot \vec{B}
  &= 0
  \\
  \vec{\nabla} \times \vec{H} 
  &= \frac{\vec{j}}{c^{2}\epsilon_0} + \frac{1}{\epsilon_0 c^{2}} \frac{\del \vec{D}}{\del t}
\end{empheq}

\subsection{Dielectric materials}
A dielectic material is a medium where there is no magnetisation ($\vec{M} = 0$), but usually is able to get a high amount of polarisation $\vec{P}$.
And in particular we have $\vec{H} = \vec{B}$ (averaged).

Experimentally, one sees that when disregarding non-linear terms, the $\vec{E}$ field and the polarisation $\vec{P}$ two follow a linear relationship
\begin{align*}
  \vec{P} = \chi \epsilon_0 \vec{E}
\end{align*}
for some factor $\chi$ known as the \textbf{electric susceptibility} of the dielectric medium.

To simplify the Maxwell equations, we alter the electric permittivity and the speed of light constants by introducing
\begin{align*}
  c_m = \frac{c}{\sqrt{1 + \chi}} \quad \text{and} \quad \epsilon = (1 + \chi) \epsilon_0
\end{align*}
after which the Maxwell equations take on the familiar form
\begin{align*}
  \vec{\nabla} \cdot \vec{E} 
  &= \frac{\rho}{\epsilon}\\
  \vec{\nabla} \times \vec{E} 
  &=  \frac{\del \vec{B}}{\del t}\\
  \vec{\nabla} \cdot \vec{B} &= 0\\
  \vec{\nabla} \times \vec{B} 
  &= \frac{\vec{J}}{\epsilon c_m^{2}} + \frac{1}{c_m^{2}} \frac{\del \vec{E}}{\del t}
\end{align*}

We can solve these how we usually would, except that we need to replace the constants $c$ and $\epsilon_0$ with their dielectric counterparts $c_m$ and $\epsilon$.

For example, we know that the electric and magnetic fields satisfy the wave equation
\begin{align*}
  \left(
    \frac{1}{c_m^{2}} \frac{\del^{2}}{\del t^{2}} - \nabla^{2}
  \right)
  \vec{E}(\vec{x},t) = 0
\end{align*}
which has solutions of the form
\begin{align*}
  \vec{E}(\vec{x},t) = \vec{E}_0 e^{i(\omega t - \vec{k} \cdot \vec{x})}
\end{align*}
which satisfies
\begin{align*}
  k^{2} = \frac{\omega^{2}}{c^{2}_m} = \frac{\omega^{2}}{c^{2}}(1 + \chi)
\end{align*}
which has a phase-velocity of $c_m = \frac{c}{\sqrt{1 + \chi}} =: \frac{c}{n}$, where $n = \sqrt{1 + \chi}$ is also known as the \textbf{refraction index} of the dielectric material.


\subsection{The dielectric susceptibility $\chi$}
We just saw that once we know $\chi$, the maxwell equations can be solved with our usual methods.

We would like to find a nice way to calculate $\chi$, or equivalently, the polarisation $\vec{P}$ of the medium when subjected to an electric field $\vec{E}$.

To do so, we can (to good approximation) view the dielectirc medium as a colllection of dipoles with positive and negative charges, where the dipoles are bound together.

These binding forces require us to calculate quantum effects which we won't address.
We can however make good approximations by modeling the bindings forces as those of a harmonic oscillator.

Because accelerating charges radiate and lose energy, we assume that the harmonic oscillator is a dampened one, i.e. that the force acting on a charge $q$ of a dipole is
\begin{align*}
  F = qE = m(\ddot{x} + \gamma \dot{x} + \omega_0^{2} x)
\end{align*}
where the electric field is given by $E = E_0 e^{i \omega t}$.

The solutions to the differential equation are of the form
\begin{align*}
  x = x_0 e^{i \omega t} \quad \text{for} \quad x_0 = \frac{\frac{qE_0}{m}}{\omega_0^{2} - \omega^{2} + i \omega \gamma}
\end{align*}
The electric dipole moment is then
\begin{align*}
  \vec{p} = q \vec{x} = \frac{\frac{q^{2}}{m}}{\omega_0^{2} - \omega^{2} + i \omega \gamma} \vec{E}
\end{align*}
so assuming a constant charge density $N$, the polarisation of the dielectirc is
\begin{align*}
  \vec{P} = N \vec{p} = N q \vec{x} = \frac{Nq^{2}}{m} \frac{1}{\omega_0^{2} - \omega^{2} + i \omega \gamma}\vec{E}
\end{align*}
and so, the dielectric susceptibility is
\begin{align*}
  \chi = n^{2} - 1 = \frac{N q^{2}}{m \epsilon_0} \frac{1}{\omega_0^{2} - \omega^{2} + i \omega \gamma}
\end{align*}
Note, the index of refraction $n$ can be a complex number, so we sometimes want to split it into its real and imaginy parts
\begin{align*}
  n = n_r + i n_I
\end{align*}
this lets us describe for example, the amplitude of a planar wave,
\begin{align*}
  \vec{E} = \vec{E}_0 e^{i \omega (t - n \hat{k} \cdot \vec{x})} \implies \abs{\vec{E}} = \abs{\vec{E}_0} e^{- \omega \frac{n_I}{c} \hat{k} \cdot \vec{x}}
\end{align*}
which, in the physical case ($n_I > 0$) decreases, as the wave penetrates further into the material.
