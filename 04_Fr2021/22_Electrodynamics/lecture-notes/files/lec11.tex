Now, using our vector potential $\vec{\nabla} \times \vec{A} = \vec{B}$, we can re-write the maxwell equations to read
\begin{align*}
  \vec{E} = - \vec{\nabla}\Phi - \frac{\del \vec{A}}{\del t}
\end{align*}

The maxwell equations also imply charge conservation. 
Given that
\begin{align*}
 \epsilon_0c^{2} \vec{\nabla} \times \vec{B} 
 &= \vec{J} + \frac{\del \vec{E}}{\del t} \epsilon_0\\
 \implies
  0
 &=
 \vec{\nabla} \cdot \vec{J} + \frac{\del \rho}{\del t} = 0
\end{align*}
which when we integrate over the universe, we get
\begin{align*}
  0 
  &= \int_{\text{universe}} d^{3} \vec{x} \vec{\nabla} \cdot \vec{J} + \frac{\del }{\del t}\int_{\text{universe}} d^{3}\vec{x} \rho(\vec{x}
  \\
  &= \underbrace{\int_{\del \text{universe}} d \vec{S} \cdot \vec{J}}_{= 0} + \frac{\del Q_{\text{universe}}}{\del t} 
\end{align*}
where we say that there is no charge flowing ``out of the universe'' meaning twhich says that the total charge in the universe is constant.

Now, with the bac-cab identity for the curl of the curl,we can re-write
\begin{align*}
  c^{2} \vec{\nabla} \times \vec{B} 
  &= 
  \frac{\vec{J}}{\epsilon_0} + \frac{\del \vec{E}}{\del t}\\
  \implies
  c^{2} \left(
    \vec{\nabla}(\vec{\nabla \cdot \vec{A}}) - \nabla^{2} \vec{A}
  \right)
  &= \frac{\vec{J}}{\epsilon_0}
  +
  \frac{\del }{\del t}
  \left(
    - \vec{\nabla}\Phi - \frac{\del \vec{A}}{\del t}
  \right)
\end{align*}
which gives us the relation
\begin{align*}
  \left[\frac{1}{c^{2}}\frac{\del^{2}}{\del t^{2}} - \nabla^{2}\right] \vec{A}
  =
  \frac{\vec{J}}{\epsilon_0c^{2}} - \vec{\nabla} \left(
    \frac{1}{c^{2}} \frac{\del \Phi}{\del t} + \vec{\nabla}\vec{A}
  \right)
\end{align*}
which is not so nice as we simultanously need the scalar potential and the vector potential.
We will see later how we can only write it in terms of the vector potential.

Using our previous equation $\vec{E} = - \vec{\nabla} \Phi - \frac{\del \vec{A}}{\del t}$ and introducing the \textbf{d'Alembert Operator}
\begin{empheq}[box=\bluebase]{align*}
  \square := \frac{1}{c^{2}}\frac{\del^{2}}{\del t^{2}} - \nabla^{2}
\end{empheq}
we will later use symmetry to get
\begin{align*}
  \square \vec{A} = \frac{\vec{J}}{\epsilon_0c^{2}} \qquad
  \square \Phi = \frac{\rho}{\epsilon_0}
\end{align*}
\subsection{Gauge invariance}
The electric and magnetic field remain invariant under a simultanoues transformation known as \textbf{gauge transformation} given by
\begin{align*}
  \Phi \mapsto  \Phi' &= \Phi + \frac{\del }{\del t}f(\vec{x},t)\\
  \vec{A} \mapsto \vec{A}' &= \vec{A} - \nabla f(\vec{x},t)
\end{align*}
and after the substitution we get
\begin{align*}
  \square \Phi'  
  &= 
  \frac{\rho}{\epsilon_0} + \frac{\del }{\del t} \left(
    \frac{1}{c^{2}} \frac{\del \Phi'}{\del t} + \vec{\nabla} \cdot \vec{A}'
  \right)
  \square \vec{A}' = \frac{\vec{J}}{\epsilon_0 c^{2}} - \vec{\nabla} \left(
    \frac{1}{c^{2}} \frac{\del \Phi'}{\del t} + \vec{\nabla} \cdot \vec{A}
  \right)
\end{align*}
The idea then is that we choose $f(\vec{x},t)$ such that
\begin{align*}
  \frac{1}{c^{2}}\frac{\del \Phi'}{\del t} + \vec{\nabla} \cdot \vec{A}' = 0
\end{align*}
We will then find that the d'Alembert operator is consistent with the Lorentz transformation. So under the transformation
\begin{align*}
  \vec{x} \mapsto \vec{x}' &= \gamma(\vec{x} + \vec{v}t )\\
  t \mapsto t' &= \gamma(t - \frac{\vec{x} \cdot \vec{v}}{c^{2}}
\end{align*}
To find such an $f$, we first look for wave solutions
\begin{align*}
  f = f(\vec{n} \cdot \vec{x} + ct)
\end{align*}
then if we set $u := \vec{n} \cdot \vec{x} + ct$, we see
\begin{align*}
  \vec{\nabla}f = \frac{\del f(u)}{\del u}  \cdot \vec{n} \quad \text{and} \quad \frac{1}{c}\frac{\del f}{\del t} = \frac{\del f}{\del u} \cdot 1
\end{align*}
which means that its d'Alembert Operator evaluates to zero.
\begin{align*}
  \square f 
  = 
  \frac{1}{c^{2}} \frac{\del^{2}}{\del t^{2}}f - \nabla^{2}f 
  = 
  \frac{\del^{2}f}{\del u^{2}} - \frac{\del^{2}f}{\del u^{2}} \underbrace{u^{2}}_{=1} 
    = 0
\end{align*}
which is just saying that $f$ is indeed a wave.

If we write the electomagnetic fields as waves
\begin{align*}
  \vec{E} = \vec{e} f(\vec{n} \cdot \vec{x} + ct), \quad \vec{B} = \vec{b} f(\vec{n} \cdot \vec{x} + ct)
\end{align*}
then the maxwell equations show that they are perpendicular to each other, as
\begin{align*}
  0 = \vec{\nabla} \cdot \vec{E} = \hat{e_i}\hat{n}_i \del_u E = 0, \quad \vec{\nabla} \cdot \vec{B} = 0
\end{align*}
each imply that
\begin{align*}
 \vec{e} \cdot \vec{n} = 0 = \vec{b} \cdot \vec{n} 
\end{align*}
and by $\vec{\nabla} \times \vec{E} = - \frac{\del \vec{B}}{\del t}$ we get that
\begin{align*}
  \abs{\vec{E}} = c \abs{\vec{B}}
\end{align*}
The maxwell equations tell us that electromagnetic waves propagate at the speed of light.

Another example of wave functions are spherical waves.
There, we do not have have any angular dependence so
\begin{align*}
  \frac{1}{r} \frac{\del^{2}}{\del r^{2}}r + \underbrace{\frac{\vec{A}(\theta,\phi)}{r^{2}}}_{=0} = \nabla^{2}
\end{align*}
and if we assume then $\square f = 0$, we get
\begin{align*}
  \square f = \left[\frac{1}{c^{2}} \frac{\del^{2}}{\del t^{2}} - \frac{1}{r}\frac{\del^{2}}{\del r^{2}}r\right]f(r,t) = 0
\end{align*}
If we look for solutions of the form $f(r,t) = \frac{\psi(r,t)}{r}$ then the general solution is of the form
\begin{align*}
  \psi(r,t) = A(r + ct) + B(r - ct) \implies f(r,t) = \frac{A(r+ct) + B(r - ct)}{r}
\end{align*}
which we can interpret as ingoing and outgoing waves that decrease inamplitude as $r$ increases.

In many physical situations however is that we only have outgoing waves ($A = 0$).

