
\subsection{Laplace in spherical coordinates}
When solving for Laplace eigenfunctions in spherical coordinate systems, we have seen in MMP-I and Physics III that the laplace operator can be decomposed in a \textbf{radial} and \textbf{angular} part
\begin{align*}
  \nabla^{2} = \frac{1}{r} \frac{\del^{2} }{\del r^{2}} + \frac{A(\theta,\phi)}{r^{2}}
\end{align*}
where the angular part may seem a bit obtuse:
\begin{align*}
  A = \frac{1}{\sin \theta} \frac{\del }{\del \theta} \sin \theta \frac{\del }{\del \theta} + \frac{1}{\sin^{2} \theta} \frac{\del^{2} }{\del \phi^{2}}
\end{align*}
but it turns out that solving it isn't as bad as it seems.
Let's denote the eigenfunctions to the $\phi$ part, with $\ell_m$, so
\begin{align*}
  \frac{\del^{2}}{\del \phi^{2}}\ell_m(\phi) = - m^{2} \ell_m(\phi)
\end{align*}
which is just a harmonic oscillator. The solution can be found using the Euler ansatz and it is just given by
\begin{align*}
  \ell_m(\phi) = \frac{1}{\sqrt{2 \pi}} e^{im \phi}
\end{align*}
Moreover, because the angle $\phi$ should be $2 \pi$ periodic, we should have
\begin{align*}
  \ell_m(\phi) = \ell_m(\phi + 2 \pi) \implies e^{0} = e^{2 \pi i m} \implies m = 0, \pm 1, \pm 2, \ldots
\end{align*}
so we can restrict $m$ to only take integer values.
We can also check for orthogonality.
\begin{align*}
  \scal{\ell_m,e_n} = \int_0^{2 \pi}\psi \ell_m^{\ast}(\phi) e_n(\phi) = \delta_{mn}
\end{align*}
For the angular operator $\hat{A}$, let's denote its Eigenfunctions with $Y$, and expand it it terms of the eigenfunctions $\ell_m$, where the coeffients may depend on $\theta$:
\begin{align*}
  \hat{A}(\theta,\phi) = Y(\theta,\phi) = \rho Y(\theta,\phi)\\
  Y(\theta,\phi) = \sum_{m \in \Z}c_m(\theta) \ell_m(\phi)
\end{align*}
Inserting $c_m(\theta)\ell_m(\phi)$, we want to solve for
\begin{align*}
  \rho c_m(\theta) \ell_m(\phi)
  &=
  \left[
    \frac{1}{\sin \theta} \frac{\del }{\del \theta} \sin \theta \frac{\del }{\del \theta} + \frac{1}{\sin^{2} \theta} \frac{\del^2}{\del \phi^{2}}
\right]
c_m(\theta) \ell_m(\phi)
\\
  &=
  \left[
    \frac{1}{\sin \theta} \frac{\del }{\del \theta} \sin \theta \frac{\del }{\del \theta} + \frac{1}{\sin^{2} \theta} (-m^{2})
  \right]
c_m(\theta) \ell_m(\phi)
\end{align*}
and we can divide by $\ell_m(\phi)$ to completely remove the $\phi$ dependence in the differential equation.
Using the insights of past mathematicians, it is easier to replace $\rho$ with $-l(l+1)$, and re-label $c_m(\theta)$ with $P_{l,m}(\theta)$ so we are solving the equation
\begin{align*}
  \left[
    \frac{1}{\sin \theta} \frac{\del }{\del \theta} \sin \theta \frac{\del }{\del \theta} - \frac{m^{2}}{\sin^{2} \theta}
\right]
P_{l,m}(\theta)
=
-l(l+1) P_{l,m}(\theta)
\end{align*}
Let's first solve it for $m = 0$ to get solutions $P_l^{0} = P_l$ and see if we can generalize from there.
Because $\theta$ must be $\pi$-periodic, we make the Ansatz $P_l(\theta) = P_l(\cos \theta)$ and then we take the taylor expansion of $P_l$.
\begin{align*}
  P_l(\cos \theta) = \sum_{n \in \N}\beta_l^{n}(\cos \theta)^{n}
\end{align*}
and we will show in the exercise classes that solutions only exist for $l = 0, 1, \ldots$.
The resulting functions are called the \textbf{Legendre Polynomials} are are given by
\begin{align*}
  P_l(x) = \frac{1}{2^{l}l!} \frac{\dd^{l}}{\dd x^{l}} (x^2 - 1 )^{l}
\end{align*}
They also are orthogonal and are complete.

Now for arbitrary $m$, the solution is similar but not covered here. The solutions $P_{l,m}$ are called the \textbf{associated Legendre Polynomials}.
\begin{align*}
  P_{l,m}(x) = \frac{1-x^{2})^{m/2}}{2^{l}l!} \frac{\dd^{m+l}}{\dd x^{m+l}} (Z^{2} - 1)^{l} \stackrel{m>0}{=} (1 - Z^{2})^{m/2} \frac{\dd^{m}}{\dd x^{m}} P_l(x)
\end{align*}
Then, for the eigenfunctions $Y(\theta,\phi)$ of $\hat{A}(\theta,\phi)$ we take the Ansatz that a complete basis of solutions can be obtained by taking a product of the $P_{l,m}$ and the $\ell_m$.
This gives us the \textbf{spherical harmonics} $Y_{l,m}(\theta,\phi)$ given by
\begin{empheq}[box=\bluebase]{align*}
  Y_{l,m}(\theta,\phi) = \frac{(-1)^{m}}{\sqrt{2 \pi}}
  \sqrt{
    \frac{2l + 1}{2} \frac{(l-m)!}{(l+m)!}
  }
  P_{l,m}(\cos \theta) e^{i m \phi}
\end{empheq}
We argued that the $e^{im \phi}$ are complete and that $P_{l,m}$ are also complete in their respective spaces.
We have proved in MMP-I that the $Y_{l,m}$ for $m = -l,\ldots, l$ form an Orthonormal Basis of $L^{2}(\mathbb{S}^{2})$, that is
\begin{align*}
  \scal{Y_{l'm'},Y_{lm}} = \int_{\mathbb{S}^{2}} d \Omega Y_{l'm'}^{\ast}Y_{lm} = \delta_{l'l} \delta_{m'm}
\end{align*}
Completeness in particular means that
\begin{align*}
  \sum_{l=0}^{\infty}\sum_{m=-l}^{l}Y_{l,m}^{\ast}(\theta,\phi)Y_{l,m}(\theta',\phi') = \delta(\phi-\phi')\delta(\cos \theta'- \cos \theta)
\end{align*}

