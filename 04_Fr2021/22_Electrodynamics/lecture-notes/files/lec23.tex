We define what it means to take the average over a field $\vec{F}(\vec{x},t)$ a distance via
\begin{empheq}[box=\bluebase]{align*}
  \scal{F(\vec{x},t)}:= \int d^{3}\vec{y} f(\vec{y}) F(\vec{x}- \vec{y},t) = (F(-,t) \ast f(-))(\vec{x})
\end{empheq}
where $f(\vec{y})$ is some smooth, non-negative \emph{weighing factor}.

This definition is just the \emph{convolution} we know from harmonic analysis.

For example, to describe the total curren density $j^{\mu}$, we split it into the atomic and free parts, where we average the atomic part.
\begin{align*}
  j^{\mu} = j_{\text{free}}^{\mu} + j_{\text{atomic}}^{\mu} \simeq j_{\text{free}}^{\mu} + \scal{j_{\text{atomic}}^{\mu}}
\end{align*}

\subsection{Charge density}
Consider a collection of free charges $(q_i)_{i \in \text{ free }}$ at positions $(\vec{x}_i)_{i \in \text{free}}$, 
aswell as charges bound to molecules at positions $(\vec{x}_n)_{n \in \text{molec}}$.
Each of these molecules has charges $q_{n,1}, \ldots, q_{n,r_n}$ for some $r_n \in \N$ at positions $(\vec{x}_n + \vec{x}_{n,k})_{k = 1,\ldots,r_n\}}$.
Where $\vec{x}_{n,k}$ is the position of the charge relative to the center of the molecule $\vec{x}_n$.

We already know what the charge density that the free charges generate looks like
\begin{align*}
  \rho_{\text{free}}(\vec{x}) = \sum_{i \in \text{ free }} q_i \delta(\vec{x} - \vec{x_i})
\end{align*}

We fix some molecule and look at its contribution to the bound charge density.

If the molecule has $r$ charges $q_1, \ldots, q_m$ bound to it, and is centered around the point $\vec{x}_{\text{mol}}$, then 
\begin{align*}
  \rho_{\text{mol}}(\vec{x}) = \sum_{k = 1}^{r} q_{k} \xi(\vec{x} - \vec{x}_{\text{mol}} - \vec{x}_k )
\end{align*}
so taking the average, on both sides, we get 
\begin{align*}
  \scal{\rho_{\text{mol}}}(\vec{x}) 
  &= \sum_{k=1}^{r} q_k\scal{\delta(\vec{x} - \vec{x}_{\text{mol}} - \vec{x}_k)}\\
  &= \sum_{k=1}^{r} q_k \int d^{3} \vec{y} f(\vec{y}) \delta(\vec{x} - \vec{x}_{\text{mol}} - \vec{x}_{k} - \vec{y})\\
  &=
  \sum_{k=1}^{r} q_k f(\vec{x} - \vec{x}_{\text{mol}} - \vec{x}_k)
\end{align*}
Performing a taylor expansion for $f$:
\begin{align*}
  f(\vec{x} - \vec{x}_{\text{mol.}} - \vec{x}_k) = f(\vec{x} - \vec{x}_{\text{mol}}) - \vec{x}_k \cdot \vec{\nabla} f(\vec{x} - \vec{x}_{\text{mol}}) + \mathcal{O}(\vec{x}_k^{2})
\end{align*}
we can write the average of the charge density as
\begin{align*}
  \scal{
    \rho_{\text{mol}}
  }
  (\vec{x}) 
  = q_{\text{tot}}f(\vec{x} - \vec{x}_{\text{mol}}) - \sum_{k=1}^{n}q_k \vec{x}_k \cdot \vec{\nabla}f(\vec{x} - \vec{x}_{\text{mol}}) + \mathcal{O}(\vec{x}_k^{2})
\end{align*}
where $q_{\text{tot}} = \sum_{k=1}^{r} q_k$ is the total charge of the molecule.
Using the definition of the \textbf{electric dipole moment} of the molecule
\begin{align*}
  \vec{p}_{\text{mol}} := \sum_{k=1}^{r} q_k \vec{x}_k
\end{align*}
this gets the compact form
\begin{align*}
  \scal{\rho_{\text{mol}}} = q_{\text{tot}} f(\vec{x} - \vec{x}_{\text{mol}}) - \vec{p} \cdot \vec{\nabla}f(\vec{x} - \vec{x}_{\text{mol}}) + \mathcal{O}(\vec{x}_k^{2})
\end{align*}
Summing over all molecules,
we get that the total bound charge density averaged over atomic distances is up to a first order approximation given by
\begin{align*}
  \scal{\rho_{\text{bound}}} 
= 
\scal{
  \sum_{n \in \text{molec}}q_{n,\text{tot}} \delta(\vec{x} - \vec{x}_n)
}
- 
  \scal{
    \sum_{n \in \text{molec}}
    \vec{p}_n \cdot \vec{\nabla} \delta(\vec{x}- \vec{x}_n)
  }
  + \mathcal{O}(\vec{x}_{n,k}^{2})
\end{align*}
By defining the \textbf{effective charge density} $\rho_{\text{eff}}$
\begin{empheq}[box=\bluebase]{align*}
  \rho_{\text{eff}} &:= 
  \scal{
    \sum_{i \in \text{free}}q_i \delta(\vec{x} - \vec{x}_i)
  }
  +
  \scal{
    \sum_{n \in \text{molec}}
    q_{n,\text{tot}}
    \delta(\vec{x} - \vec{x}_n)
  }
  \\
  &= \scal{\rho_{\text{free}}} + \scal{\rho_{\text{molecules}}}
\end{empheq}
aswell as the \textbf{polarised charge density}
\begin{empheq}[box=\bluebase]{align*}
  \rho_{\text{pol}} 
  = - \vec{\nabla} \cdot \vec{P}
  \quad \text{for} \quad 
  \vec{P} = \scal{\sum_{n \in \text{molec}} \vec{p}_n \delta(\vec{x} - \vec{x}_n)}
\end{empheq}
where $\vec{P}$ is called the \textbf{polarisation} of the medium, this lets us decompose the total charge density into the effective and polarised terms:

\begin{empheq}[box=\bluebase]{align*}
  \scal{\rho} 
  &\approx \rho_{\text{eff}} + \rho_{\text{pol}} \\
  &=
  \scal{\rho_{\text{free}}} + \scal{\rho_{\text{molecules}}}- \vec{\nabla} \cdot \vec{P}
\end{empheq}
where the approximation is up to first order with respect to $\vec{x}_{n,k}$.



\subsection{Current density}
Just like before, we consider currents generated by:
\begin{align*}
  \text{free moving charges}: \quad (q_i)_{i \in \text{free}} \text{ at positions } (\vec{x}_i)_{i \in \text{free}} \text{ with velocities } (\vec{v}_i)_{i \in \text{free}}
\end{align*}
aswell currents bound to molecules:
\begin{align*}
  \text{bound currents}: \quad \text{molecules with centers at } (\vec{x}_n)_{n \in \text{molec}} \text{ moving on average with velocity } (\vec{v}_n)_{n \in \text{molec}}
\end{align*}
where each of these molecules has:
\begin{align*}
  r_n \text{ charges } q_{n,1}, \ldots q_{n,r_n} \text{ at positions } (\vec{x}_n + \vec{x}_{n,k})_{k=1,\ldots,r_n} \text{ moving with velocities } (\vec{v}_n + \vec{v}_{n,k})_{k=1,\ldots,r_n}
\end{align*}
The freely moving charges generate a current just as we already know:
\begin{align*}
  j_{\text{free}} = \sum_{i \in \text{free}} q_i \vec{v}_i\delta(\vec{x} - \vec{x}_i)
\end{align*}
And if we again look at a single molecule at position $\vec{x}_{\text{mol}}$ with velocity $\vec{v}_{\text{mol}}$.

Then the current the molecule generates will be
\begin{align*}
  \vec{j}_{\text{mol}} = \sum_{k=1}^{r} q_k(\vec{v}_{\text{mol}} + \vec{v}_k) \delta(\vec{x} - \vec{x}_{\text{mol}} - \vec{x}_k)
\end{align*}
which means that the average is
\begin{align*}
  \scal{\vec{j}_{\text{mol}}}(\vec{x}) 
  &= \sum_{k=1}^{r}q_k(\vec{v}_{\text{mol}} + \vec{v}_k) \int d^{3}\vec{y}f(\vec{y}) \delta(\vec{x} - \vec{x}_{\text{mol}} - \vec{x}_k - \vec{y})\\
  &= \sum_{k=1}^{r}q_k (\vec{v}_{\text{mol}} + \vec{v}_k) f(\vec{x} - \vec{x}_{\text{mol}} - \vec{x}_k)
\end{align*}
Doing the same taylor expansion as in the previous section and using the identity
\begin{align*}
  (\vec{a} \times \vec{b}) \times \vec{\nabla}f = \vec{a}(\vec{b} \cdot \vec{\nabla}f) - \vec{b}(\vec{a} \cdot \vec{\nabla}f)
\end{align*}
gives us the average of the charge density of this molecule:
\begin{align*}
  \scal{j_{\text{mol}}}(\vec{x}) 
  &=
  \scal{
    q_{\text{tot}} \delta(\vec{x} - \vec{x}_{\text{mol}}) \vec{v}_{\text{mol}}
  }
  +
  \frac{d }{d t} \scal{
    \vec{p}_{\text{mol}} \vec{v}_{\text{mol}}\delta(\vec{x} - \vec{x}_{\text{mol}}
}\\
  &+
  \vec{\nabla} \times \scal{
    \sum_{k=1}^{r} \frac{q_k}{2} \left(
      \vec{x}_k \times \vec{v}_k
    \right)
  }
  + \scal{
    \big(\vec{p}_{\text{mol}} \times \vec{v}_{\text{mol}}\big)
    \vec{v}_{\text{mol}}\delta(\vec{x} - \vec{x}_{\text{mol}})
  }
  + \ldots
\end{align*}
where $q_{\text{tot}} = \sum_{k=1}^{r} q_k$ is the total charge and $\vec{p}_{\text{mol}} = \sum_{k=1}^{r}q_k \vec{x}_k$ is the dipole moment of the molecule.

Taking the sum over all molecules, the total bound current density averaged over atomic distances is
\begin{align*}
  \scal{j_{\text{bound}}} 
  &= 
  \scal{
    \sum_{n \in \text{molec}} q_{n,\text{tot}} \vec{v}_n\delta(\vec{x} - \vec{x}_n)
  }
  + \frac{d }{d t} \scal{
    \sum_{n \in \text{molec}} \vec{p}_n \vec{v}_n \delta(\vec{x} - \vec{x}_n)
  }\\
  &+ \vec{\nabla} \times \scal{
    \sum_{n \in \text{molec}} \sum_{k=1}^{r_n}
    \frac{q_k}{2} \left(
      \vec{x}_k \times \vec{v}_k
    \right)
  }
  + \scal{
    \sum_{n \in \text{molec}}
    (\vec{p}_n \times \vec{v}_n) \vec{v}_n \delta(\vec{x} - \vec{x}_n)
  }
  + \ldots
\end{align*}
To clean this up a bit, we define the \textbf{effective current density}
\begin{empheq}[box=\bluebase]{align*}
  \vec{j}_{\text{eff}} 
  &:=
  \scal{\sum_{i \in \text{free}}
  q_i \vec{v}_i \delta(\vec{x} - \vec{x}_i)
}
  + \scal{
    \sum_{n \in \text{molec}}
    q_{n,\text{tot}}\vec{v}_n \delta(\vec{x} - \vec{x}_n)
  }\\
  &= \scal{\vec{j}_{\text{free}}}
  +
  \scal{\vec{j}_{\text{molecules}}}
\end{empheq}
, the \textbf{polarised current density}
\begin{empheq}[box=\bluebase]{align*}
  \vec{j}_{\text{pol}} := \frac{\del \vec{P}}{\del t} \quad \text{for} \quad \vec{P} = \scal{\sum_{n \in \text{molec}}
    \vec{p}_n \delta(\vec{x} - \vec{x}_n)
  }
\end{empheq}
aswell as the \textbf{magnetized current density}
\begin{empheq}[box=\bluebase]{align*}
  \vec{j}_{\text{mag}} := \vec{\nabla} \times \vec{M} \quad \text{for} \quad \vec{M} := \scal{\sum_{n \in \text{molec}}\vec{m}_n \delta(\vec{x} - \vec{x}_n)}
\end{empheq}
, where $\vec{m}_n$ is the \textbf{magnetic moment} of the molecule given by
\begin{empheq}[box=\bluebase]{align*}
  \vec{m}_n := \sum_{k=1}^{r_n} \frac{q_k}{2}(\vec{x}_k \times \vec{v}_k)
\end{empheq}

In the end, we are left with the nice compact formula for the total current density averaged over atomic distances:
\begin{empheq}[box=\bluebase]{align*}
  \scal{\vec{j}} 
  &\approx
  \vec{j}_{\text{eff}}
  + \vec{j}_{\text{pol}}
  + \vec{j}_{\text{mag}}
  \\
  &=
  \scal{\vec{j}_{\text{free}}} + \scal{\vec{j}_{\text{molecules}}}
  + \frac{\del \vec{P}}{\del t}
  + \vec{\nabla}\times \vec{M}
\end{empheq}
with the approximation being good when the $\vec{v}_{n,k}$ and $\vec{x}_{n,k}$ are small.
