\section{Statistics}

\subsection{Point estimates}
Given some data $\bm{x} = (x_{1}, \ldots, x_{n})$, we want to find out what random variables $\bm{X} = (X_{1}, \ldots, X_{n})$ could have generated this data.

To do so, we consider a collection of distributions $(\mu_{\theta})_{\theta \in \Theta}$ on P-spaces $(\R^{n},\mathcal{B}(\R^{n}),\IP_{\theta})_{\theta \in \Theta}$.
 
A \textbf{point estimate} is therefore a function
\begin{align*}
  T: \R^{n} \to \Theta
\end{align*}
which to every possible dataset $\bm{x}$ attributes some P-space $(\R^{n},\mathcal{B}(\R^{n}),\IP_{\theta})$ with random variables $\bm{X}$ and their corresponding distributions $\mu_{\theta}$.

Usually, we put some structure on $\Theta$.
