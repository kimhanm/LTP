\section{Probability theory}
\subsection{Introduction}

Although even the greeks and 17-th century mathematicians 
a mathematical framework to discuss probability and statistics was only developed in the 19-th and 20-th century.

One reason this might have been the case could be the empirical nature of statistics. 
Probabily theory was dependent on data and was difficult to axiomatize and whose proofs often relied on human intuition over formal derivation.

One difficulty was that in order to find a suitable notion of probability on an infinite dimensional vector space.
For example, to think about the probability that a particle takes any given Brownian motion path is to find a notion of measure on the set $C([0,\infty), \R)$.

As a consequence, the first fields medal that was awarded for a contribution to probability was in 2002 and 2006. Almost 70 years after the first medal was awarded.

So the axiomatisation was only done in the early 20-eth century and reached the center of mathematics in the 21-st century.


\subsection{Probability Space}

\begin{dfn}[Probability space]
  A \textbf{Probability space} (or P-space) is a triple $(\Omega,\mathcal{A}, \IP)$ consisting of a non-empty set $\Omega$, 
  a $\sigma$-Algebra $\mathcal{A} \subseteq 2^{\Omega}$,
  and a $\sigma$-additive probability measure $\IP: \mathcal{A} \to [0,1]$.
  This means that the following are satisfied:
  \begin{enumerate}
    \item $\Omega \in \mathcal{A}$
    \item $\forall A \in \mathcal{A}: A^{c} \in \mathcal{A}$
    \item $\forall (A_i)_{i \in \N}, A_i \in \mathcal{A}$, $\bigcup_{i \in \N} A_i \in \mathcal{A}$
    \item $\IP[\emptyset] = 0, \IP[\Omega] = 1$
    \item $\forall (A_i)_{i \in \N}, A_i \in \mathcal{A}$, if all $A_i$ are disjoint, then
      $\IP\left(\bigsqcup_{i \in \N} A_i\right) = \sum_{i \in \N} \IP[A_i]$
  \end{enumerate}
\end{dfn}

It is useful (and sometimes necessary) to be able to talk about the ``smallest measureable sets''.
\begin{dfn}[]
A subset $A \in \mathcal{A}$ of $\Omega$ is \textbf{atomic}, if 
\begin{align*}
  B \in \mathcal{A}, B \subseteq A \implies B = \emptyset \lor B = A
\end{align*}
We denote the set of all atoms as $\text{Atom}(\mathcal{A})$. 
In particular, if $\mathcal{A} = \mathcal{P}(X)$, then the atoms are exactly the singletons.
We can uniquely decompose any element $B \in \mathcal{A}$ into a disjoint union of atoms.

\end{dfn}

\begin{rem}[]
  Often, when $\Omega$ is finite, it is useful to set $\mathcal{A} = \mathcal{P}(\Omega)$ i.e to let all subsets of $\Omega$ be measureable. In particular, the atoms are then exactly the singletons $\omega \in \Omega$. If that is the case, we can define a \textbf{weight} function
\begin{align*}
  p: \Omega \to [0,1], \quad p(\omega) := \IP[\{\omega\}]
\end{align*}
  Then for any subset $A \subseteq \Omega$, its probability measure can be calculated as follows
  \begin{align*}
    \IP[A] = \sum_{\omega \in A}p(\omega)
  \end{align*}
  or in the general case, for $B \in \mathcal{A} \neq \mathcal{P}(\Omega)$:
  \begin{align*}
    \IP[B] = \sum_{A \in \text{Atom}(\mathcal{A})} \IP[A \cap B]
  \end{align*}
\end{rem}

It can easily shown that the following axioms are satified:
\begin{lem}[] \label{lem:IP-properties}
  Let $(\Omega,\mathcal{A},\IP)$ be a P-space and $A,B \in  \mathcal{A}$. Then
\begin{enumerate}
  \item $\IP[A^{c}] = 1 - \IP[A]$
  \item $\IP[A\cup B] = \IP[A] + \IP[B] - \IP[A \cap B]$
  \item $\IP\left(\bigcup_{i = 1}^n A_i\right) = \sum_{k=1}^n (-1)^{k+1} \sum_{1 \leq i_1 <\ldots < i_k \leq n} \IP[A_{i_1} \cap \ldots \cap A_{i_k}]$
  \item $A \subseteq B \implies \IP[A] \leq \IP[B]$
\end{enumerate}
\end{lem}
\begin{proof}
  We only show the proof for $\mathcal{A} = \mathcal{P}(\Omega)$. For generalized $\mathcal{A}$, the proof is more or less the same.
  \begin{enumerate}
    \item By $\sigma$-additivity of $\IP$, and $\IP[\Omega] = 1$, we have
      \begin{align*}
        \IP[\Omega] = \IP[A \sqcup A^{c}] = \IP[A] + \IP[A^{c}] = 1
      \end{align*}
    \item With the decomposition $A \cup B = (A \setminus B) \sqcup (A \cap B) \sqcup (B \setminus A)$, we find
    \begin{align*}
      \IP[A \cup B] + \IP[A \cap B] = \IP[A \setminus B] + \IP[A \cap B] + \IP[B \setminus A] + \IP[A \cap B] = \IP[A] + \IP[B]
    \end{align*}
  \item We use induction over $n$. We find
    \begin{align*}
      \IP\left[\bigcup_{i=1}^{n+1}A_i\right]
      &\stackrel{(b)}{=}
      \IP\left[\bigcup_{i=1}^{n}A_i\right] + \IP[A_{n+1}] - \IP \left[\bigcup_{i=1}^{n}(A_i \cap A_{n+1})\right]\\
      &\stackrel{\text{ind}}{=} 
      \underbrace{\left(\sum_{k=1}^{n}(-1)^{k+1} \sum_{1 \leq i_1 < \ldots < i_k \leq n} \IP[A_{i_1} \cap \ldots \cap A_{i_k}] \right)}_{=: \Sigma_1} + \IP[A_{n+1}]\\
      &- 
      \underbrace{
      \left(
        \sum_{m=1}^{n}
        (-1)^{m+1} \sum_{1 \leq i_1 < \ldots < i_m \leq n} \IP[A_{n+1} \cap A_{i_1} \cap \ldots \cap A_{i_m}]
      \right)
    }_{=: \Sigma_2}
    \end{align*}

    Comparing this with the predicted result
    \begin{align*}
      \sum_{k=1}^{n+1} (-1)^{k+1} \sum_{1 \leq i_1 < \ldots < i_k \leq n+1} \IP[A_{i_1} \cap \ldots \cap A_{i_k}]
    \end{align*}
    we can see that they are the same:
    \begin{itemize}
      \item For $k = 1$, the summands $\IP[A_1],\ldots,[A_n]$ are present in the first sum $\Sigma_1$ and $\IP[A_{n+1}]$ is also present.
      \item For $k = n+1$, there is only one term $\IP[A_1 \cap \ldots \cap A_{n+1}]$, which is present in $\Sigma_2$ with sign $(-1)^{(n+1)+1}$.

      \item For $2 \leq k \leq n$, we can either chose $k$ elements from $A_1, \ldots,A_n$ (a summand in $\Sigma_1$), or chose always chose $A_{n+1}$ and $k-1$ other elements from $A_1,\ldots,A_n$ (a summand in $\Sigma_2$).

        The negative sign before $\Sigma_2$ ensures that the term $\IP[A_{n+1} \cap A_{i_1} \cap \ldots \cap A_{i_{k-1}}]$ has the correct sign $(-1)^{k+1}$
    \end{itemize}

  \item This follows directly from $B = (B \setminus A) \sqcup A$.
  \end{enumerate}
\end{proof}

\begin{ex}[Laplace Model]
To model many situations, we can set the space $\Omega$ such that every singleton has equal probability to occur.
In the \textbf{Laplace Model}, we set
\begin{align*}
  p(\omega) = \frac{1}{\abs{\Omega}}
\end{align*}
This obviously only works for finite $\Omega$. 
For a subset $A \subseteq \Omega$, then $\IP[A] = \frac{\abs{A}}{\abs{\Omega}}$.
\end{ex}

\begin{ex}[Bernoulli-Experiment]
  Consider the experiment where we measure the number of calls during a given time at a call center. So $\Omega = \{0,1,2,\ldots\}$. 
  We can set the weights to be 
  \begin{align*}
    p(\omega) = e^{-\lambda} \frac{\lambda^{\omega}}{\omega!}, \quad \text{for} \quad \omega \in \Omega
  \end{align*}
  , where $\lambda > 0$ is some parameter. This model is called the \textbf{Poisson-Distribution}.
  Note that the above defintiion is well defined, as
  \begin{align*}
    \IP[\Omega] = \sum_{n \in \N} e^{-\lambda} \frac{\lambda^{n}}{n!} = e^{- \lambda} \cdot e^{\lambda} = 1
  \end{align*}
  If we let $A = \{1,2, \ldots\}$ to be the outcome, where we get at least one call, then we see that
  \begin{align*}
    \IP[A] = 1 - \IP[A^{c}] = 1 - \IP[\{0\}) = 1 - e^{- \lambda}
  \end{align*}
\end{ex}

What is often very useful is to associate to each possible outcome $\omega \in \Omega$ a real value. This gives us the following definition.

\begin{dfn}
  A \textbf{random variable} is a real-valued function $X: \Omega \to \R$.


  If the image $X(\Omega)$ is also countable, then we can turn the probability measure $\IP$ on $\Omega$ into a probability measure on the image $X(\Omega)$ where for $x \in X(\Omega)$ we define
  \begin{align*}
    \IP[X = x] := \IP\left[\{\omega\in \Omega \big\vert X(\omega) = x\}\right] = \IP[X^{-1}(x)]
  \end{align*}
  which reads as: ``The probability that the random variable $X$ obtains the value $x \in X(\Omega)$ is the probability measure of the preimage of $x$ under the function $X$.''

  In the case for general $\mathcal{A} \neq \mathcal{P}(\Omega)$, we must require that $X$ be constant on every atom, or equivalently, that the preimage of singletons under $X$ are $\mathcal{A}$-measureable. 
  \begin{align*}
    \IP[X = x] := \IP \left[
      \{A \in \text{Atom}(\Omega) \big\vert X(A) = x\}
    \right]
    = \IP[X^{-1}(x)]
  \end{align*}
  Other common notation for $\IP[X = x]$ is $\mu_X(x), P(X=x), P(x), P_x$ etc.
\end{dfn}
%Given a P-space $(\Omega,\mathcal{A},P)$ we can restrict the probability measure $\IP: \mathcal{A} \to [0,1]$ to a subset $\mathcal{B} \subseteq \mathcal{A}$ to obtain a new P-space $(\Omega, \mathcal{B}, \IP_{\mathcal{B}})$.
%
%Note that a random variable $X$ with respect to the $\sigma$-Algebra $\mathcal{A}$ need not be a random variable with respect to $\mathcal{B}$, as the preimages might not be measureable anymore.
%
%Say we take a random variable $X$ and chose a random element (or Atom) of $\Omega$. What would be the best guess on the value that $X$ takes on $\omega$?

\begin{dfn}
  For a random variable $X: \Omega \to \R$ we define the \textbf{expectation value} of $X$ to be
  \begin{align*}
    \E[X] := \sum_{\omega \in \Omega}X(\omega) p(\omega) \quad \left(\text{or} \quad \E[X] := \sum_{A \in \text{Atom}(\Omega)} X(A)\IP[A]
    \right)
  \end{align*}
  If $X$ can take on negative values, the right hand side can be problematic because we can run into non-absolute convergence.
  To remedy this, we can divide $X$ into its positive and negative parts 
  \begin{align*}
    X(\omega) = X^+(\omega) - X^-(\omega) = \max\{X(\omega), 0\} + \{\min\{0, X(\omega)\}
  \end{align*}
  and then change the definition of the expectation as follows:
  \begin{align*}
    \E[X] := \E[X^+] - \E[X^-]= \sum_{X(\omega) > 0}X(\omega) p(\omega) - \sum_{X(\omega) < 0}\underbrace{-X(\omega)p(\omega)}_{\geq 0}
  \end{align*}
  as long as both sums aren't infinite.
\end{dfn}

The expectation value can also be expressed in terms of the distribution of $X$:
\begin{empheq}[box=\bluebase]{align*}
  \E[X] 
  %= \sum_{x \in X(\Omega)} \sum_{\omega: X(\omega) = x} X(\omega) p(\omega) 
  =
  \sum_{x \in X(\Omega)} x \cdot \IP[X = x]
\end{empheq}
which is often easier to compute when we know how $X$ is distributed.

\begin{ex}[]
  In the call-center example from before, let $X$ be the number of calls, so $X(\omega) = \omega$. Then
  \begin{align*}
    \E[X] 
    = \sum_{k \in \N} k \IP[X = k] 
    = \sum_{k=0}^{\infty}k e^{-\lambda} \frac{\lambda^{k}}{k!} 
    = \lambda e^{-\lambda} \underbrace{\sum_{k=1}^{\infty}\frac{\lambda^{k-1}}{(k-1)!}}_{=e^{\lambda}} = \lambda
  \end{align*}
  so the paramter $\lambda$ gives us the expected number of calls.
\end{ex}

\begin{ex}[]
  In an insurance contract the cost to the insurance company is
  \begin{align*}
    X = \left\{\begin{array}{ll}
      c & \text{if the event $A$ occurs} \\
      0 & \text{else}
    \end{array} \right.
  \end{align*}
  The premiums the insured have to pay should then be the expectation value
  \begin{align*}
    \E[X] = c \cdot \IP[A] + 0 \cdot \IP[A^{c}] = c \cdot \IP[A]
  \end{align*}
\end{ex}
More generally, we can define the \textbf{indicator function}
\begin{align*}
  I_B(\omega) := \left\{\begin{array}{ll}
      1 & \text{for } \omega \in B \\
     0 & \text{for } \omega \notin B
  \end{array} \right.
\end{align*}
  for subsets $B \subseteq \Omega$, then we can write $X = c I_B$, so the expectation value is
  \begin{align*}
    \E[cI_B] = c \IP[B] \quad \text{for} \quad c \in \R, B \in \mathcal{A}
  \end{align*}
This immediately gives us the following lemma

\begin{lem}[Linearity of the expectation value]
  Let $X,Y: \Omega \to \R$ be random variables and $a,b \in \R$. Then
  \begin{align*}
    \E[aX + bY] = a\E[X] + bE[Y]
  \end{align*}
\end{lem}
\begin{proof}
  This follows directly from the definition of the expectation value:
  \begin{align*}
    \E[aX + bY] 
    = \sum_{\omega \in \Omega}(aX + bY)(\omega) p(\omega) 
    = a\sum_{\omega \in \Omega}X(\omega) p(\omega) + b \sum_{\omega \in \Omega}Y(\omega) p(\omega)
    =a\E[X] + b\E[Y]
  \end{align*}
\end{proof}

Another useful result when calculating the expectation value is
\begin{lem}[]
If $X$ only takes values in $\N$, then
\begin{align*}
  \E[X]= \sum_{j \in \N} \IP[X > j]
\end{align*}
\end{lem}
\begin{proof}
  We can use the representation using the distribution to find
  \begin{align*}
    \E[X] &= \sum_{k=1}^{\infty} k \IP[X = k]\\
    &= \sum_{k=1}^{\infty}\sum_{j=0}^{k-1}\IP[X = k]\\
    &= (\IP[X=1] + \IP[X=2] + \ldots) + (\IP[X=2] + \IP[X=3] + \ldots) + \ldots\\
    &= \sum_{j=0}^{\infty}\sum_{k=j+1}^{\infty}\IP[X=k]\\
    &=\sum_{j=0}^{\infty} \IP[X > j]
  \end{align*}
\end{proof}

