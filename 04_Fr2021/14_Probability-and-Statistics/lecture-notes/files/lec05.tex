 
\subsection{Random Walks}
The random walk is a model for the random motion of a particle in an $n$-dimensional grid $\Z^{n}$ starting at the origin.
At every \emph{period}, the particle can move in any direction.

Let's first check out the one-dimensional case with $N$ periods:
Let $\Omega$ to be the set of all binary sequences of length $N$
\begin{align*}
  \Omega = \{\omega = (x_{1}, \ldots, x_{N}) \big\vert x_i \in \{\pm 1\}\}
\end{align*}
and let $X_k(\omega)$ to the the $k$-th component of $\omega \in \Omega$.
The position after the $n$-th period will then be
\begin{align*}
  S_n(\omega) = \sum_{k=1}^{n} X_k(\omega)
\end{align*}
For arbitrary periods ($N \to \infty$) we run into a problem. 
The set $\Omega$ is then bijective to $\mathcal{P}(\N)$, which is an uncountable set.
It is therefore difficult to find a good probability measure on $\Omega$.

For finite $N$ however, the set $\Omega$ has cardinality $\abs{\Omega} = 2^{N}$ so we can just use the equal distribution $\IP$ given by
\begin{align*}
  \IP[A] = \frac{\abs{A}}{\abs{\Omega}} = 2^{-N} \abs{A}
\end{align*}
It is easy to show the following
\begin{enumerate}
  \item $\IP[X_k = +1] = \frac{1}{2}$
  \item $\IP[X_{k_1}=x_{k_1}, \ldots X_{k_l} = x_{k_l}] = 2^{-l}$ for any $1 \leq k_{1} < \ldots < k_{l} \leq N$.
    In particular, the random variables $X_{1}, \ldots, X_{N}$ are all independent (in the sense of definition \ref{dfn:rvind})
  \item $\E[X_k] = (+1) \IP[X_k = +1] + (-1)\IP[X_k = -1] = 0$, so by linearity of the expectation value, $\E[S_n] = 0$.
\end{enumerate}

\begin{prop}[]
For a given $n$, the random variable $S_n$ takes on values $\{-n, -n + 2, \ldots, n-2, n\}$ with probality
\begin{align*}
  \IP[S_n = 2k - n] = \binom{n}{k}2^{-n} = \binom{n}{k} \frac{1}{2}^{k} \left(
    1 - \frac{1}{2}
  \right)^{n-k}, \quad k = 0,1,\ldots,n
\end{align*}
The distribution of $S_n$ is  called a \emph{linearly transformed binomial distribution} with $p = \frac{1}{2}$
\end{prop}
\begin{proof}
Let define $U_n$ the number of $+1$ steps up to period $n$. So
\begin{align*}
  U_n = \sum_{k=1}^{n}\mathds{1}_{\{X_k = +1\}}
\end{align*}
Then $S_n = U_n - (n- U_n) = 2Un - n$, so by calculating the cardinality
\begin{align*}
  \abs{\{S_n = 2k - n\}} = \abs{\{U_n = k\}} = \binom{n}{k}2^{N-n}
\end{align*}
the proability is just that divided by $\abs{\Omega} = 2^{N}$:
\begin{align*}
  \IP[S_n = 2k -n] = \binom{n}{k}2^{-n}
\end{align*}
\end{proof}
Using Stirling's formula
\begin{align*}
  n! \sim \left(
    \frac{n}{e}
  \right)^{n} \sqrt{2 \pi n}
\end{align*}
We can calculate the probability that the particle returns to the origin after $2n$ steps:
\begin{align*}
  \IP[S_{2n} = 0] = \frac{\IP[\abs{S_{2n - 1}} = 1]}{2} = \IP[S_{2n-1} = 1] = \binom{2n}{n}2^{-2n} \sim \frac{1}{\sqrt{\pi n}}
\end{align*}

\subsubsection{Reflection principle}

For $a \in \Z$, consider the time until the particle first reaches $a$. This can be done with the random variable $T_a$ given by
\begin{align*}
  T_a(\omega) = \min \{n > 0: S_n(\omega) = a\}
\end{align*}
where for $a = 0$ we are interested in the first \emph{return} to the origin and where we set $\min \emptyset = N+1$.

We might ask ourselves, what is the probability that the particle reaches position $-a$ at some point and then travels to $b$? 
The following lemma simplifies this by reflecting the trajectory from $-a$ to $b$ around $-a$.
To reach $-2a -b$ the particle obviously has to pass $-a$ at some point, so the ``extra'' condition drops.
\begin{lem}[Reflection principle]
For $a > 0$ and $b \geq -a$
\begin{align*}
  \IP[T_{-a} \leq n, S_n = b] = \P[S_n = -2a - b]
\end{align*}
\end{lem}

Although this lemma seems rather simple, it has some nice consequences. It allows us to find the probability distribution for $T_a$.
\begin{thm}[]
For $a \neq 0$ we have
\begin{align*}
  \IP[T_{-a} \leq n] = 2 \IP[S_n <-a] + \IP[S_n = -1] = \IP[S_n \notin (-a,a]]
\end{align*}
\end{thm}
\begin{proof}
Using the additivity $\IP$ on disjoint sets, we have
\begin{align*}
  \IP[T_{-a} \leq n] 
  &= 
  \sum_{b=-\infty}^{\infty}\IP[T_{-a} \leq n, S_n = b]\\
  &=
  \sum_{b=-\infty}^{-a}\IP[S_n = b] + \sum_{b=-a+1}^{\infty}\IP[S_n = -2a - b]\\
  &= \IP[S_n \leq -a] + \P[S_n \leq -a -1]
\end{align*}
\end{proof}
\begin{cor}
For ever $a \neq 0$ we have
\begin{enumerate}
  \item $\IP[T_a > N] \to 0$ for $N \to \infty$
  \item $\E[T_a] = \sum_{k=1}^{N+1}k \IP[T_a = k] \to \infty$ for $N \to  \infty$.
\end{enumerate}
\end{cor}
\begin{proof}
Let $a > 0$. Using the previous theorem and Stirling's formula we get
\begin{align*}
  \IP[T_{-a} > N] = \IP[S_{n} \in (-a,a]] \leq \frac{C}{\sqrt{\pi N}} \to  0
\end{align*}
for some constant $C$. For the expectation value, we get
\begin{align*}
  \E[T_{-a}]
  &=
  \sum_{k=0}^{N}\IP[T_{-a} > k]\\
  &=
  \sum_{k=0}^{N}\IP[S_k \in (-a,a]]\\
  &\geq \sum_{k=1}^{N}\IP[S_k \in \{0,1\}] \to  \infty \quad \text{for} \quad N \to  \infty
\end{align*}
\end{proof}
In the frequentist interpretation, this corollary says that we every position will be reached eventually, but we might need infinitely many trials $\omega \in \Omega$ to get there.


If we have a path of length $2n$ that \emph{stay above $0$}, then by omitting the very first step, we uniquely obtain a paths of length $2n - 1$ that \emph{never reaches $-1$}.
The two types of paths are in bijection, so we get the theorem
\begin{thm}[]
\begin{align*}
  \IP[T_0 > 2n] = \IP[S_{2n} = 0]
\end{align*}
\end{thm}
\begin{proof}
The comment above is pretty much the proof of the theorem. 
Written out, it says
\begin{align*}
  \IP[T_0 > 2n] &= \frac{1}{2} \IP[T_{-1} > 2n -1] + \frac{1}{2}\IP[T_1 > 2n - 1] = \IP[T_{-1} > 2n -1]\\
                &= \IP[S_{2n-1} \in (-1,]] = \IP[S_{2n-1} = 1] = \IP[S_{2n} = 0]
\end{align*}
\end{proof}
