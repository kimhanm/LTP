We have seen the theorem that given a Graph $G=(V,E)$ with adjacency matrix $A$
\begin{align*}
  \lambda_2(L) = 0 \iff G \text{ is disconnected}
\end{align*}
The proof for $\implies$ was given last week.

For the other way: if $G$ is disconnected, there exists non-trivial $S,S^{c}$ such that for $y = 1_S$, $y$ and $1$ are linearly independent, so their span is two-dimensional.

Then we can write
\begin{align*}
  y^{T}Ly = \sum_{(i,j) \in E}(y_i - y_j)^{2} = 0
\end{align*}
% I have no clue what bandeira is writing here.
and because $L \geq 0$ and $Ly = 0$ and $1,y \in \Ker(L)$, the dimension of the kernel must be $\geq 2$. So $\lambda_2(L) = 0$.


Image a network with two clusters that are almost disconnected. Is it true that $\lambda_2(L)$ is almost $0$?


\begin{thm}[]
  Let $G = (V,E)$ without isolated nodes and let $\mathcal{L}_G$ be the normalized graph Laplacian
  \begin{align*}
    \mathcal{L}_G = I - D^{-1/2}AD^{-1/2}
  \end{align*}
  then, for $\text{cut}(S)$ and $\text{Ncut}(S)$ given by
  \begin{align*}
    \text{cut}(S) = \sum_{i\in S}\sum_{j \in S^{c}} 1_{(i,j) \in E}, \quad \text{vol}(A) = \sum_{x \in A}\degree(x)
  \end{align*}
  \begin{align*}
    \lambda_2(\mathcal{L}_G) \leq \min_{0 \neq S \subset V} \left(
      \frac{\text{cut}(S)}{\text{vol}(S)}
      +
      \frac{\text{cut}(S)}{\text{vol}(S^{c})}
    \right)
  \end{align*}
\end{thm}
\begin{proof}
  By the courant'fischer variational principle of eigenvalues, we have
\begin{align*}
  \lambda_2(\mathcal{L}_G)= \sum_{\|z\|=1, z \bot v_1(\mathcal{L}_G)} z^{T}\mathcal{L}_Gz
\end{align*}
But we know that $v_1(\mathcal{L}_G) = D^{1/2}1$ so by a change of variables $z = D^{1/2}y$ we get
\begin{align*}
  \lambda_2(\mathcal{L}_G) = \min_{y^{T}D^{1/2}\mathcal{L}_GD^{1/2}y,y^{T}D1 = 0}y^{T}D^{1/2}\mathcal{L}_GD^{1/2}y = \min_{y^{T}Dy=1,y^{T}D1=0}\sum_{(i,j) \in E}(y_i - y_j)^{2}
\end{align*}
By restricing the vectors $y$ to only take two values for a set $S \subset V$
\begin{align*}
  y_i = \left\{\begin{array}{ll}
    a & \text{ if }i \in S\\
    b & \text{ if }i \notin S
  \end{array} \right.
\end{align*}
and we ask for which values of $a$ and $b$ does $y$ satisfy the constraints?
\begin{align*}
  y^{T}Dy=1 \iff \sum_{i=1}^{n}y_i^{2} \degree(i) = a^{2} \sum_{i\in S}\degree(i) + b^{2} \sum_{i \in S}\degree(i) = a^{2} \text{vol}(S) + b^{2} \text{vol}(S^{c}) = 1
\end{align*}
Which requires the two following equations
\begin{align*}
  a^{2} \text{vol}(S) + b^{2}\text{vol}(S^{c}) = 1 \quad \text{and} \quad \sum_{i=1}^{n}\degree(i)y_i = a \text{vol}(S) + b \text{vol}(S^{c})
\end{align*}
and we can show that the resulting $y_i$ must be given by
\begin{align*}
  y_i =
  \left\{\begin{array}{ll}
      \left(
        \frac{\text{vol}(S^{c})}{\text{vol}(S)\text{vol}(G)}
      \right)^{\frac{1}{2}} & \text{ if } i \in S\\
      -\left(
        \frac{\text{vol}(S)}{\text{vol}(S^{c})\text{vol}(G)}
      \right)^{\frac{1}{2}} & \text{ if } i \notin S\\
  \end{array} \right.
\end{align*}
now we need to compute $y^{T}\mathcal{L}_Gy$.
\begin{align*}
  y^{T}\mathcal{L}_Gy 
  &= \sum_{(i,j) \in E}(y_i - y_j)^{2} 
  \\
  &= 
  \sum_{i \in S}\sum_{j \in S^{c}}1_{(i,j) \in E}(y_i - y_j)^{2}
  \\
  &=
  [...]
  \\
  &= \text{cut}(S) \frac{1}{\text{vol}(G)} \left[
    \frac{\text{vol}(S^{c})}{\text{vol}S} + 2 + \frac{\text{vol}(S)}{\text{vol}(S^{c})}
  \right]
  \\
  &=
  \text{cut}(S) \left[
    \frac{1}{\text{vol}(S)}
    +
    \frac{1}{\text{vol}(S^{c})}
  \right]
  = \text{Ncut}(S)
\end{align*}
\end{proof}



Given a cluster of a graph, we want to find a partition $S,S^{c}$ with small $\text{Ncut}$.
