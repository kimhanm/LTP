
Now we are finally able to prove the following:
\begin{thm}[]
  Let $E$ be a splitting field of $f \in k[X]$.
  If $f$ is solvable by radicals, then $\Gal(E/k)$ is a resolvable group.
\end{thm}
The proof makes use of the following observation
\begin{lem}[]
  For a tower of field extensions
  $
    k = K_0 \subseteq K_1 \subseteq \ldots \subseteq K_t
  $
  such that
  \begin{enumerate}
    \item $K_t/k$ is normal
    \item $K_i/K_{i-1}$ is a pure extension of prime type $p_i$
    \item $k$ contains all $p_i$-th roots of $1$.
  \end{enumerate}
  Then $\Gal(K_t/k)$ is resolvable
\end{lem}
The sketch of the proof is as follows: If $f$ is solvable with radicals, then there exists a field $L$ with $k \subseteq E \subseteq L$ and $L/k$ radical.

Using Corollary \ref{cor:radical-normal} there exists a field $F$ with $k \subseteq E \subseteq L \subseteq F$ such that $F/k$ is radical and normal.

The lemma then says that $\Gal(F/k)$ is resovlable, and since
\begin{align*}
  \Gal(E/k) \iso \faktor{\Gal(F/k)}{\Gal(F/E)}
\end{align*}
it follows that $\Gal(E/k)$ is resolvable.

\begin{proof}[Proof Lemma]
  Set $G = \Gal(K_t/k), G_1 = \Gal(K_t/K_1), G_2 = \Gal(K_t/K_2)$ etc. and we obtain a sequence of groups
  \begin{align*}
    \{e\} = G_t \subseteq G_{t-1} \subseteq \ldots \subseteq G_2 \subseteq G_1 \subseteq G
  \end{align*}
  If $u_i$ is the adjoint of the $i$-th pure extension: $K_i = K_{i-1}(u_i)$, which by assumption satisfies $u_i^{p_i} \in K_{i-1}$, we consider the polynomial
  \begin{align*}
    f_i(X) = X^{p_i} - c_i \in K_{i-1}[X] \quad \text{where} \quad c_i = u_i^{p_i} \in K_{i-1}
  \end{align*}
  In \ref{lem:prime-polynomial} (a) we saw that either $f$ is irreducible or $c_i$ is the $p_i$-th power of an element in $K_{i-1}$.

  But by assumption, $u_i \notin K_{i-1}$, so $f_i$ is irreducible and $K_{i-1}$ contains all $p_i$-th roots of $1$. Therefore, $K_i$ is a splitting field of $f_i \in K_{i-1}[X]$. 

  By the same Lemma, $\Gal(K_i/K_{i-1})$ is either $\Z/p_i\Z$ or $\{e\}$.

  Furthermore, because $K_i/K_{i-1}$ is normal and the cyclic group is abelian,
the sequence of groups $G_t \subseteq \ldots \subseteq G_1 \subseteq G$ is indeed a subnormal sequence with abelian quotients.
\end{proof}

\begin{proof}[Proof Theorem]
  Let $f \in k[X]$ be solvable by radicals and $E$ a splitting field of $f$.
  By Corollary 3.5.1 we can assume that there exists a field $K$ such that $K/k$ is radical and normal
  Let
  \begin{align*}
    k = K_0 \subseteq K_1 \subseteq \ldots \subseteq K_t = K, \quad \text{where} \quad K_i = K_{i-1}(u_i), \text{ and } u_i^{p_i} \in K_{i-1}
  \end{align*}
  be the corresponding chain of fields.
  Set $m = \prod_{i=1}^{t}p_{i}$
  and let
  \begin{align*}
    k^{\ast} &= \text{ splitting field of } X^{m} - 1 \in k[X]\\
    K^{\ast} &= \text{ splitting field of } X^{m} - 1 \in K[X]
  \end{align*}
  In exercise sheet 6, we proved that since $K/k$ and $K^{\ast}/K$ are normal, also $K^{\ast}/k$ is normal.

  Looking at the diagram
  \begin{center}
  \begin{tikzcd}[ ] %\arrow[bend right,swap]{dr}{F}
    k \arrow[hook]{r}{} \arrow[hook]{dr}{}& K \arrow[hook]{r}{} & K^{\ast}\\
      & k^{\ast} \arrow[hook]{ur}{}
  \end{tikzcd}
  \end{center}
  we claim that $\Gal(K^{\ast}/k^{\ast})$ is resolvable.
  By setting
  \begin{align*}
    K_0^{\ast} = k^{\ast}, K_1^{\ast}= K_0^{\ast}(u_1), \ldots, K_i^{\ast} = K_{i-1}^{\ast}(u_i)
  \end{align*}
  we obtain a tower from $k^{\ast}$ to $K^{\ast}$.
  Since $u_i^{p_i} \in K_{i-1} \subseteq K_{i-1}^{\ast}$, these extensions are all pure, but their prime type might have changed from $p_i$ to something else.

Either it's still $p_i$, or $\{m \in \Z \big\vert u_i^{m} \in K_{i-1}^{\ast}\} = \Z$, in which case $K_{i}^{\ast} = K_{i-1}^{\ast}$.

So by the Lemma, $\Gal(K^{\ast}/k^{\ast})$ is resolvable.

Now we claim that $\Gal(k^{\ast}/k)$ is resolvable.
Let
\begin{align*}
  \Gamma_m(k^{\ast}) = \{\xi \in k^{\ast} \big\vert \xi^{m} = 1\}
\end{align*}
be the set of roots of $X^{m} - 1$ in $k^{\ast}$.
Then $\Gamma_m(k^{\ast})$ is a finite subgroup of $(k^{\ast})^{\times}$ and thus cyclic.


We also know that the restriction homomorphism
\begin{align*}
  \Gal(k^{\ast}/k) \to  \Aut(\Gamma_m(k^{\ast})), \quad \sigma \mapsto  \sigma|_{\Gamma_m(k^{\ast})}
\end{align*}
is injective. But since $\Aut(\Gamma_m(k^{\ast}))$ is abelian, so is $\Gal(k^{\ast}/k)$.


  Now we know that from $k \subseteq k^{\ast} \subseteq K^{\ast}$ we get
  \begin{align*}
    \faktor{\Gal(K^{\ast}/k)}{\Gal(K^{\ast}/k^{\ast})} \iso \Gal(k^{\ast}/k)
  \end{align*}
  and from proposition \ref{prop:3-13} $\Gal(K^{\ast}/k)$ is resolvable.

  And looking at $k \subseteq E \subseteq K^{\ast}$ we get
  \begin{align*}
    \Gal(E/k) \iso \faktor{\Gal(K^{\ast}/k)}{\Gal(K^{\ast})/E}
  \end{align*}
  we get $\Gal(K^{\ast}/k)$ resolvable $\implies$ $\Gal(E/k)$ resolvable.

\end{proof}


\begin{cor}[Abels-Ruffini]
For $n \geq 5$, the general polynomial
\begin{align*}
  f(X) = \prod_{i=1}^{n}(X-y_i)
\end{align*}
is not solvable by radicals.
\end{cor}
\begin{proof}
  Considering $f \in k(y_1,\ldots,y_n)[X]$ we can write
  \begin{align*}
    \prod_{i=1}^{n}(X-y_i) = X^{n} - s_1X^{n-1} + \ldots +  (-1)^{n}s_n
  \end{align*}
  so we also get $f \in k(s_1,\ldots,s_n)$.
  If we set $E = k(s_1,\ldots,s_n) \subseteq k(y_{1}, \ldots, y_{n}) = F$, then $f \in E[X]$ and $F$ is a splitting field of $f$.

  We wish to compute $\Gal(E/F)$ and notice that we get an injective group homomorphism
  \begin{align*}
    \Gal(E/F) \to S_{\{y_{1}, \ldots, y_{n}\}}, \quad \sigma \mapsto  \sigma|_{\{y_{1}, \ldots, y_{n}\}}
  \end{align*}
  Now let $s \in S_n$ and $R \in F$ and define
  \begin{align*}
    (\sigma_sR)(y_{1}, \ldots, y_{n}) = R(y_{y_{s(1)}}, \ldots, y_{s(n)}
  \end{align*}
  and it's easy to show that $\sigma_s \in \Aut F$, where $\sigma_s(s_j) = s_j$ and as such, the restriction homomorphism is a bijection
  \begin{align*}
    \sigma_s \in \Gal(F/E) \iso S_n
  \end{align*}
  So for $n \geq 5$, we know that $[S_n,S_n] = A_n$ is simple and non-abelian, thus $\Gal(F/E) = S_n$ is not solvable.

  By the theorem, it follows that $f$ is not solvable by radicals.
\end{proof}


\begin{cor}[]
  The polynomial $f = X^{5} - 4X + 2 \in \Q[X]$ is not solvable by radicals because $\Gal(f) = S_5$
\end{cor}

The abels ruffini corollary does not fully capture the power of the theorem, which says something deeper that just finding roots of polynomials.




