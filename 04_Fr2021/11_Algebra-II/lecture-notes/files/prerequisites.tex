\setcounter{section}{-1}

\section{Prerequisites}

\subsection{Ring Theory}

\begin{dfn}[]
  Let $R$ be a UFD and $f \in R[X] \setminus \{0\}$. The $\gcd$ of the coefficients of $f$ is called the \textbf{content} $I(f)$ of $f$.

  We say that $f$ is \textbf{primitive}, if $I(f) \in R^{\times}$.
\end{dfn}



\begin{thm}[Eisenstein Criterion]
  Let $R$ be a UFD and $f = a_nX^{n} + a_{n-1}X^{n-1} + \ldots + a_0 \in R[X]$ be primitive such that there exists some prime $p \in R$ with
  \begin{align*}
   p \not| a_n, \quad p|a_0, \ldots p|a_{n-1}, \quad p^{2}\not|a_0 
  \end{align*}
  then $f$ is prime in $R[X]$.
\end{thm}
We will often use the special case where $R = \Z$.

\subsection{Group Theory}
For every $g$ in a group $G$, the mapping
\begin{align*}
  \gamma_g: G \to G, \quad x \mapsto gxg^{-1}
\end{align*}
is an automorhpism. The association
\begin{align*}
  \Phi: G \to \Aut(G), \quad g \mapsto \gamma_g
\end{align*}
is a group homomorphism. Its kernel $Z_G = \Ker \Phi$ is called the \textbf{center} of the group $G$.
\begin{dfn}[]
A group $G$ is said to be \textbf{nilpotent} of order $1$, if $G$ is abelian.

We say that it is nilpotent of order $n+1$ if $G/Z_G$ is nilpotent of order $n$.
\end{dfn}
\begin{dfn}[]
A \textbf{subnormal series} in a group $G$ is a chain of subgroups such that
\begin{align*}
  \{e\} = G_0 \lhd G_1 \lhd G_2 \lhd  \ldots \lhd G_n = G
\end{align*}
such that every subgroup is normal in the next one.
We say that $G$ is \textbf{resolvable}, if such a subnormal series exists such that $G_{k+1}/G_k$ is an abelian group for all $k$.
\end{dfn}

\subsection{Fields}

\begin{dfn}[]
  Let $L$ be a field and $K \subseteq L$ a subring that is also a field. 
  If that is the case, we say that $K$ is a \textbf{subfield} of $L$ and call $L$ a \textbf{field-extension} of $K$ and write $L/K$ (read: $L$ over $K$).

  $L$ can also be seen as a vector space over $K$. Its \textbf{degree} is $[L:K] := \dim_K L$.
  If $[L:K] < \infty$, we say that $L$ is a \textbf{finite field extension} of $K$.
\end{dfn}

\begin{lem}[]
  Let $F/L$ and $L/K$ be finite field extensions. Then
  \begin{align*}
    [F:K] = [F:L] \cdot [L:K]
  \end{align*}
\end{lem}
\begin{proof}[Proof Sketch]
  If $x_1, \ldots, x_m \in F$ is a basis of $F$ over $L$ and $y_{1}, \ldots, y_{n} \in L$ is a basis of $L$ over $K$, 
  then the products $x_iy_j \in F$ for $1 \leq i \leq m, 1 \leq j \leq n$ form a basis of $F$ over $K$.
\end{proof}

\begin{dfn}[]
  Every field $k$ contains a smallest subfield called the \textbf{prime field} of $k$.
  It is either isomorphic to $\Q$ or $\F_p$ for some prime $p \in \N$.

  Define the \textbf{characteristic} of the field as
  \begin{align*}
    \charac k := \left\{\begin{array}{ll}
        p & \text{ if its prime field is } \F_p\\
        0 & \text{ if its prime field is } \Q
    \end{array} \right.
  \end{align*}

\end{dfn}


\begin{dfn}[]
  Let $L/K$ be a field extension. For $x \in L$ consider the \textbf{evaluation mapping}
  \begin{align*}
    \phi_x: K[X] \to L,\quad f \mapsto  f(x)
  \end{align*}
  \begin{enumerate}
    \item If $\phi_x$ is injective, we say that $x$ is \textbf{trancendental} over $K$.
    \item If $\phi_x$ is not injective, we say $x$ is \textbf{algebraic} over $K$.
      
      If this is the case, then $\Ker \phi_x = (m_x)$ is an ideal and we call $m_x(X)$ the \textbf{minimal polnomial} of $x$ and its degree the degree of $x$.
  \end{enumerate}
  We call $L$ \textbf{algebraic} over $K$, if every $x \in L$ is algebraic.
\end{dfn}
Note that $x$ is algebraic over $K$ if and only if $x$ is the root of a non-zero polynomial $f \in K[X]$.

\begin{prop}[]
  If $L/K$ is a finite field extension, then $L$ is algebraic over $K$.
\end{prop}
By seeing $L$ as an $n$-dimensional vector space over $K$, we can easily see that the $n+1$ vectors $1, \alpha, \alpha^{2}, \ldots, \alpha^{n}$ are linearly dependent. This gives us a non-zero polynomial for which $\alpha$ is a root.

The converse is not true.



\begin{prop}[]
  Let $L/K$ be a field extension and $x \in L$. 
  Then there exists a (up to isomorhpism) unique subfield $K(x)$ of $L$ that is the smallest subfield of $L$ containing $K$ and $x$ and:
  \begin{itemize}
    \item If $x$ is transcendent, then $K[X] = \Im \phi_x \iso L[X]$
    \item If $x$ is algebraic, then
  \end{itemize}

\end{prop}



\begin{dfn}[]
  Let $f \in K[X] \setminus \{0\}$. A field extension $L/K$ of the form $L = K(a_{1}, \ldots, a_{n}$, where $f(X) = \alpha \prod_{i=1}^{n}(X - a_i)$ for $\alpha \in L^{\times}$ is a \textbf{splitting field} of $f$ over $K$.
\end{dfn}
Every polynomial has a splitting field and they are unique up to a (non-canonical) isomorphism.


\begin{dfn}[]
  A field $K$ is called \textbf{algebraically closed}, if every
  non-zero polynomial splits into linear factors in $K$.

  If $L/K$ is a field extension such that $L$ is algebraically closed, then the set
  \begin{align*}
    E = \{x \in L \big\vert x \text{ is algebraic over }K\}
  \end{align*}
  forms an algebraically closed field that does not depend on the choice of such $L$.
  We call $E$ the \textbf{algebraic closure} $\overline{K}$ of $K$.
\end{dfn}

\begin{dfn}[]
  Let $E/k$ be a field extension and $\alpha \in E$. Then $k[\alpha]$ is the image of the evaluation homormophism
  \begin{align*}
    k[X] \to E, \quad p \mapsto  p(\alpha)
  \end{align*}
  Since $E$ is a field, $k[\alpha]$ is an integraldomain and $k(\alpha)$ is the space of rational functions of $k[\alpha]$
\end{dfn}
