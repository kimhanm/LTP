\section{The Galois Correspondence}

One might ask if the inverse of Theorem \ref{thm:3-14} is true, that is: 
If $E/k$ is a normal extension for a polynomial $f \in k[X]$ and $\Gal(E/k)$ is resolvable, does that mean that $f$ is solvable by radicals?

More generally one might ask if there is a correspondence between subgroups of $\Gal(E/k)$ and sandwiched fields $k \subseteq B \subseteq E$.

It turns out that such a correspondence exists if we impose the property that $f$ be separable.
We wish to better understand this correspondence in this chapter.

\begin{dfn}[]
  Let $E$ be a field and $H \subseteq \Aut E$ a subset. The set
  \begin{align*}
    E^{h} := \{a \in E \big\vert \sigma(a) = a \forall \sigma \in H\}
  \end{align*}
  is a subfield of $E$ called the \textbf{fixing field} of $H$.
\end{dfn}
Note that the map $H \mapsto E^{H}$ is contravariant monotonous with respect to inclusion
\begin{align*}
  H_1 \subseteq H_2 \implies E^{H_2} \subseteq E^{H_1}
\end{align*}

If $H$ is the galois group of a field extension $E/k$, then trivially $k \subseteq E^{\Gal(E/k)}$.

\begin{ex}[]
  Take for example $k = \F_p(t), f(X) = X^{p}-t$ and $E$ a splitting field of $f$.
  From the $X^{p}-c$ Lemma, we know that $\Gal(E/k) = \{e\}$, which shows $k \subsetneq E^{\Gal(E/k)} = E$.
\end{ex}
