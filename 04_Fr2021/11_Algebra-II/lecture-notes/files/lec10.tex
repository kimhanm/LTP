\begin{ex}[]
	The polynomial $f = X^{3} - 2 \in \Q[X]$ is irreducible and let $E \subseteq \C$ be the splitting field of $f$.

	Since the degree of the extension is $6$, we know that $\Gal(E/\Q) \iso S_3$.
	For $\beta = \sqrt[3]{2} \in \R$ and $\omega = e^{\frac{2 \pi i }{3}}$, the roots of $f$ are
	\begin{align*}
		\alpha_1 = \beta, \alpha_2 = \beta \omega, \alpha_3 = \beta \omega^{2}
	\end{align*}
	Since the galois group acts transitively, let $\sigma_{ij} \in S_3$ be the automorphism that transposes $\alpha_i$ and $\alpha_j$ and let $\tau \in S_3$ be the cyclic permutation of the roots.

	Then the correspondence with the subgroups and the subfields are
	\begin{align*}
		\left<\sigma_{12}\right> \sim \Q(\alpha_3), \quad \left<\sigma_{13}\right> \sim \Q(\alpha_2), \quad \left<\sigma_{23}\right> \sim \Q(\alpha_1), \quad \left<\tau\right> \sim \Q(\omega)
	\end{align*}
\end{ex}

The Galois correspondence has many simple consequences
\begin{cor}[] \label{cor:4-18}
	A finite Galois extension has only finitely many subfields.
\end{cor}


\begin{dfn}[]
	A field extension $E/k$ is called \textbf{simple} if there exists an $u \in E$ such that $E = k(u)$
\end{dfn}
\begin{prop}
	A finite extension $E/k$ is simple if and only if there exist finitely many subfields.
\end{prop}
\begin{proof}
	Assume there are only finitely many subfields.
	\begin{itemize}
		\item If $k$ is infinite, then $E$ as a $k$-vectorspace cannot be the union of its finitely mans subfields. (See Exercise sheet 8 Problem 4).

			So there exists an element $u \in E$ that is not contained in any subfield, which shows $E = k(u)$
		\item If $k$ is finite, then $k = \F_q$ and $E = \F_{q^{n}}$ for $n = [E:k]$, which means that we can an element $u$ that generates $\F_{q^{n}}^{\times}$.
	\end{itemize}
	
	On the other hand let $E = k(u)$ and $k \subseteq F \subseteq E$ an interior field. Then let
	\begin{align*}
		f_F(T) = \text{irr}(x,F)(T) = T^{n} + a_{n-1}^{T^{n-1}} + \ldots + a_0 \in F[T]
	\end{align*}
	then let $F_0 = k(a_{n-1},\ldots,a_0) \subseteq F$.
	Since $f_F$ is irreducible in $F[T]$ it is also irreducible in $F_0(T)$ and so
	\begin{align*}
		[E:F] = [F(x),F] = n, \quad [E:F_0] = [F_0(x),F_0] = n \implies F = F_0
	\end{align*}
	Note that $\text{irr}(x,F)$ divides $\text{irr}(x,k)$ in $E[T]$.
	Then the number of inteior fields $\leq$ the number of polynomials in $E[T]$ that divide $\text{irr}(x,k)$.
\end{proof}


\begin{cor}
	A finite Galois extension $E/k$ is always simple.
\end{cor}

\begin{ex}[]
  Set $E = \F_p(X,Y)$ and $k = \F_p(X^{p},Y^{p})$. 
  Clearly $[E.k] = p^{2}$ and there does not exist an $x \in E$ such that $E = k(x)$, but there are infinitely many subfields.
\end{ex}

\begin{thm}[]\label{4-24}
  Let $E/k$ be a finite Galois extension with $\charac k = 0$. If $\Gal(E/k)$ is resolvable, then $E$ is contained in a radical extension of $k$.
\end{thm}
We first show that there exists a normal divisor $N \lhd \Gal(E/k)$ of index $p$ for some prime $p$ and prove the theorem a bit later.

\begin{proof}[Proof claim]

Because $G$ is finite and resolvable then $[G,G] \nsubseteq G$. and $G/[G,G]$ is a finite abelian group $\neq \{e\}$.
  By the classification theorem of finite abelian groups (See Algebra I), it can be written as a product of $\Z/p^{n}\Z$ for primes $p$ and $n \geq 1$.

  Because any inclusion $\Z/p^{n}\Z \supseteq \Z/p^{n-1}\Z$ has index $p$, we see that $G/[G,G]$ contains a subgroup $M$ of index $p$. Let
  \begin{align*}
    \rho: G \to  G/[G,G], \quad \text{and} \quad N := \rho^{-1}(M)
  \end{align*}
Then $N \lhd G$ has index $p$ aswell.

Clearly, $E^{N}/k$ is a Galois extension of degree $p$.
\end{proof}

\begin{lem}[] \label{lem:4-25}
  Let $E/k$ be a finite Galois extension with $[E:k] = p$ for some prime $p$.
  If $k$ contains a $p$-th root $\omega$ of $1$ with $\omega \neq 1$, then
  \begin{align*}
    \exists \xi \in E \quad \text{with} \quad 
    \xi^{p} \in k \quad E = k(\xi)
  \end{align*}
\end{lem}
\begin{proof}
First note that $\Gal(E/k)$ is cyclical of order $p$, so let $\sigma \in \Gal(E/k)$ be a generator of the group.

Viewing $\sigma: E \to E$ as a $k$-linear map that satisfies $\sigma^{p} = \id_E$, we claim that $X^{p}-1$ is the minimal polynomial of $\sigma$.

If there is a smaller polynomial $P = \sum_{i=0}^{p-1}a_i X^{i}\in k[X]$ with degree $\degree P \leq p-1$ and $P(\sigma) = 0$, then this means
\begin{align*}
  \sum_{i=0}^{p-1}a_i \sigma^{i} = 0
\end{align*}
but since
$\id_E, \sigma, \ldots, \sigma^{p-1}$ are characters $\Hom(E^{\times},E^{\times})$, they are linearly independent in $F(E^{\times},E)$ (by \ref{prop:4-4}), which means $P = 0$.

Therefore, $X^{p}-1$ is the characteristic polynomial of $\sigma$ and $\omega \in k$ is an eigenvalue of $\sigma$.

Let $\xi \in E^{\times}$ be an eigenvector $\sigma(\xi) = \omega \cdot \xi$. Since $\omega \neq 1$, it must follow that $\xi \in k$ and thus $E = k(\xi)$.

Moreover, $\sigma(\xi^{p}) = \sigma(\xi)^{p} = \omega^{p} \xi^{p} = \xi^{p}$.
And since $\sigma$ generates the Galoisgroup, we have $\xi^{p} \in E^{\Gal(E/k)} = k$.
\end{proof}


Now we are ready to prove the theorem \ref{4-24}
\begin{proof}[Proof Theorem]
  Let $E/k$ be Galois with $\Gal(E/k)$ resolvable.

  We use induction on $[E:k]$. If $[E:k] = 1$ it's clear.

  Assume $[E:k] \geq 2$, so $\abs{\Gal(E/k)} = [E:k] \geq 2$.

  We have shown earlier in the claim that there exists a normal divisor $N \lhd \Gal(E/k)$ of index $p$ for some prime $p$.

  Since $k \subseteq E^{N}$, the field $E^{N}/k$ is a Galois extension of degree $p$.
  Let $k^{\ast}/k$ be the splitting field of $X^{p} - 1 \in k[X]$ and $\omega \in k^{\ast}$ be a generator of all $p$-th roots of unity. (i.e. $\omega \neq 1$).
  \begin{itemize}
    \item If $\omega \in k$, then by Lemma \ref{lem:4-25}, $E^{N}/k$ is a pure extension and $[E:E^{N}] < [E:k]$ and $\Gal(E/E^{N}) = N$.
      
      Since $N < \Gal(E/k)$ is a subgroup of a resolvable group, it is also resolvable, and by induction hypthesis we get a tower of pure extensions
      \begin{align*}
        E^{N} = K_1 \subseteq K \subset \ldots \subset K_t
      \end{align*}
      Therefore we see that
      \begin{align*}
        k \subseteq E^{N} = K_1 \subseteq K_2 \subseteq \ldots \subseteq K_t
      \end{align*}
      where $K_t$ is a radical extension of $k$ that contains $E$.

    \item If $\omega \notin k$, then we define $E^{\ast} = E(\omega)$ and we get the diagram
      \begin{center}
      \begin{tikzcd}[ ] %\arrow[bend right,swap]{dr}{F}
        k \arrow[hook]{r}{} \arrow[hook]{d}{}& k^{\ast} \arrow[hook]{d}{}\\
        E \arrow[hook]{r}{}& E^{\ast} = E(\omega)
      \end{tikzcd}
      \end{center}
  \end{itemize}
\end{proof}



