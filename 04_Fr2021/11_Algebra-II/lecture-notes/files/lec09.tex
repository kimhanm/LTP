Recall that the Galois group of a normal extension for a separable polynomial has order $\abs{\Gal(E/k)} = [E:k]$
and that $[E:E^{G}] = \abs{G}$.


\begin{thm}[]\label{thm:4-11}
  Let $E/k$ be a finite extension with galois group $G = \Gal(E/k)$. Then the following are equivalent
  \begin{enumerate}
    \item $E$ is a splitting field of a separable polynomial in $k[X]$
    \item $E^{G} = k$
    \item Every irreducible polynomial in $k[x]$ with a root in $E$ is separable and splits over $E$.
  \end{enumerate}
\end{thm}
\begin{proof}
\begin{itemize}
  \item[$1 \implies 2$] Since $E$ is a splitting field of a separable polynomial, it follows that $[E:k] = \abs{G}$ and
    since $G$ is a finite subgroup of $\Aut E$, it follows that $[E:E^{G}] = \abs{G}$. 
    Because $k \subseteq E^{G} \subseteq E$, we get
    \begin{align*}
      [E^{G}:k] = \frac{[E:k]}{[E:E^{g}]} = 1 \implies E^{G} = k
    \end{align*}
  \item[$2 \implies 3$] Let $p \in k[x]$ irreducible with a root $\alpha \in E$.
    Dividing by the leading coefficient, we can also assume that $p$ is unitary (leading coefficient $= 1$).
    We define the polynomial
    \begin{align*}
      q(X) = \prod_{\sigma \in G/\text{Stab}(\alpha)}(X - \sigma(\alpha))
    \end{align*}
    and clearly, $q$ is unitary, has degree $\abs{\text{Orb}(\alpha)}$, has no multiple roots, so is separable.
    Also, because $p \in k[X]$, we see that $p(\sigma(\alpha)) = \sigma(p(\alpha)) = 0$, so $R(q) \subseteq R(p)$.

    In execise sheet 8, Problem 1, we have shown that $q \in E^{G}[X]$.
    But because by assumption, $E^{G} = k$, so $q$ divides $p$. But they are both unitary, so $p = q$ is separable and splits over $E$.
  \item[$3 \implies 1$] Let $k \subseteq E' \subseteq E$ be maximal with the property that $E'$ is a splitting field of a separable polynomial $g \in k[X]$.

    Assume that $E \setminus E'$ is nonempty with $\alpha \in E \setminus E'$. 
    Then the minimal polynomial $\text{irr}(\alpha,k) \in k[X]$ is an irreducible polynomial with root $\alpha$.
    By assumption, $\text{irr}(\alpha,k)$ is separable and splits over $E$. Taking the product 
    $f = g \cdot \text{irr}(\alpha,k)$ we see that $f \in k[X]$ is clearly separable and splits in $E$.
    If $E''$ is a splitting field of $f$, we would get that $E' \subsetneq E'' \subseteq E$, which is a contradiction to the maximality of $E'$.
\end{itemize}
\end{proof}



\begin{dfn}[]
  A finite extension $E/k$ is called a \textbf{Galois extension} of $k$, if $E$ is a splitting field of a separable polynomial in $k[X]$. (Or as we have just shown, if $E^{G} = k$)
\end{dfn}

Consider extensions $k \subseteq B \subseteq E$. 
If $E/k$ is Galois for, say a separable polynomial $f \in k[X]$, then we can take the same polynomial $f \in B[X]$ and so $E/B$ is also Galois.
However, it does not necessarily follow that $B/k$ is Galois, as it might not even be a normal extension.



\begin{prop}[]
Let $k \subseteq B \subseteq E$ such that $E/k$ is Galois.
Then
\begin{align*}
  B/k \text{ is Galois } \iff  \sigma(B) = B  \quad \forall \sigma \in \Gal(E/k)
\end{align*}
\end{prop}

\begin{proof}
  If $B/k$ is Galois, then $B$ is a splitting field of a polynomial $f \in k[X]$.

  But by Theorem %\ref{thm:2-26}
  we immediately get $\sigma(B) = B$ for all $\sigma \in \Gal(E/k)$.

  Onthe contrary, if $\sigma(B) = B$ for all $\sigma \in \Gal(E/k)$, then consider the image $H < \Gal(B/k)$ of the restriction homomorphism
  \begin{align*}
    \Gal(E/k) \to  \Gal(B/k), \quad \sigma \mapsto  \sigma|_{B}
  \end{align*}
  Then we have
  \begin{align*}
    k \subseteq B^{\Gal(B/k)} \subseteq B^{H} \subseteq E^{\Gal(E/k)} = k \implies B^{\Gal(B/k)} = k
  \end{align*}
  So $B/k$ is Galois.
\end{proof}



Let $G$ be a group and let $\text{Sub}(G)$ be the set of subgroups of $G$, ordered via inclusion.
If $E/k$ is a field extensions, then write 
\begin{align*}
  \Int(E/k) = \{\text{fields }B \big\vert k \subseteq B \subseteq E\}
\end{align*}
for the set of intermediary field extensions between $E$ and $k$.
This set is also orded via inclusion.

\begin{thm}[Galois correspondence] \label{thm:galois-correspondence}
  Let $E/k$ be a finite Galois extension.
  \begin{enumerate}
    \item The map
      \begin{align*}
        \gamma: \text{Sub}(\Gal(E/k)) \to \Int(E/k), \quad H \mapsto E^{H}
      \end{align*}
      is a contravariant bijection with inverse map
      \begin{align*}
        \delta: \Int(E/k) \to \text{Sub}(\Gal(E/k)), \quad B \mapsto \Gal(E/B)
      \end{align*}
    \item $B \in \Int(E/k)$ is Galois if and and only if $\Gal(E/B)$ is a normal subgroup of $\Gal(E/k)$.
      If that is the case, then
      \begin{align*}
        \faktor{\Gal(E/k)}{\Gal(E/B)} \iso \Gal(B/k)
      \end{align*}
  \end{enumerate}
\end{thm}

\begin{proof}
  Recall the Corollary % \ref{cor:4-10}
  where we showed that if $E$ is a field and $H,G$ are finite subgroups of $\Aut E$, then
  \begin{align*}
    E^{G} \subseteq E^{H} \iff H < G
  \end{align*}

  \begin{enumerate}
    \item To show injectivity, let $H_1,H_2$ be subgroups of $\Gal(E/k)$. In particular, $H_1,H_2$ are finite subgroups of $\Aut E$.
      Then if $E^{H_1} = E^{H_2}$, the corollary immediately gives $H_1 = H_2$.

      For surjectivity, we show that $\gamma \circ \delta = \id_{\Int(E/k)}$.
      Let $k \subseteq B \subseteq E$. Since $E/k$ Galois $\implies E/B$ Galois, we have
      \begin{align*}
        (\gamma \circ \delta)(B) = \gamma(\Gal(E/B)) = E^{\Gal(E/B)} = B
      \end{align*}

    \item Assume first that $B/k$ is Galois. In particular, it is normal and by Theorem II-26, we know that $\sigma(B) = B \forall  \sigma \in \Gal(E/k)$.
      And since $\Gal(E/B)$ is the kernel of the restriction mapping
      \begin{align*}
        \Gal(E/k) \to  \Gal(B/k), \quad \sigma \mapsto  \sigma|_B
      \end{align*}
      it is a normal subgroup. 
      And by the first isomorphism theorem, we get $\faktor{\Gal(E/k)}{\Gal(E/B)} \iso \Gal(B/k)$.

      On the other hand, assume that $\Gal(E/B)$ is a normal subgroup. We want to show that $B/k$ is Galois, i.e. $B^{\Gal(B/k)} = k$.

      To do this, we use the characterisation from Proposition IV-14, which tells us that it suffices to show
      $\sigma(B) = B \forall \sigma \in \Gal(E/k)$.

      Since $E/k$ is Galois, we also get that $E/B$ is Galois, so $E^{\Gal(E/B)} = B$.
      Let $\sigma \in \Gal(E/k), \xi \in B, h \in \Gal(E/B)$.
      We show that
      \begin{align*}
        \sigma(\xi) \in B \iff  h(\sigma (\xi)) = \sigma(\xi)
      \end{align*}
      But that is clear because
      \begin{align*}
        h \sigma(\xi) = \sigma( (\sigma^{-1}h \sigma \xi) = \sigma(\xi)
      \end{align*}
  \end{enumerate}
\end{proof}

\begin{ex}[]
  For the finite galois extension $E/k :=\Q(\sqrt[3]{2},\zeta_3)/\Q$. Find all intermediate fields.

  To do so, set
  \begin{align*}
    a_1 := \sqrt[3]{2}, a_2 := \zeta_3 \sqrt[3]{2}, a_3 := \zeta_3^{2} \sqrt[3]{2}
  \end{align*}
  and let $T$ be complex conjugation, $R$ multiplication with $\zeta_3$.
  We already know that
  $\Gal(E/k) = \scal{T,R} \iso D_3$

  \begin{center}
  \begin{tikzcd}[column sep=0.6em] %\arrow[bend right,swap]{dr}{F}
    && \{e\} 
    \arrow[hook,swap,color=orange]{dll}{2}
    \arrow[hook,swap,color=orange]{dl}{2}
    \arrow[hook,color=orange]{d}{2}
    \arrow[hook,color=orange]{ddr}{3}
    \\
    \scal{T}
    \arrow[hook]{ddrr}{3}
    & \scal{T,R}
    \arrow[hook]{ddr}{3}
    & \scal{T,R^{2}}
    \arrow[hook]{dd}{3}
    \\
    &&& \scal{R}
    \arrow[hook,swap,color=orange]{dl}{2}
    \\
    && \Gal(E/k)
  \end{tikzcd}
  \end{center}
  and the corresponding diagram for the fixing fields is given by
  \begin{center}
  \begin{tikzcd}[column sep=0.8em] %\arrow[bend right,swap]{dr}{F}
   &&\Q(\sqrt[3]{2},\zeta_3)
   \\
   \Q(\sqrt[3]{2})
   \arrow[hook,color=orange]{urr}{2}
   &
   \Q(\zeta_3,\sqrt[3]{2})
   \arrow[hook,color=orange]{ur}{2}
   &
   \Q(\zeta_3^{2},\sqrt[3]{2})
   \arrow[hook,color=orange]{u}{2}
   \\
   &&& \Q(\zeta_3)
   \arrow[hook,color=orange]{uul}{3}
   \\
   && \Q
   \arrow[hook]{uull}{3}
   \arrow[hook]{uul}{3}
   \arrow[hook]{uu}{3}
   \arrow[hook,color=orange]{ur}{2}
  \end{tikzcd}
  \end{center}
  where the coloured morphisms are the normal subgroup inclusions, or the Galois extensions, respectively.

  Where the numbers are the index of the subgroups, or the degree of the field extension.
\end{ex}



\begin{ex}[]
  For the extension $E/k := \Q(\zeta_{15})/\Q$ we proceed as follows
  \begin{itemize}
    \item Compute $\Gal(E/k)$ and find its subgroups.
    \item Find isomorphism classes and automorphisms $\Aut(E)$
    \item Use the Galois correspondence to find the intermediate fields by computing the fixing fields.
  \end{itemize}
  Using the chinese remainder theorem, we get
  \begin{align*}
    \Gal(\Q(\zeta)/\Q) \iso (\Z/15 \Z)^{\times} = (\Z/3\Z)^{\times} \times (\Z/5\Z)^{\times} \iso (\Z/2\Z) \times \Z/4\Z
  \end{align*}
  The subgroups are
  \begin{center}
  \begin{tikzcd}[ ] %\arrow[bend right,swap]{dr}{F}
    & \{e\} \arrow[hook]{dr}{2}
    \arrow[hook]{d}{2}
    \arrow[hook,swap]{dl}{2}
    \\
    \scal{(1,0)}
    \arrow[hook]{dr}{2}
    & \scal{(0,2)}
    \arrow[hook]{d}{2}
    \arrow[hook]{dr}{2}
    \arrow[hook,swap]{dl}{2}
    & \scal{(1,2)}
    \arrow[hook,swap]{dl}{2}
    \\
    \scal{(0,1)} 
    \arrow[hook]{dr}{2}
    & \scal{(1,0),(0,2)}
    \arrow[hook]{d}{2}
    & \scal{(1,1)} \arrow[hook,swap]{dl}{2}
    \\
    & \scal{(1,0),(0,1)}
  \end{tikzcd}
  \end{center}
  For the correspondence, we first note that
  \begin{align*}
    \F_3^{\times} \times \F_5^{\times} \iso (\Z/15\Z)^{\times}, (-3) \cdot 3 + 2 \cdot 5 = 1
  \end{align*}
  The correspondence is as follows
  \begin{align*}
    \Z/2\Z \times \Z/4\Z \stackrel{\sim}{\to} \F_3^{\times} \times \F_5^{\times} \stackrel{\sim}{\to} (\Z/15\Z)^{\times} \stackrel{\sim}{\to} \Gal(E/k)\\
    (1,0) \stackrel{\sim}{\mapsto} (2,1) \stackrel{\sim}{\mapsto} (2 \cdot 2 \cdot 5 + 1 \cdot (-1) \cdot 3 = 11) \stackrel{\sim}{\mapsto} (\zeta \mapsto \zeta^{11}) =: \sigma\\
    (0,1) \stackrel{\sim}{\mapsto} (1,2) \stackrel{\sim}{\mapsto} (1 \cdot 2 \cdot 5 + 2 \cdot (-3) \cdot 2 = 7) \stackrel{\sim}{\mapsto} (\zeta \mapsto \zeta^{7}) =: \tau
  \end{align*}
  so the automorphism classes are 
  \begin{center}
  \begin{tikzcd}[ ] %\arrow[bend right,swap]{dr}{F}
    & \scal{\id}
    \\
    \scal{\sigma} 
    & \scal{\tau^{2}}
    & \scal{\sigma \tau^{2}}
    \\
    \scal{\tau}
    & \scal{\sigma,\tau^{2}}
    & \scal{\sigma \tau}
    \\
    & \scal{\sigma,\tau}
  \end{tikzcd}
  \end{center}
  We now compute fixing fields.
  \begin{itemize}
    \item[$\scal{\sigma}$]
  \begin{align*}
    \zeta \mapsto \zeta^{11} \quad \text{fixes} \quad \zeta^{3} \mapsto \zeta^{33} = \zeta^{3}
  \end{align*}
  and the minimal polynomial of $\zeta^{3} = e^{\frac{2 \pi i}{5}}$ is $\frac{X^{5} - 1}{X-1} = X^{4} + X^{3} + X^{2} + X+ 1$.
  \item[$\scal{\tau^{2}}$]
  \begin{align*}
    \zeta \mapsto \zeta^{49} = \zeta^{4}\quad \text{fixes} \quad \zeta + \zeta^{4} \mapsto \zeta^{4} + \zeta
  \end{align*}

  \item[$\scal{\sigma \tau^{2}}$]
  \begin{align*}
    \zeta \mapsto \zeta^{49 \cdot 11} = \zeta^{-1} \quad \text{fixes} \quad \zeta + \zeta^{-1} \mapsto \zeta^{-1} + \zeta
  \end{align*}
  has minimal polynomial
  \begin{align*}
    X^{2} + (\zeta + \zeta^{-1})X + 1 \in \Q(\zeta + \zeta^{-1})[X]
  \end{align*}


  \item[$\scal{\tau}$]
  \begin{align*}
    \zeta \mapsto \zeta^{7} \quad \text{fixes} \quad \zeta^{5} \mapsto \zeta^{35} = \zeta^{5}
  \end{align*}
  The minimal polynomial of $\zeta^{5} = e^{\frac{2 \pi i}{3}}$ is $\frac{X^{3} -1}{X-1} = X^{2} + X + 1$

  \item[$\scal{\sigma,\tau^{2}}$]
    \begin{align*}
      (\zeta^{3} + \zeta^{12})^{2} = \zeta^{6} + 2 + \zeta^{9}
    \end{align*}
    \ldots

  \item[$\scal{\sigma \tau}$]
    \begin{align*}
      \zeta \mapsto \zeta^{77} = \zeta^{2}
    \end{align*}
    Note that $\Image(\zeta + \zeta^{2} + \zeta^{4} + \zeta^{8}) \neq 0$.
    So it is not the fixing field of $\Q$, so it has to be the only remaining one.
\end{itemize}

  \begin{center}
  \begin{tikzcd}[ ] %\arrow[bend right,swap]{dr}{F}
    & \Q(\zeta)
    \\
    \Q(\zeta^{3})
    \arrow[hook]{ur}{2}
    & \Q(\zeta + \zeta^{4})
    \arrow[hook]{u}{2}
    & \Q(\zeta + \zeta^{-1})
    \arrow[hook]{ul}{2}
    \\
    \Q(\zeta^{5})
    \arrow[hook]{ur}{2}
    & \Q(\zeta^{3} + \zeta^{12})
    \arrow[hook]{ul}{2}
    \arrow[hook]{u}{2}
    \arrow[hook]{ur}{2}
    & \Q(\zeta + \zeta^{2} + \zeta^{4} + \zeta^{8})
    \arrow[hook]{ul}{2}
    \\
    & \Q
    \arrow[hook]{ul}{2}
    \arrow[hook]{u}{2}
    \arrow[hook]{ur}{2}
  \end{tikzcd}
  \end{center}
\end{ex}
