%In general, we have that $\abs{R(f)} \leq \degree(f)$.
\begin{ex}[]
  Take $k = \F_p(t)$ and $f = X^{p} - t \in k[X]$. Then $f$ is irreducible and $\abs{R(f)} = 1$.

  Let $E$ be a splitting field of $f$ and $\alpha \in R(f)$. So $\alpha^{p} = t$. It follows that
  \begin{align*}
    (X - \alpha)^{p} = X^{p} - \alpha^{p} = X^{p} - t = f \implies R(f) = \{\alpha\}
  \end{align*}
\end{ex}

A central goal of this section is to find out criteria for when $\abs{\Gal(E/K)} = [E:K]$
We will show that this is the case when $f$ is irreducible such that $\abs{R(f)} = \degree(f)$.

\begin{dfn}[]
  A polynomial is said to have no \textbf{multiple roots}, if in a splitting field, $\abs{R(f)} = \degree f$.
\end{dfn}

\begin{lem}[]
  Let $f \in k[x]$ and $f'\in k[x]$ the formal derivative of $f$. Then
  \begin{align*}
    f \text{ has no multiple roots } \iff \gcd(f,f') \in k[x]^{\times}
  \end{align*}
\end{lem}
We will prove this in exercise sheet $2$.

The nice thing about this is that the euclidean algorithm can calculate the $\gcd$ and we know that it is rather easy.

\begin{cor}[Rotman Lemma 3.4]
  Let $f \in k[X]$ irreducible satisfying one of the following:
  \begin{enumerate}
    \item $\charac k = 0$
    \item If $\charac k > 0$,then $\charac k \not| \degree f$
  \end{enumerate}
  then $f$ does not have multiple roots.
\end{cor}
\begin{proof}
  For $d = \degree f$ and $a_d \neq 0$ we write out $f$ and $f'$
  \begin{align*}
    f(x) &= a_xX^{d} + a_{d-1}X^{d-1} + \ldots + a_0
    f'(x) &= a_xdX^{d-1} + a_{d-1}(d-1)X^{d-2} + \ldots + a_1
  \end{align*}
  Since $f$ is irreducible and either (a) or (b) is true, we know $a_d d \neq 0$, so $\degree f' = d-1$.

  If $p \in k[x]$ divides both $f$ and $f'$, then for sure $\degree p \leq d -1$. But $f$ is irreducible, so $\degree p = 0$.
  Therefore, $p \in k$ and $\gcd(f,f') = 1$. Using the previous lemma, it follows that $f$ does not have multiple roots.
\end{proof}


\begin{dfn}[]
\begin{itemize}
  \item An irreducible polynomial is \textbf{separable}, if it has no multiple roots.
  \item A polynomial is \textbf{separable}, if all its irreducible factors are separable
\end{itemize}
\end{dfn}
\begin{ex}[]
  $X^{4} + 1 \in \Q[X]$ is irreducible, and since $\charac \Q = 0$, it follows from the previous corollary that it seperable. 
  Note that $(X^{4} + 1)^{15}$ is also separable, in the reducible sense.
\end{ex}

Let $E/k$ be a field extension, $\alpha \in E$ $\phi_{\alpha}: k[X] \to  E$ the evaluation homomorphism at $\alpha$.
Then $\Ker \phi_{\alpha}$ is an ideal in $k[X]$. There are two possibilities
\begin{itemize}
  \item $\Ker \phi_{\alpha} = \{0\}$. We then say that $\alpha$ is \textbf{transcendent} over $k$.
  \item $\Ker \phi_{\alpha} \neq \{0\}$ and we say that $\alpha$ is \textbf{algebraic} over $k$. Since $k[x]$ is a PID there exists a unique unitary (leading coefficient $= 1$) polynomial $\text{irr}(\alpha,k)$ that generates $\Ker \phi_{\alpha}$, i.e. ($\Ker \phi_{\alpha} = (\text{irr}(\alpha,k))$.
    We call this polynomial the \textbf{minimal polynomial} (and sometimes write $m_{\alpha,k}$.
\end{itemize}
Since $m_{\alpha,k}$ is irreducible and $k[X]$ is a euclidean domain, it follows that $\faktor{k[x]}{\Ker \phi_{\alpha}}$ is a field and
$\phi_{\alpha}$ induces a field isomorphism
\begin{align*}
  \overline{\phi}_{\alpha}: \faktor{k[X]}{\Ker \phi_{\alpha}} \to k(\alpha) 
\end{align*}


Let $\phi: k \to  k'$ be a field isomorphism. By the universal property of the polynomial ring, this induces a ring isomorphism
\begin{align*}
  \phi_{\ast}: k[X] \to k'[X], \quad \phi_{\ast}(a_nX^{n} + \ldots a_0) := \phi(a_n)X^{n} + \ldots \phi(a_0)
\end{align*}
Since $\phi_{\ast}$ is a ring isomorphism it follows that $p \in k[x]$ is irreducible if and only f $\phi_{\ast}(p)$ is irreducible.

\begin{lem}[Rotman 3.130]\label{lem:rot3-130}
Sei $\phi: K \to K'$ ein Körperisomorphismus, $f \in K[X]$ irreduzibel und $f_\ast = \phi_\ast(f)\in K'[X]$.

Dann gibt es für alle $\alpha \in R(f),\ \beta \in R(f_\ast)$ einen Körperisomorphismus $\widehat{\phi}: K(\alpha) \to K'(\beta)$ der $\phi$ erweitert und $\alpha$ auf $\beta$ abbildet.
\end{lem}
\begin{proof}[Beweis]
Da $K[X]$ ein Hauptidealring und $f \in K[X]$ irreduzibel ist, ist $(f)$ ein maximales Ideal.
Insbesondere ist $K[X]/(f)$ ein Körper.
Weiterin ist $\alpha$ eine Nullstelle von $f$ und da $f$ irreduzibel ist, generiert $f$ gerade den Kern des Evaluationshomomorphismus: $\Ker \ev_\alpha = (f)$.
Dasselbe gilt natürlich auch für $f_\ast$.

Darum induziert $\phi_\ast$ mit dem Ersten Isomorphiesatz einen Körperisomorphismus
\begin{center}
\begin{tikzcd}
    K[X] \arrow[]{r}{\phi_\ast}
    \arrow[]{d}{\pi_1}
    & 
    K'[X]
    \arrow[]{d}{\pi_2}
\\
    \faktor{K[X]}{\Ker \ev_\alpha} 
    \arrow[]{r}{\overline{\phi_\ast}}
    &
    \faktor{K'[X]}{\Ker \ev_{\beta}}
\end{tikzcd}
\end{center}
Diese Abbildung ist wohldefiniert, da die Abbildung $\pi_2 \circ \phi_\ast :K[X] \to \faktor{K'[X]}{\Ker \ev_\beta}$ für $\pi_2$ die kanonische Projektion einen Ringhomomorphismus ist und als Kern gerade $\Ker \pi_2 \circ \phi_\ast = \Ker \ev_\alpha$ hat.

Zuletzt benutzen wir gerade die Definition der Adjunktion von Nullstellen an einem Körper und erhalten somit Körperisomorphismen  $\Phi,\Psi$:
\begin{align*}
    \Phi: \faktor{K[X]}{\Ker \ev_\alpha} \to K(\alpha), \quad \text{und} \quad \Psi: \faktor{K'[X]}{\Ker \ev_\beta} \to K'(\beta)
\end{align*}
und man beachte, dass dies wegen $\phi_\ast(f) = f_\ast$ gerade $\alpha$ auf $\beta$ abbildet.

Wir können die ganze Argumentation in einem Diagramm zusammenfassen.
\begin{center}
\begin{tikzcd}[row sep=1em]
    K 
    \arrow[]{r}{\phi}
    \arrow[hookrightarrow]{d}{}
    & 
    K'
    \arrow[hookrightarrow]{d}{}
\\
    K[X] \arrow[]{r}{\phi_\ast}
    \arrow[]{d}{\pi_1}
    & 
    K'[X]
    \arrow[]{d}{\pi_2}
\\
    \faktor{K[X]}{\Ker \ev_\alpha} 
    \arrow[]{r}{\overline{\phi_\ast}}
    \isoarrow{d}
    \arrow[]{d}{\ \Phi}
    &
    \faktor{K'[X]}{\Ker \ev_{\beta}}
    \isoarrow{d}
    \arrow[]{d}{\ \Psi}
\\
    K(\alpha) \arrow[]{r}{\widehat{\phi} = \Psi\circ\overline{\phi_\ast}\circ\Phi^{-1} }
    & 
    K'(\beta)
\end{tikzcd}
\end{center}
und sehen, dass $\widehat{\phi} = \Psi \circ \overline{\phi_\ast} \circ \Phi^{-1}$ gerade die gewünschten Eigenschaften hat.
\end{proof}


A stronger version of this lemma is the following proposition:
\begin{prop}[]\label{prop:2-16}
  Let $k \subseteq B \subseteq E$ be field extensions where $E$ is the splitting field of some $g \in k[x]$.
  Then, any automorphism $\sigma: B \to B$ can be extended to an automorphism $\Sigma: E \to  E$
\end{prop}
\begin{proof}
  We prove this using induction on $[E:k]$. If $[E:k]=1$, then $f$ splits in linear factors in $k[X]$ and the same is true for $f_\ast$.
\end{proof}
