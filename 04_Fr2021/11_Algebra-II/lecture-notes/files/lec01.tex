\section{Introduction}
The main topic of the lecture will be \textbf{Galois theory}.
The motivating problem in Galois theory is to find a formula for solutions to the function $x^{n} a_{n-1}x^{n-1} + \ldots + a_0 = 0$.
Formulae for linear and quadratic polynomials were already known to Babylonian mathematicians ($\sim$ 1700 B.C).

Euclid ($\sim$ 300 BC) was able to translate the problem for quadratic equations into a geometric problem.

Arabian mathematician al-Khwarizmi (780-850) wrote a book ``al-gabr'' where we would present a way to systematically solve linear and quadratic equations. From the name al-gabr came our modern anglizied name ``Algebra''.

In 16-th century Italy, Scipione del Ferro was able to solve equations of degree $3$, with the degree $4$ case being solved by Ludovico Ferrari.
These solutions were initially kept secret until they were publicised by Cardano.
Cardano's transformed equations of the form $x^{3} + ax^{2} + bx + c =0$ into simpler equations of the form
\begin{align*}
  \xi^{3} + p \xi + q = 0 \quad \text{for} \quad \xi = x - \frac{a}{3}
\end{align*}
The idea is to then write $\xi = y + u$, where we set $u$ later to simplify the equation. Substituting gives us
\begin{align*}
  \xi^{3} + p \xi + q &= y^{3} + 3y^{2}u + 3yu^{2} + u^{3} + p(y+u) + q\\
                      &= y^{3} + (y+u)(3yu + p) + u^{3} + q
\end{align*}
By setting $u$ such that $3yu + p = 0$, i.e. $u = -\frac{p}{3y}$, we obtain a simple formula for $y$
\begin{align*}
  y^{3} - \frac{p^{3}}{27y^{3}} + q = 0 \implies y^{6} + qy^{3} - \left(\frac{p}{3}\right)^3
\end{align*}
Which is quadratic in $z = y^{3}$ and can easily be solved. The solution only uses the arithmetic operations of addition, subtraction, multiplication, division and taking roots.


Many people tried to solve the equation for degree $5$, but were unable to do so.

Substantion contribution was made by Lagrange (1736 - 1813) which found htat if $\xi_1, \xi_2, \xi_3$ were solutions to the equation $\xi^{3} + p \xi + q = 0$, then the six solutions to the resolvent $y^{6} + qy^{3} - \left(
  \frac{p}{3}
\right)^{3} = 0$
can be expressed in terms of the Permutation group $S_3$
\begin{align*}
  y_{\sigma} = \frac{1}{3} \left(
    \xi_{\sigma(1)} + \omega \xi_{\sigma(2)}+ \omega^{2} \xi_{\sigma(3)}
  \right)
\end{align*}
where $\omega = e^{\frac{2 \pi i}{3}}$ is the primary third root of unity.

Paolo Ruffini then wanted to show that the general solution to the equation of degree $5$ had no closed formula.
He studied rational functions $f(\zeta_1, \ldots \zeta_5)$, where $\xi_i$ are solutions to the equation of degree $5$ and realized that the permutations $\sigma \in S_5$ that keep $f(\sigma(\zeta))$ invariant form a subgroup of $S_5$

He then classified the subgroups of $S_5$, and Nils Henrik Abel finished the proof the theorem
\begin{thm}[Abel-Ruffini]
  The general equation of fifth degree 
  \begin{align*}
    x^{5} + ax^{4} + bx^{3} + cx^{2} + dx + e = 0
  \end{align*}
  does not have a formula for the roots that only uses finite arithmetic operations.
\end{thm}
The statement of the theorem is very intuitive, but it is hard to give a precise mathematical definition of it. This makes it also quite hard to prove.
We will prove this as a side-result of Galois-theory, where we have seen that the alternating group $A_5$ is anabelian and simple (a group whose normal divisors are only the trivial group and the group itself).

What we will do is to every polynomial $f \in k[X]$ we associate a group $\Gal(f) < S_n$ and show that if $k$ is ``nice enough'', then $f(X) = 0$ is solvable if and only if $\Gal(f)$ is resolvable.



\section{Galois Groups of a field extension}

Let $E$ be a field. Then the set of field isomorphisms
\begin{align*}
  \Aut(E) := \{\phi: E \to  E \big\vert \phi \text{ is a field isomorphisms}\}  
\end{align*}
is a group under composition.

If $k \subseteq E$ is a subfield, we call $E$ a field-extension of $k$.

\begin{dfn}[]
  For a given field extension $E/k$, the associated \textbf{Galois group} is the subgroup
  \begin{align*}
    \Gal(E/k) := \{\phi \in \Aut(E): \phi(x) = x \forall x \in k\} < \Aut(E)
  \end{align*}
\end{dfn}
We know from Algebra I that $E$ can be seen as a $k$-vector space. As we will see in the exercise classes, we can show that every $\phi \in \Gal(E/k)$ is an isomorphism of $E$ as a $k$-vector space.

For example, we can show that $\Gal(\C,\R) = \{id_{\C}, \overline{\cdot}\}$, where $\overline{\cdot}$ is the complex conjugation. 

\begin{lem}[]\label{lem:2.4}
  Let $f \in k[X]$ be a polynomial and $E/k$ be a field extension such that $f$ splits in $E$ (i.e. $f$ can be written as a product of linear factors).
  If we set $R(f) \subseteq E$ to be the set of roots of $f$, then every $\sigma \in \Gal(E/k)$ induces a permutation on $R(f)$.
\end{lem}
\begin{proof}
  Let $f(X) = a_nX^{n} + a_{n-1}X^{n-1} + \ldots + a_0$, where $a_i \in k$, $\alpha \in R(f)$ and $\sigma \in \Gal(E/k)$.
  Then we can write
  \begin{align*}
    0 &= f(\alpha) = \sigma(f(\alpha)) = \sigma(a_na^{n} + \ldots + a_0) = \sigma(a_n)\sigma(\alpha)^{n} + \ldots + \sigma(a_0)\\
      &= a_n \sigma(\alpha)^{n} + \ldots + a_0 = f(\sigma(\alpha))
  \end{align*}
  so $\sigma(\alpha) \in R(f)$ and $\sigma(R(f)) \subseteq R(f)$.
  Since $\sigma: E \to  E$ is injective and $\abs{R(\phi)}\leq n$, it follows that $\sigma(R(f)) = R(f)$
\end{proof}

\begin{dfn}[]
  Let $f \in k[X]$. The \textbf{Galois group} of $f$ is the Galois group $\Gal(E/k)$ associated to the field extension of the splitting field $E$ of $f$.
\end{dfn}
We know from Algebra I that the splitting field always exists and is unique up to isomorphism. So the Galois group is also unique up to isomorphism.

For a set $X$, let $S_X$ denote the set of permutations of $X$.

\begin{lem}\label{lem:2.4}
  Let $E/k$ be a splitting field of a polynomial $f \in k[X]$ and $R(f) \subseteq E$ its roots. Then the restriction mapping
  \begin{align*}
    \Gal(E/k) \to S_{R(f)}, \quad \sigma \mapsto  \sigma_{|R(f)}
  \end{align*}
  is an injective group homomorphism.
\end{lem}
\begin{proof}
  From Lemma \ref{lem:2.4} we know that $\sigma(R(f)) = R(f)$ for all $\sigma \in \Gal(E/k)$.
  For injectivity let $R(f) = \{\alpha_{1}, \ldots, \alpha_{n}\}$. From Algebra I, we know tha $E = k[\alpha_{1}, \ldots, \alpha_{n}]$, where $k[\alpha_{1}, \ldots, \alpha_{n}]$ is the image of the restriction mapping
  \begin{align*}
    k[X_{1}, \ldots, X_{n}] \to E, \quad p \mapsto p(\alpha_{1}, \ldots, \alpha_{n})
  \end{align*}
  We now show that the kernel of the ring homomorphism is the identity $\id_E$. 
  Assume $\sigma \in \Gal(E/k)$ such that $\sigma|_{R(f)} = \id_{R(f)}$, which means $\sigma(\alpha_i) = \alpha_i$.
  Let $\xi \in E$ and chose $p \in k[X_{1}, \ldots, X_{n}]$ such that $p(\alpha_{1}, \ldots, \alpha_{n}) = \xi$.
  Since $\sigma(x) = x \forall x \in K$ it follows that
  \begin{align*}
    \sigma(\xi) = \sigma(p(\alpha_{1}, \ldots, \alpha_{n})) = p(\sigma(\alpha_1), \ldots \sigma(\alpha_n)) = p(\alpha_{1}, \ldots, \alpha_{n}) = \xi
  \end{align*}
  which shows $\sigma = \id_E$.

  An alternate proof would be to write
  \begin{align*}
    E = k(\alpha_{1}, \ldots, \alpha_{n}) = \left\{\frac{p(\alpha_{1}, \ldots, \alpha_{n})}{q(\alpha_{1}, \ldots, \alpha_{n})} \big\vert p,q \in k[X_{1}, \ldots, X_{n}], q(\alpha_{1}, \ldots, \alpha_{n}) \neq 0\right\}
  \end{align*}
  and applying such a $\sigma$ on any rational function obviously keeps it invariant, so $\sigma = \id_E$.
\end{proof}
