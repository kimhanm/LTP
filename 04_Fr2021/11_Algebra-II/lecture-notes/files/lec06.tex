\begin{lem}[]
In the context of the previous Lemma, assume that the $u_i$ are ordered such that there exist $m_i \in \Z$ such that
\begin{align*}
  u_1^{m_1} \in K, \quad u_2^{m_2} \in k(u_1), \quad \ldots \quad u_t^{m_t} \in k(u_1, \ldots, u_{t-1})
\end{align*}
Then $E/k$ is a radical extension
\end{lem}
\begin{proof}
  For $\sigma_r \in \Gal(E/k)$,
  set $B_0 = k, B_1 = k(\sigma_1(u_1), \ldots \sigma_l(u_1))$ and inductively, set $B_j = B_{j-1}(\sigma_1(u_j), \ldots \sigma_l(u_j))$.
This results in a tower of extensions
\begin{align*}
  k = B_0 \subseteq B_1 \subseteq B_2 \subseteq \ldots \subseteq B_t = E
\end{align*}
Now we show that $B_1/k$ is a radical extension.

Using the assumption, we find that for any $1 \leq r \leq l$:
\begin{align*}
  \sigma_r(u_1)^{m_1} = \sigma_r(\underbrace{u_1^{m_1}}_{\in k}) = u_1^{m_1} \in k \subseteq k(\sigma_1(u_1), \ldots, \sigma_{r-1}(u_1))
\end{align*}
So for all $1 \leq r \leq t$, the extension
\begin{align*}
  k(\sigma_1(u_1), \ldots, \sigma_{r-1}(u_1)) \subseteq k(\sigma_1(u_1), \ldots, \sigma_r(u_1))
\end{align*}
is pure. Therefore, we get a tower or pure extensions
\begin{align*}
  k \subseteq k(\sigma_1(u_j)) \subseteq k(\sigma_1(u_j),\sigma_2(u_j)) \subseteq \ldots \subseteq k(\sigma_1(u_j), \ldots, \sigma_{r-1}(u_j)) \subseteq k(\sigma_1(u_j),\ldots, \sigma_r(u_j)) \subseteq \ldots = B_1
\end{align*}
which means that $k \subseteq B_1$ is radical.

The same argument goes for the other $B_j/B_{j-1}$, so $E/k$ is radical.
\end{proof}

\begin{cor}[] \label{cor:radical-normal}
  Let $K/k$ be a radical extension. Then there exists a field $F$ with $k \subseteq K \subseteq F$ such that $F/k$ is radical and normal.
\end{cor}

Recall the definition for resolvable groups. A group $G$ is resolvable, if there exists a subnormal sequence
\begin{align*}
  \{e\} = G_0 \lhd G_1 \lhd \ldots \lhd G_t = G
\end{align*}
such that the factor groups $G_{i+1}/G_i$ are abelian.

And we also saw that $A_n,S_n$ were resolvable for $n \geq 4$.
There is a criterion for when a group is resolvable, that uses iterated commutator groups:

For a group $G$, let $[G,G]$ be the subgroup generated by $\{[a,b] \big\vert a,b \in G\}$, where $[a,b] = aba^{-1}b^{-1}$.
This subgruop $[G,G]$ is a normal divisor of $G$ and also characteristic, which means that it is invariant under automorphisms of $G$ ($\forall \alpha \in \Aut(G), \alpha([G,G]) = [G,G]$).

In Algebra I, we inductively defined
\begin{align*}
  G^{(1)} := [G,G], \quad G^{(j+1)} := [G^{(j)}, G^{(j)}]
\end{align*}
and saw that $G$ is resolvable if and only if there exists an $n$ such that $G^{(n)} = \{e\}$.

We sometimes write $G_{\text{ab}} := G/[G,G]$, which is the largest abelian quotient of $G$ in the sense that for any homomorphism to an abelian group $\phi: G \to  A$, then $\phi$ factors through $G_{\text{ab}}$.
That is, there exists a unique group homomorhpism $\overline{\phi}:G_{\text{ab}} \to A$ such that the following diagram commutes
\begin{center}
\begin{tikzcd}[column sep=0.8em] %\arrow[bend right,swap]{dr}{F}
  G \arrow[]{rr}{\phi} 
  \arrow[swap]{dr}{\pi}
  & & A\\
 & G_{\text{ab}}\arrow[swap]{ur}{\overline{\phi}}
\end{tikzcd}
\end{center}



\begin{prop}[]\label{prop:3-13}
\begin{enumerate}
  \item If $H < G$, then $G$ resolvable $\implies$ $H$ resolvable.
  \item If $N \lhd G$, then $G$ is resolvable if and only if $N$ and $G/N$ are resolvable.
\end{enumerate}
\end{prop}
\begin{proof}
Note that if $\phi: G \to  L$ is a homomorphism, then
\begin{align*}
  \phi([a,b]) = [\phi(a),\phi(b)], \quad \forall a,b \in G
\end{align*}
so $\phi([G,G]) \subseteq [L,L]$.
In particular, if $\phi$ is surjective, then $\phi([G,G]) = [L,L]$.

\begin{enumerate}
  \item Since $H < G$, we can consider the embedding $\phi: H \to G$ to get $[H,H] \to [G,G]$. Inductively, we also get $H^{(j)} \subseteq G^{(j)}$, so if $G$ is resolvable for $G^{(n)} = \{e\}$, then $H^{(n)} \subseteq G^{(n)} = \{e\}$.
  \item If $G$ is resolvable, then, by (a), $N$ is resovable.
    Taking the surjective quotient map $\pi: G \to G/N$, we get $\pi([G,G]) = [G/N,G/N]$, and also inductively, 
    $\pi(G^{j}) = (G/N)^{(j)}$.

    Now, assume that $N$ and $G/N$ are resolvable. Then there exists an $n \geq 1$ such that 
    \begin{align*}
      \{e\} = (G/N)^{(n)} = \pi(G^{n}) \implies G^{(n)} \subseteq N
    \end{align*}
    but since $N$ is resolvable, there exists a $l \geq 1$ such that $N^{(l)} = \{e\}$. Therefore
    \begin{align*}
      G^{(n+l)} = (G^{(n)})^{(l)} \subseteq N^{l} = \{e\}
    \end{align*}
\end{enumerate}
\end{proof}

