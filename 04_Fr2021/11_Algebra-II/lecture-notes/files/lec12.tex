
\begin{cor}[]
  If $\gcd(p,n) = 1$, then $[\F_p[n]:\F_p]$ is a power of $p$ modulo $n$.
\end{cor}
To better understand what the group $\Z/n\Z^{\times}$ is like we show that we can decompose it into smaller copies, if $n$ is not a prime.

\begin{thm}[]
  If $\gcd(n,m) = 1$, then
  \begin{align*}
    (\Z/mn\Z)^{\times} \iso (\Z/m\Z)^{\times} \times (\Z/n\Z)^{\times}
  \end{align*}
  and if $p$ is prime $> 2$, then
  \begin{align*}
    (\Z/p^{r}\Z)^{\times} \iso \Z/p^{r-1} \times \Z/(p-1)\Z
  \end{align*}
  in particular, we have
  \begin{align*}
    (\Z,2^{r}\Z)^{\times} \iso \Z/2\Z \times (\Z/2^{r-2}\Z)
  \end{align*}
\end{thm}

For example we can calculate
$(\Z/7\Z)^{\times} = \{1,2,3,4,5,6\} \iso \Z/6\Z$
The possible exponents are $1,2,3,6$.
\begin{itemize}
  \item For $p = 2$, we have $2^{3} = 8 = 1 \mod 7$ and thus $[\F_p[7]:\F_2]= 3$ with 
    \begin{align*}
      \Phi_7(T) = (T^{3} + T + 1)(T^{3} + T^{2} + ) \in \F_[T]
    \end{align*}
  \item For $p = 3$, we need exponent $6$, and $\Phi_7$ is irreducible mod $3$.
  \item The prime $p = 13$, is of order $2$, because $13^{2} = 24 \cdot 7 + 1$. We then have
    \begin{align*}
      \Phi_7(T) = (T^{2} + 3T + 1)(T^{2} + 5T + 1)(T^{2} + 6T + 1)
    \end{align*}
  \item $p = 29$ is of order $1$ and
    \begin{align*}
      \Phi_7(T) = (T-7) (T-16)(T-20)(T-23)(T-24)(T-25)
    \end{align*}
\end{itemize}


\begin{thm}[]
  If $\gcd(p,n) = 1$, then the unitary irreducible factors of $\Phi_n$ if $\F_p[X]$ are all different and have the same degree of order $p \mod n$ in $(\Z(n\Z)^{\times}$.
\end{thm}

\begin{proof}
  Because $X^{n} -1$ has no multiple roots the same holds for $\Phi_n(X)$, therefore all irreducible factors have to be different.

  Since $\F_p[n]$, is a splitting field of $X^{n}-1$, we have that $\Phi_n(X)$ splits into linear factors in $\F_p[n][X]$, so the irreducible factors of $\Phi_n(X)$ must be of the FOrm $\text{irr}(\alpha,\F_p)$ for some $\alpha \in \F_p[n]$ with $\Phi_n(\alpha) = 0$.

  So it suffices to show that if $\Phi_n(\alpha) = 0$, then $\alpha$ must be a primitive $n$-th root of unity.
  From this, it would follow that $\F_p[n] = \F_p(\alpha)$ and the minimal polynomial $\text{irr}(\alpha,\F_p)$ has degree $[\F_p[n] : \F_p]$.

  Now assume that $\alpha$ is \emph{not} a primitive $n$-th root of unity with $\Phi_n(\alpha) = 0$.

  Then there exists a $1 \leq m <n$ with $\alpha^{m} = 1$ and suc that $m$ divides $n$.
  And we can write
  \begin{align*}
    0 = X^{m} - 1 = \prod_{l | m} \Phi_d(X) 
  \end{align*}
  so there has to be a $d_0 | m$ with $\Phi_{d_0}(\alpha) = 0$.
  But we can also write
  \begin{align*}
    X^{m} -1 = \Phi_n(X) \prod_{d|n, d < n} \Phi_d(X)
  \end{align*}
  so since $d_0$ divides $m$ and is a true divisor of $n$ we have that $\alpha$ must be a root of multiplicity at least two.
\end{proof}


Dirichlet proved that given some $a \in (\Z/n\Z)^{\times}$, there are finitely primes $p$ with $p = a \mod n$.


\begin{center}
Missing Second Half 28.05.21
\end{center}
