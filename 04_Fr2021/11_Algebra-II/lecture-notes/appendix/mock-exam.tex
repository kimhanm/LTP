
\section*{Mock Exam FS 2021}


\subsection*{2.a}
Das Lemma über Fixpunkte sagt, dass für eine $p$-Gruppe $G$ und eine $G$-Menge $T$ gilt
$$
\abs{\text{Fix}_G(T)} \underset{\mod p}{=} \abs{T}.
$$
Wir betrachten die Gruppenwirkung 
$$
\rho: G \times G \to G, (g,h) \mapsto ghg^{-1}
$$
und sehen, dass 
$$
h \in \text{Fix}_G(G) \iff \forall g \in G: ghg^{-1} = h \iff h \in Z_G.
$$
Nach dem Lemma gilt also
$$
\abs{Z_G} =
\abs{\text{Fix}_G(G)} 
\underset{\mod p}{=} \abs{G} = p^{m} \underset{\mod p}{=} 0
$$
Nun ist das neutrale Element $1 \in G$ im Zentrum, also ist $\abs{Z_G}$ ein vielfaches von $p$ und darum nicht-trivial.

\subsection*{2.b}
Nach (a) hat $Z_G$ Ordnung $p^k$ für ein $1 \leq k \leq m$ und darum hat
$G^{(1)} := G/Z_G$ Ordnung $p^{m-k} < p^{m}$ und ist somit entweder trivial (und somit $G$ nilpotent), oder wiederum eine $p$-Gruppe.

Wiederholt man im zweiten Fall das Argument für $G^{(1)}$, so sieht man dass $G^{(2)} := G^{(1)}/Z_{G^{(1)}}$ wieder entweder trivial oder eine $p$-Gruppe ist und kleinere Ordnung als $G^{(1)}$ hat.

Macht man das für $G^{(k+1) := G^{(k)}/Z_{G^{(k)}}}$, falls $G^{(k)}$ nicht trivial, immer so weiter, so verkleinert sich die Ordnung.
  Da $G$ endlich ist, muss dieser Prozess aufhören, d.h. $\exists n \in \N: G^{(n)} = \{1\}$.
  Also ist $G$ nilpotent.


\subsection*{3}
  
Der Klassifikationssatz sagt folgendes aus:

Sei $R$ ein Hauptidealring und $M$ ein endlich erzeugtes Modul über $R$. 

Dann ist $M$ isomorph zu $R^n \times M_{\text{tors}}$, wobei $n$ der Rang von $M/M_{\text{tors}}$ ist.

Zum Beweis betrachten wir die Abbildung
$$
\Psi: R^{n} \times M_{\text{tors}} \to M
$$
$$
(a,m) \mapsto \sum_{i=1}^{n}a_i (x_i + M_{\text{tors}})
$$



\subsection*{4.a}
Für $\xi$ eine primitive $n$-te Wurzel von $1$ ist das $n$-te Zyklotomische Polynom definiert als
$$
\Phi_n(X) = \prod_{a \in A_n}(X - \xi^{a})
,
A_n = \{a \in \{1,\ldots,n-1\}\big\vert \ggT(a,n) = 1\}
$$
Wir zeigen 





\subsection*{7.a}
Sei $G$ eine Gruppe der Ordnung $8 = 2^{3}$.
Wir gehen alle Teiler von $8$ durch und schauen was passiert wenn $G$ Elemente solcher Ordnung enthält.

\begin{itemize}
  \item Enthält $G$ ein Element der Ordnung $8$, so ist $G \cong \mathbb{Z}/8 \mathbb{Z}$.
  \item Angenommen, ein Element $g\in G$ hat Ordnung $4$. 
    Für $N := \langle g\rangle$ ist $|G|/|H| = 2$ d.h. $N$ ist ein Normalteiler und $G/N \cong \mathbb{Z}/2 \mathbb{Z}$.

    Für ein $x \in G \setminus N$ gilt dann $x^{2}N = (xN)(xN) = N$, also $x^{2} \in N$.

    Weil $N$ durch $g$ generiert wird, wird $N$ auch von $xgx^{-1}$ generiert, da
    $$
    N = xNx^{-1} = \{1,xgx^{-1},xg^{2}x^{-1},xg^{3}x^{-1}\}
    $$
    des weiteren hat $xgx^{-1}$ auch Ordnung $4$ und ist darum entweder $xgx^{-1} = g$ oder $xgx^{-1} = g^{3}$.
    Wir wissen nun
    $$
    x^{2} \in \{1,g,g^{2},g^{3}\}, \text{ und } xgx^{-1} \in \{g, g^{3}\}
    $$
    (Man beachte dass nicht jede der $4 \cdot 2$ Möglichkeiten unbedingt alle Gruppenaxiome erfüllt z.B. $x^{2} = g, xgx^{-1} = g^{3}$ vertragen sich nicht. Wir betrachten nur die, die das tun.)
    \begin{itemize}
      \item Hat $x$ die Ordnung $8$, so ist $G \cong \mathbb{Z}/8 \mathbb{Z}$.
      \item Hat $x$ Ordnung $2$ und $xgx^{-1} = g$, so ist $G$ abelsch und $G \cong \mathbb{Z}/4 \mathbb{Z} \times \mathbb{Z}/2 \mathbb{Z}$.

        Ist $xgx^{-1} = g^{3}$, so beschreibt
        $$
        G = \langle x,g \big\vert x^{2} = 1, xgx^{-1} = g^{-1}\rangle
        $$
        gerade die Diedergruppe $D_{2 \cdot 4}$.

         \item Hat $x$ Ordnung $4$, so ist $x^{2} = g^{2}$.
           Im Falle $xgx^{-1} = g$ ist $G$ abelsch und
           $$(xg)^{2} = xgx^{-1} x^{2}g = gx^{2}g = g^{4} = 1
           $$
           d.h. $xg$ ist ein Element der Ordnung $2$ und somit $G \cong \mathbb{Z}/4 \mathbb{Z} \times \mathbb{Z}/2 \mathbb{Z}$.

           Im Fall $xgx^{-1} = g^{3}$, so beschreibt
           $$
           G = \langle x,g \big\vert g^{4} = 1, x^{2} = g^{2}, xgx^{-1} = g^{-1}\rangle
           $$
           die Quaternionengruppe $Q_8$.

    \end{itemize}

  \item Haben alle Elemente von $G$ Ordnung $\leq 2$, so ist $G \cong (\mathbb{Z}/2 \mathbb{Z})^{3}$.
\end{itemize}


\subsection*{7.b}
Sei $G$ eine abelsche Gruppe der Ordnung $50$.
Wir wissen vom Klassifikationssatz über endlich erzeugte abelsche Gruppen, dass 
$$
G \cong \mathbb{Z}/(d_1) \times \ldots \times \mathbb{Z}/(d_n)
$$
für $1 \leq d_1 | \ldots | d_n \neq 0$.
Die Grupppen sind also
$$
\mathbb{Z}/50 \mathbb{Z}, \quad \mathbb{Z}/5 \times \mathbb{Z}/10 \mathbb{Z}
$$


Alternativ betrachtet man die Primfaktorzerlegung $50 = 2 \cdot 5^{2} = 2 \cdot 5 \cdot 5$, und erhält
$$
\mathbb{Z}/2 \mathbb{Z} \times \mathbb{Z}/25 \mathbb{Z} \left(
  \cong \mathbb{Z}/50\mathbb{Z}
\right), \quad \mathbb{Z}/2 \mathbb{Z} \times \mathbb{Z}/5 \mathbb{Z} \times \mathbb{Z}/ 5 \mathbb{Z} \left(
  \cong \mathbb{Z}/10 \mathbb{Z} \times \mathbb{Z}/5 \mathbb{Z}
\right)
$$



\subsection*{8}
Angenommen $[K(\alpha) : K] = n$. Seien $g,h \in R[X]$ mit $f = gh$.






\subsection*{11}
Wir zeigen, dass $f := X^{4} - X - 1 \in \mathbb{Z}[X]$ irreduzibel und primitiv ist und wenden dann den Gauss'schen Satz an.

$f$ ist offensichtlich primitiv. Für Irreduzibilität verwenden wir folgende Proposition:
Sei $R$ ein faktorieller Ring, $p \in R$ prim, und $f \in R[X] \setminus \{0\}$.
Gilt für $f_{\mathrm{mod }p} \in R/(p)[X]$, dass $\mathrm{deg} f = \mathrm{deg} f_{\mathrm{mod }p}$ und $f_{\mathrm{mod }p} \in R/(p)[X]$ irreduzibel ist,
so ist $f \in R[X]$ prim.

Wir setzen $p=2$ und zeigen, dass $f_{\mathrm{mod }p} = X^{4} + X + 1 \in \mathbb{F}_2[X]$ irreduzibel ist.
Angenommen $f_{\mathrm{mod }p} = gh$ für $g,h \in \mathbb{F}_2[X]$. 
Dann ist o.B.d.A $\deg g \leq 2$.
Die irreduziblen Polynome in $\mathbb{F}_2[X]$ mit grad $\leq 2$ sind gerade
$$
X, X + 1, X^{2} + X + 1
$$
Den Faktor $x$ ist offensichtlich nicht möglich und
den Fall $\deg g = \deg h = 2$ können wir ausschliessen, da $(X^{2} + X + 1)^{2} = X^{4} + X^{2} + 1 \neq f_{\mathrm{mod }p}$.

Des weiteren ist ein Polynom genau dann durch $X+1$ teilbar, wenn die Summe der Koeffizienten gerade ist. 
Das ist bei $f_{\mathrm{mod }p}$ nicht der Fall, also ist $f_{\mathrm{mod }p}$ irreduzibel in $\mathbb{F}_2[X]$.

Nach der Proposition ist $X^{4} - X - 1 \in \mathbb{Z}[X]$ irreduzibel und nach dem Gauss'schen Satz auch in $\mathbb{Q}[X]$.

\subsection*{12}
Nach dem Eisenstein Kriterium für $p = 5$ ist $f := X^{5} - 20 X + 5 \in \mathbb{Q}[X]$ irreduzibel.

Des weiteren ist 
$$f' = 5X^{4} - 20X = 5(X^{4} - 4)$$
mit $2$ reellen Nullstellen $\pm \sqrt[4]{4} = \pm \sqrt{2}$.
$f$ hat also $3 = p-2$ reelle Nullstellen und nach Satz \ref{thm:p-minus-two-roots} ist $\Gal(f) \cong S_5$.





