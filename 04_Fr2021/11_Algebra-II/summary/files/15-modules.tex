\section{Modules}
Modules are to ring what vector spaces are to fields.

\begin{dfn}
  For a ring $R$, an \textbf{$R$-module} $M$ is an abelian group with scalar multiplication
\begin{align*}
  R \times M \to  M, \quad (a,m) \mapsto a \cdot m
\end{align*}
For an index set $I$, we define the \textbf{free $R$-module}
\begin{align*}
  R^{(I)} := \{x: I \to R \big\vert x_i = 0 \text{ for almost all $i$}\}
\end{align*}
Any free module is isomorphic to $R^{(I)}$ for some set $I$.

For $R$-modules $M,N$, \textbf{module homomorphism} over $R$ is a group homomorphism $\Phi: M \to N$ that satisfies
\begin{align*}
  \Phi(am) = a \Phi(m) \quad \forall a \in R, m \in M
\end{align*}
\end{dfn}

\begin{dfn}
Let $M$ be an $R$-module.
An element $m \in M$ is called a \textbf{torsion element} of $M$, if there exists an $a \in R \setminus \{0\}$ with $a \cdot m = 0$.

Write $M_{\text{tor}}$ for the set of torsion elements of $M$.

We say that $M$ is a \textbf{torsion-module}, if $M_{\text{tor}} = M$ and we say that $M$ is \textbf{torsion-free}, if $M_{\text{tor}} = \{0\}$.
\end{dfn}


\begin{itemize}
  \item Every ideal $\mathfrak{a} \subseteq R$ is an $R$-module.
  \item If $R$ is a PID, then $\mathfrak{a}$ is a free $R$-module.
  \item An abelian group is a $\Z$-module with $n \cdot g = g^{n}$.
    Taking $a = \text{ord}(g)$, we see that $G$ is a torsion-module.

  \item $M = \Q/\Z$ is a torsion module over $\Z$.
  \item If $R$ is an integral domain and $M$ is a free $R$-module, then $M$ is torsion-free.
\end{itemize}




\begin{thm}[Classification theorem]
  Let $R$ be a PID and $M$ a finitely generated $R$-module.

  Then there exist $d_1|d_2| \ldots |d_n \in R \setminus \{0\}$ such that
  \begin{align*}
    M \iso R^{r} \times R/(d_1) \times \ldots \times R/(d_n)
  \end{align*}

  alternatively, we can write
  \begin{align*}
    M \iso R^{r} \times \prod_{j=1}^{n} M_{\text{tors}}^{(p_i)}
  \end{align*}
  where $p_1,\ldots,p_n$ are non-conjugate primes in $R$ and
  \begin{align*}
    M_{\text{tors}}^{(p_i)} &:= \{m \in M_{\text{tors}} \big\vert \exists k \in \N \text{ with } p_i^{k}m = 0 \} \\
                            &\iso \faktor{R}{(p_j^{n_j,1}} \times \ldots \times
                            \faktor{R}{(p_j^{n_j,k})}
  \end{align*}
\end{thm}


