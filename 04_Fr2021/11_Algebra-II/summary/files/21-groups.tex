\section{Groups}

Notation
\begin{itemize}
  \item $G \iso H$: $G$ is isomorphic to $H$
  \item $H < G$: $H$ is a subgroup of $G$.
\end{itemize}

Examples of groups
\begin{itemize}
  \item $\GL(n,K), \SL(n,K), \text{O}(n),\SO(n),\text{U}(n),\text{SU}(n), \text{SP}(2n), O(p,q)$
  \item $S_n$, Dihedral group $D_{2n}$ of order $2n$
  \item $\Aut(k), \Aut(G), \text{Bij}(X)$.
  \item Vector spaces, $R^{\times}$, $\pi_1(X,x_0)$.
\end{itemize}

\begin{ex}[Dihedral group]
  For $n \in \N$, the dihedral group $D_{2n}$ (in physics $D_n$) is the symmetry group of a regular $n$-gon embedded in $\R^{2}$ and has order $2n$.

  If $R$ is rotation with angle $\frac{2\pi}{n}$ and $T$ is mirroring around the $x$-axis,
  the dihedral group can be written as
  \begin{align*}
    D_{2n} 
    &= \{1,R,R^{2},\ldots,R^{n-1},T,RT,R^{2}T,\ldots,R^{n-1}T\}\\
    &= \scal{R,T\big\vert T^{2} = 1, R^{n} = 1,RT = R^{-1}}
  \end{align*}

\end{ex}

\begin{dfn}
Let $G$ be a group and $A \subseteq G$ a subset. 
The \textbf{subgroup generated by} $A$ is the smallest subgroup that contains $A$:
\begin{align*}
  \scal{A} := \bigcap_{X \subseteq H < G} H
\end{align*}
\end{dfn}
It can alternatively be written as the set
\begin{align*}
  \scal{A} = \{a_1^{k_1} \dots a_nk^{n} \big\vert n \in \N, a_{1}, \ldots, a_{n} \in A, k_i = \pm 1\}
\end{align*}


\begin{dfn}
  The \textbf{commutator} of two elements $g,h \in G$ is $[g,h] := gh g^{-1} h^{-1}$.
  The \textbf{commutator group} of $G$ is the subgroup
  \begin{align*}
    [G,G] := \scal{\{[g,h] \big\vert g,h \in G\}}
  \end{align*}
\end{dfn}

\begin{dfn}
  For every $g \in G$, the mapping
  \begin{align*}
    \gamma_g: G \to G, \quad x \mapsto gxg^{-1}
  \end{align*}
  is an automorphism, called a \textbf{inner automorphism}.

  This induces a mapping
  \begin{align*}
    \Phi: G \to \Aut(G), \quad g \mapsto \gamma_g
  \end{align*}
  The kernel of $\Phi$ is called the \textbf{center}
  \begin{align*}
    Z(G) = \{g \in G \big\vert \forall x \in G: [x,g] = 1\}
  \end{align*}
  We say that two elements $x,y \in G$ are \textbf{conjugate}, if there exists a $g \in G$ such that $\gamma_g(x) = gxg^{-1} = y$.
\end{dfn}

\begin{itemize}
  \item The center is obviously commutative, and the commutator group is not.
  \item Two matrices are conjugate, if and only if they have the same normal form.
  \item If the group is abelian, then every inner automorphism is trivially the identity $\id_G$.
\end{itemize}

\begin{dfn}[]
  Let $X,Y \subseteq G$ be subsets and $g \in G$. We define
  \begin{align*}
    X Y &= \{xy \big\vert x \in X,y \in Y\}\\
    gX &= \{gx \big\vert x \in X\}\\
    Xg &= \{xg \big\vert x \in X\}\\
    X_g &= \{\gamma_g(x) \big\vert x \in X\}\\
    g_X &= \{\gamma_x(g) \big\vert x \in X\}\\
    X^{-1}&= \{x^{-1} \big\vert x \in X\}
  \end{align*}

  For a subgroup $H < G$, we define the set of \textbf{left-subclasses} (Linksnebenklassen)
  \begin{align*}
    G/H := \{gH \big\vert g \in G\} 
  \end{align*}
  and analogously the right-subclasses $H \backslash G$.

  The \textbf{index} of the subgroup is
  \begin{align*}
    [G:H] := \abs{G/H} = \abs{H \backslash G}
  \end{align*}
\end{dfn}

\begin{prop}[]
Let $g,g' \in G$, $H < G$. Then
\begin{align*}
  gH = g'H
  \iff gH \cap g'H \neq \emptyset
  \iff g \in g'H
\end{align*}
\end{prop}

\begin{thm}[Lagrange]
If $\abs{G} < \infty$, then $\abs{G} = \abs{G/H} \cdot \abs{H}$.
\end{thm}
\begin{proof}[Proof sketch]
  Show that the map
  \begin{align*}
    \Phi: G/H \times H \to G, \quad (xH,h) \mapsto xh
  \end{align*}
  is bijective.
\end{proof}
As a corollary, the index of every subgroup is a divisor of the order of the group.

\subsection{Normal divisors}
The set of left-subclasses is not always a group. For example in $G = D_{2 \cdot 3}$, we have  $R\scal{T} R \scal{T} \neq R^{2} \scal{T}$.


\begin{dfn}
  A subgroup $H < G$ is called a \textbf{normal divisor} (write $H \lhd G$) if
  \begin{align*}
    \pi: G \to G/H, \quad g \mapsto  gH
  \end{align*}
  is a group homomorphism.

  We call $G$ simple, if only $\{e\}$ and $G$ itself are the only normal divisors of $G$.
\end{dfn}
\begin{itemize}
  \item Every subgroup of an abelian group is normal.
  \item Every subgroup of index $2$ is normal.
\end{itemize}

\begin{thm}
Let $N < G$ be a subgroup. Then the following are equivalent
\begin{enumerate}
  \item $N \lhd G$
  \item $xN = Nx$ for all $x \in G$
  \item There exists a group homomorphism $\phi: G \to S$ with $\Ker \phi = N$
  \item $(xH)(yH) = (xy)H$ for all $x,y \in G$
\end{enumerate}
\end{thm}

\begin{prop}[Universal property of Normal divisors]
  Let $\phi: G \to H$ and $N \lhd  G$ with $N \subseteq \Ker \phi$. Then there exists a unique group homomorphism
  $\overline{\phi}: G/N \to H$ such that the following diagram commutes
  \begin{center}
  \begin{tikzcd}[column sep=0.8em] %\arrow[bend right,swap]{dr}{F}
    G \arrow[]{rr}{\phi} \arrow[]{dr}{\pi} & & H\\
                                           & G/N \arrow[]{ur}{\exists! \overline{\phi}}
  \end{tikzcd}
  \end{center}
\end{prop}


\begin{thm}[First isomorphism Theorem]
  Let $\phi: G \to H$ be a group homomorphism.

  Then $\phi$ induces an isomorphism $\overline{\phi}: G/\Ker \phi \to \Image \phi$ such that the following diagram commutes
  \begin{center}
  \begin{tikzcd}[ ] %\arrow[bend right,swap]{dr}{F}
    G \arrow[]{r}{\phi} \arrow[]{d}{\pi}& H \\
    G/\Ker \phi \arrow[]{r}{\overline{\phi}} & \Image \phi < H \arrow[]{u}{\iota}
  \end{tikzcd}
  \end{center}
  where $\pi$ is the canonical projection and $\iota$ is the inclusion mapping.
\end{thm}

\begin{cor}[Second Isomorphism Theorem]
  Let $N \lhd G$ and $H < N$. Then
  \begin{align*}
    N \cap H \lhd H, &\quad N \lhd HN\\
    H/(N \cap H) &\iso HN/N = NH/N < G
  \end{align*}
\end{cor}
And in particular, $N \lhd G$, $N < H < G \implies N \lhd H$.

\begin{cor}[Third Isomorphism Theorem]
  Let $N \lhd G$. 
  Then there exists a correspondence between subgroups that contain $N$ and subgroups of $H/N$.

  For such subgroups $N < H < G$
  \begin{align*}
    H/N \lhd G/N \iff H \lhd G
  \end{align*}
  and we have an isomorphism
  \begin{align*}
    \faktor{G/N}{H/N} &\iso G/H\\
    (gN)(H/N) &\leftrightsquigarrow gH
  \end{align*}
\end{cor}
This corollary mirrors the one for ideals in a ring.



\begin{cor}[]
Let $N \lhd G$. For any other group $H$, there exists a natural isomorphism
\begin{align*}
  \Hom(G/N,H) \iso \{\phi \in \Hom(G,H) \big\vert \phi|_N = e_H\}
\end{align*}
\end{cor}
 
