\section{Ideals}
\begin{dfn}[]
  Let $R$ be a commutative ring. A subset $\mathfrak{a} \subseteq R$ is called an \textbf{ideal} if
  \begin{enumerate}
    \item $\mathfrak{a} \neq 0$
    \item $\forall a,b \in \mathfrak{a}: a + b \in \mathfrak{a}$
    \item $\forall a \in \mathfrak{a}, r \in R: r a \in \mathfrak{a}$
  \end{enumerate}
\end{dfn}

Trivially, $R$ itself and $\{0\}$ are ideals. The kernel of a ring homomorphism is an ideal.


\begin{dfn}[]
For a commutative ring $R$ and elements $a_1,\ldots,a_n$, define the \textbf{ideal generated by} $a_1,\ldots,a_n$ as
\begin{align*}
  (a_1,\ldots,a_n) = \{\sum_{k=1}^{n}a_i x_i \big\vert x_i \in R\}
\end{align*}
An ideal $\mathfrak{a}$ is called a \textbf{principal ideal} (Hauptideal), if it can be generated by a single element $\mathfrak{a} = (a)$.

If every ideal in $R$ is a principal ideal, then $R$ is called a \textbf{principal ideal domain} (PID).
\end{dfn}

A non-principal ideal is $(X,Y) \subseteq Z[X,Y]$

\begin{dfn}[]
For ideals $\mathfrak{a},\mathfrak{b}$ and an element $r \in R$ define
\begin{enumerate}
  \item $r \cdot \mathfrak{a} := \{r a \big\vert a \in \mathfrak{a}\} \subseteq \mathfrak{a}$
  \item $\mathfrak{a} + \mathfrak{b} := \{a + b \big\vert a\in \mathfrak{a}, b \in \mathfrak{b}\} \subseteq \mathfrak{a},\mathfrak{b}$
  \item $\mathfrak{a} \mathfrak{b} := \{\sum_{k=1}^{n}a_kb_k \big\vert a_k \in \mathfrak{a}, b_k \in \mathfrak{b}\} \subseteq \mathfrak{a},\mathfrak{b}$.
\end{enumerate}
\end{dfn}



\begin{thm}[]
The relation $a \sim b \iff a-b \in \mathfrak{a}$ defines an equivalence relation on $R$ and we write $a \equiv b \mod \mathfrak{a}$.

The quotient $R/\mathfrak{a}$ 
is called the \textbf{factor ring} (Faktorring) ``$R$ modulo $\mathfrak{a}$'' with induced addition and multiplication.
It allows a surjective ring homomorphism called the canonical projection
\begin{align*}
  \rho: R \to R / \mathfrak{a}, \quad x \mapsto x + \mathfrak{a}
\end{align*}
\end{thm}

\begin{lem}
Let $\mathfrak{a},\mathfrak{b} \subseteq R$ be ideals in a commutative ring. Then
\begin{enumerate}
  \item $I = R \iff 1 \in I \iff I \cap R^{\times} \neq \emptyset$
  \item $(a) \subseteq (b) \iff b | a$
\end{enumerate}
\end{lem}


\begin{prop}
  Let $\phi: R \to S$ be a ring homomorphism and $\mathfrak{a} \subseteq \Ker \phi$ an ideal.

  This induces a ring homomorphism
  $\overline{\phi}: \faktor{R}{\mathfrak{a}} \to  S$
  such that the following diagram commutes.
  \begin{center}
    \begin{tikzcd}[column sep=0.8em] %\arrow[bend right,swap]{dr}{F}
      R \arrow[]{rr}{\phi} \arrow[]{dr}{\rho}& & S\\
                                             & \faktor{R}{\mathfrak{a}} \arrow[]{ur}{\overline{\phi}}
    \end{tikzcd}
  \end{center}
  and if $\mathfrak{a} = \Ker \phi$, $\overline{\phi}$ is an isomorphism.
\end{prop}
For example, the map
\begin{align*}
  \phi: \R[X] \to \C, X \mapsto  i
\end{align*}
has kernel $(X^{2} + 1)$ and gives us the isomorphism $\faktor{\R}{(X^{2}+1)} \iso \C$.


\begin{dfn}[]
An ideal $\mathfrak{p} \subseteq R$ is called a \textbf{prime ideal}, if $\mathfrak{p} \neq R$ and for all $a,b \in R$ we have
\begin{align*}
  ab \in \mathfrak{p} \implies a \in \mathfrak{p} \text{ or } b \in \mathfrak{p}
\end{align*}.
An ideal $\mathfrak{m} \subseteq R$ is a \textbf{maximal ideal}, if $\mathfrak{m} \neq R$ and any other ideal containing $\mathfrak{m}$ is either $\mathfrak{m}$ or $R$.

Equivalently, we have
\begin{enumerate}
  \item $\mathfrak{p}$ is a prime ideal if and only if $R/\mathfrak{p}$ is an integral domain.
  \item $\mathfrak{m}$ is a maximal ideal if and only if $R/\mathfrak{m}$ is a field.
\end{enumerate}
\end{dfn}

\begin{enumerate}
  \item $\Z/(0)$ is a prime ideal, but not a maximal ideal.
  \item For $R = \Z[X]/(X^{2})$ we have
    \begin{align*}
      R/(X) \iso \faktor{\Z[X]}{(X^{2},X)} \iso \Z
    \end{align*}
    so $(X) \subseteq R$ is a prime ideal.
\end{enumerate}


\begin{prop}[]
  Let $\mathfrak{a}_0 \subseteq R$ be an ideal.
  There exists a correspondence between ideals that contain $\mathfrak{a}_0$ and ideals in $R/\mathfrak{a}_0$ given by
  \begin{align*}
    \mathfrak{a}_0 \subseteq \mathfrak{a} \subseteq R \leftrightsquigarrow \mathfrak{a} + \mathfrak{a}_0 \subseteq R/\mathfrak{a}_0
  \end{align*}
\end{prop}

Cliv-hanger: Does every ring have a maximal ideal?
