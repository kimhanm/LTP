
\subsection{Polynomial Rings II}

Let $R$ be a factorial ring and $K = \text{Quot}(R)$ it's quotient field.
Then
\begin{align*}
  f,g \in K, f \sim_R g \iff \frac{f}{g} \in R^{\times}
\end{align*}


\begin{dfn}[]
  Let $R$ be a UFD and $f = \sum_{i=0}^{n}a_i X^{i} \in R[X] \setminus \{0\}$.

  The \textbf{content} (Inhalt) of $f$ is defined as
  \begin{align*}
    I(f) := \gcd(a_{1}, \ldots, a_{n})
  \end{align*}
  we say $f$ is \textbf{primitive}, if $I(f) \in R^{\times}$.

\end{dfn}

\begin{lem}
  For $f \in K[X] \setminus \{0\}$, there exists a $d \in K \setminus \{0\}$ such that $f = d f^{\ast}$ for $f^{\ast} \in R[X]$ primitive.
  We call $d$ the \textbf{content} of $f$.

  Furthermore
  \begin{enumerate}
    \item $I(a f) \sim a I(f)$
    \item $I(fg) \sim I(f) I(g)$
    \item $I(f) \in R \iff f \in R [X]$.
  \end{enumerate}
\end{lem}


\begin{thm}[Gauss]
  Let $R$ be a UFD. Then $R[X]$ is a UFD and $R[X]$ has exactly two types of prime elmements.
  \begin{itemize}
    \item $f = p \in R$ prime
    \item $f \in R[X]$ primitive such that $f$ is irreducible as an element of $K[X]$.
    \item Let $f\in R[X]$ primitive. Then $f$ is irreducible in $R[X]$ if and only if it is irreducible in $K[X]$.
  \end{itemize}
\end{thm}

Let $R$ be a UFD and $p$ prime.
The inclusion $\iota: R \to R/(p), a \mapsto \overline{a}= a + (p)$
induces a ringhomomorphism
\begin{align*}
  R[X] \to R/(p)[X], \quad f = \sum_{k=0}^{n}a_kX^{k} \mapsto  \overline{f} = \sum_{k=0}^{n}\overline{a}_n X^{k}
\end{align*}

\begin{prop}[]
  If $f \in R[X] \setminus \{0\}$ satisfies $\degree(f) = \degree(\overline{f})$ and $\overline{f} \in R/(p)[X]$ is irreducible, then $f$ is irreducible.
\end{prop}


\begin{thm}[Eisenstein Criterion]
  Let $R$ be a UFD and $p \in R$ prime, $f = \sum_{i=1}^{n}a_i X^{i}$ primitive such that
  \begin{align*}
    p\not|a_n, p|a_i, 0 \leq i < n, p^{2} \not|a_0
  \end{align*}
  then $f$ is irreducible.
\end{thm}
\begin{proof}
Let $f = gh$ be a non-trivial decomposition.
Since $f$ is primitive and $I(gh) \sim I(g)I(h)$ both $g$ and $h$ must be primitive.

Take the equation $f = gh$ modulo $p$.
Because all non-leading coefficients of $f$ vanish, we are left with
\begin{align*}
  \overline{f} = \overline{g} \overline{h} = a_n X^{n}
\end{align*}
so $\overline{g},\overline{h}$ must be of the form
\begin{align*}
  \overline{g} = b_k X^{k}, \quad \overline{h} = c_l X^{l}
\end{align*}
with $k,l > 0$.
Because the constant terms of $g,h$ vanished, it means that $p$ must divide both $b_0,c_0$.
But $a_0 = b_0 c_0$, which contradicts $p^{2} \not | a_0$.
\end{proof}


A common trick is to take a polynomial $f(X)$ and use the substitution $Y = X + 1$ and look at $f(Y)$.

This trick is commonly used with the Eisenstein criterion to show irreducibility.

