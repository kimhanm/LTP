\section{Fields}
Instead of understanding fields intrinsically by studying their elements, it is often better to understand fields extrinsically through their relations with other fields.

\subsection{Field Extensions}

\begin{dfn}[]
  Let $L$ be a field and $k$ a subring that is also a field.

  Then say that $k$ is a \textbf{subfield} of $L$ and we call $L$ a \textbf{field extension} of $k$ and write $L/k$ to denote this fact.

  Because $L$ is also a $k$-vector space, write $[L:K] := \dim_k L$ for its \textbf{degree}.

  If $[L:K] < \infty$, we say that $L$ is a \textbf{finite} field extension of $k$.

\end{dfn}


\begin{lem}[Multiplicity of degree]
  Let $F/L/K$ be finite field extensions. Then
  \begin{align*}
    [F:K] = [F:L] [L:K]
  \end{align*}
\end{lem}
\begin{proof}[Proof sketch]
  If $x_{1}, \ldots, x_{m} \in F$ are a basis of $F$ over $L$ and $y_{1}, \ldots, y_{n} \in L$ are a basis of $L$ over $K$, then the products $x_iy_j \in F$ are a basis of $F$ over $k$.
\end{proof}

\begin{dfn}[]
  Every field $k$ contains a smallest subfield callled the \textbf{prime field} of $k$.
  It is either isomorphic to $\Q$ or $\F_p$ for some prime $p$.

  The \textbf{characteristic} of a field $k$ is define as
  \begin{align*}
    \charac k := 
    \left\{\begin{array}{ll}
        p & \text{ if its prime field is } F_p\\
        0 & \text{ if its prime field is } \Q
    \end{array} \right.
  \end{align*}
\end{dfn}

\begin{prop}[]
  A ring homomorphism between two fields is always injective and only exists if the two fields have the same characteristic.
\end{prop}
That is because the kernel is an ideal and the only ideals are $K$ and $\{0\}$.

\begin{dfn}[]
Let $L/k$ and $A \subseteq L$.
Write $k(A)$ for the smallest intermediate field between $k$ and $L$ that contains $A$.
If $A = \{\alpha_{1}, \ldots, \alpha_{n}\}$, write $K(a_{1}, \ldots, a_{n})$.
\end{dfn}

\subsection{Polynomial rings over fields}
\begin{dfn}[]
  Let $L/k$. For $\alpha \in L$ let
  \begin{align*}
    \ev_{\alpha} : K[X] \to L, \quad f \mapsto f(\alpha)
  \end{align*}
  be the evaluation mapping. $\alpha \in L$ is called
  \begin{itemize}
    \item \textbf{transcendent} over $k$, if $\ev_{\alpha}$ is injective.
    \item \textbf{algebraic} over $k$, otherwise.
      The kernel $\Ker \phi_{\alpha}$ is an ideal.
      Since $K[T]$ is a PID, it has a unique normed (leading coefficient $=1$) generator which we call the \textbf{minimal polynomial} of $\alpha$
      \begin{align*}
        m_{\alpha} = \text{irr}(\alpha,k) \in K[X]
        \quad \text{with} \quad \Ker \phi_{\alpha} = (\text{irr}(\alpha,k))
      \end{align*}
      in particular, if a polynomial $f(X)$ has $f(\alpha) = 0$, then the minimal polynomial divides $f$.
  \end{itemize}
\end{dfn}

\begin{itemize}
  \item $e,\pi \in \R$ are transcendent over $\Q$.
  \item $\sqrt{2} \in \R$ is algebraic with minimal polynomial $\text{irr}(\sqrt{2},\Q) = X^{2} - 2$.
\end{itemize}

If $\alpha$ is algebraic, then $\irr(\alpha,k)$ is irreducible and the ideal $(\irr(\alpha,k)) = \Ker \ev_{\alpha}$ is maximal.
Therefore, $k[X]/(\irr(\alpha,k)$ is a field and $\ev_{\alpha}$ induces a field isomorphism
\begin{align*}
  \overline{\ev_{\alpha}} : \faktor{k[X]}{\ev_{\alpha}} \to k(\alpha)
\end{align*}



\begin{itemize}
  \item $\Q(i) =\{ a + b i \big\vert a,b \in \Q\} \iso Q[X]/(X^{2}+1) = \{aX + b \big\vert a,b \in \Q\}$.
  \item $\Q(e) = \{\frac{f(e)}{g(e)} \big\vert f,g \in Q[X], g(e) \neq 0\}$.
\end{itemize}



\begin{cor}[Wantzel]
  With ruler and compass, neither $\sqrt[3]{2}$ nor an angle $\frac{\pi}{9}$ can be constructed.

  If $p \in \N$ is an odd prime number and the regular $p$-gon is constructable with ruler and compass, then $p$ must be of the form $p = 2^{2^{n}} + 1$.
\end{cor}
\begin{proof}[Proof Sketch]
  Start at $0 \in \R^{2}$.
  Then the set of constructable points is a field.

  Let $k_n$ be the field generated by the points obtained after $n$ steps.
  Then $[k_{n+1}:k_n] \leq 2$.
  So $[k_n:1]$ must be a power of $2$.
  But $[Q(\sqrt[3]{2}):\Q] = 3$.

  Moreover, $\cos(\frac{\pi}{9})$ has a minimal polynomi that is not $2^{k}$.
\end{proof}

\begin{dfn}[]
  A field extension $L/k$ is called \textbf{algebraic}, if every $\alpha \in L$ is algebraic over $k$
\end{dfn}

If $L/k$ is a finite field extension of degree $n$, then the $n+1$ elements $1,x,x^{2}, \ldots,x^{n}$ are linearlydependent. So every finite field extension is algebraic.


\begin{cor}[]
  If in a field extension $L/k$, $x,y \in L$ algebrac over $k$, then $x+y,x-y,xy, \frac{x}{y}$ are algebraic over $k$.
\end{cor}

\begin{cor}[]
  Let $F/L/k$ be field extensions. Then
  \begin{align*}
    F/k \text{ algebraic } \iff F/L \text{ and } L/k \text{ are algebraic}
  \end{align*}
\end{cor}



\begin{thm}[Kronecker]
  Let $k$ be a field, $f \in K[X]$ with $n = \degree f > 0$. 
  Then there exists a field extension $L/k$ such that
  \begin{align*}
    f(X) = a \prod_{i=1}^{n}(X - \alpha_i)
  \end{align*}
  where $a \in K^{\times}, \alpha_i \in L$.
\end{thm}
\begin{proof}[Proof Sketch]
  Let $p$ be an irreducible divisor of $f$.
  By induction over $n$, define
  \begin{align*}
    K_1 = \frac{K[X_1]}{p(X_1)}
  \end{align*}
  then $f(X)$ has a root $\alpha_1 := X_1 + (p(X_1)) \in K_1$.
  Divide $f$ by $\alpha_1$ which has smaller degree and keep going until $f_n \in K^{\times}$.
\end{proof}


\begin{itemize}
  \item $f = X^{2} + 1 \in \R[X]$ has the extension  $\C$
  \item For $f = X^{3} - 2 \in \Q[X]$ we have the extension $\Q(\sqrt[3]{2},e^{\frac{2 \pi i }{3}})$.
\end{itemize}


