\subsection{Algebraic Closure}

\begin{dfn}[]
  Let $k$ be a field, $f \in k[X]$ with $\degree f > 0$.
  A \textbf{splitting field} of $f$ over $k$ is a field extension $L/k$ such that
  \begin{enumerate}
    \item $f$ splits into linear factors in $L[X]$
    \item In any intermediate proper subfield $k \subseteq E \subsetneq L$, $f$ does not split over $E$.
  \end{enumerate}
\end{dfn}
\begin{itemize}
  \item Such a splitting field always exists and is unique up to isomorphism.
  \item A splitting field is an algebraic field extension of $k$.
  \item For $f \in k[X]$ and $L$ a splitting field of $f$ over $k$
    \begin{align*}
      [L:k] \leq (\degree f)!
    \end{align*}
\end{itemize}
\begin{dfn}[]
  A field $k$ is called \textbf{algebraically closed}, if every polynomial $f \in k[X]$ with $\degree f > 0$ has a root in $k$.
\end{dfn}
Every finite field is not algebraically closed, because we can take the polynomial
\begin{align*}
  f(X) = 1 + \prod_{\lambda \in k}(X - \lambda)
\end{align*}



\begin{prop}[]
  Let $L/k$ be a field extension and $L$ algebraically closed.
  Then
  \begin{align*}
    E = \{x \in L \big\vert x \text{ is algebraic over }k\} \subseteq L
  \end{align*}
  is also an algebraically closed field extension of $k$.
\end{prop}
\begin{proof}[Proof Sketch]
  Since $x,y$ algebraic means that $x+y,xy, \frac{x}{y}$ are algebraic for $y \neq 0$, $E$ is a field.

  For $f \in E[X], \degree f > 0$, by Kronecker's theorem, there exists a field $F \supseteq E$ over which $f$ splits.
  But $L$ is algebraically closed, so any root $\alpha \in F$ of $f$ is also present in $L$.
  Which means $\alpha \in L$ is algebraic over $k$, so by construction $\alpha \in E$.
\end{proof}

The construction of $E$ is in fact not dependent on the choice of $L$.

\begin{thm}[]
  For any field $k$, there exists a field extension $L/k$ with $L$ algebraically closed.

  Such extensions are unique up to isomorphism.
\end{thm}
\begin{proof}[Proof sketch]
  For any $f \in k[X], \degree f >0$, we take a free variable $Y_f$.
\end{proof}



