\section{Rings}





\begin{dfn}
  An element $a \in R \setminus \{0\}$ is called a \textbf{zero divisor} (Nullteiler), if there exists a $b \in R \setminus \{0\}$ with $a b = 0$.

  A ring $R \neq \{0\}$ is called an \textbf{integral domain} (Integritätsbereich), if it has no zero divisors.
  This is equivalent to asking that the following holds
  \begin{align*}
    ab = ac \land a \neq 0 \implies b = c
  \end{align*}
\end{dfn}

\begin{prop}[]
  \phantom{a}
  \begin{itemize}
    \item Every subring of an integral domain is again an integral domain.
    \item Every field is an integral domain.
    \item $Z/n\Z$ is an integral domain $\iff$ $n$ is prime.
  \end{itemize}
\end{prop}

\begin{dfn}
  In a commuative ring $R$, $a,b \in R$ we say that $a$ \textbf{divides} $b$, (write $a | b$) if there exists a $c \in R$ with $b = ac$.
  Define the \textbf{group of units} (Einheitengruppe)
  \begin{align*}
    R^{\times} := \{a \big\vert a \text{ divides } 1\}
  \end{align*}

  If $b = ac$ for some unit $c \in R^{\times}$, write $b \sim a$.
\end{dfn}

\begin{prop}[]
  \phantom{a}
\begin{itemize}
  \item $a \sim b \implies a|b$ and $b|a$
  \item If $R$ is an integral domain, then $a \sim b \Leftarrow a | b$ and $b | a$.
\end{itemize}
\end{prop}

\begin{dfn}[]
  Let $R$ be an integral domain. It's \textbf{quotient field} (Quotientenkörper) is the field
  \begin{align*}
    \text{Quot}(R) := \faktor{R \times (R \setminus \{0\})}{\sim}, \quad (a,b) \sim (p,q) \iff  aq = bp
  \end{align*}
  and write $\frac{a}{b} = [(a,b)]_{\sim}$.
  There is a canonical inclusion
  \begin{align*}
    \iota: R \hookrightarrow \text{Quot}(R), \quad x \mapsto \frac{x}{1}
  \end{align*}
\end{dfn}

\begin{itemize}
  \item $\text{Quot}(\Z) = \Q$
  \item Because $i^{2}, \sqrt{2}^{2} \in \Z$ we have $\text{Quot}(\Z[i]) = \text{Quot}(Z)[i]$, $\text{Quot}(\Z[\sqrt{2}]) = \text{Quot}(\Z)[\sqrt{2}]$
\end{itemize}





\begin{dfn}
  For a commutative ring $R$, the \textbf{polynomial ring} (with variable $X$) is the collection of finite power series
  \begin{align*}
    R[X] := \left\{
      \sum_{k=0}^{n} a_k X^{k} \big\vert a_k \in R n \in \N
    \right\}
  \end{align*}
  with coefficient-wise addition and Cauchy-multiplication
  \begin{align*}
    \left(
      \sum_{k=0}^{n}a_k X^{k}
    \right)
    \left(
      \sum_{k=0}^{m}b_k X^{k}
    \right)
    =
    \sum_{k=0}^{n+m}
    c_k X^{k}
    , \quad c_k = \sum_{i+j = k}a_ib_j
  \end{align*}
\end{dfn}

To construct this ring, we start with the set of all sequences $(a_n)_{n \in \N} \in R^{\N}$ and identify $(0,1,0,\ldots) =: X$.


Every polynomial $f \in R[X]$ induces a function $f:R \to R, x \mapsto  f(x)$, but the mapping
\begin{align*}
  R[X] \to \End_{\Set}(R), f \mapsto (x \mapsto f(x))
\end{align*}
is not injective. (i.e $X^{2} + X \in \F_2[X]$)


The ring of formal power series is denoted by $R\dbrack{X}$


\begin{dfn}
  For $f \in R[X]$ define its \textbf{degree}
  \begin{align*}
    \degree (f) = \sup \{n \in \N \big\vert a_n = 0\}
  \end{align*}
  in particular $\degree (0) = - \infty$.
\end{dfn}

\begin{prop}[]
  If $R$ is an integral domain, then so is $R[X]$ and
  \begin{itemize}
    \item $\degree(fg) = \degree(f) + \degree(g)$
    \item $\degree(f+g) \leq \max \{\degree (f), \degree(g)\}$
    \item $(R[X])^{\times} = R^{\times}$. (In general, only $R^{\times}  \subseteq R[X]^{\times}$,
      For example $2 X + 1 \in \Z/4\Z[X]$ is invertible.)
  \end{itemize}
\end{prop}

\begin{dfn}
For $n \in \N$, define the polynomial ring in $n$-variables inductively as
\begin{align*}
  R[X_1,\ldots,X_n] = \left\{\begin{array}{ll}
    R & n = 0\\
    R[X_1,\ldots,X_{n-1}][X_n] &  n > 0
  \end{array} \right.
\end{align*}
This ring has multiple degree functions, $\degree_{X_1}, \ldots, \degree_{X_n}$ or $\degree_{\text{tot}}$.

For a field $K$, define the field of \textbf{rational functions} in $n$-variables as
\begin{align*}
  K(X 1,\ldots,X_n) &:= \text{Quot}(K[X_1,\ldots,X_n])\\
                    &= \{\frac{f}{g} \big\vert f,g \in K[X_1,\ldots,X_n], g \neq 0\}
\end{align*}
\end{dfn}

\begin{thm}
  For the canonical inclusion $\iota: R \to R[X_1,\ldots,X_n]$, $n$-elements $x_1,\ldots,x_n \in S$, any ringhomormophism $\phi: R \to  S$ induces a unique ringhomomorphism $\overline{\phi}: R[X_1,\ldots,X_n] \to S$ such that the following diagram commutes
  \begin{center}
  \begin{tikzcd}[column sep=0.8em] %\arrow[bend right,swap]{dr}{F}
    R \arrow[]{dr}{\iota} \arrow[]{rr}{\phi} & & S\\
      & R[X_1,\ldots,X_n] \arrow[dotted]{ur}{\exists! \overline{\phi}}
  \end{tikzcd}
  \end{center}
  and $\overline{\phi}(X_i) = x_i$.

\end{thm}
This ringhomomorphism is given by
\begin{align*}
  &\overline{\phi}\left(
    \sum_{k_1,\ldots,k_n = 0}^{m}a_{k_1,\ldots,k_n}X_1^{k_1} \dots X_n^{k_n}
  \right)\\
  &=\sum_{k_1,\ldots,k_n = 0}^{m}
  \phi(a_{k_1,\ldots,k_n}) x_1^{k_1}\dots x_n^{k_n} \in S
\end{align*}


