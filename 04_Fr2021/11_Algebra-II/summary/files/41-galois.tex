\section{Galois Theory}

\begin{dfn}[]
  Let $L/k$ be a field extension. The \textbf{Galois group} of the extension is the subgroup
  \begin{align*}
    \Gal(E/k) := \{\sigma \in \Aut(L) \big\vert \sigma|_k = \id_k\} < \Aut(L)
  \end{align*}
\end{dfn}
\begin{itemize}
  \item $\Gal(\C/\R) = \{\id, \overline{\cdot}\}$
  \item $\Gal(\Q(\sqrt{2})/\Q) = \{\id, (\sqrt{2} \mapsto  - \sqrt{2})\}$
  \item $\Gal(\Q(\sqrt[3]{2})/\Q)) = \{\id\}$, because $X^{3} - 2)$ only has one root in $\Q(\sqrt[3]{2})$.
  \item $\Gal(\Q(\sqrt[3]{2},e^{\frac{2 \pi i }{3}})/\Q) = \{\id, \rho,\rho^{2},\sigma, \sigma \rho, \sigma \rho^{2}\} \iso D_{2 \cdot 3}$, where $\sigma$ is complex conjugation and
    \begin{align*}
      \rho: \sqrt[3]{2} \mapsto e^{\frac{2 \pi i }{3}}\sqrt[3]{2}
    \end{align*}
\end{itemize}


\begin{dfn}[]
  Let $f \in k[X]$ and $L/k$ such that $f$ splits in $L$. Write
  \begin{align*}
    R(f) := \{x \in L \big\vert f(x) = 0\} 
  \end{align*}
  for its collection of roots.
\end{dfn}

\begin{lem}[]
  Let $E/k$ be a splitting field of a polynomial $f \in k [X]$.
  Then every element $\sigma \Gal(E/k)$ induces a permutation on the roots of $f$ and the restriction mapping
  \begin{align*}
    \Gal(E/k) \to  S_{R(f)}, \quad \sigma \mapsto  \sigma|_{R(f)}
  \end{align*}
\end{lem}
\begin{proof}
  Let $R(f) = \{\alpha_{1}, \ldots, \alpha_{n}\}$. Then write
  \begin{align*}
    E = k(\alpha_{1}, \ldots, \alpha_{n}) = \left\{
      \frac{p(\alpha_{1}, \ldots, \alpha_{n})}{q(\alpha_{1}, \ldots, \alpha_{n})} \big\vert p,q \in k[X_{1}, \ldots, X_{n}], q(\alpha_{1}, \ldots, \alpha_{n}) \neq 0
    \right\}
  \end{align*}
  applying any $\sigma \in \Gal(E/k)$ on a rational function obviously keeps it invariant.
\end{proof}




\subsection{Separablility}
Let $p$ be prime and consider
\begin{align*}
  f(X) = X^{p} - t \in \F_p(t)[X]
\end{align*}
If $a$ is a root in some splitting field, we have
\begin{align*}
  (X - a)^{p} = \text{Fr}_p(X-a) = X^{p} - a^{p} = X^{p} -t
\end{align*}
which means it only has one root $R(f) = \{a\}$.


\begin{dfn}[]
  A polynomial $f \in k[X]$ is said to have no \textbf{multiple roots} (mehrfachen Nullstellen), if in a splitting field $\abs{R(f)} = \degree f$.
\end{dfn}


\begin{lem}[]
  For $f \in k[X]$ and $f'\in k[X]$ its formal derivative
  \begin{align*}
    f \text{ has no multiple roots } \iff \gcd(f,f') \in k[X]^{\times}
  \end{align*}
\end{lem}
\begin{proof}
  For $f(X) = a \prod_{i=1}^{n}(X - \lambda_i) \in \overline{k}[X]$ we have
  \begin{align*}
    f'(X) = a \sum_{i=1}^{n} \prod_{\underset{j\neq i}{j = 1}}^{\infty} (X - \lambda_j)
  \end{align*}
\end{proof}
\begin{cor}[]
  Let $f \in k[X]$ irreducible.
  Then $f$ has no multiple roots if and only if $f' \neq 0$.
\end{cor}

\begin{dfn}[]
An irreducible polynomial is called \textbf{separable}, if it does not have multiple roots.

A reducible polynomial is called \textbf{separable}, if all its irreducible factors are separable.

\end{dfn}

\begin{itemize}
  \item By the corollary, every polynomial in $\Q[X]$ is separable because $\charac \Q = 0$.
  \item 
\end{itemize}


\begin{thm}[]
  Let $\phi: k \to k_{\ast}$ be a field isomorphism, $f \in k[X], f_{\ast} = \phi_{\ast}(f) \in k_{\ast}[X]$, $E/k$ a splitting field of $f$ and $E_{\ast}$ a splitting field of $f_{\ast}$.

  If $f$ is separable, then there are exactly $[E:k]$ isomorphisms $\Phi: E \to E_{\ast}$ that extend $\phi$.

  In particular, when taking $k_{\ast} =k, \phi = \id_k$ we find
  \begin{align*}
    [E:k] = \abs{\Gal(E/k)}
  \end{align*}
\end{thm}


\begin{cor}[]
  Let $E/k$ be a splitting field of a separable polynomial $f \in k[X]$.
  If $f$ is irreducible, then
  \begin{align*}
    \degree f | \abs{\Gal(E/k)} | (\degree f)!
  \end{align*}
\end{cor}


\begin{thm}[]
Let $p$ be prime, $n \geq 1\in \N$. Then
\begin{align*}
  \Gal(\F_{p^{n}}/\F_p) \iso \Z/n\Z
\end{align*}
and a generating elment of the galois group is the \textbf{Frobenius homomorphism}
\begin{align*}
  \text{Fr}: \F_{p^{n}} \to \F_{p^{n}}, \quad x \mapsto  x^{p}
\end{align*}
\end{thm}


\begin{dfn}[]
  Let $E/k$ be a field extenion, $\alpha \in L$.
  \begin{itemize}
    \item We say $\alpha$ is \textbf{separable} if $\irr(\alpha,k)$ is separable.
    \item We call the extension \textbf{separable}, if every $\alpha \in L$ is separable.
  \end{itemize}
\end{dfn}

\begin{itemize}
  \item If $a$ is a root of $X^{p} -t$, then $(\F_p(t))(a)/\F_p(t)$ is not separable.
\end{itemize}

\begin{prop}[]
  The extension $k(A)/k$ is separable if and only if every $a \in A$ is separable.

  In particular, every field extension with caracteristic $0$ is separable.
  Every algebraic field extension of a finite field is separable.
\end{prop}






