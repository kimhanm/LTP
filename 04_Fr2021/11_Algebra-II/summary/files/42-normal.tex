\subsection{Normal Field extensions}


\begin{dfn}[]
  A field extension $E/k$ is called \textbf{normal}, if $E$ the splitting field of some polynomial $f \in  k[X]$.
\end{dfn}

\begin{prop}[]
  Let $L = k(\alpha_{1}, \ldots, \alpha_{n})$ and $\overline{L}$ an algebraic closure of $L$.
  Then the following are equivalent
  \begin{enumerate}
    \item $L$ is normal
    \item For ally $\alpha \in L$, the minimal polynomial $\irr(\alpha,k)$ splits over $L$.
    \item The minimal polynomial of all $\alpha_i$ splits over $L$
    \item For $\phi \in \Hom(L,\overline{L})$ with $\phi|_k = \id_k$ it holds $\phi(L) \subseteq L$
    \item For $\phi \in \Hom(L,\overline{L})$ with $\phi|_k = \id_k$ it holds $\phi(L) = L$
    \item Every irreducible polynomial in $k[X]$ with one root in $L$ splits over $L$
  \end{enumerate}
\end{prop}

\begin{itemize}
  \item $\Q(\sqrt[3]{2})/\Q$ is not normal. Because $X^{3} -2$, has a root, but doesn't split.
\end{itemize}


\begin{prop}[]
Let $B/E/K$ be algebraic extensions. Then
\begin{align*}
  B/k \text{ normal } \implies B/E \text{ normal}
\end{align*}
\end{prop}
Other implications are not true. Take for example
\begin{align*}
  \Q(\sqrt[3]{2},e^{\frac{2 \pi i}{3}})/\Q(\sqrt[3]{2})/\Q
\end{align*}
shows that $E/k$ is not normal.

The tower
\begin{align*}
  \Q(\sqrt[4]{2})/\Q(\sqrt[2]{2})/\Q
\end{align*}
has $B/E$ and $E/k$ normal (with $X^{2} -2$ and $X^{2} - \sqrt{2}$, but $B/k$ is not normal.



\begin{dfn}[]
  Let $k$ be a field and $H \subseteq \Aut(k)$ a subset.
  The \textbf{fixing field} (Fixkörper) of $H$ is
  \begin{align*}
    k^{H} := \{x \in k \big\vert \sigma(x) = x \quad \forall  \sigma \in H\} \subseteq E
  \end{align*}
\end{dfn}
The map $H \mapsto k^{H}$ is contravariant with respect to inclusion.
For $E/k$, we always have $k \subseteq E^{\Gal(E/k)}$.

\begin{dfn}[]
  A finite field extension $L/k$ is called \textbf{Galois} (galoissch), if $L$ is the splitting field of a separable polynomial in $k[X]$.
\end{dfn}

\begin{prop}[]
  Let $E/k$ be a finite field extension. 
  Then the following are equivalent

  \begin{enumerate}
    \item $E/k$ is a Galois extension
    \item $E/k$ is normal a separable extension.
    \item $E^{\Gal(E/k)} = k$
    \item $[E/k] = \abs{\Gal(E/k)}$
  \end{enumerate}
\end{prop}


\begin{prop}[]
  Given a tower $B/E/K$
  \begin{align*}
    B/k \text{ Galois } \implies B/E \text{ galois}
  \end{align*}
\end{prop}
The counterexamples for the other implications are the same as for normal towers.


\begin{dfn}[]
For a group $G$ define the collection of subgroups of $G$
\begin{align*}
  \text{Sub}(G) := \{H \text{ group} \big\vert H < G\}
\end{align*}
Fr a field extension $E/k$ define the collection of intermediate fields
\begin{align*}
  \text{Int}(E/k) := \{B \text{ field} \big\vert E/B/k\}
\end{align*}
\end{dfn}


\begin{thm}[Galois Correspondence]
  Let $E/k$ be a finite Galois extension.
  \begin{enumerate}
    \item There are bijective maps (that are contravariant with respect to inclusion)
      \begin{align*}
        \text{Sub}(\Gal(E/k)) &\leftrightarrow \text{Int}(E/k)\\
        H &\mapsto  E^{H}\\
        B &\mapsfrom \Gal(E/B)
      \end{align*}
    \item $B \in \text{Int}(E/k)$ is Galois if and only if $\Gal(E/B)$ is a normal subgroup of $\Gal(E/k)$. 
      If that is the case, then
      \begin{align*}
        \faktor{\Gal(E/k)}{\Gal(E/B)} \iso \Gal(B/k)
      \end{align*}
  \end{enumerate}
\end{thm}

\begin{ex}[]
  For the finite galois extension $E/k :=\Q(\sqrt[3]{2},\zeta_3)/\Q$. Find all intermediate fields.

  To do so, set
  \begin{align*}
    a_1 := \sqrt[3]{2}, a_2 := \zeta_3 \sqrt[3]{2}, a_3 := \zeta_3^{2} \sqrt[3]{2}
  \end{align*}
  and let $T$ be complex conjugation, $R$ multiplication with $\zeta_3$.
  We already know that
  $\Gal(E/k) = \scal{T,R} \iso D_3$

  \begin{center}
  \begin{tikzcd}[column sep=0.6em] %\arrow[bend right,swap]{dr}{F}
    && \{e\} 
    \arrow[hook,swap,color=orange]{dll}{2}
    \arrow[hook,swap,color=orange]{dl}{2}
    \arrow[hook,color=orange]{d}{2}
    \arrow[hook,color=orange]{ddr}{3}
    \\
    \scal{T}
    \arrow[hook]{ddrr}{3}
    & \scal{T,R}
    \arrow[hook]{ddr}{3}
    & \scal{T,R^{2}}
    \arrow[hook]{dd}{3}
    \\
    &&& \scal{R}
    \arrow[hook,swap,color=orange]{dl}{2}
    \\
    && \Gal(E/k)
  \end{tikzcd}
  \end{center}
  and the corresponding diagram for the fixing fields is given by
  \begin{center}
  \begin{tikzcd}[column sep=0.8em] %\arrow[bend right,swap]{dr}{F}
   &&\Q(\sqrt[3]{2},\zeta_3)
   \\
   \Q(\sqrt[3]{2})
   \arrow[hook,color=orange]{urr}{2}
   &
   \Q(\zeta_3,\sqrt[3]{2})
   \arrow[hook,color=orange]{ur}{2}
   &
   \Q(\zeta_3^{2},\sqrt[3]{2})
   \arrow[hook,color=orange]{u}{2}
   \\
   &&& \Q(\zeta_3)
   \arrow[hook,color=orange]{uul}{3}
   \\
   && \Q
   \arrow[hook]{uull}{3}
   \arrow[hook]{uul}{3}
   \arrow[hook]{uu}{3}
   \arrow[hook,color=orange]{ur}{2}
  \end{tikzcd}
  \end{center}
  where the coloured morphisms are the normal subgroup inclusions, or the Galois extensions, respectively.

  Where the numbers are the index of the subgroups, or the degree of the field extension.
\end{ex}




