\subsection{Nilpotent and resolvable groups}

\begin{dfn}[]
A group is called \textbf{nilpotent} of \textbf{order} $1$, if $G$ is abelian.

We say $G$ is nilpotent of order $n+1$, if $G/Z(G)$ is nilpotent of order $n$.
\end{dfn}


\begin{dfn}[]
Let $G$ be a group and $p \in \N$ prime. We say $G$ is a $\bm{p}$-\textbf{group} if $\abs{G} = p^{k}$ for some $k \geq 0 \in \N$.
\end{dfn}

\begin{prop}[]
Every $p$-group is nilpotent.
\end{prop}
\begin{proof}[Proof Sketch]
  Under conjgation, $G$ admits a group action on itself. Then the center $Z(G)$ is exactly the fixpoints of the group action. 
  With
  \begin{align*}
    \abs{G} = \abs{\text{Fix}_G(G)} + \sum_{\text{non-trivial orbits}} [G: \text{Stab}_G(x)]
  \end{align*}
  since non-trivial orbits divide $\abs{G} = p^{k}$ and have index $> 1$, when factoring out the center, the group $G/Z(G)$ becomes strictly smaller.
\end{proof}

\begin{dfn}[]
A sequence of chains of normal subgroups
\begin{align*}
  \{e\} = G_0 \lhd G_1 \lhd G_2 \lhd  \ldots \lhd  G_n = G
\end{align*}
is called a \textbf{subnormal series} (Subnormalreihe).

A group $G$ is called \textbf{resolvable} (auflösbar), if there exists a subnormal series such that the $G_{k+1}/G_k$ are abelian groups.
\end{dfn}

\begin{itemize}
  \item The dihedral group $D_{2n}$ is resolvable with $\{e\} \lhd \scal{R} \lhd D_{2n}$.
  \item The affine group $A_k = \{\begin{psmallmatrix}
   a &b \\
   0& 1
  \end{psmallmatrix} \big\vert a \in R^{\times}, b \in R
  \}
  $ is resolvable and is not nipotent if $\abs{R^{\times}} > 1$.
\item $S_4$ is resolvable with $\{1\} \lhd  \scal{(12)(34),(13)(24)} \lhd A_4 \lhd  S_4$.
\end{itemize}

\begin{prop}[]
  Let $G$ be a group. Then $[G,G] \lhd G$ and $G/[G,G]$ is abelian.

  Moreover, it is the ``largest'' abelian factor group:
  If $H$ is an abelian group and $\phi: G \to  H$ is a group homomorphism, then
  $[G,G] \subseteq \Ker \phi$ and there exists a group homomorphism $\overline{\phi}: G/[G,G] \to  H$ such that the following diagram commutes

  \begin{center}
  \begin{tikzcd}[ ] %\arrow[bend right,swap]{dr}{F}
    G \arrow[]{r}{\phi} \arrow[]{d}{\pi} & H\\
    \faktor{G}{[G,G]} \arrow[]{ur}{\overline{\phi}}
  \end{tikzcd}
  \end{center}
\end{prop}

\begin{cor}[]
  A group $G$ is resolvable if and only if the series
  \begin{align*}
    G^{(0)} := G, \quad G^{(n+1)} := [G^{(n)},G^{(n)}]
  \end{align*}
  reaches the trivial subgroup $\{e\}$.
\end{cor}

\begin{prop}[Lego property]
  Let $N \lhd G$ be a normal subgroup.
  Then $G$ is resolvable if and only if $N$ and $G/N$ are resolvable.

\end{prop}
\begin{proof}[Proof sketch]
  By the third isomorphism theorem, there is a correspondence between subgroups of $G/N$ and subgroups that contain $N$.

  So we get subnormal series
  \begin{align*}
    \{e\} \lhd  G_1 \lhd \ldots \lhd N\\
    \{e\} \lhd  H_1/N \lhd  \ldots \lhd G/N
  \end{align*}
  which we can combine to get a subnormal series for $G$.
\end{proof}


